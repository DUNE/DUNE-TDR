\section{Physics Requirements and Goals}
\label{sec:dp-pds-requirements}

The \dword{pds} of the \dune \dword{fd} provides information on three key detection aspects of the experiment:

\begin{enumerate}
\item Event (and sub-event) time reconstruction;
\item Event triggering;
\item Event energy reconstruction.
\end{enumerate}

As discussed below, these detection aspects enabled by the \dword{pds} affect the entire primary physics program of \dune: long-baseline neutrino oscillations, nucleon decay, and \dword{snb}. In the following, we list the scientific requirements the \dword{pds} will meet  (Sec.~\ref{subsec:dp-pds-requirements_requirements}), as well as additional scientific opportunities \dword{pds} offers (Sec.~\ref{subsec:dp-pds-requirements_opportunities}). Among requirements, we only list how \dword{pds} primarily affects the \dune primary physics program, namely: 
\begin{enumerate}
\item A capability or measurement uniquely provided by the \dword{pds};
\item A capability or measurement redundant with \dword{tpc} information, but where redundancy is considered essential;
\item A capability or measurement competitive with \dword{tpc} information.
\end{enumerate}

%%%%%%%%%%%%%%%%%%%%%%%%%%%%%%%%%%%%%%%%%%%%%%%%%%%%%%%%%%%%%%%%%%%%

\subsection{Scientific Requirements}
\label{subsec:dp-pds-requirements_requirements}

\subsubsection{Event fiducialization for non-beam events}

The accelerator complex for beam events provides absolute event time in the \dune \dword{tpc}, but the \dword{pds} must provide absolute event time for non-beam events. Event time reconstruction is necessary to determine the drift distance within the \dword{tpc}. This is essential for a number of reasons, particularly for defining a fiducial volume along the \dword{tpc} drift direction, which is the vertical one for \dword{dp}. The cosmic ray (CR) \fixme{This abbreviation is already used for capacitance resistor. We either need another abbreviation or no abbreviation at all.} activity in one \dword{fd} module is of order \SI{0.05}{\Hz}, or about \SI{1e8}{\per(\Mtyr)}. Although smaller, the atmospheric neutrino event rate in one \dword{fd} module is also significant, of order \SI{1e5}{\per(\Mtyr)}. These rates can be compared with the background levels target for nucleon decay (NDK) \fixme{This abbreviation should be added to the common glossary.} searches at \dune. For the NDK channels where \dune can provide the best sensitivities, the goal is to operate in nearly background-free conditions even after an exposure of several years, that is, with background levels of order \SI{10}{\per(\Mtyr)} or less. Hence, the CR activity must be suppressed by at least seven orders of magnitude to enable optimal NDK searches. \dword{pds} information is essential to reach the required levels of suppression of CR-induced (and atmospheric neutrino-induced) backgrounds. Because most CR activity will enter from the top of the detector, a NDK candidate must be fully contained within the detector. For the most relevant direction, namely the vertical one, the fiducial requirement can only be imposed if \dword{pds} information is available. Because this capability is uniquely provided by the \dword{pds} and because it critically affects the NDK program, {\bf we consider event fiducialization of NDK candidates to be the most important requirement of the \dword{pds}}. High signal efficiency is critical in rare event searches like NDK, so {\bf we require a $\boldsymbol{>90\%}$ efficient event time determination via the \dword{pds} throughout the \dword{tpc} active volume for NDK signal events}. NDK event fiducialization may be prevented not only if NDK-induced \dshort{lar} scintillation flashes are undetected, but also if the \dword{pds} detects spurious \dshort{lar} scintillation flashes within the same event window, uncorrelated with the NDK activity. Additional flashes are produced, typically, by radiologically-induced detector activity. In this case, an ambiguity in the drift time determination of the NDK may arise, and the correct (NDK-induced) flash must be associated to the event. Hence, {\bf we also require a $\boldsymbol{>90\%}$ NDK flash purity among all reconstructed NDK-like flashes within the same NDK event readout window}. Similarly, the \dword{pds} is also needed to determine the full containment of atmospheric neutrino interactions in \dune.
 
\subsubsection{Supernova burst triggering}

A burst of several hundred neutrino interactions per \dune \dword{fd} module is expected over a timescale of about \SI{10}{\s} from a core-collapse supernova at a \SI{10}{\kilo\parsec} distance from the Earth, near the center of our galaxy. As the \dword{snb} interaction rate scales as $1/r^2$, where $r$ is the supernova distance from us, \dword{snb} detection at \dune is in principle possible up to distances of order \SI{50}{\kilo\parsec}. This would be the case for an \dword{snb} occurring in the Large Magellanic Cloud, for example. The dominant neutrino interaction channel in \dune is the \dword{cc} electron neutrino interaction on an argon nucleus, with typical deposited energies from electrons and nuclear de-excitation gammas of order 20~MeV, producing short tracks throughout the \dword{tpc} active volume.

In \dune, one can in principle trigger on \dwords{snb} using either \dword{tpc} or \dword{pds} information. In both cases, the trigger scheme exploits the time coincidence of multiple signals (\dword{tpc} stubs or \dword{pds} optical clusters\footnote{Throughout this chapter, we use the term "optical flash" for the process of scintillation light production in \dshort{lar}. We use "optical hit" to identify a reconstructed optical signal in a single photo-detector channel. We use the term "optical cluster" to name the reconstructed object given by a collection of optical hits seen by separate photo-detectors and that are correlated in time/space.}) over a timescale matching the SN luminosity time evolution. In the case of an \dword{snb} trigger being generated, the data acquisition system would write to disk the full (non-zero-suppressed) detector information for a time range spanning several tens of seconds surrounding the trigger timestamp. Considering the rarity of \dwords{snb} in our galactic neighbourhood, and given the importance of \dword{snb} detection, we consider it essential to develop both a redundant and a highly efficient \dword{snb} triggering scheme in \dword{dune}. Concerning redundancy, {\bf we require that the \dune \dword{fd} may trigger on an \dword{snb} independently using \dword{tpc} or \dword{pds} information}, hence minimizing \dword{snb} inefficiencies stemming from downtime or malfunctioning of the \dword{tpc}, the \dword{pds}, or their associated trigger schemes. As for trigger efficiency, {\bf we require the \dword{pds} alone to be able to trigger with $\boldsymbol{>90\%}$ efficiency on an \dword{snb} at a \SI{20}{\kilo\parsec} distance from Earth}, hence up to distances covering the entire Milky Way. Considering the very large amount of detector information being generated by an \dword{snb} trigger, we also require that the \dword{snb} trigger efficiency be reached for a {\bf fake trigger rate not exceeding \SI{1}{\per month}}.

\subsubsection{Scintillation-based calorimetry}

The most important physics goal of the \dune \dword{fd} is to carry out a comprehensive program of neutrino oscillation measurements using \numu and \anumu beams from Fermilab. Because of this, the \dune \dword{fd} performance in terms of neutrino energy reconstruction and neutrino flavour classification is of paramount importance. As detailed in this volume, the \dword{tpc} should provide a 10--15\% neutrino energy resolution for \nue \dword{cc} interactions in the few-GeV energy range, using charge information alone. At these energies, however, the \dword{pds} will also detect thousands of PEs. Thus, a competitive calorimetric measurement of the event energy can also, in principle, be obtained using the light intensity detected by the \dword{pds}, if spatial non-uniformities in \dword{pds} response can largely be corrected for. {\bf We require the \dword{pds} alone to be able to reconstruct the neutrino energy of a few-GeV electron neutrino \dword{cc} interaction with 20\% RMS resolution or better}, thus providing a measurement competitive to the \dword{tpc}-based one. Such an independent and competitive energy measurement would also be useful as a risk mitigation measure for degraded \dword{tpc} performance. The \dword{tpc}- and \dword{pds}-based energy measurements may be combined, possibly providing better energy resolution than \dword{tpc} alone. We give two arguments supporting this possibility. On one hand, the charge and light signals may be combined to reduce electron-ion recombination fluctuations, ensuring a more compensatory \dword{lar} calorimeter response. On the other hand, charge and light readout planes are located at opposite detector ends in the \dword{dp} design, providing maximal complementarity between the two sub-systems. Neutrino interactions occurring near the light readout at the bottom of the cryostat will be maximally affected by electron attachment of the charge signal, while they will have the highest light detection probabilities possible. The situation is reversed for neutrino interactions near the charge readout plane. The \dword{pds} may provide a competitive energy measurement not only for beam neutrinos, but also at lower energies, particularly for \dword{snb} neutrinos. For a discussion of why an accurate energy measurement is beneficial also for \dword{snb} events, see Sec.~\ref{subsubsec:dp-pds-requirements_attachment}.

%%%%%%%%%%%%%%%%%%%%%%%%%%%%%%%%%%%%%%%%%%%%%%%%%%%%%%%%%%%%%%%%%%%%

\subsection{Additional Scientific Opportunities}
\label{subsec:dp-pds-requirements_opportunities}

\subsubsection{Electron attachment correction for the charge-based energy measurement in non-beam events}
\label{subsubsec:dp-pds-requirements_attachment}

Given the maximum drift distance of \dpmaxdrift in the \dual \dword{tpc} and assuming a (minimally-required) \SI{3}{\ms} electron lifetime, the charge could be attenuated by more the one order of magnitude along drift due to electron attachment on electro-negative impurities. If left uncorrected, electron attachment would therefore cause great deterioration in the charge-based energy measurement in such relatively short electron lifetimes. Because of the longer drift distance, this effect is much more dramatic in the \dword{dp} than the \dword{sp}. For non-beam events, the only possible way of correcting for the electron attachment is through \dword{pds}-based event timing\footnote{This is true for a single physics event (e.g., a neutrino interaction) per \dword{tpc} event. Other than background flashes, the association between \dword{tpc} tracks and \dword{pds} flashes has no ambiguities.}. This will be exploited in \dune to better discern, for example, the spectral features in the energy spectrum of the \dword{snb} flux. An improved charge-based energy measurement would improve both the determination of the pinched-thermal spectral parameters of the progenitor, and the detection of the sharp, time-dependent spectral features from collective neutrino oscillations. A similar argument applies to the charge-based energy measurement in the study of atmospheric neutrino oscillations, where good neutrino energy reconstruction is also very important.

\subsubsection{Event timing in high multiplicity supernova burst events}

The early time distribution carries important information for core-collapse stellar models, which has not been observed thus far. The first neutrino release after core bounce is the \nue-rich neutronization burst, lasting about \SI{10}{\ms}. For a supernova at \SI{10}{\kilo\parsec} distance, several tens of interactions induced by the neutronization flux should be detected in one \dword{fd} module. In other words, the mean interval between successive neutrino interactions should be less than \SI{1}{\ms} in this case. For closer supernovae, information at even earlier (pre-bounce) times may be studied during the core infall phase. A \SI{1}{\ms} wide notch in the luminosity curve may be visible, corresponding to neutrino trapping in the ultra-dense matter. Such timescales can be compared with the typical time resolutions obtained with \dword{tpc} and \dword{pds} information. For \dword{dp} \dword{tpc} and for a \SI{1.6}{mm/\micro\s} electron drift velocity, the drift time can be as large as \SI{7.5}{\ms}, with foreseen readout windows of \dpreadout. Considering the uniform spatial distribution of neutrino interactions in the \dword{tpc}, the intrinsic absolute time resolution expected in this case is of order $\dpreadout/\sqrt{12}\simeq \SI{5}{\ms}$. The \dword{tpc}-based time resolution is therefore not sufficient to resolve these early time features, even in the case of sufficient neutrino statistics. On the other hand, the \dword{pds} may reconstruct one flash per \dword{snb} neutrino interaction at \SI{1}{\micro\s}-scale time resolution or more. Hence, \dword{pds}-based timing is definitely an added value for a nearby supernova.

\subsubsection{Improved event identification via scintillation-based Michel electron tagging}

There is another physics case where the \dword{pds} may detect more than one physics-induced flash per \dword{tpc} event\footnote{We use the qualifier 'physics-induced' to distinguish those light flashes from the ones induced by radiological or cosmogenic activity in the detector or by electronics noise.}. This can happen in the presence of particle decays in the final state of an event of interest, such as a neutrino interaction or a nucleon decay. In this case, multiple, correlated sub-events may be separated in time. Michel electron tagging from muon decay at rest is the most obvious opportunity for sub-event identification using \dword{pds} timing. Approximately one million scintillation photons are produced in the liquid argon per Michel electron. In addition, the timescale of muon decay is long, comparable to the timescale of the slow component of argon scintillation light. Although Michel electrons can also be identified through \dword{tpc} track imaging, in some event topologies,  \dword{pds} may outperform the \dword{tpc}, for example where the muon and electron tracks are nearly parallel. In summary, the \dword{pds} may better identify the event final state than \dword{tpc}-only information. Considering the large fraction of $\mu^-$ capture on argon (about 75\%), Michel electron tagging also provides valuable $\mu^-/\mu^+$ separation. The latter can, in turn, provide separation between muon neutrinos and antineutrinos. 
