\section{Project Management}
\label{sec:dp-pds-management}

The \dual \dword{pd} consortium was formed in 2017 and it is composed of eleven institutes from France, Peru, Spain, UK and USA. The charge of the \dual \dword{pd} consortium is to plan and execute the construction, installation and commissioning of the \dword{dp} \dword{pds}.
% I think we should check with Peru and UK if they are still interested in being included in the consortium

%%%%%%%%%%%%%%%%%%%%%%%%%%%%%%%%%
\subsection{Consortium Organization}

The current \dual \dword{pd} consortium Leader (CL) is %In\'{e}s Gil-Botella
 from CIEMAT (Spain) and the Technical Lead (TL) is %Dominique Duchesneau 
 from LAPP (France). They are members of the \dword{dune} Technical Board and they represent the consortium to the overall \dword{dune} collaboration. The CL is responsible for the subsystem deliverables and for the effective management of the consortium. The TL acts as the overall project manager and is the interface to the International Project Office (IPO); he is responsible for monitoring and reporting on progress with respect to the agreed schedule and for issues related to interface documentation.

The institutions participating in the consortium are responsible for the design and/or construction of a particular subsystem. It is hoped that the national groups within the consortia will be able to approach relevant funding agencies with a specific construction-phase proposal, such that a likely funding line can be established in 2019. The \dual \dword{pd} consortium is open to any new institution willing to join the current effort.

The current institutions participating in the \dual \dword{pd} consortium are LAPP (France); PUCP (Peru); IFAE, CIEMAT, and IFIC (Spain); UCL (UK); and ANL, Duke U., U. of Iowa, SDSMT, and UTA (USA).

The \dual \dword{pd} consortium is divided into four working groups: photosensors and electronics, calibration system, mechanics and integration, and simulation and physics. The corresponding current WG convener institutions are:
\begin{itemize}
% AH REMOVING NAMES
\item WG1: Photosensors and Electronics -  CIEMAT %A. Verdugo
\item WG2: Calibration System -  CIEMAT %C. Cuesta
\item WG3: Mechanics and Integration - U. of Iowa %B. Bilki 
\item WG4: Sim. \& Phys. - %K. Scholberg (
Duke U., %M. Sorel (
IFIC, %L. Zambelli (
LAPP
\end{itemize}

%%%%%%%%%%%%%%%%%%%%%%%%%%%%%%%%%
%\subsection{Planning Assumptions}
%\label{sec:fddp-pd-12.2}

%The optimization and final design of the \dual \dword{pd} system will be driven by the \dword{pddp} data, expected by Summer 2019.

%\dword{pddp} operation and data analysis are fundamental steps to understanding whether the current \dword{pds} design considered as baseline, based on cryogenic \dwords{pmt} with \dword{tpb} coating, is able to provide $t_0$ for non-beam events, background rejection and triggering on non-beam events. These data will be used to tune the \dword{mc} simulations and extrapolate the performance of the system to the \dword{dpmod}.


\subsection{Planning Assumptions}
\label{sec:fddp-pd-12.2}

The baseline design of the \dual \dword{pds} is a complex optimization based on the previous results and experience with the first tonne scale dual phase \dword{lar} \dword{tpc} demonstrator (\dword{wa105}), critical evaluation of the design and construction of \dword{wa105} and \dword{pddp}, physics objectives of the \dune experiment, and the detailed simulation studies of \dune \dword{dp} as well as \dword{wa105} and \dword{pddp}.

The previous design and construction of the \dual modules have enabled us to critically evaluate several scenarios and develop the \dune \dword{pds} design such that it is optimized for maximum physics performance; simple construction, transportation, handling and installation; and easy and robust operation.

Simulations are effectively used in the design and optimization of the \dual \dword{pd} system to meet the physics requirements in terms of:
\begin{itemize}
\item light collection efficiency,
\item number of channels,
\item photosensor requirements,
\item dynamic range of readout electronics and timing resolution, and 
\item trigger strategy on non-beam events.
\end{itemize}

Although no major design modifications are foreseen, the \dword{pddp} operations will be closely monitored to fine-tune the \dune \dual design.

%%%%%%%%%%%%%%%%%%%%%%%%%%%%%%%%%
\subsection{WBS and Institutional Responsibilities}

The \dual \dword{pd} consortium has developed a detailed breakdown of deliverables and responsibilities \citedocdb{5606} included in the overall \dword{dune} collaboration \dword{wbs} \citedocdb{5594}%\cite{bib:docdb5594} 
coordinated by the IPO. The main deliverables are %based on the \dword{pddp} \dword{pds}  and are 
divided into seven topics. These topics are listed below:
%These are listed along with the participating institutions below: 

%\dword{pddp} \dword{pds} 
\begin{enumerate}
\item Management \dual \dword{pds} (includes milestones and review dates)
\item Physics and Simulations
\item Design, Engineering, R\&D and validation tests
\item Production Setup (includes tooling)
\item Production (includes component production, assembly, testing, and \dword{qc})
\item Integration (contributions to activities at global integration facility)
\item Installation (contributions to activities at \surf)

%\item Management \dual \dword{pds} (includes milestones and review dates) \textit{- LAPP, CIEMAT }
%\item Physics and Simulations \textit{- Duke, LAPP, IFIC, SDSMT, CIEMAT, PUCP, UCL, Texas-Austin}
%\item Design, Engineering, R\&D and validation tests \textit{- Iowa, CIEMAT, IFIC, UCL, Texas-Austin, IFAE, SDSMT}
%\item Production Setup (includes tooling) \textit{- UCL}
%\item Production (includes component production, assembly, testing, and \dword{qc}) \textit{- Iowa, CIEMAT, IFAE, IFIC, UCL, Texas-Austin, Duke, SDSMT, LAPP}
%\item Integration (contributions to activities at global integration facility) \textit{- SDSMT}
%\item Installation (contributions to activities at \surf) \textit{- CIEMAT, IFIC, SDSMT, Iowa}
\end{enumerate}

\subsection{High Level Schedule}

%The \dual \dword{pds} consortium's main activities during the next \num{16} months are focused on developing the \dword{tdr}. 
%The main high-level milestones are detailed in Table~\ref{tab:dppd_t_12_5} for the pre-\dword{tdr} period. The plan for the activities in the post-\dword{tdr} period is summarized in Table~\ref{tab:dppd_t_12_6}.

The main high-level milestones for the pre-installation and installation periods are summarized in Tables~\ref{tab:dppd_t_12_n1} and \ref{tab:dppd_t_12_n2} respectively. The installation period schedule is expressed as the start month and the end month with respect to the start of installation of the \dune \dual.

%\begin{dunetable}
%[Pre-\dword{tdr} key milestones]
%{|l|l| p{0.8\textwidth}}
%{tab:dppd_t_12_5}
%{Pre-\dword{tdr} key milestones (TO BE UPDATED)}

%Milestone & End date \\ \toprowrule
%Simulations and physics: %Finalize the 
%Implementation of \dual optical & \\
%simulation in \larsoft for \dword{pddp} & 08/2018 \\ \colhline
%Simulations and physics: Optimization of the & \\
%\dword{dpmod} performance to fulfill the physics requirements and & \\
%definition of a trigger strategy & 05/2019 \\ \colhline
%Photosensors: Components selection and final design & 03/2019 \\ %\colhline
%\dword{pmt} calibration system design and selection of components & 03/2019 \\ \colhline
%Cabling definition and design of flange & 03/2019 \\ \colhline
%Design review in light of \dword{pddp} calibration data & 03/2019 \\ %\colhline
%\dword{qc} plan & 06/2018 \\ \colhline
%Identification of Interfaces & 06/2018 \\ \colhline
%Integration, installation and commissioning plans & 12/2018 \\ \colhline
%\dword{dpmod} \dword{tdr} & 06/2019 \\ 
%\end{dunetable}

%\fixme{Remove table with pre-TDR milestones, Tab.~\ref{tab:dppd_t_12_5}?}

%\begin{dunetable}
%[Post-\dword{tdr} key milestones]
%{|l|l|l| p{0.8\textwidth}}
%{tab:dppd_t_12_6}
%{Post-\dword{tdr} key milestones}

%Milestone & Start date & End date \\ \toprowrule
%\textbf{\dword{pmt} preparation and installation} (can be done in batches) & & \\ \colhline
%\dword{pmt} procurement procedure and production & 01/2021 & 12/2022 \\ %\colhline
%\dword{pmt} base design and manufacturing & 01/2022 & 12/2022 \\ %\colhline
%\dword{pmt} support structure production and assembly & 08/2022 & 01/2023 \\ \colhline
%\dword{pmt} characterization - \num{10} \dwords{pmt}/week (two facilities) & 02/2023 & 12/2023 \\ \colhline
%\dword{tpb} coating (two facilities similar to that for CERN ICARUS) & 01/2024 & 12/2024 \\ \colhline
%Splitter production and tests & 05/2024 & 12/2024 \\ \colhline
%\textbf{Installation at \surf} & & \\ \colhline
%\dword{pmt} cable and fiber routing in cryostat from flange to bottom & & \\
%                  (depends on \dword{fc} and flange installation) & 09/2024 & 09/2024 \\ \colhline
%\dword{pmt} testing, installation in cryostat and cabling (\num{72} %\dwords{pmt}/month) & 10/2024 & 07/2025 \\ \colhline
%\dword{pmt} support installation on the membrane & & \\
%                  (in parallel by sector with \dword{pmt} installation) & 10/2024 & 07/2025 \\ \colhline
%Splitter installation & & \\
%                  (in parallel with \dword{pmt} installation to test cabling and connections) & 10/2024 & 07/2025 \\ \colhline
%\textbf{Light calibration system} & & \\ \colhline
%Fibers, light source tests and procurement & 06/2023 & 05/2024 \\ %\colhline
%Fiber calibration system installation & & \\
%                  (in parallel with \dword{pmt} installation with validation test) & 09/2024 & 07/2025 \\ 
%\end{dunetable}



\begin{dunetable}
[Pre-Installation Period Key Milestones]
{|l|l|l| p{0.8\textwidth}}
{tab:dppd_t_12_n1}
{Pre-Installation Period Key Milestones}

Milestone & Start date & End date \\ \toprowrule
\textbf{\dword{pmt} preparation and installation} (can be done in batches) & & \\ \colhline
\dword{pmt} procurement procedure and production & 01/2021 & 12/2022 \\ \colhline
\dword{pmt} base design and manufacturing & 01/2022 & 12/2022 \\ \colhline
\dword{pmt} support structure production and assembly & 08/2022 & 01/2023 \\ \colhline
\dword{pmt} characterization - \num{10} \dwords{pmt}/week (two facilities) & 02/2023 & 12/2023 \\ \colhline
\dword{tpb} coating (two facilities similar to that for CERN ICARUS) & 01/2024 & 12/2024 \\ \colhline
Splitter production and tests & 05/2024 & 12/2024 \\ \colhline
\textbf{Light calibration system} & & \\ \colhline
Fibers, light source tests and procurement & 06/2023 & 05/2024 \\ \colhline
\end{dunetable}

\begin{dunetable}
[Installation Period Key Milestones]
{|l|c|c| p{0.8\textwidth}}
{tab:dppd_t_12_n2}
{Installation Period Key Milestones}

Milestone & Start month & End month \\ \toprowrule
\dword{pmt} cable and fiber routing in cryostat from flange to bottom & & \\
                  (depends on \dword{fc} and flange installation) & 1 & 2 \\ \colhline
\dword{pmt} testing, installation in cryostat and cabling (\num{120} \dwords{pmt}/month) & 2 & 8 \\ \colhline
\dword{pmt} support installation on the membrane & & \\
                  (in parallel by sector with \dword{pmt} installation) & 2 & 8 \\ \colhline
Splitter installation & & \\
                  (in parallel with \dword{pmt} installation to test cabling and connections) & 2 & 8 \\ \colhline
Fiber calibration system installation & & \\
                  (in parallel with \dword{pmt} installation with validation test) & 2 & 8 \\ 
\end{dunetable}


%\fixme{Update Tab.~\ref{tab:dppd_t_12_6} according to new schedule, with installation of 2nd module starting on 08/2025.}

\subsection{High Level Cost of Baseline Design}

An initial cost estimate of the \dual \dword{pd} photon detection system for one 10kt DUNE detector was developed in 2018 (Table~\ref{tab:dppd_t1.7}). This is based on protoDUNE-DP costs.

\begin{dunetable}
[Dual Phase Photon Detection System Cost Summary]
{|l|l| p{0.8\textwidth}}
{tab:dppd_t1.7}
{Dual Phase Photon Detection System Cost Summary (TO BE COMPLETED)}

Item & Core Cost (k\$ US)\\ \toprowrule
Item 1 & 1.0 \\
Item 2 & 1.0 \\
Item 3 & 1.0 \\
... & ... \\
\end{dunetable}

\fixme{Complete costs table, Tab.~\ref{tab:dppd_t1.7}.}