\section{Project Management}
\label{sec:dp-pds-management}

The \dual \dword{pd} consortium was formed in 2017;  it comprises eleven institutions from France, Peru, Spain, UK, and USA. The charge of the \dual \dword{pd} consortium is to plan and execute the construction, installation, and commissioning of the \dual \dword{pds}.
% I think we should check with Peru and UK if they are still interested in being included in the consortium

%%%%%%%%%%%%%%%%%%%%%%%%%%%%%%%%%
\subsection{Consortium Organization}

The current \dual \dword{pd} consortium leader is %In\'{e}s Gil-Botella
 from CIEMAT (Spain) and the technical lead is %Dominique Duchesneau 
 from LAPP (France). They are members of the \dword{dune} technical board and represent the consortium in the overall \dword{dune} collaboration. The consortium leader is responsible for the subsystem deliverables and for effectively managing the consortium. The technical lead acts as the overall project manager and is the interface to the international project office; he is responsible for monitoring and reporting on progress on the agreed schedule and for interface documentation.

The institutions participating in the consortium are responsible for the design and/or construction of a particular subsystem. The national groups within the consortium plan to approach relevant funding agencies with a specific construction-phase proposal to obtain a likely funding line in 2019. The \dual \dword{pd} consortium is open to any new institution willing to join the current effort.

The current institutions participating in the \dual \dword{pd} consortium are listed in Table \ref{tab:dppd_instit}.

\begin{dunetable}
[Participating institutes]
{ll}
{tab:dppd_instit}
{Institutions participating in the \dual \dword{pds} consortium.}   
Institution & Country \\ \toprowrule
LAPP & France \\ \colhline 
PUCP & Peru \\ \colhline
IFAE & Spain \\ \colhline
CIEMAT & Spain \\ \colhline
IFIC & Spain \\ \colhline%
UCL & United KIngdom \\ \colhline
Argonne National Laboratory & USA \\ \colhline
Duke University & USA \\ \colhline
University of Iowa & USA \\ \colhline
SDSMT & USA \\ \colhline
University of Texas at Austin & USA \\
\end{dunetable}

The \dual \dword{pd} consortium is divided into four working groups: photosensors and electronics; calibration system; mechanics and integration; and simulation and physics. The corresponding current Working Group convener institutions are
\begin{itemize}
% AH REMOVING NAMES
\item WG1: Photosensors and Electronics -  CIEMAT, %A. Verdugo
\item WG2: Calibration System -  CIEMAT, %C. Cuesta
\item WG3: Mechanics and Integration - University of Iowa%, and %B. Bilki 
\item WG4: Simulation and Physics - %K. Scholberg (
Duke University, %M. Sorel (
IFIC, %L. Zambelli (
LAPP.
\end{itemize}

%%%%%%%%%%%%%%%%%%%%%%%%%%%%%%%%%
%\subsection{Planning Assumptions}
%\label{sec:fddp-pd-12.2}

%The optimization and final design of the \dual \dword{pd} system will be driven by the \dword{pddp} data, expected by Summer 2019.

%\dword{pddp} operation and data analysis are fundamental steps to understanding whether the current \dword{pds} design considered as baseline, based on cryogenic \dwords{pmt} with \dword{tpb} coating, is able to provide $t_0$ for non-beam events, background rejection and triggering on non-beam events. These data will be used to tune the \dword{mc} simulations and extrapolate the performance of the system to the \dword{dpmod}.


\subsection{Planning Assumptions}
\label{sec:fddp-pd-12.2}

The baseline design of the \dual \dword{pds} is a complex optimization based on the results from and experience with the first tonne scale dual phase \dword{lar} \dword{tpc} demonstrator (\dword{wa105}), critical evaluation of the design and construction of \dword{wa105} and \dword{pddp}, physics objectives of the \dune experiment, and the detailed simulation studies of \dune \dword{dp} as well as \dword{wa105} and \dword{pddp}.

The previous design and construction of the \dual modules have enabled us to critically evaluate several scenarios and develop the \dune \dword{pds} design optimized for maximum physics performance; simple construction, transportation, handling, and installation; and easy and robust operation.

Simulations are effectively used in designing and optimizing the \dual \dword{pd} system to meet the physics requirements:
\begin{itemize}
\item light collection efficiency,
\item number of channels,
\item photosensor requirements,
\item dynamic range of readout electronics and timing resolution, and 
\item trigger strategy on non-beam events.
\end{itemize}

Although no major design modifications are foreseen, the \dword{pddp} operations will be closely monitored to fine-tune the \dune \dual design. The baseline design will be validated with \dword{pddp} data in March \num{2020}. The final components of the system will be installed, tested and validated during the second \dword{pddp} run at \dword{cern} in March \num{2023}. Internal design reviews will be conducted after accomplishing these milestones.  

%%%%%%%%%%%%%%%%%%%%%%%%%%%%%%%%%
%\subsection{\dword{wbs} and Institutional Responsibilities}
\subsection{Work Breakdown Structure and Institutional Responsibilities}

The \dual \dword{pd} consortium has developed a detailed breakdown of deliverables and responsibilities \citedocdb{5606} included in the overall \dword{dune} collaboration \dword{wbs} \citedocdb{5594}%\cite{bib:docdb5594} 
coordinated by the international project office. The main deliverables are %based on the \dword{pddp} \dword{pds}  and are 
divided into seven topics: 
%These are listed along with the participating institutions below: 

%\dword{pddp} \dword{pds} 
\begin{enumerate}
\item Management \dual \dword{pds} (includes milestones and review dates),
\item Physics and simulations,
\item Design, engineering, R\&D, and validation tests,
\item Production set up (includes tooling),
\item Production (includes component production, assembly, testing, and \dword{qc}),
\item Integration (contributions to activities at global integration facility), and
\item Installation (contributions to activities at \surf).

%\item Management \dual \dword{pds} (includes milestones and review dates) \textit{- LAPP, CIEMAT }
%\item Physics and Simulations \textit{- Duke, LAPP, IFIC, SDSMT, CIEMAT, PUCP, UCL, Texas-Austin}
%\item Design, Engineering, R\&D and validation tests \textit{- Iowa, CIEMAT, IFIC, UCL, Texas-Austin, IFAE, SDSMT}
%\item Production Setup (includes tooling) \textit{- UCL}
%\item Production (includes component production, assembly, testing, and \dword{qc}) \textit{- Iowa, CIEMAT, IFAE, IFIC, UCL, Texas-Austin, Duke, SDSMT, LAPP}
%\item Integration (contributions to activities at global integration facility) \textit{- SDSMT}
%\item Installation (contributions to activities at \surf) \textit{- CIEMAT, IFIC, SDSMT, Iowa}
\end{enumerate}

\subsection{High Level Schedule}


The main high-level milestones of the \dual \dword{pds} and the dates by which they will be accomplished are summarized in Table~\ref{tab:Xsched}.

\begin{dunetable}
[\dual \dshort{pd} consortium schedule]
{p{0.65\textwidth}p{0.25\textwidth}}
{tab:Xsched}
{\dual \dword{pd} Consortium Schedule}   
Milestone & Date (Month YYYY)   \\ \toprowrule
Initial design validation with \dword{pddp} data & March 2020 \\ \colhline
%Technology Decision Dates &      \\ \colhline
%Final Design Review Dates &      \\ \colhline
%Start of module 0 component production for ProtoDUNE-II &      \\ \colhline
%End of module 0 component production for ProtoDUNE-II &      \\ \colhline
\rowcolor{dunepeach} Start of \dword{pdsp}-II installation& \startpduneiispinstall      \\ \colhline
\rowcolor{dunepeach} Start of \dword{pddp}-II installation& \startpduneiidpinstall      \\ \colhline
% \dword{prr} dates &      \\ \colhline
%Start of  (component 1) production  &      \\ \colhline
%Start of (component 2) production  &      \\ \colhline
%Start of  (component 3) production  &      \\ \colhline
\rowcolor{dunepeach}South Dakota Logistics Warehouse available& \sdlwavailable      \\ \colhline
\rowcolor{dunepeach}Beneficial occupancy of cavern 1 and \dword{cuc}& \cucbenocc      \\ \colhline
Final design validation with \dword{pddp} II data & March 2023 \\ \colhline
\rowcolor{dunepeach} \dword{cuc} counting room accessible& \accesscuccountrm      \\ \colhline


\rowcolor{dunepeach}Top of \dword{detmodule} \#1 cryostat accessible& \accesstopfirstcryo      \\ \colhline

\dword{pmt} procurement procedure and production & June 2024 \\ \colhline
\dword{pmt} base design and manufacturing &  June 2024 \\ \colhline
\dword{pmt} support structure production and assembly & July 2024 \\ \colhline


%End of  (component 1) production  &      \\ \colhline
%... & ...                       \\ \colhline

\rowcolor{dunepeach}Start of \dword{detmodule} \#1 TPC installation& \startfirsttpcinstall      \\ \colhline


\rowcolor{dunepeach}Top of \dword{detmodule} \#2 accessible& \accesstopsecondcryo      \\ \colhline

\dword{pmt} characterization & April 2025 \\ \colhline
Fibers, light source tests and procurement & April 2025 \\ \colhline
\rowcolor{dunepeach}End of \dword{detmodule} \#1 TPC installation& \firsttpcinstallend      \\ \colhline
Splitter production and tests & June 2025 \\ \colhline
\dword{tpb} coating of the \dwords{pmt} & July 2025 \\ \colhline
\dword{tpb} coating of the reflector/\dword{wls} panels & July 2025 \\ \colhline

 \rowcolor{dunepeach}Start of \dword{detmodule} \#2 TPC installation& \startsecondtpcinstall      \\ \colhline
 
\dword{pmt} cable and fiber routing in cryostat from flange to bottom & October 2025 \\ \colhline
\dword{pmt} testing, installation in cryostat and cabling & April 2026 \\ \colhline
\dword{pmt} support installation on the membrane & April 2026 \\ \colhline
Splitter installation & April 2026 \\ \colhline
Fiber calibration system installation & April 2026 \\ 
 
\rowcolor{dunepeach}End of \dword{detmodule} \#2 TPC installation& \secondtpcinstallend      \\ \colhline

%last item & ...                         \\
\end{dunetable}



%The \dual \dword{pds} consortium's main activities during the next \num{16} months are focused on developing the \dword{tdr}. 
%The main high-level milestones are detailed in Table~\ref{tab:dppd_t_12_5} for the pre-\dword{tdr} period. The plan for the activities in the post-\dword{tdr} period is summarized in Table~\ref{tab:dppd_t_12_6}.

%The main high-level milestones for the pre-installation are summarized in Table~\ref{tab:dppd_t_12_n1} and installation periods in  Table~\ref{tab:dppd_t_12_n2}. The installation period schedule is expressed as the start month and the end month with respect to the start of installation of the \dune \dual.

%\begin{dunetable}
%[Pre-\dword{tdr} key milestones]
%{|l|l| p{0.8\textwidth}}
%{tab:dppd_t_12_5}
%{Pre-\dword{tdr} key milestones (TO BE UPDATED)}

%Milestone & End date \\ \toprowrule
%Simulations and physics: %Finalize the 
%Implementation of \dual optical & \\
%simulation in \larsoft for \dword{pddp} & 08/2018 \\ \colhline
%Simulations and physics: Optimization of the & \\
%\dword{dpmod} performance to fulfill the physics requirements and & \\
%definition of a trigger strategy & 05/2019 \\ \colhline
%Photosensors: Components selection and final design & 03/2019 \\ %\colhline
%\dword{pmt} calibration system design and selection of components & 03/2019 \\ \colhline
%Cabling definition and design of flange & 03/2019 \\ \colhline
%Design review in light of \dword{pddp} calibration data & 03/2019 \\ %\colhline
%\dword{qc} plan & 06/2018 \\ \colhline
%Identification of Interfaces & 06/2018 \\ \colhline
%Integration, installation and commissioning plans & 12/2018 \\ \colhline
%\dword{dpmod} \dword{tdr} & 06/2019 \\ 
%\end{dunetable}

%\fixme{Remove table with pre-TDR milestones, Table~\ref{tab:dppd_t_12_5}?}

%\begin{dunetable}
%[Post-\dword{tdr} key milestones]
%{|l|l|l| p{0.8\textwidth}}
%{tab:dppd_t_12_6}
%{Post-\dword{tdr} key milestones}

%Milestone & Start date & End date \\ \toprowrule
%\textbf{\dword{pmt} preparation and installation} (can be done in batches) & & \\ \colhline
%\dword{pmt} procurement procedure and production & 01/2021 & 12/2022 \\ %\colhline
%\dword{pmt} base design and manufacturing & 01/2022 & 12/2022 \\ %\colhline
%\dword{pmt} support structure production and assembly & 08/2022 & 01/2023 \\ \colhline
%\dword{pmt} characterization - \num{10} \dwords{pmt}/week (two facilities) & 02/2023 & 12/2023 \\ \colhline
%\dword{tpb} coating (two facilities similar to that for CERN ICARUS) & 01/2024 & 12/2024 \\ \colhline
%Splitter production and tests & 05/2024 & 12/2024 \\ \colhline
%\textbf{Installation at \surf} & & \\ \colhline
%\dword{pmt} cable and fiber routing in cryostat from flange to bottom & & \\
%                  (depends on \dword{fc} and flange installation) & 09/2024 & 09/2024 \\ \colhline
%\dword{pmt} testing, installation in cryostat and cabling (\num{72} %\dwords{pmt}/month) & 10/2024 & 07/2025 \\ \colhline
%\dword{pmt} support installation on the membrane & & \\
%                  (in parallel by sector with \dword{pmt} installation) & 10/2024 & 07/2025 \\ \colhline
%Splitter installation & & \\
%                  (in parallel with \dword{pmt} installation to test cabling and connections) & 10/2024 & 07/2025 \\ \colhline
%\textbf{Light calibration system} & & \\ \colhline
%Fibers, light source tests and procurement & 06/2023 & 05/2024 \\ %\colhline
%Fiber calibration system installation & & \\
%                  (in parallel with \dword{pmt} installation with validation test) & 09/2024 & 07/2025 \\ 
%\end{dunetable}



%\begin{dunetable}
%[Pre-Installation Period Key Milestones]
%{|l|l|l| p{0.8\textwidth}}
%{tab:dppd_t_12_n1}
%{Pre-Installation Period Key Milestones.}
%
%Milestone & Start date & End date \\ \toprowrule
%\textbf{\dword{pmt} preparation and installation} (can be done in batches) & & \\ \colhline
%\dword{pmt} procurement procedure and production & 01/2021 & 12/2022 \\ \colhline
%\dword{pmt} base design and manufacturing & 01/2022 & 12/2022 \\ \colhline
%\dword{pmt} support structure production and assembly & 08/2022 & 01/2023 \\ \colhline
%\dword{pmt} characterization - \num{10} \dwords{pmt}/week (two facilities) & 02/2023 & 12/2023 \\ \colhline
%\dword{tpb} coating (two facilities similar to that for CERN ICARUS) & 01/2024 & 12/2024 \\ \colhline
%Splitter production and tests & 05/2024 & 12/2024 \\ \colhline
%\textbf{Light calibration system} & & \\ \colhline
%Fibers, light source tests and procurement & 06/2023 & 05/2024 \\ \colhline
%\end{dunetable}

%\begin{dunetable}
%[Installation Period Key Milestones]
%{|l|c|c| p{0.8\textwidth}}
%{tab:dppd_t_12_n2}
%{Installation Period Key Milestones.}
%
%Milestone & Start month & End month \\ \toprowrule
%\dword{pmt} cable and fiber routing in cryostat from flange to bottom & & \\
%                  (depends on \dword{fc} and flange installation) & 1 & 2 \\ \colhline
%\dword{pmt} testing, installation in cryostat and cabling (\num{120} \dwords{pmt}/month) & 2 & 8 \\ \colhline
%\dword{pmt} support installation on the membrane & & \\
%                  (in parallel by sector with \dword{pmt} installation) & 2 & 8 \\ \colhline
%Splitter installation & & \\
%                  (in parallel with \dword{pmt} installation to test cabling and connections) & 2 & 8 \\ \colhline
%Fiber calibration system installation & & \\
%                  (in parallel with \dword{pmt} installation with validation test) & 2 & 8 \\ 
%\end{dunetable}


%\fixme{Update Table~\ref{tab:dppd_t_12_6} according to new schedule, with installation of 2nd module starting on 08/2025.}

%\subsection{High Level Cost of Baseline Design}


%\fixme{I just added the new template -- the autogeneration is not yet in place, so this is the best place to start. 4/4/19. Anne}

%An initial cost estimate of the \dword{pds} for one \SI{10}{kt} \dword{dune} \dword{dpmod} was developed and presented in Table~\ref{tab:dp-pds-costsumm}. This is based on \dword{pddp} costs.
 
%\begin{dunetable}
%[\dual \dword{pds} Cost Summary]
%{p{0.5\textwidth}p{0.2\textwidth}p{0.2\textwidth}}
%{tab:dp-pds-costsumm}
%{\dual \dword{pds} Cost Summary (TO BE COMPLETED)}   
%Cost Item & M\&S (k\$ US) & Labor Hours \\ \toprowrule

%Project Management &     &             \\ \colhline
%Physics and Simulations &     &             \\ \colhline

%\rowcolor{dunepeach} Design, Engineering and R\&D &  &     \\ %\colhline
% Photosensors &     &             \\ \colhline
% Mechanics &     &             \\ \colhline
% Electronics, Cables and \dword{hv} &     &             \\ %\colhline
% Calibration System &     &             \\ \colhline
% Integration and Installation &     &             \\ \colhline
% \rowcolor{dunepeach} Production Setup &  &     \\ \colhline
% Photosensors  &     &             \\ \colhline
% Mechanics &     &             \\ \colhline 
% Electronics, Cables and \dword{hv} &     &             \\ %\colhline
% Calibration System &     &             \\ \colhline 
% Integration and Installation &     &             \\ \colhline
%\rowcolor{dunepeach} Production &  &     \\ \colhline
% Photosensors  &     &             \\ \colhline
% Mechanics &     &             \\ \colhline 
% Electronics, Cables and \dword{hv} &     &             \\ %\colhline
% Calibration System &     &             \\ \colhline 
% Integration and Installation &     &             \\ \colhline
%\rowcolor{dunepeach} DUNE FD Integration \& Installation  &  &  %   \\ %\colhline

%\dword{pmt} Testing at \dword{ctsf} &     &             \\ \colhline
%\dword{wls} Production &     &             \\ \colhline
%\dword{wls} Testing &     &             \\ \colhline

%Signal/Optical Flanges Installation &     &             \\ \colhline
%\dword{pmt} Installation &     &             \\ \colhline
%\dword{pmt} Cabling and Optical Fibers to Feedthrough  &     &             \\ \colhline
%Cabling from Feedthrough to Splitter  &     &             \\ \colhline
%Cabling from Splitter to \dword{hv} Power Supply &     &             \\ \colhline
%Cabling from Splitter to \dword{utca} \dword{fe} Electronics &     &             \\ \colhline
%Light Electronic Rack Installation &     &             \\ \colhline
%Installation Tests &     &             \\ \colhline
%Commissioning &     &             \\ \colhline

%\end{dunetable}