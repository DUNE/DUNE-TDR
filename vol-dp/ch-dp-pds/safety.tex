\section{Safety}
\label{sec:dp-pds-safety}

\dual \dword{pd} Consortium will observe and comply with the institutional and national safety regulations of all of its production/assembly sites. The relevant safety documents for these sites will be reviewed by the consortium, and the regulations will be implemented by the responsible of the local operations.

The production/assembly site is the place where the high voltage cables, high voltage splitter boxes, calibration fibers and the \dword{pmt} mechanics will be prepared for final installation or assembly. The \dwords{pmt} will be received, and the bases and the mounting structures will be connected to the \dwords{pmt}. The output of the production/assembly procedure will be the \dwords{pmt} in their final mechanical and electronic structure. The assembly procedure will require multiple steps at the end of which there will be specific quality control tests. The main safety risks during production are excessive heat due to electrical/mechanical processing, chemicals for cleaning the components, electrocution, and to a limited extent, heavy lifting and tripping hazards. Dedicated handling, cleaning, equipment utilization procedures will be developed by the production site host institute. The main safety risks during assembly are mechanical hazards such as sharp edges, heavy tools and small parts. The procedure for the utilization of safety equipment like safety goggles, gloves and safety shoes will be developed by the production/assembly site institutional representatives. Delicate material handling and transportation instructions will also be developed by the institute. These instructions will be incorporated with the assembled structure for any downstream operation site.

The testing locations at the \dword{itf} and underground cleanroom  will be responsible for the quality control of the assembled \dword{pmt} structure. These stations will also be responsible for testing the \dword{hv} cables, \dword{hv} splitters and the calibration fibers. The previously developed handling instructions will be obeyed during the testing procedures. Possible safety concerns are electrocution and heavy lifting. The functional tests of the \dwords{pmt} will involve powering up the \dwords{pmt} for a predefined period. Therefore, the testing procedures and the relevant safety regulations will be developed for and incorporated with all the future \dword{pmt} tests.

The production/assembly sites and the \dword{itf} will also serve as transportation sites. The \dwords{pmt} will be packed for transportation to the \dword{itf} and \surf. The content of the transportation boxes will be individual carton boxes for the \dwords{pmt} with their bases, mounting assemblies and short cables, placed into the larger plastic pallet boxes in \num{4} x \num{3} x \num{3} arrays for the transportation from the production/assembly sites to the \dword{itf}; and three stacked custom design structures that can hold an array of \num{4} x \num{3} \dwords{pmt} for the transportation from the \dword{itf} to \surf. At this stage, the main safety concern is heavy lifting, at the same time obeying the delicate detector handling procedures. Personnel to move the \num{36}-\dword{pmt} plastic pallet box both indoors and outside for truck loading will have special training. The loading/unloading procedures for the \dword{pmt} transportation boxes will be used at the production/assembly sites, \dword{itf} and the ground and underground stations of the experiment site. Similar procedures will be developed for the transportation of the \dword{hv} cables and splitters, and calibration fibers.

The \dword{tpb} coating of the \dword{pmt} windows will be performed at the \dword{itf}. The \dwords{pmt} will be taken out of the transportation boxes, will go through window cleaning and \dword{tpb} evaporation, and will be stored in the shelves before and after the coating. The \dwords{pmt} will then go through the functional testing procedure and will be installed in their transportation assembly to be transported to the underground hall. Main safety risks at the \dword{itf} are electrocution, exposure to excessive heat and chemicals, and heavy lifting. Workplace safety regulations will be developed for the \dword{itf} \dual \dword{pd} work area. This will include general electrical and mechanical safety rules. The gantry crane operator will be required to have the necessary training. The transportation of the \dword{pmt} boxes in the \dword{itf} will be performed in accordance with the general safety regulation of the \dword{itf}.

The underground operation and installation safety rules will follow the general facility rules. The main risks at the underground clean rooms, cryostat roof and inside the cryostat are working in confined spaces and oxygen deficiency hazard. The \dual \dword{pd} specific risks include electrocution for the pre-installation testing, heavy lifting and tripping.

Safety is the highest priority at all stages of \dual \dword{pd} operations. The safety procedures are developed by the particular institutes involved in specific operations, and are discussed and approved by the consortium. The guidelines and procedures for handling and transportation of the \dual \dword{pd} materials will be made part of the \dword{itf} and underground facility safety regulations.








