\section{Safety}
\label{sec:dp-pds-safety}

\dual \dword{pd} Consortium will observe and comply with the institutional and national safety regulations of all of its production/assembly sites. The relevant safety documents for these sites will be reviewed by the consortium, and the regulations will be implemented by those responsible for local operations.

The production/assembly site is where high voltage cables, high voltage splitter boxes, calibration fibers, and the \dword{pmt} mechanics will be prepared for final installation or assembly. The \dwords{pmt} will be received, and the bases and mounting structures will be connected to the \dwords{pmt}. The output of the production/assembly procedure will be the \dwords{pmt} in their final mechanical and electronic structure. The assembly procedure requires several steps followed by specific quality control tests. The main safety risks during production are excessive heat from electrical/mechanical processing, chemicals for cleaning components, electrocution, and to some extent, heavy lifting and tripping hazards. Dedicated handling, cleaning, and equipment procedures will be developed by the production site host institution. The main safety risks during assembly are mechanical hazards such as sharp edges, heavy tools, and small parts. Procedures for using safety equipment like safety goggles, gloves, and safety shoes will be developed by representatives of the production/assembly site institution. Delicate material handling and transportation instructions will also be developed by the institution. These instructions will be incorporated into the assembled structure for any downstream operation site.

The testing locations at the \dword{itf} and underground cleanroom  will be responsible for the \dword{qc} of the assembled \dword{pmt} structure. These stations will also be responsible for testing the \dword{hv} cables, \dword{hv} splitters, and the calibration fibers. The previously developed handling instructions will be respected during the testing procedures. Possible safety concerns are electrocution and heavy lifting. The functional tests of the \dwords{pmt} will involve powering up the \dwords{pmt} for a predefined period. The testing procedures and the relevant safety regulations will be developed for and incorporated into all the future \dword{pmt} tests.

The production/assembly sites and the \dword{itf} will also serve as transportation sites. The \dwords{pmt} will be packed for transportation to the \dword{itf} and \surf. The contents of the transportation boxes will be individual cartons for the \dwords{pmt} with their bases, mounting assemblies, and short cables, placed into larger plastic pallet boxes in \num{4} $\times$ \num{3} $\times$ \num{3} arrays for transportation from the production/assembly sites to the \dword{itf}. Three stacked custom design structures that can hold an array of \num{4} $\times$ \num{3} \dwords{pmt} will serve to transport \dwords{pmt} from the \dword{itf} to \surf. At this stage, the main safety concern is heavy lifting while at the same time obeying the delicate detector handling procedures. Personnel to move the \num{36}-\dword{pmt} plastic pallet boxes both indoors and outside for truck loading will have special training. The loading/unloading procedures for the \dword{pmt} transportation boxes will be used at the production/assembly sites, \dword{itf}, and the ground and underground stations of the experiment site. Similar procedures will be developed for transporting the \dword{hv} cables and splitters and calibration fibers.

The \dword{tpb} coating of the \dword{pmt} windows will be done at the \dword{itf}. The \dwords{pmt} will be placed on shelves before they are removed from the transportation boxes, the windows will be cleaned, and \dword{tpb} evaporation will be done before the \dwords{pmt} are stored on shelves after the coating. The \dwords{pmt} will then go through functional testing and placed in their transportation assembly to be taken to the underground hall. The main safety risks at the \dword{itf} are electrocution, exposure to excessive heat and chemicals, and heavy lifting. Workplace safety regulations will be developed for the \dword{itf} \dual \dword{pds} work area. This will include general electrical and mechanical safety rules. The gantry crane operator must have the necessary training. Transporting the \dword{pmt} boxes in the \dword{itf} must follow the general safety regulations of the \dword{itf}.

The underground operation and installation safety rules will follow general facility rules. The common risks at the underground clean rooms, cryostat roof, and inside the cryostat are working in confined spaces and oxygen deficiency hazard. The \dual \dword{pds} specific risks include electrocution for pre-installation testing, heavy lifting, and tripping.

Safety is the highest priority at all stages of \dual \dword{pds} operations. The safety procedures are developed by the particular institutions involved in specific operations and are discussed and approved by the consortium. The guidelines and procedures for handling and transportation of the \dual \dword{pds} materials will be made part of the \dword{itf} and underground facility safety regulations.








