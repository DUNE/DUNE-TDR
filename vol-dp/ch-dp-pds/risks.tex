\section{Risks}
\label{sec:dp-pds-risks}

Table%~\ref{tab:dppd_t_12_1} 
\ref{tab:risks:DP-FD-PDS} summarizes the risks associated with the \dword{dp} \dword{pds}. Severity level assigned for each risk is indicated as L (low), M (medium), or H (high). Severity levels are split into three columns: probability for a risk to occur (P), and impact in terms of cost (C) and schedule (S) that the materialization of a risk would incur. Below, we discuss these risks and mitigation plans for each risk item.


% risk table values for subsystem DP-FD-PDS
\begin{footnotesize}
%\begin{longtable}{p{0.18\textwidth}p{0.20\textwidth}p{0.32\textwidth}p{0.02\textwidth}p{0.02\textwidth}p{0.02\textwidth}}
\begin{longtable}{P{0.18\textwidth}P{0.20\textwidth}P{0.32\textwidth}P{0.02\textwidth}P{0.02\textwidth}P{0.02\textwidth}} 
\caption[DP PDS risks]{Risks for DP-FD-PDS (P=probability, C=cost, S=schedule) More information at \dshort{riskprob}. \fixmehl{ref \texttt{tab:risks:DP-FD-PDS}}} \\
\rowcolor{dunesky}
ID & Risk & Mitigation & P & C & S  \\  \colhline
RT-DP-PDS-01 & Insufficient light yield due to inefficient PDS design & Increase PMT photo-cathode coverage and/or WLS reflector foils coverage. & L & M & L \\  \colhline
RT-DP-PDS-02 & Poor coating quality for \dshort{tpb} coated surfaces and \dshort{lar} contamination by \dshort{tpb} & Test quality and ageing properties of TPB coating techniques. Elaborate improved techniques if needed. & L & L & L \\  \colhline
RT-DP-PDS-03 & \dshort{pmt} channel loss due to faulty \dshort{pmt} base design & Optimize clustering algorithms. Improve \dshort{pmt} base design from analysis of possible failure modes in \dshort{pddp}. & L & L & L \\  \colhline
RT-DP-PDS-04 & Bad \dshort{pmt} channel due to faulty connection between \dshort{hv}/signal cable and \dshort{pmt} base & Optimize clustering algorithms. Connectivity tests in \lntwo prior to installation. & L & L & L \\  \colhline
RT-DP-PDS-05 & \dshort{pmt} signal saturation & Tuning of \dshort{pmt} gain. In worst case, redesign front-end to adjust to analog input range of ADC. & M & L & L \\  \colhline
RT-DP-PDS-06 & Excessive electronics noise to distinguish \dshort{lar} scintillation light & Measurement of noise levels during commissioning prior to \lar filling. Modifications to grounding, shielding, or power distribution schemes. & M & L & L \\  \colhline
RT-DP-PDS-07 & Availability of resources for work at the installation/integration site less than planned & Move people temporarily from institutions involved in the \dshort{pds} consortium to the integration/installation site. & L & L & L \\  \colhline
RT-DP-PDS-08 & Damage of \dshorts{pmt} during shipment to the experiment site & Special packaging to avoid possible \dshort{pmt} damage during shipment. Contingency of 10\% spare  \dshorts{pmt}. & L & L & L \\  \colhline
RT-DP-PDS-09 & Damage of optical fibers during installation & Fibers will be last DP-PDS item to be installed. Detailed documentation for all DP-PDS installation tasks.   & L & L & L \\  \colhline
RT-DP-PDS-10 & Excessive exposure to ambient light of \dshort{tpb} coated surfaces, resulting in degraded performance & \dshort{tpb} coated surfaces temporarily covered until cryostat closing. Detailed installation procedure to minimize exposure to ambient light. & L & L & L \\  \colhline
RT-DP-PDS-11 & \dshort{pmt} implosion during \dshort{lar} filling & No mitigation necessary, considering \SI{7}{bar} pressure rating of \dshorts{pmt} and experience with same/similar \dshorts{pmt} in other large liquid detectors. & L & L & L \\  \colhline
RT-DP-PDS-12 & Insufficient light yield due to poor \dshort{lar} purity & Procurement of \dshort{lar} from the manufacturer will require less than 3 ppM in nitrogen. & M & L & L \\  \colhline
RT-DP-PDS-13 & \dshort{pmt} channel or \dshort{pds} sector loss due to failures in \dshort{hv}/signal rack & Ease of maintenance outside cryostat and availability of spares for all components of at least one \dshort{hv}/signal rack. & L & L & L \\  \colhline
RT-DP-PDS-14 & Unstable response of the photon detection system over the lifetime of the experiment & Channel-level instabilities corrected via light calibration system. Detector-level instabilities corrected via cosmic-ray muon calibration data. & L & L & L \\  \colhline
RT-DP-PDS-15 & Bubbles from heat in \dshorts{pmt} or resistors cause \dshort{hv} discharge of the cathode & Verify the power density of the \dshort{pmt} bases are within specifications. Monitor and interlock \dshort{pmt} power supply currents. & L & L & L \\  \colhline
RT-DP-PDS-16 & Reflector/\dshort{wls} panel assemblies together with the \dshort{fc} walls can swing in the fluid flow & Allow appropriate open areas within/between reflector/\dshort{wls} panel assemblies to minimize drag. & L & L & L \\  \colhline

\label{tab:risks:DP-FD-PDS}
\end{longtable}
\end{footnotesize}


%%%%%%%%%%%%%%%%%%%%%%%%%%%%%%%%%%%%%%%%%%%%%%%%%%%%%%%%%%%%%%%%%%%%

\subsection{Design and Construction Risks}
\label{sec:dp-pds-risks_design}

\begin{itemize}

\item Because of the long drift distance and the position of the cathode and \dword{gg} on top of the \dwords{pmt}, the number of photons detected by the \dwords{pmt} might not be sufficient at some geometrical acceptances (RT-DP-PDS-01). This is estimated as a low-probability risk, given that the current \dword{pds} design has been optimized based on detailed simulations. The \dword{pds} baseline design includes \dword{wls} reflector foils precisely to address the risk of insufficient light levels (see Section~\ref{sec:dp-pds-simulation}). The detailed understanding of the light levels observed in the \dword{wa105} and particularly in \dword{pddp} will further mitigate this risk. The \dune \dword{fd} \dword{pds} design is essentially a large-scale replica of the recently installed \dword{pddp} \dword{pds}. 
The most notable exception is the addition of \dword{wls} reflector foils, %that is considered to be tested 
which we plan to test in a \dword{pddp} \dword{pds} upgrade. The largest variation in light yield occurs along the drift direction, where the extrapolation from \dword{pddp} size to the  \dword{dpmod} size is only a factor of two. If higher \dword{pds} detection efficiency turns out to be necessary based on \dword{pddp} experience and simulated projection to \dune scale, we may increase the number of \dwords{pmt} %may be increased 
and/or the surface area of the \dword{wls} reflector foils. % may be increased.

\item Past experience shows that the \dword{tpb} coating might not be sufficiently stable (RT-DP-PDS-02). \dword{tpb} emanation from coated surfaces (\dword{pmt} photo-cathodes and \dword{wls} reflector foils) into the bulk \dword{lar} may cause \dword{wls} behavior of the (contaminated) \dword{lar} bulk, or create a loss in optical performance of the coatings over time. This is estimated as a low-probability risk. 
Certain coating techniques have shown %to emanate 
\dword{tpb} emanation of up to tens of parts per billion by mass into argon \cite{Asaadi:2018ixs}, although such impurity levels are not expected to be problematic. %In addition, o
Other \dword{lar} experiments using \dword{tpb} coatings, most notably DarkSide-50 \cite{Agnes:2018fwg} and DEAP-3600 \cite{Ajaj:2019imk}, have already shown percent-level stability in \dword{pds} light yields over timescales of one to two years. Slow and long-term degradation of the \dword{tpb} coating from exposure to intense \dword{vuv} radiation has not been tested or experienced. Nevertheless, different coating techniques vary greatly in terms of coating quality and stability. As a risk mitigation measure, we plan to test the quality and ageing properties specifically of the %exact 
selected coating procedure %to be followed 
for the \dword{pmt} photo-cathodes and \dword{wls} reflector foils in \dword{pddp} and in dedicated R\&D setups. We will %elaborate
develop improved coating techniques if needed.

\item An inadequate \dword{pmt} base design may result in dead \dword{pmt} channels during the long lifetime of the experiment (RT-DP-PDS-03). Such channel losses would be impossible to recover. %could not be recovered. 
This risk is estimated as low-probability. On the one hand, the loss of single \dword{pmt} channels is expected to have a minimal impact on detector performance. Current \dword{pmt} clustering algorithms (see Section~\ref{sec:dp-pds-performance}) already allow for non-responsive \dwords{pmt} along one row or column within one optical cluster. On the other hand, the \dword{pmt} base design is a mature and simple design already tested in the \dword{wa105} and being %to be 
tested in \dword{pddp}. \dword{pddp} operations will provide additional risk mitigation. If \dword{pddp} experience with \dword{pmt} bases proves to be unsatisfactory, we will introduce and test modifications to the design. % will be introduced and tested.

\item \dword{pmt} cables are soldered to the \dword{pmt} base and tested before installation. If the soldering is done poorly, some channels could show a bad waveform or no signal at all (RT-DP-PDS-04). Such noisy or dead channels would be impossible to repair. % could not be repaired. 
This risk is estimated as low-probability. On the one hand, and as discussed above, the loss of single \dword{pmt} channels is expected to have a minimal impact on detector performance. On the other hand, the \dword{pmt} with final base plus soldered cable will be tested in \lntwo during the \dword{pmt} characterization tests (see Section~\ref{sec:dp-pds-selection-characterization}). In addition, several tests on the base are planned before the \dwords{pmt} are installed: impedance, \dword{hv} tests in gas argon to avoid sparks, and full test of the \dword{pmt} to verify that the signals are correct.

\item \dword{pmt} signal saturation at the front-end input (RT-DP-PDS-05) may occur as a result of operation at a higher-than-anticipated \dword{pmt} gain, e.g., to compensate for a poor \dword{s/n} ratio. \dword{pmt} signals may also saturate due to an incorrect signal amplitude estimate. This risk is estimated to be medium-probability, also considering the highly non-uniform \dword{pds} spatial response. \dword{pmt} signal saturation would have minimal impact on \dword{pds} $t_0$ reconstruction and triggering capabilities. It would have more impact on \dword{pds} energy reconstruction capabilities, although some amount of saturation can be tolerated in this case, as well (see Section~\ref{sec:dp-pds-performance}). To mitigate this risk, we will avoid operating the \dwords{pmt} at very high voltages. A balance between gain and saturation will be chosen.

\item Excessive noise on \dword{pmt} waveforms could appear because of poor grounding design, insufficient cable shielding or noisy power distribution (RT-DP-PDS-06). The risk is estimated as medium-probability. This problem could be detected during \dword{pds} commissioning before, during, or after filling. Doing so prior to \dword{lar} filling is essential, in case the necessary modifications to grounding, shielding, or power distribution  schemes would affect components inside the cryostat.

\end{itemize}

%%%%%%%%%%%%%%%%%%%%%%%%%%%%%%%%%%%%%%%%%%%%%%%%%%%%%%%%%%%%%%%%%%%%

\subsection{Risks During Installation}
\label{sec:dp-pds-risks_installation}

\begin{itemize}

\item Availability of resources for work at the installation/integration site may be less than planned (RT-DP-PDS-07). This risk is estimated to be low-probability. The \dword{fd} construction cost estimate assumes that qualified local labor can be identified for %certain 
specific activities.  The cost will increase if external labor is required.  Labor costs might need to increase to attract qualified candidates. %Labor resources from laboratories may need to be housed and used. 
Labor resources from institutions involved in the \dword{pds} consortium %laboratories 
may be required, and need housing. We will ensure that sufficient funding is available to maintain  people temporarily %from institutions involved in the \dword{pds} consortium 
at the integration/installation site.

\item If \dwords{pmt} are not packaged properly, they could be damaged during shipment (RT-DP-PDS-08). If a \dword{pmt} is damaged, it cannot be used in the detector. This risk is estimated to be low-probability. We will use special packaging to avoid possible damage to the \dwords{pmt} during shipment. In any case, we will have a \num{10}\% contingency of spare \dwords{pmt}.

\item Fibers are fragile and could break during installation because of the high number of fibers in the detector (RT-DP-PDS-09). This risk is estimated to be low-probability. Personnel in charge of assembling the different parts of the detector must take special care during fiber installation.

\item \dword{tpb} coated surfaces (\dword{pmt} photo-cathodes and \dword{wls} reflector foils) are sensitive to ultraviolet light exposure, which may cause degraded optical performance (RT-DP-PDS-10), see for example \cite{Jones:2012hm}. This risk is estimated to be low-probability. \dword{tpb} coated surfaces will be covered with plastic bags or filters to avoid photo-degradation until closing of the cryostat. The installation procedure for \dwords{pmt} and \dword{wls} reflector foils will be defined in great detail, to minimize exposure to ambient light. 

\end{itemize}

%%%%%%%%%%%%%%%%%%%%%%%%%%%%%%%%%%%%%%%%%%%%%%%%%%%%%%%%%%%%%%%%%%%%

\subsection{Risks During Commissioning}
\label{sec:dp-pds-risks_commissioning}

\begin{itemize}

\item %Filling the detector with \dword{lar} could become a critical issue in the occurrence of a \dword{pmt} implosion (RT-DP-PDS-11). 
A \dword{pmt} implosion that occurs while the detector is being filled  with \dword{lar} could become a critical issue (RT-DP-PDS-11). If this were to happen, a chain reaction could develop, as in the \dword{sk} detector accident of 2001, destroying several other \dwords{pmt}. It would require emptying the detector, and the \dwords{pmt} would have to be reinforced or removed. This risk is estimated to be low-probability. 
The Hamamatsu R5912 \dwords{pmt} are rated for \SI{7}{bar} pressure and the hydrostatic pressure on the cryostat floor will be about \SI{2}{bar}. Similar \SI{20.3}{cm} (\SI{8}{inch}) tubes  with the same pressure rating  have been used successfully in other large detectors, such as \dword{microboone}, \dword{sno}, \dword{icarus} T600, and DEAP-3600, and under similar pressure conditions. In addition, a computational model to calculate the shock wave pressure resulting from tube implosion was developed for the MiniBooNE experiment, based on the stored energy in an evacuated tube at a given depth \cite{Brice:2006ny}. (It should be noted that the energy stored in a \SI{20.3}{cm} (\SI{8}{inch})  \dword{pmt} is over an order of magnitude less than in a \SI{50.8}{cm} (\SI{20}{inch}) \dword{pmt} as used in \dword{sk}.) That %study 
model indicated a large safety factor against a chain reaction of \dword{pmt} implosions for the MiniBooNE \dwords{pmt}, which were more densely packed than the \dword{dpmod}'s will be.

\end{itemize}

%%%%%%%%%%%%%%%%%%%%%%%%%%%%%%%%%%%%%%%%%%%%%%%%%%%%%%%%%%%%%%%%%%%

\subsection{Risks During Operation}
\label{sec:dp-pds-risks_operation}

\begin{itemize}

\item A high level of \dword{lar} contamination may result in an unacceptably short absorption length for argon scintillation light, leading to an unacceptably low detected light yield, despite a satisfactory \dword{pds} design (RT-DP-PDS-12). This risk is estimated as medium-probability. A particularly harmful contaminant, as far as light detection is concerned, is nitrogen. The simulation studies presented in this chapter assume an argon scintillation light absorption length of \SI{20}{m}, corresponding to a N$_2$ impurity concentration of about \SI{3}{ppm}. %Such argon purity specification 
This level of argon purity has already been achieved in large \dword{lartpc} detectors. In the \dword{dpmod}, this risk will be mitigated both at the \dword{lar} procurement stage %level 
and through the purification system.

\item The \dword{hv}/signal racks will contain the \dword{hv} crates, the \dword{hv}/signal splitters, the \dword{utca} crates for the \dword{fe} electronics, and the calibration \dword{led} driver and %the 
its associated electronics for \num{36} \dwords{pmt}. Failure in %some 
any of these components may result in failure of a single \dword{pmt} channel or of an entire sector of \num{36} \dwords{pmt} (RT-DP-PDS-13). This risk is estimated as low-probability. The loss of an entire \dword{pmt} sector would have a major impact on detector performance. However, all of these components are located outside the cryostat and are easily replaceable. A number of spares corresponding to at least one \dword{hv}/signal rack will be available for all components.

\item The \dword{pds} response may vary over time (RT-DP-PDS-14) as a result of things %a variety of causes, 
such as time variations in the \dword{lar} purity, the quality of \dword{tpb} coatings, or  \dword{pmt} gains. This risk is estimated as low-probability. The quantum efficiency of the \dword{tpb}-coated \dwords{pmt} and their gain will be continuously monitored using the light calibration system (see Section~\ref{sec:dp-pds-calibration}). Once such channel-level time variations are corrected for, possible global changes in \dword{lar} purity or in the optical response of the \dword{wls} reflector foils will be addressed using cosmic-ray muon calibration data. 

\item If the \dword{pmt} bases have overheating areas, bubbles might form, which may then cause \dword{hv} discharge of the cathode (RT-DP-PDS-15). This risk is estimated as low-probability. The power density of the \dword{pmt} bases will be validated to be within the specifications during the production stage. In addition, a large area of the cathode consists of high resistance rods, delaying the energy release, and a global interlock/alarm mechanism will %exist to 
be implemented to signal %possible 
any discharges. 

\item The reflector/\dword{wls} panels mounted on the inner side of the \dword{fc} walls might block the \dword{lar} convective flow and cause movement of \dword{fc} -- reflector/\dword{wls} panel assemblies (RT-DP-PDS-16). This risk is estimated as low-probability. We have developed a design that allows appropriate open areas within and between reflector/\dword{wls} panel assemblies in order to minimize drag. The design will be validated with \dword{cfd} simulations.


\end{itemize}
