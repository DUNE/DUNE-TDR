\section{Risks}
\label{sec:dp-pds-risks}

Table~\ref{tab:dppd_t_12_1} shows the summary of the risks associated with the \dual \dword{pds}. Severity level assigned for each risk is indicated as L (low), M (medium) and H (high). Below, we discuss these risks and the mitigation plan for each risk item.

\begin{dunetable}
[Dual Phase Photon Detector Risk Summary]
{p{0.05\textwidth}p{0.7\textwidth}p{0.1\textwidth}}
{tab:dppd_t_12_1}
{Dual Phase Photon Detector Risk Summary.}
ID & Risk & Severity \\
1 & The photon detection system is not stable over the lifetime of the experiment (\dword{tpb}, channel) & L \\
2 & Implosion of \dwords{pmt} & L \\
3 & Fibers damaged during installation & L \\
4 & Background level (noise, light sources) is too high to distinguish signal & M \\
5 & \dword{pmt} signals are saturated & M \\
6 & Availability of resources for work at the far site is less than planned & M \\
7 & \dwords{pmt} damaged during shipment to the far site & M \\
8 & Not enough photons arriving to \dwords{pmt} & L \\
9 & Problems with the choosing the correct type of screws and feedthroughs for the cabling and installation on the cryostat & M \\
10 & Excess noise due to poor grounding or not following grounding rules & M \\
11 & Problem with the connection of the cable to the \dword{pmt} base & L \\
12 & \dword{tpb} coating not sufficiently stable, contaminates \dword{lar} & L \\
\end{dunetable}

\begin{enumerate}

\item During the operation of the \dword{fd}, the photon detection system could show performance fluctuations. For example, the \dword{pmt} gains could change. Continuous monitoring with the \dword{pmt} calibration system will enable us to resolve these issues at the individual \dword{pmt} level.

\item A critical process could be the filling of the detector with \dword{lar} because a \dword{pmt} could implode. If this happens, the detector must be emptied. Special care must be taken during the filling of the detector with \dword{lar}. Pressure tests will be performed to quantify this.

\item Fibers are fragile and they could be broken during installation because of the high number of fibers in the detector. Special care during fibers installation must be taken by personnel in charge of the assembly of the different parts of the detector.

\item If external noise or light sources affect the \dwords{pmt}, it could be difficult to measure the signals without oscillations over the baseline. Grounding, shielding and power distributions are critical to the success of the experiment. Mitigation will be by using proper shielding techniques on all cables and validation of the noise performance of all equipment during the installation and commissioning phases.

\item If we have to increase the \dword{pmt} \dword{hv} in order to have higher gains, \dword{pmt} front-end inputs could be saturated. Then, we would loose a number of events in the acquisition. To mitigate this risk we will avoid to operate the \dwords{pmt} at very high voltages. A balance between gain and saturation will be chosen.

\item The far site is remote. The \dword{fd} construction cost estimate assumes that qualified local labor can be identified for certain activities. The cost will increase if external labor is required. Labor costs might need to increase to attract qualified candidates. Labor resources from laboratories may need to be housed and used. We will ensure that sufficient funding is available to move people temporarily from institutions involved in the \dword{pds} consortium to the integration/installation site.

\item If \dwords{pmt} are not packaged properly, they could be damaged during shipment. In this case, the \dword{pmt} cannot be used in the detector. We will use special packaging to avoid possible damages to the \dwords{pmt} during shipment. In any case, we will have a 10 \% contingency spare \dwords{pmt}.

\item Due to the long drift distance and the position of the cathode and the ground grid on top of the \dwords{pmt}, the number of photons reaching the \dwords{pmt} might not be sufficient at some geometrical acceptances. Additional light collection enhancement tools are being considered to mitigate this effect. The most feasible option is the installation of a wavelength shifting film on the inner surface of the field cage.

\item Choosing a wrong feedthrough means we could have a higher noise level or oscillations in the signals. Wrong parts might cause mechanical problems. Special care must be taken during the design of the screws and \dword{hv} feedthroughs needed for cabling in the detector. \dword{pddp} experience will be employed to mitigate this risk.

\item During commissioning of the detector prior to \dword{lar} filling, it could be observed that an excessive noise appears due to some components failing to fulfill the grounding rules. We will ensure that grounding rules are enforced and review any proposed modification of the detector design.

\item \dword{pmt} cables are soldered to the \dword{pmt} base and tested before installation. In case of a bad soldering, some channels could show a bad waveform or no signal. Several tests will be done in the base before the installation of the \dword{pmt}: impedance, \dword{hv} tests in gas Ar to avoid sparks and full test of the \dword{pmt} to verify the signals are correct.

\item Limited past experience demonstrates that the \dword{tpb} coating might not be sufficiently stable and can contaminate the \dword{lar} in the long term. We will carefully examine results from \dword{pddp} and other laboratory tests as quickly as possible. We will elaborate improved coating techniques if needed.

\end{enumerate}