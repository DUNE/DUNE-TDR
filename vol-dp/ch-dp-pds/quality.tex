\section{Quality Control and Quality Assurance}
\label{sec:dp-pds-quality}

The \dword{qa} and \dword{qc} procedures for the \dual \dword{pds} are based on our experience with the \dword{pddp}. The \dwords{pmt} of the \dual \dword{pds} will go through a series of performance and functionality tests at various locations until commissioning. The mechanical assembly of the \dwords{pmt} will go through mechanical tests, the high voltage cables will be tested for continuity and resistance, the high voltage/signal splitters will be tested for electronics integrity, and the calibration fibers will be tested for high quality transmission. Below  are the details of the quality assurance and control procedures.

\subsection{Quality Control During Production and Assembly}

The \dword{qc} performed at the different institutions includes reception of the \dwords{pmt} from the manufacturer and execution of the \dword{qc} tests to accept or return the \dwords{pmt} according to the acceptance and rejection criteria; \dword{qc} of the construction of the base boards, support structures, cables, \dword{hv}/signal splitters and light calibration units. The testing will involve the mechanical tests of the assembly, performance tests of the \dwords{pmt} and electronics both at room temperature and when necessary, at cryogenic temperatures. The institutes that perform these activities are hereby referred to as production and assembly sites.

\begin{itemize}
\item The \dwords{pmt} will be visually inspected as they are received from the manufacturer.

\item 
%The baseboards will be tested for electronics integrity as they are produced, before they are connected to the \dwords{pmt}. Once they are connected, mechanical integrity will be tested. 
The \dword{pmt} support structure design is already validated by immersing its mounted \dword{pmt} in cryogenic temperatures and at an over-pressure equivalent to a depth of \SI{12}{m} in \lar{}. Design validation tests are carried out to confirm that the \dword{pmt} base design fulfills the specifications at room and cryogenic temperatures. A cable with SHV connector is soldered to each \dword{pmt} base to facilitate the different base and \dword{pmt} tests and the final \dword{pmt} connection during the installation. The \dword{pmt} bases are labeled (on the cable) in order to keep track of them. After production of the \dword{pmt} base boards they are individually tested before connecting to the \dwords{pmt} to verify that components are correctly mounted. Later they are cleaned and tested at maximum voltage in an argon gas environment to confirm that there are no sparks. After mounting the bases on the \dwords{pmt}, they are tested again %in argon gas 
at maximum voltage to confirm that there are no sparks due to bad soldering.

\item All the light readout units (\dword{pmt} + base + support) will be tested and characterized in liquid nitrogen in order to check their performance at cryogenic temperature and to obtain a database with the most important parameters for each \dword{pmt} (gain versus voltage, dark counts, etc.). The \dword{pmt} base number attached to each \dword{pmt} will also be included in the database.

\item The wrapping materials and techniques are studied with one fully assembled light readout unit. The handling, transportation and installation scenarios are carefully studied and the transportation box design is validated. The transport box and \dword{pmt} wrapping must  ensure complete darkness.

\item The light output of the \dwords{led} and the fibers' light transmission from the light calibration system will be measured with a power meter.

\item The high voltage cables will be tested for continuity and for their resistance.

\item The high voltage/signal splitters will be tested for electronics integrity followed by their performance with a reference \dword{pmt}.

\end{itemize}

\subsection{Quality Control at the Integration and Testing Facility}

The \dwords{pmt} will be transported to the \dword{itf}, which is planned to be constructed in South Dakota, in the proximity of the experiment site. As the \dword{pmt} boxes are received, they will be visually inspected to validate safe transportation. The \dword{tpb} coating of the \dword{pmt} windows will be performed at the \dword{itf}. The \dword{pmt} windows will be cleaned prior to the coating process. The \dword{qc} of the cleaning will be based on experience in a qualitative manner. The \dwords{pmt} will then be placed in the evaporator and the coating operations will be performed. The first few samples will undergo microscopic examination and surface uniformity tests, and the coating procedure will be validated. The production \dwords{pmt} will be randomly sampled for basic coating \dword{qa}.

Once the coating is finished, the \dwords{pmt} will be placed in the special underground transport assembly which will be placed inside the original transportation boxes. The \dwords{pmt} will not be placed in the carton boxes. They will have protective covers over the windows. Before being placed in the transportation assembly, the \dwords{pmt} will undergo functionality tests individually. Once all the \dwords{pmt} in a box are validated, the box will be closed and transported to the underground hall.

The \dword{itf} is also the primary reception point for the other \dword{dp} \dword{pds} equipment including the cables, fibers, light calibration systems and \dword{hv}/signal splitters. These boxes will be inspected for validating safe transportation. Unless there is a sign of obvious damage to the transportation boxes, no \dword{qc} inspection will be performed on the individual items. If potential damage is identified, the boxes will be opened and visual/functional \dword{qc} tests will be performed. In case the damaged items can be repaired, this operation will be performed at the \dword{itf}. In case the damage is not repairable, the items will be returned to the manufacturer or the production/assembly sites.

\subsection{Quality Control at the Underground Areas}

After  transportation from the \dword{itf} to \surf, the \dwords{pmt} will be tested for proper functionality in a dedicated light-tight box in the clean room. During the installation, the \dword{pmt} database will be updated with the position of each \dword{pmt} (identified by its serial number and base number) in the \dword{detmodule}. After installation, the full connection from the \dword{fe} to the \dwords{pmt} will be checked. The \dword{fe} channel and splitter number connected to each \dword{pmt} will be included in the \dword{pmt} database.

The \dword{hv}/signal cables and the calibration fibers will be transported to the cryostat roof long before the \dwords{pmt} themselves. As they arrive, they will be checked for potential damage during transportation. These tests will include continuity test for the \dword{hv}/signal cables and qualitative transmission test for the calibration fibers.

The last step of \dword{qc} will be to test the entire system using the full readout chain (Section~\ref{subsec:dp-pds-commissioning}).
