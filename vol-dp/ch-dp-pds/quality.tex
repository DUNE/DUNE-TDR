\section{Quality Control and Quality Assurance}
\label{sec:dp-pds-quality}

The \dwords{pmt} of the \dual \dword{pds} will go through a series of performance and functionality tests at various locations until commissioning. The mechanical assembly of the \dwords{pmt} will go through mechanical tests, the \dword{hv} cables will be tested for continuity and resistance, the \dword{hv}/signal splitters will be tested for electronics integrity, and the calibration fibers will be tested for high quality transmission. These \dword{qa} and \dword{qc} procedures for the \dword{dp} \dword{pds} are based on our experience with the \dword{pddp}. Below  are the details of the \dword{qa} and control procedures.

\subsection{Quality Control During Production and Assembly}

Upon receipt of the \dwords{pmt} from the manufacturer, the \dword{qc} performed at the receiving institutions begins with acceptance \dword{qc} testing %before accepting or returning the \dwords{pmt} 
using established acceptance and rejection criteria. %\dword{qc} also includes 
Checking the construction of the base boards, support structures, cables, \dword{hv}/signal splitters, and light calibration units follows. \dword{qc} also includes mechanical tests of the assembly and performance tests of the \dwords{pmt} and electronics, both at room temperature and when necessary, at cryogenic temperatures. The institutions that perform these tests are referred to as production and assembly sites. The itemized \dword{qc} steps for the \dwords{pmt} are:

\begin{itemize}
\item Perform visual inspection upon receipt. %The \dwords{pmt} will be visually inspected as they are received from the manufacturer.

\item Conduct design validation tests. 
%The baseboards will be tested for electronics integrity as they are produced, before they are connected to the \dwords{pmt}. Once they are connected, mechanical integrity will be tested. 
The \dword{pmt} support structure design is already validated by immersing its mounted \dword{pmt} in a cryogenic bath %cryogenic temperatures and 
at an over-pressure equivalent to a depth of \SI{12}{m} in \dword{lar}. %Design validation tests are carried out to c
Confirm that the \dword{pmt} base design fulfills the specifications at room and cryogenic temperatures. 
Solder a cable with a \dword{shv} connector %is soldered 
to each \dword{pmt} base to facilitate the different base and \dword{pmt} tests and the final \dword{pmt} connection during the installation. 
Label the \dword{pmt} bases %are labeled (on the cable) 
to track them. After production of the \dword{pmt} base boards, test each individually %is individually tested 
before connecting to the \dwords{pmt} to verify that components are correctly mounted. Later clean and test them %they are cleaned and tested 
at maximum voltage in an argon gas environment to confirm that no sparks occur. After mounting the bases on the \dwords{pmt}, test them again %they are tested again %in argon gas 
at maximum voltage to confirm that no sparks occur due to poor soldering.

\item Test and characterize all the light readout units (\dword{pmt} + base + support) %will be tested 
 in liquid nitrogen to check performance at cryogenic temperature and to obtain a database with the most important parameters for each \dword{pmt} (e.g., gain versus voltage, dark counts). Include the \dword{pmt} base number attached to each \dword{pmt} %will also be included 
 in the database.

\item Study the wrapping materials and techniques %are studied 
with one fully assembled light readout unit. Carefully study the handling, transportation, and installation scenarios,% are carefully studied, 
and validate the transportation box design. % is validated. 
The transport box and \dword{pmt} wrapping must ensure complete darkness.

\item Measure the light output of the \dwords{led} and fiber light transmission from the light calibration system %will be measured 
with a power meter.

\item Test he \dword{hv} cables %will be tested 
for continuity and for resistance.

\item Test the \dword{hv}/signal splitters %will be tested 
for electronics integrity, then conduct % followed by  
performance tests with a reference \dword{pmt}.

\end{itemize}

%\subsection{Quality Control at the Integration and Testing Facility}
\subsection{Quality Control at the Coating, Testing and Storage Facility}

The \dwords{pmt} will be transported to the \dword{ctsf}, a facility that we expect to establish %envisaged to be established %in South Dakota 
near the experiment site where the \dword{tpb} coating of the \dword{pmt} windows will be performed. As the \dword{pmt} boxes arrive, %are received, 
they will be visually inspected to verify that they were safely transported. %The \dword{tpb} coating of the \dword{pmt} windows will be performed at the \dword{ctsf}. 
We will clean the \dword{pmt} windows %will be cleaned 
before applying the coating. % is applied. 
The \dword{qc} of the cleaning will be qualitative, based on experience. The \dwords{pmt} will then be placed in the evaporator, and the coating will be applied. The first few samples will undergo microscopic examination and surface uniformity tests, and the coating procedure will be validated. Subsequent production \dwords{pmt} will be randomly sampled for basic coating \dword{qa}.

Once the coating is finished, the \dwords{pmt} will be packed in special underground transport assembly and placed inside the original transportation boxes. The \dwords{pmt}, however, will not be placed in the cartons. They will have protective covers over the windows. Before being placed in the transportation assembly, the \dwords{pmt} will undergo functionality tests individually. Once all the \dwords{pmt} in a box are validated, the box will be closed and transported to the underground hall. 
\fixme{Previous pgraph is confusing.  clarify and distinguish `transport assembly' and `transportation box' and `carton', also order of operations. I attempt a rewrite below, but it needs work because I don't fully understand... Anne} 

Once the coating is finished, we will put protective covers over the \dword{pmt} windows. The  \dwords{pmt} will then undergo functionality tests individually before being assembled into  special underground transport assemblies and placed inside the original transportation boxes.  Once all the \dwords{pmt} in a box are validated, the box will be closed and transported to the underground hall. 

The \dword{ctsf} is also the primary reception point for the other \dword{dp} \dword{pds} equipment, including cables, fibers, light calibration systems, and \dword{hv}/signal splitters. The boxes containing these items will undergo inpection to verify that %These boxes will be inspected for verify 
they were safely transported. Unless signs of  damage to the transportation boxes are obvious, no \dword{qc} inspection will be performed on the individual items. If  damage is identified, the boxes will be opened and visual/functional \dword{qc} tests will be performed. If damaged items can be repaired, that will be done at the \dword{ctsf}. If the damage is not repairable, the items will be returned to the manufacturer or the production/assembly sites.

\subsection{Quality Control at the Underground Areas}

After  transportation from the \dword{ctsf} to \dword{surf}, the \dwords{pmt} will be tested for proper functionality in a dedicated light-tight box in the clean room. During installation, the \dword{pmt} database will be updated with the position of each \dword{pmt} (identified by its serial number and base number) in the \dword{detmodule}. After installation, the full connection from the \dword{fe} to the \dwords{pmt} will be checked. The \dword{fe} channel and splitter number connected to each \dword{pmt} will be included in the \dword{pmt} database.

The \dword{hv}/signal cables and the calibration fibers will arrive %be transported to 
at the cryostat roof %long 
some time before the \dwords{pmt}. % themselves. 
As they arrive, we will conduct %they will be checked for potential damage during transportation. These tests will include 
continuity tests for the \dword{hv}/signal cables and qualitative transmission tests for the calibration fibers to verify that no damage occurred during transportation.

The last step of \dword{qc} will be to test the entire system using the full readout chain (Section~\ref{subsec:dp-pds-commissioning}).
