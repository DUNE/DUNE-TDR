%\fixme{SG: working on this; noted a few points below that require updating; created two interface documents for DP-PDS and CRP and added them to references and email sent to Luke to add to global list. JM: Checked and edited. Signing off.}
Interfaces between calibration and other consortia have been identified, and appropriate documents are being developed. %We plan to produce full first drafts of the interface documents by June 2019. 
The documents are currently maintained in the \dword{cern} Engineering and Equipment Data Management Service (EDMS) database, with a \dword{tdr} snapshot kept in the \dword{dune} document database (DocDB).
A brief summary is provided in this section. Table~\ref{tab:fdgen-calib-interfaces-dp} lists the 
interfaces and corresponding DocDB document numbers. 
The main systems interfaces are \dword{crp}, \dword{pds}, \dword{hv}, and \dword{daq}, and the important issues that must be considered are listed below.


\begin{dunetable}
[Calibration system interface links]
{p{0.14\textwidth}p{0.40\textwidth}p{0.14\textwidth}}
{tab:fdgen-calib-interfaces-dp}
{Calibration consortium interface links.}   
\small
Interfacing System & Description & Reference \\ \toprowrule
\dword{hv}	&
effect of calibration hardware (laser and radioactive source) on \efield and field cage; laser light effect on \dword{cpa} materials;
%, field cage penetrations; 
attachment of positioning targets to HV supports 
& \citedocdb{7066} 
\\ \colhline
\dword{pds}	& 
effect of laser light on \dword{pds}, reflectors on the \dword{fc} wall (if any); validation of light response and triggering for low energy signals 
& \citedocdb{7060}
\\ \colhline
\dword{daq}	& 
DAQ constraint on total volume of the calibration data; receiving triggers from DAQ
& \citedocdb{7069}  
\\ \colhline
\dword{cisc} &
multi-functional \dword{cisc}/calibration ports; space sharing around ports; fluid flow validation; slow controls and monitoring for calibration quantities
& \citedocdb{7072} 
\\ \colhline
TPC Electronics	         &  
Noise, electronics calibration
& \citedocdb{7063}  
\\ \colhline
\dword{crp}	&
Attenuation of neutrons by \dword{crp} materials, position of calibration systems relative to \dword{crp} 
%Interaction with \dword{crp} ??
& \citedocdb{7057} 
\\ \colhline
Physics	&
tools to study impact of calibrations on physics
& \citedocdb{6865}  
\\ \colhline
Software \& Computing	  &
Calibration database design and maintenance
& \citedocdb{6868} 
\\ \colhline
TC Facility              &   
Significant interfaces at multiple levels   
& \citedocdb{6829}   \\ \colhline
TC Installation     	  &     
Significant interfaces at multiple levels
& \citedocdb{6847}    \\ 

\end{dunetable}


\begin{description}
    \item[HV] Evaluate the effect of the calibration hardware, especially the laser system periscopes, on the \efield. 
    Integrate the hardware of the %alternative 
    photoelectron laser system (targets) and the \dword{lbls} (diodes) within the \dword{hv} system components. Ensure the radioactive source deployment is in a safe field region.
    %and cannot do mechanical harm to the \dword{fc}.
    \item[PDS] Evaluate long-term effects of laser light, even if diffuse or reflected, on the scintillating components (\dword{tpb} reflector panels) of the \dword{pds}. Given the wide area of these panels, they will be hard to avoid, so we will establish a laser run plan to minimize exposure. Validate light response model and triggering for low energy signals. 
     \item[CRP] The preferred locations for the laser periscopes are in the gaps between the \dword{fc} and the \dword{crp}, so checks will be necessary to confirm sufficient space between laser periscopes and the \dword{fc}/\dword{crp}, taking into account different thermal shrinking modes.
     \item[DAQ] Evaluate \dword{daq} constraints on the total volume of calibration data that can be acquired; develop strategies to maximize the efficiency of data taking with data reduction methods; study how calibration systems can receive trigger signals from \dword{daq} to maximize supernova live time. More details on this are presented in Section~\ref{sec:dp-calib-daqreq}.
  
\end{description}

Integrating and installing calibration devices will interfere with other devices, requiring coordination with the appropriate consortia. Similarly, calibration has significant interfaces at several levels with facilities in coordinating resources for assembly, integration, installation, and commissioning (e.g., networking, cabling, safety). Rack space distribution and interaction between calibration and other modules from other consortia will be managed by \dword{tc} in consultation with those consortia.

