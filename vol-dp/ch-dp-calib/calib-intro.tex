\dword{dune} plans to build two primary systems dedicated to calibrate the \dword{dpmod} -- a laser system
and a \dlong{pns} system -- both of which require interfaces with the cryostat, that are described in Section~\ref{sec:dp-calib-cryostat}.

The laser system is intended to determine the essential detector model parameters with high spatial and time granularity. The primary goal is to provide maps of the drift velocity and \efield, following a position-based technique already proven in other \dword{lartpc} experiments. Two laser subsystems are planned. With high-intensity coherent laser pulses, charge can be created in long straight tracks in the detector by direct ionization of \dword{lar} with the laser beams. This is described in Section~\ref{sec:dp-calib-sys-las-ion}. An auxiliary system aimed at an independent measurement and cross-check of the laser track direction is described in Section~\ref{sec:dp-calib-sys-las-loc}. 
Laser excitation of targets placed on the cathode creates additional charge from well defined locations that can be used 
as a general \dword{tpc} monitor and to measure the integrated drift time. This is described in Section~\ref{sec:dp-calib-sys-las-pe}. 

The \dword{pns} system provides a ``standard candle'' neutron capture signal (\SI{6.1}{\MeV} multi-gamma cascade) across the entire \dword{dune} \dword{fd} volume that is directly relevant to the supernova physics signal characterization thus validating the performance of the detector in the low-energy regime. The \dword{pns} system is described in detail in Section~\ref{sec:dp-calib-sys-pns}.  

The physics motivation, requirements and design of these systems are described in the following subsections. The proposed radioactive source deployment system is described in the Appendix in Section~\ref{sec:dp-calib-sys-rsds}. 

