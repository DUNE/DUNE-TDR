%\section{Safety}
%\label{sec:sp-calib-safe}

%We consider here only personal risk to humans, that apply in the prototyping phase, including ProtoDUNE deployment, and also during integration and commissioning at the DUNE far detector site. Risks of damaging the systems and/or other DUNE detector components are discussed in Section~\ref{sec:sp-calib-risks}.
%\fixme{SG: I reviewed this section for DP and made minor edits.}
This section discusses risks to personnel safety. Detector safety and risks involving damage to detector components are discussed in Section~\ref{sec:dp-calib-risks}.

Human safety is of critical importance during all phases of the calibration work, including R\&D, laboratory testing, prototyping (including \dword{pddp} deployment), and integration and commissioning at the \dword{dune} \dword{fd} site. \dword{dune} \dword{esh} personnel review and approve the work planning for all phases of work as part of the initial design review, as well as before implementation. All documentation of component cleaning, assembly, testing, installation, and operation will include hazard analysis and work planning documentation and will be reviewed appropriately before production begins. In addition, in the case of planned \dword{protodune2} tests, the consortium will interface with \dword{cern} safety system to ensure all requirements are met.

Several areas are of particular importance to calibration are
\begin{itemize}
\item {\bf Underground laboratory safety:} All personnel working underground or in other installation facilities must follow all appropriate safety training and be provided with the required \dword{ppe}. Risks associated with installing and operating the calibration devices include, among others, working at heights, confined space access, falling objects during overhead operations, and electrical hazards. Appropriate safety procedures will include aerial lift and fall protection training for working at heights. For falling objects, the corresponding safety procedures, including hard hats (brim facing down) and a well restricted safety area, will be part of the safety plan. More details on \dword{ppe} are provided in \dword{tdr} \tcchesh{}. % Anne fixed per defs.tex.  \fixme{Chapter 10/Volume 5 would need to be in the appropriate code.}.

\item {\bf Laser safety:} The laser system requires operating a class 4 laser~\cite{FNAL:Class4Lasers,CERN:Class4Lasers}. This requires an interlock on the laser box enclosure for normal operation, with only trained and authorized personnel present in the cavern for the one-time alignment of the laser upon installation in the feedthroughs. The trained personnel must wear appropriate laser protective eye wear. Standard operating procedures will be required for the laser, and those procedures will be reviewed and approved by the \dword{fnal} laser safety officer. 

\item {\bf Radiation safety for \dword{pns}:} A $DD$ neutron generator will be used as a calibration device. The design of safety systems for this system include key control, interlock, moderator, and shielding. Lithium-Polyethylene (\SI{7.5}{\%}) is chosen to be the material for the neutron shield which is rich in hydrogen. The gammas from neutron capture on hydrogen in the shielding material could cause potential radiation hazards. The design of the radiation safety systems (custom shielding and moderator) will be designed to meet \dword{fnal} Radiological Control Manual (FRCM) safety requirements and will be reviewed and approved by the \dword{fnal} radiological control organization. Material safety data sheets will be submitted to the \dword{dune} \dword{esh} to understand other safety hazards such as fire. Before beginning any operations at \dword{pddp}, the entire system will be assembled in a neutron shielded room and tested to confirm no leaking of neutrons will occur. The system will also have a neutron monitor that can provide an interlock. 

%\item {\bf Radiation safety for radioactive source system:} A composite source is used for the radioactive source system that consists  of \isotope{Cf}{252}, a strong neutron emitter, and \isotope{Ni}{58}, which, via the \isotope{Ni}{58}(n,$\gamma$)\isotope{Ni}{59} process, converts one of the \isotope{Cf}{252} fission neutrons, suitably moderated, to a monoenergetic \SI{9}{\MeV} gamma. This system also poses a radiation risk, which will be mitigated with a purge-box for handling, and a shielded storage box and an area with lockout-tagout procedures, also applied to the gate-valve on top of the cryostat. Material safety data sheets will be submitted to DUNE ES\&H and specific procedures will be developed for storage and handling of sources to meet FRCM requirements. These procedures will be reviewed and approved by SURF and Fermilab radiation safety officers. Sources that get deployed will be checked monthly to ensure they are not leaking. A designated shielded storage area will be assigned for sources and proper handling procedures will be reviewed periodically. A custodian will be assigned to each shielded source.

\item {\bf High voltage safety:} Some calibration devices will use high voltage. Fabrication and testing plans will show compliance with local \dword{hv} safety requirements at any institution or laboratory that conducts testing or operation, and this compliance will be reviewed as part of the design process.

\item {\bf Hazardous chemicals:} Hazardous chemicals (e.g., epoxy compounds used to attach components of the system) and cleaning compounds will be documented at the consortium management level using a material safety data sheet and with approved handling and disposal plans in place.

\item {\bf Liquid and gaseous cryogens:} Cryogens (e.g., liquid nitrogen and \dword{lar}) will most likely be used in testing  of calibration devices . Full hazard analysis plans will be in place at the consortium management level for full module or module component testing that involves cryogens. These safety plans will be reviewed appropriately by \dword{dune} \dword{esh} personnel before and during production.

\end{itemize}

%We consider risks to the calibration systems themselves, and also to other DUNE materials or systems. \fixme{This may be a shared concern. We want to avoid bumping/breaking components as they are checked, installed and commissioned in DUNE. Special care will need to be taken to install components and do checks stepwise.} 
%• Other equipment (DSS, HV, \dword{apa}) hitting laser periscopes, if already installed inside cryostat


%{\bf Damage to the photon detection system by the laser:}  To mitigate possible damage to the PD system, software will be used to block the beam while the mirrors are stopped or when laser light is directed at the PD system. Initial discussion with PDS indicates that this may not be a significant issue. \fixme{relationship between this and interface with PD?}

%{\bf Radiation damage to DUNE components:} The activation caused by the PNS is being studied and will be known by ProtoDUNE testing for the PNS at neutron flux intensities and durations well above the run plan. \fixme{May also need to reference background TF. Add RS system.}

%\todo{We have started discussions about electrical safety and grounding, and will update this once formal documents are prepared for that.} 
%Slides: https://indico.fnal.gov/event/16764/session/8/contribution/39/material/slides/2.pdf