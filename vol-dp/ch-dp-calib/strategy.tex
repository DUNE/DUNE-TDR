Once the far detector is filled and at the desired high voltage, it immediately becomes live for all non-beam physics signals, so it is important to tune the detector response model with calibration data as early as possible. Moreover, since both beam and non-beam physics data will have a fairly uniform rate, regular calibrations in order to monitor space and time dependencies are also needed.

Following those considerations, the strategy for calibration data taking will be organized around three specific periods:
\begin{description}
\item[Commissioning] As soon as the detector is full, with \dword{hv} on and the \dword{daq} operational, it is useful to take laser calibration data. The main goal is to help identify problems in the \dword{crp} modules or the electronics channels, or large cool-down distortions. Depending on how long the ramp-up will take, it could be useful to take data before the \dword{hv} reaches the nominal level, because we can identify problems earlier and possibly learn about dependency of various detector parameters with \efield.
\item[Early data] During the early stages of data-taking, the goal is to do the fullest possible fine-grained laser and neutron calibration (\efield map, lifetime, low energy scale/resolution response) as early as possible, so that all the physics can benefit from a calibrated detector from day 1. 
These results should be combined at a later stage with detector-wide average measurements with cosmics. 
\item[Stable data-taking] The main goal of %regular 
calibrations during stable data-taking is to track possible variations of detector response parameters, and contribute to constraining the %response 
detector systematics. We expect to combine fine-grained, high statistics scans at regular time intervals -- twice a year for laser, six times for the pulsed neutron source -- with more frequent coarse-grained scans (e.g., Photoelectron laser, large voxel ionization laser scan). These, combined with analysis of cosmic ray and radiological backgrounds data, can alert to the need of additional fine scans in particular regions.
\end{description}