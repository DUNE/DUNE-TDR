%Text following very closely the SP version

This section describes the installation plans for calibration systems. Most of the hardware is to be installed outside the cryostat, so space on the mezzanine surrounding each calibration port is important for powering and operating the calibration systems. However, some subsystems have internal components that will be installed %once the associated components are installed inside the cryostat (e.g., for the laser, after the periscopes are installed)  
following a specific installation sequence, coordinated with other consortia.

\subsubsection{Ionization Laser System} 

Checking the alignment of the optical components is an essential step of the ionization laser system installation. The system includes a low-power visible laser that can be used for the several mirror alignment operations, but %before that use, 
both the UV and the visible lasers in the laser box need to be aligned beforehand.

Alignment of the visible and UV (Class 4) lasers requires special safety precautions and must be carried out once for each periscope/laser system before installing further \dword{tpc} components. For that reason, the laser boxes must be installed on the cryostat roof as soon as that area becomes accessible.  

The ionization laser system periscopes will be inside the cryostat, but they will be installed from the top of the cryostat and not from the \dword{tco}. However, this installation should be done very carefully in the presence of an operator inside the cryostat, to ensure there are no collisions of the long laser periscopes with other detector components, especially \dword{crp} modules and \dword{fc} elements. In fact installation of the laser periscopes should proceed in sequence with the assembly of other components, with the furthest from \dword{tco} assembled first, but the periscopes should always be installed after the relevant close-by \dword{crp} modules and top \dword{fc} elements.

 %\fixme{Please check the previous sentence for accuracy. It was not clear if the reference in the sentence was to periscopes or \dword{tpc} components.}

In addition, the \dlong{lbls} (\dshort{lbls}) has sets of PIN diodes that will be placed on a support bracket close to the cryostat membrane below the \dword{gg}. The only other step that must be done inside the cryostat
%\dword{tco} 
is connecting the cabling to the available flange; work is still underway to decide how to route cables and which flanges to use. 

The relevant \dword{qc} is essentially an alignment test.
The \dword{lbls} can be used to align the periscopes as they are installed, so it is important that the \dword{lbls} is installed in the same sequence as the periscopes.



\subsubsection{Laser Beam Location System}
This system has several parts that need to be installed through the \dword{tco}, and some must be integrated with the 
\dword{hv} system during installation underground. 

The PIN diode system, that follows the mini-\dword{captain} experience, uses a set of diodes that fire when the laser beam hits them. Because the laser shoots from above and the diodes must be in a low voltage region, the plan is to place the diodes below the bottom ground grid, facing upward. They can be pre-mounted on a support bracket close to the cryostat membrane. 

For the pointing measurement, the beams will pass through the cathode and ground grid electrodes and hit the diodes below. At least \num{20} of these diode clusters would be installed. The installation will consist of positioning the cluster trays in pre-determined locations, and routing the cables to the respective feedthroughs; work is still underway to decide how to route cables and which flanges to use.


The second \dlong{lbls} consists of a set of \num{30} mirror clusters: a plastic or aluminum piece holding four to six small mirrors \SI{6}{\mm} in diameter, each at a different angle; the ionization laser will point to these mirrors to obtain an absolute pointing reference. These clusters will be attached to the wall \dword{fc} profiles facing into the \dword{tpc}. 
This attachment or assembly of the mirror clusters on corresponding \dword{fc} profiles will be done during \dword{fc} assembly underground.

\subsubsection{Photoelectron Laser System} 
A large number of photoelectric targets (about \num{4000}) must be fixed to the cathode grid. Experience from other experiments indicates that targets can be glued to the cathode surface; this
%can be done after cathode assembly but 
must be done before the cathode is installed in the cryostat. 

Once the 
%\dwords{cpa} are 
cathode grid is in place, the photoelectric target locations will need a high-precision survey %, %which is necessary 
for the absolute calibration of the \efield with the photoelectron laser. 

The third part of the installation is placing quartz optical fibers close to the \dword{crp} %, which are
%anode plane, needed 
to illuminate the photoelectric targets with a Nd:YAG laser.  
Fiber tips must be properly fastened and oriented for effective illumination, and fiber bundle routing will bring the fiber bundles to the outside of the cryostat where Nd:YAG laser injection points will be located. 
%\todo{SG: need text for this.}

\subsubsection{Pulsed Neutron Source System} 
The \dlong{pns} will be installed after the human-access ports are closed because the source sits above the cryostat. Installing the system should take place in two stages. In the first stage, the assembly of the system would be independent of the \dword{tpc} installation. The whole system will be installed on the ground outside the cryostat at a dedicated radiation safe facility. Once assembled, the neutron source will be lifted by crane and integrated with the cryostat structure. Final \dword{qc} testing for the system will involve operating the source and measuring the flux with an integrated monitor and dosimeter.


