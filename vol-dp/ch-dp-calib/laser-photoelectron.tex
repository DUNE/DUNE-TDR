
Well localized electron sources represent excellent calibration tool for the study of electron transport in the \dshort{lartpc}. 
A photoelectron laser system can provide such sources at predetermined locations on the cathode, leading to precise  measurements of total drift time and integrated spatial distortions when the charge is not collected in the expected channels. These are achieved by simply measuring the time difference between the laser pulse trigger time and the time when the electron cloud reaches the anode. Such measurement will result in an improved spatial characterization of the \efield, and consequent reduction of detector instrumentation systematic errors.

Being an operationally simpler system compared to the ionization laser system, the photoelectron laser can be used as a ``wake-up'' system to quickly diagnose if the detector is alive, and to provide indications of detector regions that may require a fine-grained check with the ionization system. This is especially important due to the low cosmic ray environment in the detector underground. The photoelectric laser system will utilize the ionization laser for target illumination, thus eliminating the additional cost associated with the laser purchase.


\subsubsection{Design}
\label{sec:dp-calib-sys-las-pe-des}

In order to produce localized clouds of electrons using a photoelectric effect, small metal discs will be placed on the cathode grid and used as targets. Photoelectric laser systems have been successfully used at T2K~\cite{Abgrall:2010hi} and in the Brookhaven National Laboratory (BNL) \dword{lar} test-stand~\cite{Li:2016ods} to generate well-localized electron clouds for \efield calibration. 

The baseline material choice for the metal targets is aluminum, while silver is being considered as an alternative. At \SI{266}{\nano\m} (Nd:YAG quadrupled wavelength) the single photon energy of \SI{4.66}{\eV} is sufficient to generate photoelectrons from aluminum and silver. However, aluminum and silver are prone to oxidization.
%which changes their work function. 
In the case of aluminum, a thick layer of aluminum oxide forms the surface, but this does not increase the work function of the material. Table~\ref{tab:metalphotoelectric} lists the relevant features of metals under consideration.  

The main factor driving the electron yield from the photoelectric targets is the quantum efficiency of the material. Although electrons will be released from the metal whenever photons hit the metal surface, most of the ejected electrons carry forward momentum and therefore are never released from the metal. Only a small fraction of released electrons back-scatters or knocks another electron out of the surface. The quantum efficiency for various metals is typically between \num{e-5}  and \num{e-6}, thus quite low.  All material candidates will be studied in the lab to verify the electron yield, and tested in \dword{pddp} in order to verify the quantum efficiency for different materials.

\begin{dunetable}
[Work function and other features of candidate metal targets for laser photoelectron system]
{cccccc}
{tab:metalphotoelectric}
{Work function and other features of candidate metal targets for laser photoelectron system.}
 Target Material & Work function (\SI{}{\eV}) & $\lambda_{max}$ (\SI{}{\nano\m}) & $\lambda_{laser}$ & Oxidizing & Type of \\ 
\rowtitlestyle 
  & & & required (\SI{}{\nano\m}) & in air & oxidization \\ \toprowrule
 %Gold & 5.1 & 243 & 213 & No & None\\ \colhline
 %Nickel & 5.04 &  246 & 213 & Yes & Surface layer \\ \colhline
 Aluminum & 4.06 & 305 & 266 & Yes & Surface layer \\ \colhline
 Silver & 4.26-4.73 & 291 & 266 & Yes & Surface layer \\ 
  & (lattice dependent) & & & & \\ %\colhline
\end{dunetable}

Several photoelectric strips will complement the circular targets to calibrate the amount of transverse diffusion in \dword{lar}. Based on the experience from T2K and \dword{bnl} \dword{lar} test-stands~\cite{Li:2016ods}, \SIrange{8}{10}{\milli\m} diameter targets are sufficient. Targets placed close to the cathode and the distance between the dots will be determined based on the calibration needs and simulation outcomes. Nominally, dot spacing should be \SI{1}{\m} with photocathode strips every \SI{5}{\m}. The layout of the photoelectric dots and strips will be further refined using the calibration requirements and performance simulation results.  Conducting a survey of the photocathode disc locations on the cathode  after installation and before detector closing will be essential. In this way, the absolute spatial calibration of the electric field can be achieved.

A few thousand electrons are required per spill from each dot to produce the signal above the noise level on the anode and this number will be achieved with high intensity lasers (pulses of the order of \SI{100}{\milli\joule}). The laser beams used to illuminate the targets will be injected into the cryostat via cryogenic optical fibers. The fiber injection points will be mounted on the \dword{fc} or \dword{crp} to achieve the most efficient illumination of the targets on the cathode with minimal numbers of injection points. 

If aluminum is chosen, then single photon absorption is enough, and a lower laser intensity is required, opening up the possibility for rather efficient calibration.
Because the photoelectron clouds from different dots are very well localized, calibrating the \efield distortion in the entire drift volume can be done with a single laser trigger if the light is distributed to all injection fibers at the same time. 
The photoelectron system will use the same lasers used for argon ionization. 
The beam would be redirected into a fiber coupler and then distributed among the fiber bundles.


Lasers will also send a forced trigger signal to the \dword{daq} based on the photodiode triggered on the fraction of the light passing through the dielectric mirror. Special mirrors reflective to \SI{266}{\nano\m} light will be used. 

The photoelectron system will require the following tasks to complete the design: test the mounting of the targets on the cathode grid; survey the dot positions to the required level of precision; and study the target thickness and photoelectron yield as a function of target choice, laser power, and attenuation of the laser light in the optical fibers.

\subsubsection{Measurement Program}
\label{sec:sp-calib-sys-las-pe-meas}

Photoelectron systems have been used in other experiments to diagnose electronics issues by using the known time period between the triggered laser signal and read out times, and to perform rapid checks of the readout of the TPC itself. 

A photoelectron laser is an effective diagnostic and calibration tool, that can quickly and accurately sample the electron drift velocity in the entire detector.
In addition, it can be used to identify electric field distortions due to space charge effects. Exact knowledge of the timing and position of the generated electron clouds is useful for vertex calibration.
%The reciprocal of the \efield (integrated along the drift direction) is also measured.
In addition to electronics issues, discrepancies between the measured and expected drift time can point to either distortions in the position of the detector elements or to a different drift velocity magnitude. 

Another planned measurement is the comparison between the expected and measured $y$, $z$ position of the collected charge, that can point to transverse distortions of the \efield.

