%\subsection{Photoelectron Laser System}
%\label{sec:dp-calib-sys-las-pe}
%\subsubsection{Physics Motivation}
%\label{sec:sp-calib-sys-las-pe-phys} 
Well localized electron sources are excellent calibration tools for studying electron transport in the \dword{lartpc}, identifying inhomogeneities in the \efield in all directions, and precisely determining the electron drift velocity. Verification and calibration of the \efield distortions are important for particle vertex reconstruction and identification and affect the associated systematic errors, leading to increased rate of mis-identification and poorer energy reconstruction. Photoelectron lasers can provide well localized electron sources on the cathode at predetermined locations for improved characterization of the \efield and consequent reduction of detector instrumentation systematic errors. Also, a photoelectron laser system, a simpler system operationally than ionization laser systems, can be used to quickly diagnose whether large sections of the detector are operational.
%KMTDRREADME, replaced above%can be used as a wake-up system to quickly diagnose if the detector is alive. 
This is especially important because of the low cosmic ray rates in the underground detector. 

\subsubsection{Design}
\label{sec:dp-calib-sys-las-pe-des}

To produce localized clouds of electrons using a photoelectric effect, small aluminum discs or thin discs with evaporated gold film, will be used as targets. As stated in reference~\cite{Li:2016ods}, gold film can be just \SI{22}{\nano\m} thick. Several photoelectric strips will complement the circular targets to calibrate the amount of transverse diffusion in \dword{lar}. Based on the experience from T2K and \dword{bnl} \dword{lar} test-stands~\cite{Li:2016ods}, \SIrange{8}{10}{\milli\m} diameter targets are sufficient. Targets placed close to the cathode and the distance between the dots will be determined based on the calibration needs and simulation outcomes. Nominally, dot spacing should be \SI{1}{\m} with photocathode strips every \SI{5}{\m}. The layout of the photoelectric dots and strips will be further refined using the calibration requirements and performance simulation results.  Conducting a survey of the photocathode disc locations on the cathode  after installation and before detector closing will be essential. In this way, the absolute spatial calibration of the electric field can be achieved.

At \SI{266}{\nano\m} Nd:Yag quadrupled wavelength, the single photon energy of \SI{4.66}{\eV} is sufficient to generate photoelectrons from aluminum, but not from gold, which requires two-photon absorption and therefore higher intensities.
%At S\I{266}{\nano\m} Nd:Yag quadrupled wavelength, the photon energy of\SI{4.66}{\eV} is sufficient to generate photoelectrons from both aluminum and gold. 
While aluminum has a lower associated cost, a gold film surface is easier to protect from contamination. 
%A couple of hundred electrons 
%A couple of thousand electrons 
A few thousand electrons are expected per spill from each dot. The laser beam will be 
%coming from the anode injection points, used as sources, 
guided to injection points via cryogenic optical fibers with defocusing elements on the other end. The fiber injection points will be mounted on the \dword{fc} or \dword{crp} to achieve the most efficient illumination of the targets on the cathode with minimal numbers of injection points. 

%\fixme{Jose M to Jelena M: Where can these injection points be? close to the CRP/FC gap? Check edits below...  }

If aluminum is chosen, then single photon absorption is enough, and a lower laser intensity is required, opening up the 
%Much lower energy required for photoelectric laser, opens up the
possibility for rather efficient calibration.
%Namely, laser pulse can be distributed to two drift volumes at the time in order, while illuminating the entire cathode assembly. 
Because the photoelectron clouds from different dots are very well localized, calibrating the \efield distortion in the entire drift volume can be done with a single laser trigger if the light is distributed to all injection fibers at the same time. 

The photoelectron system will use the same lasers used for argon ionization. 
The beam would be redirected into a fiber coupler and then distributed among the fiber bundles.

%Stability of the laser pulses will be monitored  with  power meters.

%\fixme{Jose: Not sure I follow this. Just above we said the light is guided to the injection points by cryogenic fibers. So these mirrors will, at most, guide the light to fiber couplers, right? Also, about this idea of putting the PM behind the dielectric mirror, we should say what is roughly the fraction of light we expect to be transmitted.

%Jelena: redirect light as early as possible} 


%Dielectric mirrors will guide the laser light to injection points, but a fraction of the light will be transmitted instead of reflected to the power meter behind the mirror. 


Lasers will also send a forced trigger signal to the \dword{daq} based on the photodiode triggered on the fraction of the light passing through the dielectric mirror. Special mirrors reflective to \SI{266}{\nano\m} light will be used. 

The photoelectron system will require the following tasks to complete the design: test the mounting of the targets on the cathode plane assembly; survey the dot positions to the required level of precision; and study the target thickness and photoelectron yield as a function of target choice, laser power, and attenuation of the laser light in the optical fibers.

%The first thing that needs to be tested is the mounting of the targets on the cathode plane assembly. In addition, survey of the dots position to the required level of precision is needed. Thickness of the target and photoelectron yield as a function of target choice, laser power and attenuation of the laser light in the optical fibers.

\subsubsection{Measurement Program}
\label{sec:sp-calib-sys-las-pe-meas}

Photoelectron systems have been used in other experiments to diagnose electronics issues by using the known time period between triggered laser signal and read out and to perform rapid checks of the readout of the \dword{tpc} itself. The electric field (integrated along the drift direction) is also measured.
