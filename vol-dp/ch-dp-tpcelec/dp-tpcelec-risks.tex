\section{Risks}
\label{sec:dp-tpcelec-risks}

%\fixme{new standard risks table for autogenerating latex as of 3/25. I will send email. Anne}


% risk table values for subsystem DP-FD-TPC
\begin{footnotesize}
%\begin{longtable}{p{0.18\textwidth}p{0.20\textwidth}p{0.32\textwidth}p{0.02\textwidth}p{0.02\textwidth}p{0.02\textwidth}}
\begin{longtable}{P{0.18\textwidth}P{0.20\textwidth}P{0.32\textwidth}P{0.02\textwidth}P{0.02\textwidth}P{0.02\textwidth}} 
\caption[Risks for DP-FD-TPC]{Risks for DP-FD-TPC (P=probability, C=cost, S=schedule) More information at \dshort{riskprob}. \fixmehl{ref \texttt{tab:risks:DP-FD-TPC}}} \\
\rowcolor{dunesky}
ID & Risk & Mitigation & P & C & S  \\  \colhline
RT-DP-TPC-01 & Component obsolescence over the experiment lifetime & Monitor component stocks and procure an adequate number of spares at the time of production & L & M & L \\  \colhline
RT-DP-TPC-02 & Modification to the LRO FE electronics due to evolution in design of PD design & A strict and timely following of the evolution of DP PDS & L & L & M \\  \colhline
RT-DP-TPC-03 & Damage to electronics due to HV discharges or other causes & FE analog electronics is protected with TVS diodes. Electronics can be easily replaced. & L & L & L \\  \colhline
RT-DP-TPC-04 & Problems with FE card extraction due to insufficient overhead clearance & Addressed by imposing a clearance requirement on \dshort{lbnf} & L & L & L \\  \colhline
RT-DP-TPC-05 & Overpressure in the \dshorts{sftchimney} & The \dshorts{sftchimney} are equipped with overpressure release valves & L & L & L \\  \colhline
RT-DP-TPC-06 & Leak of nitrogen inside the \dshort{dpmod} via cold flange & Monitor chimney pressure for leaks and switch to argon cooling in case of a leak & L & L & L \\  \colhline
RT-DP-TPC-07 & Data flow increase due to inefficient compression caused by higher noise & Have a sufficiently large (a factor of \num{5}) margin in the available bandwidth & L & L & L \\  \colhline
RT-DP-TPC-08 & Damage to \dshort{utca} crates due to presence of water on the roof of the cryostat & \dshort{lbnf} requirement that the cryostat top remains dry & L & L & L \\  \colhline
RT-DP-TPC-09 & Clogging ventilation system of \dshort{utca} crates due to bad air quality & \dshort{lbnf} requirement that the air quality is comparable to a standard industrial environment & L & L & L \\  \colhline

\label{tab:risks:DP-FD-TPC}
\end{longtable}
\end{footnotesize}

\begin{comment} %%% replaced with generated
\begin{dunetable}
[TPC Electronics System Risk Summary]
{p{0.15\textwidth}p{0.75\textwidth}}
{tab:dp-tpcelec-risks}
{TPC Electronics System Risk Summary.}
ID & Risk \\ \toprowrule
1 & Obsolescence of electronic components over the period of the experiment \\ \colhline
2 & Modification to \dword{fe} electronics due to evolution in design of \dwords{pd} \\ \colhline
3 & Damage to electronics due to \dword{hv} discharges or other causes \\ \colhline
4 & Problems with \dword{fe} card extraction due to insufficient clearance \\ \colhline
5 & Overpressure in the \dwords{sftchimney} \\ \colhline
6 & Leak of nitrogen inside the \dword{dpmod} via cold flange \\ \colhline
7 & Data flow increase due to inefficient compression caused by higher noise \\ \colhline
8 & Damage to \dword{utca} crates due to presence of water on the roof of the cryostat \\ \colhline
9 & Clogging ventilation system of \dword{utca} crates due to bad air quality \\ \colhline
\end{dunetable}
\end{comment}

The design of the \dual electronics system takes into account several risks, summarized in Table~\ref{tab:risks:DP-FD-TPC}. While the actual design was completed in 2016, the production for the \dword{dpmod} is expected to start as early as 2022. In the interim, the \dword{dp} \dword{tpc} electronics consortium team that performed the original R\&D for the electronics design will continuously monitor the market to ensure that all single component parts needed to assemble the electronics cards remain in stock on the timescale required for production of the cards for \dword{dune}. Any elements identified as at risk will be replaced in the executive design with functionally equivalent parts. While this may necessitate some minor design adaptations (e.g., modifying the PCB layout to accommodate a different component footprint), the overall functionality of the system will remain unchanged. An industry survey will also be undertaken for the \dword{fe} \dword{asic} manufacturing process, such that, if a real risk is identified, the \dword{asic} production can be re-qualified for a different production process. The long-term component obsolescence (RT-DP-TPC-01) over the lifetime of the experiment ($>$ \num{20} years) will be mitigated by procuring an adequate number of spares at the time of production.

A strict and timely reassessment of the \dword{fe} requirements necessitated by an evolution of the \dual \dword{pds} should allow us to modify the \dword{lro} \dword{fe} electronics design accordingly (RT-DP-TPC-02).

We believe that the \dword{fe} cards have suitable protection to prevent damage caused by \dword{crp} discharges (RT-DP-TPC-03) since the  TVS diodes have provided sufficient protection in \dword{wa105}. By design, the \dword{fe} cards remain accessible and could be replaced. However, the \dwords{sftchimney} require sufficient overhead clearance (RT-DP-TPC-04) to extract and re-insert the blades holding the \dword{fe} cards. This is addressed by imposing a clearance requirement on \dword{lbnf}.

The \dwords{sftchimney} are equipped with safety valves that vent excess gas if pressure suddenly rises (RT-DP-TPC-05). The over-pressure threshold must be set low enough that the flanges suffer no significant damage. The pressure of the nitrogen inside the \dwords{sftchimney} must be monitored to detect potential leaks. In the event of a leak via the cold flange (RT-DP-TPC-06), the chimney volume will be filled with argon gas to mitigate the risk of nitrogen contamination quenching the scintillation light.  

To ensure the continuous flow of digitized data to the \dword{daq} (RT-DP-TPC-07), the data rate must stay below the \dword{mch} bandwidth. In the event data compression becomes less efficient because of higher noise, this bandwidth could be exceeded. To mitigate this risk, we currently have a factor of \num{5} margin in the available bandwidth: \SI{10}{Gbit/s} \dword{mch} compared to the expected rate of \SI{1.8}{Gbit/s} (Table~\ref{tab:dp-utcabandwidth}).

%Potential damage to the \dword{utca} crates due to water condensation on the roof of the cryostat (Risk 8) or poor air quality (Risk 9)should be mitigated by the \dword{lbnf} requirements that the top of the cryostat remain dry and the air quality similar to any industrial environment (e.g., at CERN or \dword{fermilab}). Large quantities of dust in the detector caverns at SURF should be avoided after the \dword{utca} crates are deployed.
\dword{lbnf} requirements that the top of the cryostat remain dry and the air quality remain comparable to a standard industrial environment (e.g., at \dword{cern} or \dword{fermilab}) mitigate the risk of potential damage to the \dword{utca} crates from water condensation (RT-DP-TPC-08) or poor air quality (RT-DP-TPC-09). It is essential to avoid large quantities of dust in the detector caverns at \dword{surf} after the \dword{utca} crates are deployed.
