\section{Risks}
\label{sec:dp-tpcelec-risks}

\begin{dunetable}
[TPC Electronics System Risk Summary]
{p{0.15\textwidth}p{0.75\textwidth}}
{tab:dp-tpcelec-risks}
{TPC Electronics System Risk Summary}
ID & Risk \\ \toprowrule
1 & Obsolescence of electronic components over the period of experiment \\ \colhline
2 & Modification to \dword{fe} electronics due to evolution in design of \dwords{pd} \\ \colhline
3 & Damage to electronics due to \dword{hv} discharges or other causes \\ \colhline
4 & Problems with \dword{fe} card extraction due to insufficient clearance \\ \colhline
5 & Overpressure in the \dwords{sftchimney} \\ \colhline
6 & Leak of nitrogen inside the \dword{dpmod} via cold flange \\ \colhline
7 & Data flow increase due to inefficient compression caused by higher noise \\ \colhline
8 & Damage to \dword{utca} crates due to presence of water on the roof of the cryostat \\ \colhline
9 & Clogging ventilation system of \dword{utca} crates due to bad air quality \\ \colhline
\end{dunetable}

The design of the \dual electronics system takes into account several risk factors, summarized in Table~\ref{tab:dp-tpcelec-risks}, in order to mitigate their impact. The obsolescence of electronic components (ID 1) can be addressed by creating a stock of sufficient number of spare elements (preferably complete cards rather than components). Strict and timely follow-up of the \dword{fe} requirements from the \dual \dword{pds} should allow to quickly address any potential modification of the \dword{lro} \dword{fe} electronics design (ID 2).

The \dword{fe} cards should include suitable protection components to prevent damage caused by CRP discharges (ID 3). TVS diodes used in the current design have shown to be be sufficient for the protection of the \dword{fe} electronics in \dword{wa105}. By design the \dword{fe} cards are also accessible and could be replaced. However, sufficient overhead clearance (ID 4) around the \dwords{sftchimney} should be ensured so that the blades holding the \dword{fe} cards can be extracted and re-inserted. This should be addressed by imposing an appropriate requirement for LBNF.

The \dwords{sftchimney} are equipped with safety valves that vent the excess gas in case of the sudden pressure rise (ID 4). The over-pressure threshold must be set low enough such that no significant damage could happen to the flanges. The pressure of the nitrogen inside the \dwords{sftchimney} should be monitored in order to detect potential leaks. In the event of a leak towards the inner volume of the \dword{dpmod} via the cold flange (ID 5), the chimney volume should be filled with argon gas to mitigate the impact the nitrogen contamination carries for quenching of the scintillation light.  

To ensure the continuous flow of digitized data to \dword{daq}, the data rate must stay below the \dword{mch} bandwidth, which could be a potential risk if the data compression becomes less efficient due to higher noise. To mitigate this risk currently there is a factor of \num{5} margin in the available bandwidth with \SI{10}{Gbit/s} \dword{mch} compared to the expected rate of \SI{1.8}{Gbit/s} (Table~\ref{tab:dp-utcabandwidth}).

Potential damage to the \dword{utca} crates due to water condensation on the roof of the cryostat (ID 8) or poor air quality (ID 9) should be mitigated by the requirements to the LBNF that the top of the cryostat stays dry and the air quality remains similar to any industrial environment (e.g., at CERN or Fermilab). Liberating large quantities of dust in the detector caverns at SURF should be avoided after the deployment of the \dword{utca} crates.
