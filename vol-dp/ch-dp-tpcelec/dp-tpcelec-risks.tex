\section{Risks and Vulnerabilities}
\label{sec:dp-tpcelec-risks}

The design of the \dual electronics system takes into account several risk factors:
\begin{itemize}
\item{\textbf{Obsolescence of electronic components over the period of experiment}: allocation of enough spares (preferably complete cards instead of components) should be sufficient to address this issue. }
\item{\textbf{Modification to \dword{fe} electronics due to evolution in design of \dwords{pd}}: Strict and timely follow-up of the \dword{fe} requirements from the \dual \dword{pds} is required.}
\item{\textbf{Damage to electronics due to \dword{hv} discharges or other causes}: 
The \dword{fe} cards should include suitable protection components. The TVS diodes used in the current design  have been sufficient to protect the electronics in the \dword{wa105}. 
 In addition, the cards are accessible and could be replaced if damaged. }
\item{\textbf{Overpressure in the \dwords{sftchimney}}: The \dwords{sftchimney} are equipped with safety valves that vent the excess gas in case of the sudden pressure rise. The overpressure threshold must be set low enough such that no significant damage could happen to the flanges. }
\item{\textbf{Leak of nitrogen inside the \dword{dpmod} via cold flange}: The chimney volume is filled with argon gas instead of nitrogen.}
\item{\textbf{Mechanical problems with \dword{fe} card extraction due to insufficient overhead clearance}: This is addressed by imposing a requirement for LBNF to ensure enough overhead clearance to extract the blades from the \dwords{sftchimney}.}
\item{\textbf{Data flow increase due to inefficient compression caused by higher noise}: Currently there is a factor of \num{5} margin in the available bandwidth with \SI{10}{Gbit/s} \dword{mch}.} 
\item{\textbf{Damage to \dword{utca} crates due to presence of water on the roof of the cryostat}: This is addressed by imposing a requirement for LBNF to ensure that the top cryostat surface remains dry.}
\item{\textbf{Problems with the ventilation system of the \dword{utca} crates due to bad air quality}: Normal conditions similar to any industrial environment (e.g., at CERN or Fermilab) is expected to be sufficient for proper crate functioning. It is important to avoid liberation of large quantities of dust in the detector caverns at SURF.} 
\end{itemize}
