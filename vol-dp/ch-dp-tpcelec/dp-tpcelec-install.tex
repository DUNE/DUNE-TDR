%%%%%%%%%%%%%%%%%%%%%%%%%%%%%%%%%
\section{Installation, Integration, and Commissioning}
\label{sec:dp-tpcelec-install}

The installation of the TPC electronics systems proceeds in several stages. In order to cable the \dwords{crp} to the \dwords{sftchimney}, the chimneys are installed first, prior to the start of the \dword{crp} installation inside the cryostat. Next the \dword{fe} cards are mounted on the blades and inserted. The installation of the digital electronics and \dword{utca} crates is postponed until all of the heavy work finishes on top of the cryostat in order to prevent damage to the fragile components (e.g., optical fibers) due to movement of material and traffic. 
Once the \dword{utca} crates are installed and all the digital cards are inserted, the \dwords{amc} are cabled to the warm flanges of the \dwords{sft} for the charge readout and are connected to the \dword{pmt} signal cables for the light readout. Finally, to complete the installation and integrate the system with the \dword{daq}, the \SI{10}{Gbit/s} and \SI{1}{Gbit/s} optical links to the \dword{daq} and \dword{wr} timing network are connected. At this stage the full system is ready for commissioning. 

%%%%%%%%%%%%%%%%%%%%%%%%%%%%%%%%%
\subsection{SFT Chimneys}
\label{sec:dp-tpcelec-install-sft}

The installation of the \dwords{sftchimney} requires a compact gantry crane with %the supports 
movable supports along the length of the cryostat. The crane itself moves along the transverse direction. The crates containing the \dwords{sftchimney} are placed along the edges of the cryostat roof. An unpacked chimney is hoisted and transported to %its respective 
the appropriate penetration crossing pipe for installation. Once in place, the chimney is fastened to the flange on the crossing pipe. %The length of each chimney is about \SI{2.4}{m}. 
Enough overhead room to accommodate a chimney's \SI{2.4}{m} length is required
%should therefore be foreseen 
to allow to free movement of the chimney with the crane along the direction transverse to the beam axis. 
%\fixme{why is transverse direction relevant here?} BECAUSE THERE IS NO CRANE IN THAT DIRECTION

In parallel with the \dword{sftchimney} installation, the \dword{fe} cards are unpacked on top of the cryostat and mounted on the blades prior to their insertion in the chimneys.  
% \fixme{Is this necessary: ``This work is performed on the roof of the cryostat to avoid repackaging the blades after the assembly in order to bring them on top of the cryostat.''?} 
With \dwords{sftchimney} secured in the cryostat structure, the blades with mounted \dword{fe} cards %can be inserted and the chimney can be sealed. 
are inserted prior to sealing the chimney.
%At this stage, the connections with the pipes for the \lar and gas nitrogen delivery could also be made, if these latter have already been installed. The pressure probes and temperature sensors can be connected to the slow control system.
At this stage, the \lar and gas nitrogen delivery pipes are already installed, and  
%\fixme{will the sequence be known?} THIS IS TO BE PROVIDED BY THE CRYOSTAT INFRASTRACTURE 
it is possible to make the connections with them. The pressure probes and temperature sensors are also connected to the slow control system.

%%%%%%%%%%%%%%%%%%%%%%%%%%%%%%%%%
\subsection{Digital $\mu$TCA Crates}
\label{ssec:dp-tpcelec-install-utca}

The installation of the \dword{utca} crates with the digital electronics takes place in the final stage of the \dword{dpmod} installation to avoid damaging the fragile equipment. The crates are placed in their designated positions on the cryostat and connected to the power distribution network. The \dword{amc} cards and \dword{wrmch} modules are inserted in their slots. The VHDCI cables are then attached connecting the \dword{cro} \dwords{amc} to the warm flange interface of the \dwords{sftchimney}.  The fibers from the timing system are connected to \dword{wrmch}. 

%%%%%%%%%%%%%%%%%%%%%%%%%%%%%%%%%
\subsection{Integration within the DAQ}
\label{ssec:dp-tpcelec-install-daq}

The integration of the \dual TPC electronics with the \dword{daq} system requires connecting the \SI{10}{Gbit/s} fiber links to each of \num{245} \dword{utca} crates. The connection of the timing system to the synchronization \dword{wrgm} is done via a single \SI{1}{Gbit/s} fiber link. 

The necessary software for the \dword{daq} to read and decode the data packets sent by each \dword{utca} crate would also be provided by the electronics consortium.  

%%%%%%%%%%%%%%%%%%%%%%%%%%%%%%%%%
\subsection{Integration with the Photon Detection System}
\label{ssec:dp-tpcelec-install-pmt}

The cables carrying the \dword{pmt} signals from the splitter boxes are connected to the \dword{lro} analog electronics in each \dword{utca} crate. The position of the crates is optimized with respect to the layout of \dword{pmt} cables. In addition, the calibration system of the \dword{pds} is connected to specified inputs on the cards.


%%%%%%%%%%%%%%%%%%%%%%%%%%%%%%%%%
\subsection{Commissioning}
\label{ssec:dp-tpcelec-install-comission}

The \dwords{sftchimney} are commissioned as a first step. This consists of evacuating and then filling them with nitrogen gas at slight overpressure. It is necessary to check the leak rate when the chimney is under vacuum and to monitor the nitrogen pressure once it is filled in order to verify that no damage occurred to the flange interfaces during installation.

The electronics system is commissioned after completing the installation of the \dword{utca} crates with the \dwords{amc}, and the timing system. The functionality of the full \dword{daq} system is not strictly required at this stage. The data from each crate is read with a portable computer connected to the crate \dword{mch} \SI{10}{Gbit/s} or \SI{1}{Gbit/s} interface. The non-functioning channels are identified by pulsing the \dword{crp} strips and the data quality is examined to ensure the correct functioning of the digital electronics and the temporal alignment of the data segments.   
