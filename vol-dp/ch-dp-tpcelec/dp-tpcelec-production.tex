%%%%%%%%%%%%%%%%%%%%%%%%%%%%%%%%%
\section{Production and Quality Assurance}
\label{sec:dp-tpcelec-production}

%%%%%%%%%%%%%%%%%%%%%%%%%%%%%%%%%
\subsection{Cryogenic Analog FE Electronics}
\label{ssec:dp-tpcelec-prod-cryofe}
The production of the cryogenic \dwords{asic} and analog \dword{fe} cards is envisioned to be split between several sites located in France and Japan at the moment. The delivered cards are then split between five institutions in France (IPNL), Japan (KEK, NITKC, IU), and USA (SMU), where they are tested for various performance parameters such as noise levels, dead channels, hot channels, gain and its uniformity across channels, etc., at both room and operating cold temperature. An appropriate and common database will be developed and populated with test results. 

%%%%%%%%%%%%%%%%%%%%%%%%%%%%%%%%%
\subsection{Signal Feedthrough Chimneys}
\label{ssec:dp-tpcelec-prod-sft}

A number of items require manufacture in order to produce the \dwords{sftchimney}. These include 
\begin{itemize}
\item the PCB flanges for the warm and cold \fdth flange interfaces, 
\item the stainless steel pipe structure, 
\item the flanges containing the interfaces to the gas and liquid lines and slow control, 
\item the blades and railing, and 
\item the heat exchanger system. 
\end{itemize}
The flat cables that connect the \dword{fe} cards to the warm flange are commercially available products and are part of the \dword{sftchimney} procurement process. 

The manufactured components are delivered to designated institutions participating in the \dual electronics consortium where teams verify the signal continuity for both cold and warm flanges, then assemble them into \dwords{sftchimney} and test for leaks. They also check the blade insertion, test the flat cables,  then, once verified, pack the assembled \dwords{sftchimney}  and ship them to SURF. 

%%%%%%%%%%%%%%%%%%%%%%%%%%%%%%%%%
\subsection{Timing System and $\mu$TCA}
\label{ssec:dp-tpcelec-prod-utca}

The timing system components, the  \num{16} \dword{wr} switches and the \num{245} \dword{utca} crates containing the power modules, carrier hubs (\dword{mch}), and fan units are commercially available. The manufacturer takes the responsibility for the necessary quality control and quality assurance of these components, requiring no further testing on the part of the \dual electronics consortium. Once the components are delivered to the designated institutions, they can be sent to SURF for the installation. 

The commerical VHDCI signal cables (connecting the \dwords{amc} to the \dwords{sftchimney}) are procured and tested with the \dword{sftchimney} warm flanges.

%%%%%%%%%%%%%%%%%%%%%%%%%%%%%%%%%
\subsection{Charge Readout Electronics}
\label{ssec:dp-tpcelec-prod-cro}

The production of the \dword{amc} cards for the charge readout as well as the \dword{wrmch} slave cards for synchronization is currently shared between four institutions (IPNL, KEK, NITKC, IU). The cards ordered and delivered to each respective institution are subjected to quality assurance tests agreed upon by all participants. 

%%%%%%%%%%%%%%%%%%%%%%%%%%%%%%%%%
\subsection{Light Readout Electronics}
\label{ssec:dp-tpcelec-prod-lro}

he production of the \dword{lro} \dword{amc} cards is occurs in the same manner as the cards for \dword{pddp} since the number of cards to be produced and the channels to test are both small. The cards' electronic components, meeting required specifications, are purchased commercially. The project will be managed by a qualified engineer, working with a specialist in \dword{qa}.

The produced cards are delivered to designated consortium institutions. Upon delivery, teams conduct basic quality tests, including visual inspection and electrical testing, to ensure conformity of production. Another series of tests is performed on the cards to ensure their correct functionality and to evaluate their performance. Measurements include: linearity measurements (DNL and INL) of each \dword{adc} channel, and linearity of response of the \dword{asic}. The level of cross-talk on the \dword{asic} is also quantified.

A dedicated single-channel setup, with \dword{pmt} (Hamamatsu R5912-02-mod), and identical cabling and splitter as in the \dword{fd}, can be used to characterize the expected noise level of each channel, and response to single \phel{}s up to saturation. 
Multiple cards are operated in a \dword{utca} crate with the \dword{daq}.

After shipment to SURF and installation on-site, a small series of tests is performed with a pulse generator to verify the good working condition of the cards. Noise-level measurements are included in the integration effort.
