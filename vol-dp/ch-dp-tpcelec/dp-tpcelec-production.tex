%%%%%%%%%%%%%%%%%%%%%%%%%%%%%%%%%
\section{Production and Quality Assurance}
\label{sec:dp-tpcelec-production}

\fixme{A brief introductory statement would clarify the section logic for readers.}

%%%%%%%%%%%%%%%%%%%%%%%%%%%%%%%%%
\subsection{Cryogenic Analog FE Electronics}
\label{ssec:dp-tpcelec-prod-cryofe}
We envision splitting production of the cryogenic \dwords{asic} and analog \dword{fe} cards among several sites in France and Japan. The delivered cards are then split among five institutions in France (IPNL), Japan (KEK, NITKC, IU), and USA (SMU), where they are tested for various performance parameters such as noise levels, dead channels, hot channels, and gain and its uniformity across channels, at both room and operating cold temperature. An appropriate and common database will be developed and populated with test results. 

%%%%%%%%%%%%%%%%%%%%%%%%%%%%%%%%%
\subsection{Signal Feedthrough Chimneys}
\label{ssec:dp-tpcelec-prod-sft}

A number of items must be manufactured to produce the \dwords{sftchimney}. These include 
\begin{itemize}
\item the PCB flanges for the warm and cold \fdth flange interfaces, 
\item the stainless steel pipe structure, 
\item the flanges containing the interfaces to the gas and liquid lines and slow control, 
\item the blades and railing, and 
\item the heat exchanger system. 
\end{itemize}
The flat cables that connect the \dword{fe} cards to the warm flange are commercially available products and part of the \dword{sftchimney} procurement process. 

The manufactured components are delivered to designated institutions participating in the \dual electronics consortium where teams verify signal continuity for both cold and warm flanges, then assemble the components into \dwords{sftchimney} and test for leaks. They also check blade insertion, test flat cables, and then, once verified, pack the assembled \dwords{sftchimney} and ship them to \dword{surf}. 

%%%%%%%%%%%%%%%%%%%%%%%%%%%%%%%%%
\subsection{Timing System and $\mu$TCA}
\label{ssec:dp-tpcelec-prod-utca}

The timing system components, the  \num{16} \dword{wr} switches and the \num{245} \dword{utca} crates containing the power modules, carrier hubs (\dword{mch}), and fan units, are commercially available. The manufacturer is responsible for the necessary quality control and quality assurance of these components, requiring no further testing on the part of the \dual electronics consortium. Once the components are delivered to the designated institutions, they can be sent to \dword{surf} for installation. 

The commercial VHDCI signal cables (connecting the \dwords{amc} to the \dwords{sftchimney}) are procured and tested with the \dword{sftchimney} warm flanges.

%%%%%%%%%%%%%%%%%%%%%%%%%%%%%%%%%
\subsection{Charge Readout Electronics}
\label{ssec:dp-tpcelec-prod-cro}

Production of the \dword{amc} cards for the charge readout and the \dword{wrmch} slave cards for synchronization is currently shared among four institutions (IPNL, KEK, NITKC, IU). The cards ordered and delivered to each institution are subjected to quality assurance tests agreed upon by all participants. 

%%%%%%%%%%%%%%%%%%%%%%%%%%%%%%%%%
\subsection{Light Readout Electronics}
\label{ssec:dp-tpcelec-prod-lro}

Both \dword{lro} \dword{amc} cards are produced like the cards for \dword{pddp} because the number of cards produced and the channels to test are both small. The electronic components of the cards, which meet the required specifications, are purchased commercially. This part of the project will be managed by a qualified engineer working with a specialist in \dword{qa}.

The cards are delivered to designated consortium institutions. Upon delivery, teams conduct basic quality tests, including visual inspection and electrical testing, to ensure conformity of production. Another series of tests will ensure the cards function correctly functionality and will also evaluate their performance. Measurements include linearity measurements (DNL and INL) of each \dword{adc} channel and linearity of response of the \dword{asic}. The level of cross-talk on the \dword{asic} is also quantified.

A dedicated single-channel set up, with \dword{pmt} (Hamamatsu R5912-02-mod), with identical cabling and splitter as in the \dword{fd}, can be used to characterize the expected noise level of each channel and the response to single \phel{}s up to saturation. 
Several cards operate in each \dword{utca} crate with \dword{daq}.

After shipment to \dword{surf} and on-site installation, a series of tests are performed with a pulse generator to verify the cards are in good working condition. Noise-level measurements are also included during integration.
