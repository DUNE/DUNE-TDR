%%%%%%%%%%%%%%%%%%%%%%%%%%%%%%%%%
\section{Organization and Management}
\label{sec:dp-tpcelec-org}
This section discusses the organization of the \dword{dp} \dword{tpc} electronics consortium responsible for delivering the \dword{cro} and \dword{lro} electronics for \dword{dpmod}. We present the planning assumptions, and give the schedule for production, integration, and installation of the system components. In addition, we provide a detailed breakdown of the core costs for principal components. %sub-components. 

%\fixme{Again, a brief introductory statement here would clarify the logic of the section.}%%%DONE

%%%%%%%%%%%%%%%%%%%%%%%%%%%%%%%%%
\subsection{Consortium Organization}
\label{ssec:dp-tpcelec-org-consortium}
%The \dual \dword{tpc} electronics consortium currently includes %consists of 
%seven participating institutions: from France (\num{3}), Japan (\num{3}), and the USA (\num{1}). Following the structure established within \dword{dune}, the consortium management team consists of a leader (France) and a technical lead (Japan). The consortium presently includes members from APC, IPNL, and LAPP laboratories in France; Iwate University, KEK, and NITKC in Japan; and Southern Methodist University in the USA. 
The \dual \dword{tpc} electronics consortium currently includes 
eight participating institutions: APC, IPNL, LAL and LAPP laboratories from France, Iwate University, KEK, and NITKC from Japan, and Southern Methodist University from the  USA. Following the structure established within \dword{dune}, the consortium management team consists of a leader (France) and a technical lead (Japan). 


%%%%%%%%%%%%%%%%%%%%%%%%%%%%%%%%%
\subsection{Planning Assumptions}
\label{ssec:dp-tpcelec-org-assmp}
The design of the \dual TPC electronics system relies on elements already developed for \dword{pddp} and tested in the \dword{wa105}.  Commissioning of \dword{pddp} in summer 2019 should provide some additional information but is not expected to alter the design of principal components. Potential improvements related to the increase in channel density supported by the \dwords{amc}  is possible to further reduce costs. 

%%%%%%%%%%%%%%%%%%%%%%%%%%%%%%%%%
\subsection{Institutional Responsibilities}
\label{ssec:dp-tpcelec-org-wbs}
The description of the \dword{wbs}, including the assignments of the responsible institutions is documented in \citedocdb{5594}.

%%%%%%%%%%%%%%%%%%%%%%%%%%%%%%%%%
\subsection{High-level Cost and Schedule}
\label{ssec:dp-tpcelec-org-cs}
The key schedule milestones are listed in Table~\ref{tab:dp-tpcelec-milestones}. The given dates assume  the \dual TPC will be the second \dword{fd} \dword{detmodule} to be installed.

\begin{dunetable}[\dual TPC electronics consortium key milestones]
{p{0.65\textwidth}p{0.25\textwidth}}
{tab:dp-tpcelec-milestones}
{\dual TPC electronics system schedule.}
Milestone & Date (Month YYYY)\\ \toprowrule
%March 2019 & Number of \dword{lro} channels finalized \\ \colhline
%Final routing for \dword{lro} \dwords{amc} for production & MONTH 2019 \\ \colhline
Costing model for \dword{tdr} finalized & March 2019 \\ \colhline
%Firmware for \dword{cro} \dwords{amc} finalized & MONTH 2019 \\ \colhline
Start of \dword{pddp} data-taking & July 2019 \\ \colhline
Start of component production and procurement & January 2024 \\ \colhline
\dword{utca} infrastructure components produced & July 2024 \\ \colhline
Components of \dword{wr} system delivered and validated & July 2024 \\ \colhline
\dword{sft} chimneys produced and tested & January 2025 \\ \colhline
Cryogenic \dword{fe} analog electronics produced and tested & January 2025 \\ \colhline
\dwords{amc} for \dword{cro} and \dword{lro} produced and tested & January 2025 \\ \colhline
Cryostat of the second detector module is ready & August  2025 \\ \colhline
\dword{sft} chimneys installed & November 2025\\ \colhline
Cryogenic \dword{fe} electronics installed & December 2025 \\ \colhline
\dword{utca} crates and \dword{wr} network installed & December 2025 \\ \colhline
Installation of \dwords{amc} completed & January  2026 \\ \colhline
Commissioning of the \dual TPC electronics system & January  2026 \\ \colhline
Closure of the cryostat \dword{tco} & August 2026 \\
\end{dunetable}

The detailed cost model for the \dual TPC electronics system has been developed based on scaling the costs for the \dword{pddp} electronics system. The core costs for producing the electronics system to instrument a single \dword{dpmod} are summarized Table~\ref{tab:dp-tpcelec-costs}. 

\begin{dunetable}[\dual TPC electronics system cost summary]
{p{0.7\textwidth}p{0.2\textwidth}}
{tab:dp-tpcelec-costs}
{\dual TPC electronics system cost summary. \fixme{To be filled ...}}
Item & Core Cost (k\$ US) \\ \toprowrule
%\dword{cro} cryogenic \dwords{asic} & \\ \colhline
\dword{cro} cryogenic analog \dword{fe} cards &  \\ \colhline
\dwords{sftchimney} & \\ \colhline
VHDCI cables & \\ \colhline
\dword{utca} crates & \\ \colhline
\dword{wrmch} units & \\ \colhline
\dword{cro} \dwords{amc} & \\ \colhline
\dword{lro} \dwords{amc} with analog \dword{fe} & \\ \colhline
WR switches and optical transceivers & \\ \colhline
Low voltage power supplies & \\ \colhline
\end{dunetable}
