\begin{table}[htp]
  \caption{Specification for SP-APA-2 (apa-active-area)}
  \centering
  \begin{tabular}{p{0.2\textwidth}p{0.75\textwidth}} 
     \rowcolor{dunesky}
    \newtag{SP-APA-2}{ spec:apa-active-area } \fixme{apa-active-area}
                & Name: Active area    \\ 
    Description & APAs should be sensitive over most of the full area of an APA frame, limiting dead regions in the detector volume.   \\  \colhline
    
    Specification &  Maximize total active area. \\   \colhline
    
    Rationale &  { The footprint of boards/electronics/cabling should be minimized. APAs are double-sided with induction plane wires that wrap in a helical fashion around the long edge.  A single induction wire can thus sense signals from both sides of the APA. Wire wrapping for the induction planes allows to keep the readout boards at the top or bottom of an APA, thus minimizing dead spaces on the long edge sides. In a 2-APA assembly, with the lower APA hanging from the upper APA, readout boards and cold electronics boxes are located at the top of the upper APA and at the bottom of the lower APA, with a minimized dead region between the 2 APAs. } \\ \colhline
    Validation &{ ProtoDUNE-SP data can confirm performance of the regions between APAs and the helical wrapped induction plane design.  } \\    
   \colhline
  \end{tabular}
  \label{tab:spectable:SP-APA}
\end{table}