\begin{table}[htp]
  \caption{Specification for SP-PDS-1 \fixmehl{ref \texttt{tab:spec:ly-uniformity}}}
  \centering
  \begin{tabular}{p{0.2\textwidth}p{0.75\textwidth}} 
     \rowcolor{dunesky}
    \newtag{SP-PDS-1}{ spec:ly-uniformity } 
                & Name: Light Yield Uniformity    \\ 
    Description & The uniformity in the light yield  helps significantly in the rejection of low energy background, especially the 39Ar (beta spectrum with end point at 565 keV). This isotope is present at a level of 1 Bq/kg in natural argon and will cause a large number of signals in the PD system, originated by nearby decays.   \\  \colhline
    
    Specification &   \\   \colhline
    
    Rationale &  { Light yield uniformity within the active volume. The uniformity in the light yield  helps significantly in the rejection of low energy background, especially the 39Ar (beta spectrum with end point at 565 keV). This isotope is present at a level of 1 Bq/kg in natural argon and will cause a large number of signals in the PD system, originated by nearby decays. Uniformity allows to improve energy resolution of the detector through calorimetric measurements based on light.  } \\ \colhline
    Validation &{ Need text here. } \\    
   \colhline
  \end{tabular}
  \label{tab:spec:ly-uniformity}
\end{table}