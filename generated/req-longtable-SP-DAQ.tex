% This file is generated, any edits may be lost.

% It defines macros which expand to corresponding
% specification values for subsystem SP-DAQ



\begin{longtable}{p{0.25\textwidth}p{0.7\textwidth}}   
\caption{Specification for SP-DAQ \fixmehl{ref \texttt{tab:specs:SP-DAQ}}} \\

\rowcolor{dunesky}
\newtag{SP-DAQ-1}{ spec:trigger-high-energy } & Name: Off-beam High-energy Trigger \\ 
    Description & The detector shall trigger on the visible energy of underground physics events from decays or interactions within the active volume with high efficiency.   \\  \colhline
    
    Specification &  $>$\SI{100}{\MeV} \\   \colhline
    
    Rationale &   Study of these events (atmospheric neutrinos, baryon-number-violating events) is part of the DUNE mission.  Cosmic rays are also essential for calibration.  \\ \colhline
    Validation & Refer to physics TDR. 100 MeV is an achievable parameter; lower thesholds are possible.  \\
   \colhline
\rowcolor{dunesky}
\newtag{SP-DAQ-2}{ spec:trigger-low-energy } & Name: Off-beam Low-energy Trigger \\ 
    Description & The detector shall be capable of triggering on the visible energy of single low energy neutrino interactions inside the active volume.   \\  \colhline
    
    Specification &  $>$\SI{10}{\MeV} \\   \colhline
    
    Rationale &   Study of these events (solar, supernova neutrinos) enables other physics studies and provides monitoring of detector.   \\ \colhline
    Validation & Refer to physics TDR. 10 MeV is an achievable parameter; lower thresholds are possible.  \\
   \colhline
\rowcolor{dunesky}
\newtag{SP-DAQ-3}{ spec:trigger-beam } & Name: Trigger for Beam \\ 
    Description & The detector shall trigger on the visible energy of beam interactions within the active volume with efficiency high enough that it has a sub-dominant impact on physics sensitivity.   \\  \colhline
    
    Specification &  $>$\SI{100}{\MeV} \\   \colhline
    
    Rationale &   Study of these events is the primary DUNE mission  \\ \colhline
    Validation & Techniques for doing this have been run succesfully in MINOS, NOvA and T2K. 100 MeV is an achievable parameter; lower thresholds are possible.  \\
   \colhline
\rowcolor{dunesky}
\newtag{SP-DAQ-4}{ spec:trigger-calibration } & Name: Trigger for Calibration \\ 
    Description & The detector shall provide triggers to and trigger on calibration stimuli and tag the data from these triggers as such   \\  \colhline
    
    Specification &   \\   \colhline
    
    Rationale &   Calibration is essential to attain required detector performance  \\ \colhline
    Validation &   \\
   \colhline
\rowcolor{dunesky}
\newtag{SP-DAQ-5}{ spec:trigger-snb } & Name: Trigger for Supernova Burst \\ 
    Description & A trigger shall be generated when a collection of signals is detected that constitute a candidate supernova burst with high galactic coverage, while meeting offline storage requirements and overall bandwidth limitations.   \\  \colhline
    
    Specification &   \\   \colhline
    
    Rationale &   Study of supernova bursts is part of the DUNE mision, this is one of the ways to collect such data  \\ \colhline
    Validation & Refer to physics TDR.  \\
   \colhline
\rowcolor{dunesky}
\newtag{SP-DAQ-6}{ spec:data-record } & Name: Physics Event Record \\ 
    Description & The DAQ shall merge data into a form suitable for offline analysis. Furthermore, tags shall be provided to allow the data collection conditions at the time and the livetime to be determined.   \\  \colhline
    
    Specification &   \\   \colhline
    
    Rationale &   Traditionally these things are referred to offline as an 'event' and a 'run'  \\ \colhline
    Validation &   \\
   \colhline
\rowcolor{dunesky}
\newtag{SP-DAQ-7}{ spec:daq-deadtime } & Name: DAQ Deadtime \\ 
    Description & The DAQ shall operate with deadtime that does not contribute significantly to overall loss of detector livetime.   \\  \colhline
    
    Specification &   \\   \colhline
    
    Rationale &   Zero deadtime makes physics analysis bookeeping for overall livetime much simpler, and allows acquisition of neutrino events even with accidentals from backgrounds such as radiologicals.  \\ \colhline
    Validation & Zero deadtime is an achievable inter-event deadtime but a small deadtime would not significantly compromise physics sensitivity.  \\
   \colhline


\end{longtable}
