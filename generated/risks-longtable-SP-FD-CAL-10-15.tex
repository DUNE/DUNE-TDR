
% risk table values for subsystem SP-FD-CAL-RSDS
\begin{footnotesize}
%\begin{longtable}{p{0.18\textwidth}p{0.20\textwidth}p{0.32\textwidth}p{0.02\textwidth}p{0.02\textwidth}p{0.02\textwidth}}
\begin{longtable}{x{0.18\textwidth}x{0.20\textwidth}x{0.32\textwidth}x{0.02\textwidth}x{0.02\textwidth}x{0.02\textwidth}} 
\caption[Risks for SP-FD-CAL-RSDS]{Risks for SP-FD-CAL-RSDS (P=probability, C=cost, S=schedule) More information at \dword{riskprob}. \fixmehl{ref \texttt{tab:risks:SP-FD-CAL-RSDS}}} \\
\rowcolor{dunesky}
ID & Risk & Mitigation & P & C & S  \\  \colhline
RT-SP-CAL-10 & Radioactive source swings into detector elements & Constrain the system with guide-wires & L & L & L \\  \colhline
RT-SP-CAL-11 & Radioactivity leak & Obtain rigorous source certification under high pressure and cryogenic temperatures & L & L & M \\  \colhline
RT-SP-CAL-12 & Source stuck or lost & Safe engineering margins, stronger fish-line and a torque limit in deployment system & L & M & L \\  \colhline
RT-SP-CAL-13 & Oxygen and nitrogen contamination & Leak checks before deployments & L & M & M \\  \colhline
RT-SP-CAL-14 & Light leak into the detector through purge-box & Light-tight purge box with an infrared camera for visual checks & L & L & L \\  \colhline
RT-SP-CAL-15 & Activation of the cryostat insulation & Activation studies and simulations & L & L & L \\  \colhline

\label{tab:risks:SP-FD-CAL-RSDS}
\end{longtable}
\end{footnotesize}