\begin{table}[htp]
  \caption{Specification for SP-FD-1 \fixmehl{ref \texttt{tab:spec:min-drift-field}}}
  \centering
  \begin{tabular}{p{0.2\textwidth}p{0.75\textwidth}} 
     \rowcolor{dunesky}
    \newtag{SP-FD-1}{ spec:min-drift-field } 
                & Name: Minimum drift field    \\ 
    Description & The drift field in the TPC shall be greater than 250 V/cm, with a goal of 500 V/cm.   \\  \colhline
    Specification (Goal) &  $>$\,\SI{250}{ V/cm}  ( $>\,\SI{500}{ V/cm}$ ) \\   \colhline
    
    Rationale &   Limits impacts of electron-ion recombination (on particle ID via $dE/dx$ versus range), reduces effect of finite electron lifetime on S/N ratio (with implications on tracking and calorimetry), and limits electron diffusion and to a lower degree space charge effects.  \\ \colhline
    Validation & ProtoDUNE will demonstrate if the present HVS design allows reaching the nominal electric field in the drift volume.  Initial data taking will be with the maximum obtainable electric field setting, but additional studies at lower fields to study the effect on particle ID will also be targeted. Detector simulation will take advantage of the experimental data collected with ProtoDUNE.   Additional runs collected at lower field settings will allow for more fine tuning of the models.   \\
   \colhline
  \end{tabular}
  \label{tab:spec:min-drift-field}
\end{table}