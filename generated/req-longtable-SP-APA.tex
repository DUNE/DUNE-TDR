% This file is generated, any edits may be lost.

% It defines macros which expand to corresponding
% specification values for subsystem SP-APA



\begin{longtable}{p{0.25\textwidth}p{0.7\textwidth}}   
\caption{Specification for SP-APA \fixmehl{ref \texttt{tab:specs:SP-APA}}} \\

\rowcolor{dunesky}
\newtag{SP-APA-1}{ spec:apa-unit-size } & Name: Unit size \\ 
    Description & Overall dimensions of a single anode plane assembly   \\  \colhline
    
    Specification &  \SI{6.0}{m} tall $\times$ \SI{2.3}{m} wide \\   \colhline
    
    Rationale &   APA size should be as large as possible to minimize inactive regions within the detector, with the maximum size limited by logistical factors regarding fabrication, transportation, and installation at the 4,850 ft. level at SURF.   \\ \colhline
    Validation & ProtoDUNE-SP APAs of this size were successfully contructed and transported to CERN from the US and UK and installed in the ProtoDUNE cryostat.   \\
   \colhline
\rowcolor{dunesky}
\newtag{SP-APA-2}{ spec:apa-active-area } & Name: Active area \\ 
    Description & APAs should be sensitive over most of the full area of an APA frame, limiting dead regions in the detector volume.   \\  \colhline
    
    Specification &  Maximize total active area. \\   \colhline
    
    Rationale &   The footprint of boards/electronics/cabling should be minimized. APAs are double-sided with induction plane wires that wrap in a helical fashion around the long edge.  A single induction wire can thus sense signals from both sides of the APA. Wire wrapping for the induction planes allows to keep the readout boards at the top or bottom of an APA, thus minimizing dead spaces on the long edge sides. In a 2-APA assembly, with the lower APA hanging from the upper APA, readout boards and cold electronics boxes are located at the top of the upper APA and at the bottom of the lower APA, with a minimized dead region between the 2 APAs.  \\ \colhline
    Validation & ProtoDUNE-SP data can confirm performance of the regions between APAs and the helical wrapped induction plane design.   \\
   \colhline
\rowcolor{dunesky}
\newtag{SP-APA-3}{ spec:apa-wire-tension } & Name: Wire tension \\ 
    Description & APA wires shall not touch during operation and break risk must be kept to a minimum.    \\  \colhline
    
    Specification &  \SI{5}{N} $\pm$ \SI{1}{N} \\   \colhline
    
    Rationale &   Chosen to limit sag to <0.5mm, and stay below yield tension of a \SI{150}{\micro\meter} CuBe wire ($\sim$\,\SI{22}{N}).  \\ \colhline
    Validation & ProtoDUNE tension data and operational experience will confirm this requirement and inform the tolerance.   \\
   \colhline
\rowcolor{dunesky}
\newtag{SP-APA-4}{ spec:apa-bias-voltage } & Name: Wire plane bias voltages \\ 
    Description & APAs should produce optimal and uniform induction and collection signal shapes.   \\  \colhline
    
    Specification &  The setup, including boards, must hold 150\% of max operating voltage. \\   \colhline
    
    Rationale &   Headroom in case anode voltages need to be adjusted to ensure transparency.  \\ \colhline
    Validation & E-field simulation sets wire bias voltages. ProtoDUNE-SP data will confirm performance.  \\
   \colhline
\rowcolor{dunesky}
\newtag{SP-APA-5}{ spec:apa-frame-planarity } & Name: Frame planarity \\ 
    Description & Overall twist of the APA frame.   \\  \colhline
    
    Specification &  $<$\SI{5}{mm} \\   \colhline
    
    Rationale &   To maintain anode plane transparency. An overall twist of the frame of $<$5 mm will ensure a wire plane spacing change of $<$0.5 mm.   \\ \colhline
    Validation &   \\
   \colhline
\rowcolor{dunesky}
\newtag{SP-APA-6}{ spec:apa-bad-channels } & Name: Missing/unreadable channels \\ 
    Description & Number of channels incapable of recording signals.   \\  \colhline
    
    Specification &  $<$1\%, with a goal of $<$0.5\% \\   \colhline
    
    Rationale &   Missing and unreadable channels affect reconstruction efficiency. Missing wires also impact the anode field, and thus, reconstruction.  \\ \colhline
    Validation &   \\
   \colhline


\end{longtable}
