% This file is generated, any edits may be lost.

% It defines macros which expand to corresponding
% specification values for subsystem SP-PDS



\begin{longtable}{p{0.25\textwidth}p{0.7\textwidth}}   
\caption{Specification for SP-PDS \fixmehl{ref \texttt{tab:specs:SP-PDS}}} \\

\rowcolor{dunesky}
\newtag{SP-PDS-1}{ spec:ly-uniformity } & Name: Light yield uniformity \\ 
    Description & The light yield uniformity shall remain less than 50\%.    \\  \colhline
    Specification (Goal) &  < \num{50}\%  ( < \num{20}\% ) \\   \colhline
    
    Rationale &   Light yield uniformity within the active volume. The uniformity in the light yield  helps significantly in the rejection of low energy background, especially the 39Ar (beta spectrum with end point at 565 keV). This isotope is present at a level of 1 Bq/kg in natural argon and will cause a large number of signals in the PD system, originated by nearby decays. Uniformity allows to improve energy resolution of the detector through calorimetric measurements based on light.   \\ \colhline
    Validation & Need text here.  \\
   \colhline
\rowcolor{dunesky}
\newtag{SP-PDS-2}{ spec:spatial-localization } & Name: Spatial localization \\ 
    Description & Events inside the active volume shall be localized in 3D  to within < \SI{100}{\cm} using light signals.   \\  \colhline
    Specification (Goal) &  < \SI{100}{\cm}  ( < \SI{50}{\cm} ) \\   \colhline
    
    Rationale &   This facilitates TPC track-light signal matching and allows to restrict the portion of the TPC information to be acquired/saved. Need text here.   Is localization needed/helpful for the trigger?  \\ \colhline
    Validation & Need text here.  \\
   \colhline
\rowcolor{dunesky}
\newtag{SP-PDS-3}{ spec:env-light-exposure } & Name: Environmental light exposure \\ 
    Description & Blue/UV Light exposure to the PD modules should be minimized.  No exposure to sunlight at any time.  UV-Filtered light (>\SI{400}{nm}) during all exposure.   \\  \colhline
    Specification (Goal) &  \num{0} sunlight; ALARA other sources  ( ALARA ) \\   \colhline
    
    Rationale &   WLS-coated filters and reflective surfaces are destroyed by prolonged exposure to light <400nm for extended periods.  Filtering is critical.  \\ \colhline
    Validation & Need text here.  \\
   \colhline
\rowcolor{dunesky}
\newtag{SP-PDS-4}{ spec:env-humidity-limit } & Name: Environmental humidity limit \\ 
    Description & All working environments with exposed TPB coatings must maintain <\SI{50}{\%} Relative Humidity (RH) at  \SI{70}{\degree F}.   \\  \colhline
    Specification (Goal) &  < \SI{50}{\%} RH at \SI{70}{\degree F}  ( ALARA ) \\   \colhline
    
    Rationale &   TPB coated WLS surfaces may be degraded by even short-term unprotected exposure to high humidity environments.    \\ \colhline
    Validation & Need text here.  \\
   \colhline
\rowcolor{dunesky}
\newtag{SP-PDS-5}{ spec:light-tightness } & Name: Light-tight cryostat \\ 
    Description & Noise rate in photon detectors due to external sources must be less that <\SI{10}{\%} of that induced by radiological background.   \\  \colhline
    Specification (Goal) &  <\SI{10}{\%}  ( ALARA ) \\   \colhline
    
    Rationale &   All openings and flanges must be as light-tight as possible to avoid introducing false triggers into the PD system.  \\ \colhline
    Validation & Need text here.  \\
   \colhline
\rowcolor{dunesky}
\newtag{SP-PDS-6}{ spec:ed-light } & Name: Light from electrical discharge \\ 
    Description & Induced PD single-PE event rate due to flashing from HV electrical discharging or corona effect shall be less than <\SI{10}{\%} of that induced by radiological background.   \\  \colhline
    Specification (Goal) &  <\SI{10}{\%}  ( ALARA ) \\   \colhline
    
    Rationale &   HV discharging must be minimized to avoid spurious signals in the SP-PD system.   \\ \colhline
    Validation & Need text here.  \\
   \colhline
\rowcolor{dunesky}
\newtag{SP-PDS-7}{ spec:mech-deflection } & Name: Mechanical deflection (static) \\ 
    Description & The PDS shall move no more than \SI{5}{\mm} relative to  the  horizontal and vertical orientation of APA (or move in any direction at all?)   \\  \colhline
    Specification (Goal) &  $<$\SI{5}{\milli\meter}  ( ALARA ) \\   \colhline
    
    Rationale &   PD mechanical support system must be sufficieltly rigid to avoid damaging APA grid wires  \\ \colhline
    Validation & Need text here.  \\
   \colhline
\rowcolor{dunesky}
\newtag{SP-PDS-8}{ spec:apa-install } & Name: Clearance for installation through APA side tubes \\ 
    Description & PD modules must fit and be secured to the APA through slots in one side of the APA, as designed in concert with APA group.   \\  \colhline
    
    Specification &  $>$\SI{1}{\milli\meter} \\   \colhline
    
    Rationale &   Photon detector design must allow for installation and cabling inside APA modules after the  APA wire wrapping is complete.  \\ \colhline
    Validation & Need text here.  \\
   \colhline
\rowcolor{dunesky}
\newtag{SP-PDS-9}{ spec:pds-compatible } & Name: No mechanical interference with APA, SP-CE and SP-HV detector elements (clearance) \\ 
    Description & SP-PD system, including cables and mechanical supports, must fit within APA and not interfere with SP-CE.   \\  \colhline
    
    Specification &  $>$\SI{1}{\milli\meter} \\   \colhline
    
    Rationale &   PD system dimensions and tolerances must be specified and maintained to fit with the APA mechanical constraints.  PD system components must not interfere with CE cable routing or mechanical support boxes.  \\ \colhline
    Validation & Need text here.  \\
   \colhline
\rowcolor{dunesky}
\newtag{SP-PDS-10}{ spec:pds-cable } & Name: PD cable routing APA intrusion \\ 
    Description & The SP-PD cable system must be installed prior to APA wire wrapping.  Module connection to the cable system must occur without impinging into the APA side tubes more than \SI{6}{\milli\meter}.   \\  \colhline
    
    Specification &  $<$\SI{6}{\milli\meter} \\   \colhline
    
    Rationale &   The APA side tubes will be mostly filled with CE cables during integration/installation into the detector.  PD cables and connectors must not impinge into this space.  \\ \colhline
    Validation & Need text here.  \\
   \colhline
\rowcolor{dunesky}
\newtag{SP-PDS-11}{ spec:pds-cablemate } & Name: Upper-Lower APA junction gap \\ 
    Description & The PD cabling system must allow for the upper and lower APAs to mate with a maximum \SI{6}{\milli\meter} gap between APA frame footer tubes.   \\  \colhline
    
    Specification &  $<$\SI{6}{\milli\meter} \\   \colhline
    
    Rationale &   During detector integration, APA stacks of 2 APAs are made up.  As the PD cabling will already be mounted inside the APAs, a method to allow those connections to be made while allowing the APAs to mate to within 6mm must be designed and implemented.  \\ \colhline
    Validation & Need text here.  \\
   \colhline
\rowcolor{dunesky}
\newtag{SP-PDS-12}{ spec:pds-location } & Name: Maintain detector location at LAr temperature.  \\ 
    Description & No absolute number is required for physics. The requirement is driven by engineering to ensure no damage occurs.   \\  \colhline
    
    Specification &  (Dave: what is required to avoid damage) \\   \colhline
    
    Rationale &   Accommodate shrinkage of the detector and APA itself. SP-PD must accommodate variation of cable, cable connector, and mecchanical support locations.  \\ \colhline
    Validation & Need text here.  \\
   \colhline
\rowcolor{dunesky}
\newtag{SP-PDS-13}{ spec:pds-datarate } & Name: Data transfer from SP-PD to DAQ \\ 
    Description & <8 Gbps per DAQ input steady state per APA.  Burst rate TBD.     \\  \colhline
    
    Specification &  $<$\SI{8}{Gbps} \\   \colhline
    
    Rationale &   Maximum data throughput from a ganged set of CE-FEE boards.    \\ \colhline
    Validation & Need text here.  \\
   \colhline


\end{longtable}
