
% risk table values for subsystem SP-FD-CISC
\begin{longtable}{p{0.18\textwidth}p{0.20\textwidth}p{0.32\textwidth}p{0.02\textwidth}p{0.02\textwidth}p{0.02\textwidth}} 
\caption{Risks for SP-FD-CISC \fixmehl{ref \texttt{tab:risks:SP-FD-CISC}}} \\
\rowcolor{dunesky}
ID & Risk & Mitigation & P & C & S  \\  \colhline
RT-SP-CISC-01 & Baseline design from ProtoDUNEs for an instrumentation device is not adequate for DUNE far detectors & Focus on early problem discovery in ProtoDUNE so any needed redesigns can start as soon as possible. & L & M & L \\  \colhline
RT-SP-CISC-02 & Swinging of long instrumentation devices (T-gradient monitors or PrM system) & Add additional intermediate constraints to prevent swinging. & L & L & L \\  \colhline
RT-SP-CISC-03 & High E-fields near instrumentation devices cause dielectric breakdowns in \dword{lar} & CISC systems placed as far from cathode and FC as possible. & L & L & L \\  \colhline
RT-SP-CISC-04 & Light pollution from purity monitors and camera light emitting system & Use PrM lamp and camera lights outside PDS trigger window; cover PrM cathode to reduce light leakage. & L & L & L \\  \colhline
RT-SP-CISC-05 & Temperature sensors can induce noise in cold electronics & Check for noise before filling and remediate, repeat after filling. Filter or ground noisy sensors. & L & L  & L \\  \colhline
RT-SP-CISC-06 & Disagreement between lab and \em{in situ} calibrations for ProtoDUNE-SP dynamic T-gradient monitor & Investigate and improve both methods, particularly laboratory calibration. & M & L & L \\  \colhline
RT-SP-CISC-07 & Purity monitor electronics induce noise in TPC and PDS electronics. & Operate lamp outside TPC+PDS trigger window. Surround and ground light source with Faraday cage. & L & L & L \\  \colhline
RT-SP-CISC-08 & Discrepancies between measured temperature map and CFD simulations in ProtoDUNE-SP & Improve simulations with additional measurements inputs; use fraction of sensors to predict others   & L & L & L \\  \colhline
RT-SP-CISC-09 & Difficulty correlating purity and temperature in ProtoDUNE-SP impairs understanding cryo system. & Identify causes of discrepancy, modify design. Calibrate PrM differences, correlate with RTDs. & L & L & L \\  \colhline
RT-SP-CISC-10 & Cold camera R\&D fails to produce prototype meeting specifications \& safety requirements & Improve insulation and heaters. Use cameras in ullage or inspection cameras instead. & M & M & L \\  \colhline
RT-SP-CISC-11 & HV discharge caused by inspection cameras & Study E-field in and on housing and anchoring system. Test in HV facility. & L & L & L \\  \colhline
RT-SP-CISC-12 & HV discharge destroying the cameras & Ensure sufficient redundancy of cold cameras. Warm cameras are replaceable. & L & M & L \\  \colhline
RT-SP-CISC-13 & Insufficient light for cameras to acquire useful images & Test cameras with illumination similar to actual detector. & L & L & L \\  \colhline
RT-SP-CISC-14 & Cameras may induce noise in cold electronics & Continued R\&D work with grounding and shielding in realistic conditions. & L & L & L \\  \colhline
RT-SP-CISC-15 & Light attenuation in long optic fibers for purity monitors  & Test the max.\ length of usable fiber, optimize the depth of bottom PrM, number of fibers. & L & L & L \\  \colhline
RT-SP-CISC-16 & Longevity of purity monitors & Optimize PrM operation to avoid long running in low purity. Technique to protect/recover cathode. & L & L & L \\  \colhline
RT-SP-CISC-17 & Longevity: Gas analyzers and level meters may fail. & Plan for future replacement in case of failure or loss of sensitivity.  & M & M & L \\  \colhline
RT-SP-CISC-18 & Problems in interfacing  hardware devices (e.g. power supplies) with slow controls & Involve slow control experts in choice of hardware needing control/monitoring.
 & L & L & L \\  \colhline

\label{tab:risks:SP-FD-CISC}
\end{longtable}