% This file is generated, any edits may be lost.

% It defines macros which expand to corresponding
% specification values for subsystem SP-CISC



\begin{longtable}{p{0.25\textwidth}p{0.7\textwidth}}   
\caption{Specification for SP-CISC \fixmehl{ref \texttt{tab:specs:SP-CISC}}} \\

\rowcolor{dunesky}
\newtag{SP-CISC-1}{ spec:inst-noise } & Name: Noise from Instrumentation devices \\ 
    Description & The instrumentation devices shall contribute no more than \SI{1000}{enc} of noise, with a goal of ALARA. This requirement is on total system noise;   \\  \colhline
    Specification (Goal) &  $\ll\,\SI{1000}{enc}$  ( ALARA ) \\   \colhline
    
    Rationale &   Based on extrapolations from SBND and as quoted by the CE consortium, a maximum acceptable noise level is ~1100e- for a 5/1 signal/noise ratio for a MIP passing near the cathode, expected to be dominated by cold electronics. ALARA=As Low As Reasonably Achievable.  \\ \colhline
    Validation &   \\
   \colhline
\rowcolor{dunesky}
\newtag{SP-CISC-2}{ spec:inst-efield } & Name: Max. E-field near instrumentation devices \\ 
    Description & The maximum field near instrumentation devices should be $<\,\SI{30}{kV/cm}$ to avoid dielectric breakdowns.   \\  \colhline
    Specification (Goal) &  $<\,\SI{30}{kV/cm}$  ( $<\,\SI{15}{kV/cm}$ ) \\   \colhline
    
    Rationale &   The HV Consortium is using 30 kV/cm as the maximum field that the LAr can maintain without undergoing dielectric breakdown. Need to discuss with the HV group if < 15 kV/cm is a realistic goal or not.  \\ \colhline
    Validation &   \\
   \colhline
\rowcolor{dunesky}
\newtag{SP-CISC-3}{ spec:elec-lifetime-prec } & Name: Precision in electron lifetime \\ 
    Description & The precision on the measurement of the electron lifetime needs to sufficient to ensure $<$ 0.5\% uncertainty in charge readout.   \\  \colhline
    Specification (Goal) &  $<\,$1.4\% ($<$4\%)  ( $<\,$1\% ) \\   \colhline
    
    Rationale &   Based on the DUNE-FD Task Force final report, to keep the bias on the charge readout in the TPC below 0.5\% at a 3 ms electron lifetime. If the electron lifetime is 9 ms then the precision drops to 4\% to maintain the better than 0.5\% bias in the charge readout.  \\ \colhline
    Validation &   \\
   \colhline
\rowcolor{dunesky}
\newtag{SP-CISC-4}{ spec:elec-lifetime-range } & Name: Range in electron lifetime \\ 
    Description & The purity monitors inside the cryostat should be capable of measuring a lifetime range between 0 and 10 ms. The goal for the inline purity monitors is to measure a range of 0 to 30 ms for the drift electron lifetime.   \\  \colhline
    Specification (Goal) &  \SIrange{0}{10}{ms} (\SIrange{0}{30}{ms})  ( \SIrange{0}{10}{ms} (\SIrange{0}{30}{ms}) ) \\   \colhline
    
    Rationale &   10 ms drift lifetime is better than any that has been observed in a large LAr TPC and by keeping the range to a minimum, then the precision of the purity monitors can be better. The LAr coming from the filter for the 35t reads ~25 ms, but has never reached 30 ms.  \\ \colhline
    Validation &   \\
   \colhline
\rowcolor{dunesky}
\newtag{SP-CISC-11}{ spec:temp-repro } & Name: Precision: temperature reproducibility \\ 
    Description & The RMS of the distribution of independent temperature offsets between two sensors in successive immersions in LAr should be $<$ 5  mK   \\  \colhline
    Specification (Goal) &  $<\,\SI{5}{mK}$  ( \SI{2}{mK} ) \\   \colhline
    
    Rationale &   These numbers are based on ProtoDUNE design and extrapolation to DUNE to achieve high precision 3D temperature map. Physics motivation here is the CFD simulation validation.  \\ \colhline
    Validation &   \\
   \colhline
\rowcolor{dunesky}
\newtag{SP-CISC-14}{ spec:temp-stability } & Name: Stability \\ 
    Description & The thermometers should match precision requirement at all places, at all times   \\  \colhline
    Specification (Goal) &  $<\,\SI{2}{mK}$ at all places and times  ( Match precision requirement at all places, at all times ) \\   \colhline
    
    Rationale &   The stability of the thermometers should be such that they meet the required precision and can hold this for the entire duration of their operations. Their longevity is defined as the time they are stable.  \\ \colhline
    Validation &   \\
   \colhline
\rowcolor{dunesky}
\newtag{SP-CISC-27}{ spec:camera-cold-coverage } & Name: Coverage \\ 
    Description & The cold cameras are required to cover at least 80\% of the exterior of HV surfaces.   \\  \colhline
    Specification (Goal) &  $>\,$80\% of HV surfaces  ( \num{100}\% ) \\   \colhline
    
    Rationale &   To enable detailed inspection of any issues near HV surfaces.  \\ \colhline
    Validation &   \\
   \colhline
\rowcolor{dunesky}
\newtag{SP-CISC-51}{ spec:slowcontrol-alarm-rate } & Name: Alarm rate \\ 
    Description & The total number of alarms/day seen by operators need to be less than 150.   \\  \colhline
    Specification (Goal) &  $<\,$150/day  ( $<\,$50/day ) \\   \colhline
    
    Rationale &   Systems expected to define alarms so as to get manageable alarm rate, suggested as less than 150 per day. Intended to allow experiment operators to "respond" to every alarm.  \\ \colhline
    Validation &   \\
   \colhline
\rowcolor{dunesky}
\newtag{SP-CISC-52}{ spec:slowcontrol-num-vars } & Name: Total No. of variables \\ 
    Description & This is the total number of variables monitored by slow controls from all subsystems of the detector.   \\  \colhline
    Specification (Goal) &  $>\,\num{150000}$  ( \SIrange{150000}{200000} ) \\   \colhline
    
    Rationale &   The total number of variables estimated here is based on extrapolation from ProtoDUNE-SP. The selected base software framework needs to be flexible enough to be able to handle large number of variables of the order of few 100k and monitor/archive them at the rates needed.  \\ \colhline
    Validation &   \\
   \colhline
\rowcolor{dunesky}
\newtag{SP-CISC-54}{ spec:slowcontrol-archive-rate } & Name: Archiving rate \\ 
    Description & Slow control quantities will need to archived at a rate that ranges from 0.02 Hz to 1 per few minutes, depending on the slow controls quantity.   \\  \colhline
    Specification (Goal) &  \SI{0.02}{Hz}  ( Broad range \SI{1}{Hz} to \num{1} per few min. ) \\   \colhline
    
    Rationale &   Slow Controls will consist of tens of thousands of variables. These variables will be archived at a different rate depending on the rate of change expected for a given variable and sensitivity of the variable being monitored. The archiving rate will also be limited by how much data per second that one wishes to store.  \\ \colhline
    Validation &   \\
   \colhline


\end{longtable}
