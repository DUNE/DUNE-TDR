\begin{table}[htp]
  \caption{Specification for SP-FD-14 \fixmehl{ref \texttt{tab:spec:sp-signal-saturation}}}
  \centering
  \begin{tabular}{p{0.2\textwidth}p{0.75\textwidth}} 
     \rowcolor{dunesky}
    \newtag{SP-FD-14}{ spec:sp-signal-saturation } 
                & Name: Signal saturation level    \\ 
    Description & The signal saturation level shall be set so as to see saturation in less than 10\% of beam-produced events. The chosen value corresponds to one stopping and two highly-ionizing protons from one primary vertex, with trajectories at 45 degrees relative to the beam axis.   \\  \colhline
    
    Specification &  \spsignalsat \\   \colhline
    
    Rationale &   The largest signals correspond to events with multiple protons produced in the primary event vertex, in particular, when the trajectories of one or more of those particles are parallel to the wire, causing the charge over a long path length to be collected within a short time period.    \\ \colhline
    Validation &   \\
   \colhline
  \end{tabular}
  \label{tab:spec:sp-signal-saturation}
\end{table}