The huge amount of charged current muon-neutrino argon interactions produced in 
the near detector will be an important data sample to understand better neutrino-argon 
interactions in the relevant energy range for the DUNE far detector. The next chapters
will give examples of scenarios where detailed understand of such interactions with 
precision measurements will have a significant impact on the physics reach for some
topics. Effects of final state interactions, event topology and kinematics, neutron production and more will be can be studied in detail with such large statistics 
data samples.

The collection of the expected statistics and the determination of the neutrino and
antineutrino fluxes to unprecedented precision would solve  two main limitations of past neutrino experiments. At the same time, we can then exploit the unique
properties of the (anti)neutrino probe for the study of fundamental interactions with a broad program of precision Standard Model measurements. These potential measurements
have not yet been studied in detail in this Technical Design Report, as the capabilities
depend critically on the final design choice of the Near Detector, and this is still under discussion.

Neutrinos and anti-neutrinos are the most effective probes for investigating Electroweak 
physics. A precise determination of the weak mixing angle (sin$^2\theta_W$) in (anti)neutrino
scattering at the DUNE energies is twofold: (a) it provides a direct measurement of neutrino
couplings to the Z boson and (b) it probes a different scale of momentum transfer than LEP
did by virtue of not being at the Z boson mass peak. The unprecedented large statistics
of deep inelastic scattering events will allow for significant measurements of the mixing 
angle. Other Standard Model measurements include those of nucleon structure functions, the strange content of nucleons, and a precise verification of a number of sum rules. 
Some of these measurements would need cross section measurements 
on hydrogen targets.  These expected sensitivity of these measurements will be addressed
in future studies.


