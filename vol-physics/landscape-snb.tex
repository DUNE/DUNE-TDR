
%%%%%%%%%%%%%%%%%%%%%%%%%%%%%%%%%%%%%%%%%%%%%%%%%%%%%%%%%%%%%%%%
%\section{Low-Energy Neutrinos from Supernovae and Other Sources}
%\label{sec:landscape-snb}


The burst of neutrinos from the celebrated core-collapse supernova 1987A in the Large Magellanic Cloud, about
50~kpc from Earth, heralded the era of extragalactic neutrino
astronomy.  The few dozen recorded $\bar{\nu}_e$ events
have confirmed the basic physical
picture of core collapse and yielded constraints on a wide range of new
physics~\cite{Schramm:1990pf, Vissani:2014doa}.   This sample has nourished physicists and
astrophysicists for many years, but has
by now been thoroughly picked over.  The community anticipates a
much more sumptuous feast of data when the next nearby star collapses.

Core-collapse supernovae within a few hundred kiloparsecs of Earth --
within our own Galaxy and nearby -- are quite rare on a human
timescale.  They are expected once every few decades in the Milky Way
(within about 20~kpc), and with a similar rate in Andromeda, about
700~kpc away.  However core collapses should be common enough to have
a reasonable chance of occurring during the few-decade long lifetime
of a typical large-scale neutrino detector.  The rarity of these
spectacular events makes it all the more critical for the community to
be prepared to capture every last bit of information from them.

The information in a supernova neutrino burst available in principle
to be gathered by experimentalists is the \textit{flavor, energy and
  time structure} of several-tens-of-second-long, all-flavor,
few-tens-of-MeV neutrino burst~\cite{Mirizzi:2015eza, Horiuchi:2017sku}.  Imprinted on
the neutrino spectrum as a function of time is information about the
progenitor, the collapse, the explosion, and the remnant, as well as
information about neutrino parameters and potentially exotic new
physics.  Neutrino energies and flavor content of the burst can be
measured only imperfectly, due to intrinsic nature of the weak
interactions of neutrinos with matter, as well as due to imperfect
detection resolution in any real detector.  For example, supernova
burst energies are below charged-current threshold for $\nu_\mu$,
$\nu_\tau$, $\bar{\nu}_\mu$ and $\bar{\nu}_{\tau}$ (collectively
$\nu_x$), which represent two-thirds of the flux; so these flavors are
accessible only via neutral-current interactions, which tend to have
low cross sections and indistinct detector signatures. These issues make a
comprehensive unfolding of neutrino flavor, time and energy structure
from the observed interactions a challenging problem.

Much has occurred since 1987, both for experimental and theoretical
aspects of supernova neutrino detection.
There has been huge progress in the modeling of supernova explosions,
and there have been many new theoretical insights about
neutrino oscillation and exotic collective effects that may occur in
the supernova environment.    Experimentally,
worldwide detection capabilities have increased enormously, such that
we now expect several thousands of events from a core collapse at the center
of the Galaxy.

\subsection{Current Experimental Landscape}
At the time of this writing, Super-Kamiokande is the leading supernova
neutrino detector; it expects $\sim$8000 events at 10~kpc.  As for
the 1987A sample, these will be primarily $\bar{\nu}_e$ flavor via
inverse beta decay (IBD) on free protons.  Super-K will soon be
enhanced with the addition of gadolinium, which will aid in IBD
tagging.  IceCube is another water detector, with a different kind of
supernova neutrino sensitivity -- it cannot reconstruct individual
neutrino events, given that any given interaction in the ice rarely
leads to more than one photoelectron detected.  However it can measure
the overall supernova neutrino ``light curve'' as a glow of photons
over background counts.  Scintillator detectors, made of hydrocarbon,
also have high IBD rates.  There are several kton-scale scintillator
detectors online currently: these are KamLAND, LVD, and Borexino.
There is one small lead-based detector, HALO.  Some surface or
near-surface detectors will also usefully record counts even in the
presence of significant cosmogenic background: these include NOvA,
Daya Bay, and MicroBooNE.

In the world's current supernova neutrino flavor sensitivity
portfolio~\cite{Scholberg:2012id, Mirizzi:2015eza}, the sensitivity is primarily to electron antineutrino
flavor, via IBD. There is only minor sensitivity to the $\nu_e$
component of the flux, which carries with it particularly interesting
information content of the burst (e.g., neutronization burst neutrinos
are created primarily as $\nu_e$).  While there is some $\nu_e$
sensitivity in other detectors via elastic scattering on electrons and
via subdominant channels on nuclei, statistics are relatively small,
and it can be difficult to disentangle the flavor content.
Neutral-current channels are also of particular interest, given their
sensitivity to the entire supernova flux; the only way to access the
$\nu_x$ component is via NC.  NC channels are subdominant in large
neutrino detectors, and typically difficult to tag, although
scintillator has some sensitivity via NC excitation of $^{12}$C as
well as elastic scattering on protons.  Dark matter detectors have
access to the entire supernova flux via NC coherent elastic
neutrino-nucleus scattering on nuclei, with statistics at the level of
 of $\sim$10 events per ton at 10~kpc.

\subsection{Projected Landscape in the DUNE Era}
The next generation of supernova neutrino detectors, in the era of
DUNE, will be dominated by Hyper-Kamiokande, JUNO and DUNE.  Hyper-K
and JUNO are sensitive primarily to $\bar{\nu}_e$, and will have
potentially enormous statistics.  The next-generation long-string
water detectors, IceCube and KM3Net, will bring their timing
strengths.   New tens-of-ton scale
noble liquid detectors such as DARWIN will bring new full-flux NC
sensitivity. 
DUNE will bring unique $\nu_e$ sensitivity: it will offer
a new opportunity to measure the $\nu_e$ content of the burst with
high statistics and good event reconstruction.

The past decade has also brought rapid evolution of
\textit{multi-messenger astronomy}.  With the advent of gravitational
waves detection, and high-energy extragalactic neutrino detection in
IceCube, a broad community of physicists and astronomers are now
collaborating to extract maximum information from observation in a
huge range of electromagnetic wavelengths, neutrinos, charged particles
and gravitational waves.  This collaboration resulted in the
spectacular multimessenger observation of a kilonova~\cite{kilonova}.  The
next core-collapse supernova will be a similar multimessenger
extravaganza.  Worldwide neutrino detectors are currently participants
in SNEWS, the SuperNova Early Warning System~\cite{snews}, which will be
upgraded to have enhanced capabilities over the next few
years.  Information from DUNE will enhance the SNEWS
network's reach.

Neutrino pointing information is vital for prompt multi-messenger
capabilities.  Only some supernova neutrino detectors have the ability
to point back to the source of neutrinos.  Imaging water Cherenkov
detectors like Super-K can do well at this, via directional
reconstruction of neutrino-electron elastic scattering events. However other detectors
lack pointing ability, due to intrinsic quasi-isotropy of the neutrino
interactions, combined with lack of detector sensitivity to
final-state directionality.  Like Super-K, DUNE is capable of pointing
to the supernova via its good tracking ability.

\subsection{The Role of DUNE}
Supernova neutrino detection is more of a collaborative than a
competitive game.  The more information gathered by detectors
worldwide, the more extensive the knowledge to be gained; the whole is
more than the sum of the parts.  The flavor sensitivity of DUNE is
highly complementary to that of the other detectors, and will bring
critical information for reconstruction of the entire burst's flavor and
spectral content as a function of time~\cite{Ankowski:2016lab}.

\subsection{Beyond Core Collapse}
While a core-collapse burst is a known source of a
low-energy ($<$100 MeV) neutrinos, there are other potential
interesting sources of neutrinos in this energy range.  Nearby
thermonuclear or pair instability supernova events may create bursts
as well, although they are expected to be fainter in neutrinos than
core-collapse supernovae.  Mergers of neutron stars and black holes
will be low-energy neutrino sources, although the rate of these nearby
enough to detect will be small.  There are also interesting
steady-state sources of low-energy neutrinos -- in particular, there
may still be useful oscillation and solar physics information to
extract via measurement of the solar neutrino flux. DUNE will have the
unique capability of measuring solar neutrino energies event by event
with the $\nu_e$CC interactions with large statistics, in contrast to
other detectors primarily make use of recoil spectra.  The technical
challenge for solar neutrinos is overcoming radiological and
cosmogenic backgrounds, although preliminary studies are promising.
The diffuse supernova neutrino background neutrinos are another target
which have a bit higher energy, but which are much more challenging due to very low
event rate.  There may also be surprises in store for us, both from burst
and steady-state signals, enabled by unique DUNE liquid argon tracking
technology.

