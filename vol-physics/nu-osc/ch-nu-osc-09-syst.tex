\section{Detector Model and Uncertainties}
\label{sec:physics-lbnosc-syst}\label{sec:nu-osc-09}

%Systematic uncertainties can be broadly divided into three categories: flux, neutrino interactions, and detector effects. Flux uncertainties are described in Section~\ref{sec:nu-osc-04}. Neutrino cross section uncertainties are described in Section~\ref{sec:nu-osc-05}. Detector-related uncertainties are described in Section~\ref{sec:detSysts}. The remainder of this section will cover the ways in which uncertainties are constrained in the near-far oscillation fit, and the individual uncertainties with the largest impact on oscillation parameter sensitivities.

Detector effects impact the event selection efficiency as well as the reconstruction of quantities used in the oscillation fit, such as neutrino energy. The main sources of detector systematic uncertainties are limitations of calibration and modeling of particles in the detector. While neutrino interaction uncertainties can also affect reconstruction, this section is focused on effects that arise from the detectors.

The near \dword{lartpc} detector uses a similar technology as the far detector, namely they are both \dwords{lartpc}. However, important differences lead to uncertainties that do not fully correlate between the two detectors. First, the readout technology is different, as the near \dword{lartpc} uses pixels as well as a different, modular photon detector. Therefore, the charge response to particle types (e.g., muons and protons) will be different between near and far due to differences in electronics readout, noise, and local effects like alignment.  Second, the high-intensity environment of the \dword{nd} complicates associating detached energy deposits to events, a problem which does not exist in the \dword{fd}. Third, the calibration programs will be different. For example, the \dword{nd} has a high-statistics calibration sample of through-going, momentum-analyzed muons from neutrino interactions in the upstream rock, which does not exist for the \dword{fd}. 
%Conversely, the local \efield in the \dword{nd} will be different than the \dword{fd} due to the difference in detector technology. by ion accumulation (space charge) as it has much less overburden than the \dword{fd}. %KM removed because this may be small due to the differnet drift. but need to be studied.
Finally, the reconstruction efficiency will be inherently different due to the relatively small size of the \dword{nd}. Containment of charged hadrons will be significantly worse at the \dword{nd}, especially for events with energetic hadronic showers or with vertices near the edges of the fiducial volume. Detector systematic uncertainties in the \dword{gartpc} at the near site will be entirely uncorrelated to the \dword{fd}.

\subsection{Energy Scale Uncertainties}
\label{sec:EnergyScaleSysts}

An uncertainty on the overall energy scale is included in the analysis presented here, as well as particle response uncertainties that are separate and uncorrelated between four species: muons, charged hadrons, neutrons, and electromagnetic showers. In the \dword{nd}, muons reconstructed by range in \dword{lar} and by curvature in \dword{mpd} are treated separately. The energy scale and particle response uncertainties are allowed to vary with energy; each term is described by three free parameters:

\begin{equation}
\label{eq:escale_unc}    
E^{\prime}_{rec} = E_{rec} \times (p_{0} + p_{1}\sqrt{E_{rec}} + \frac{p_{2}}{\sqrt{E_{rec}}})
\end{equation}

\noindent
where $E_{rec}$ is the nominal reconstructed energy, $E^{\prime}_{rec}$ is the shifted energy, and $p_{0}$, $p_{1}$, and $p_{2}$ are free fit parameters that are allowed to vary within \textit{a priori} constraints. The energy scale and resolution parameters are conservatively treated as uncorrelated between the \dword{nd} and \dword{fd}. With a better understanding of the relationship between \dword{nd} and \dword{fd} calibration and reconstruction techniques, it may be possible to correlate some portion of the energy response. The full list of energy scale uncertainties is given as Table~\ref{tab:EscaleSysts}. Uncertainties on energy resolutions are also included and are taken to be 2\% for muons, charged hadrons, and EM showers and 40\% for neutrons.

\begin{dunetable}[Energy scale systematics]{c|ccc}{tab:EscaleSysts}
{Uncertainties applied to the energy response of various particles. $p_{0}$, $p_{1}$, and $p_{2}$ correspond to the constant, square root, and inverse square root terms in the energy response parameterization given in Equation~\ref{eq:escale_unc}. All are treated as uncorrelated between the \dword{nd} and \dword{fd}.}
    Particle           & $p_{0}$ & $p_{1}$ & $p_{2}$ \\ \toprowrule
    all (except muons) & 2\%   & 1\%   & 2\%   \\
    $\mu$ (range)      & 2\%   & 2\%   & 2\%   \\
    $\mu$ (curvature)  & 1\%   & 1\%   & 1\%   \\
    p, $\pi^{\pm}$     & 5\%   & 5\%   & 5\%   \\
    e, $\gamma$, $\pi^{0}$ & 2.5\%   & 2.5\%   & 2.5\%   \\
    n                  & 20\%  & 30\%  & 30\%  \\
    \hline
\end{dunetable} 

The scale of these uncertainties is derived from recent experiments, including calorimetric based approaches (\dword{nova}, \dword{minerva}) and \dwords{lartpc} (\dword{lariat}, \dword{microboone}, \dword{argoneut}). On \dword{nova}~\cite{NOvA:2018gge}, the muon (proton) energy scale achieved is $<1$\% (5\%). Uncertainties associated to the pion and proton re-interactions in the detector medium are expected to be controlled from \dword{protodune} and \dword{lariat} data, as well as the combined analysis of low density (gaseous) and high density (\dword{lar}) \dwords{nd}. Uncertainties in the \efield also contribute to the energy scale uncertainty, and calibration is needed (with cosmics at \dword{nd}, laser system at \dword{fd}) to constrain the overall energy scale. The recombination model will continue to be validated by the suite of \dword{lar} experiments and is not expected to be an issue for nominal field provided minimal \efield distortions. Uncertainties in the electronics response are controlled with dedicated charge injection system and validated with intrinsic sources, Michel electrons and \Ar39.

The response of the detector to neutrons is a source of active study and will couple strongly to detector technology. The validation of neutron interactions in \dword{lar} will continue to be characterized by dedicated measurements (e.g., CAPTAIN~\cite{Berns:2013usa,Bhandari:2019rat}) and the \dword{lar} program (e.g., \dword{argoneut}~\cite{Acciarri:2018myr}).  However, the association of the identification of a neutron scatter or capture to the neutron's true energy has not been demonstrated, and significant reconstruction issues exist, so a large uncertainty (20\%) is assigned comparable to the observations made by \dword{minerva}~\cite{Elkins:2019vmy} assuming they are attributed entirely to the detector model. Selection of photon candidates from $\pi^0$ is also a significant reconstruction challenge, but a recent measurement from \dword{microboone} indicates this is possible and the $\pi^0$ invariant mass has an uncertainty of 5\%, although with some bias~\cite{Adams:2018sgn}.

\subsection{Acceptance and Reconstruction Efficiency Uncertainties}

The \dword{nd} and \dword{fd} have different acceptance to \dword{cc} events due to the very different detector sizes. The \dword{fd} is sufficiently large that acceptance is not expected to vary significantly as a function of event kinematics. However, the \dword{nd} selection requires that hadronic showers be well contained in \dword{lar} to ensure a good energy resolution, resulting in a loss of acceptance for events with energetic hadronic showers. The \dword{nd} also has regions of muon phase space with lower acceptance due to tracks exiting the side of the \dword{tpc} but failing to match to the \dword{mpd}.

Uncertainties are evaluated on the muon and hadron acceptance of the \dword{nd}. The detector acceptance for muons and hadrons is shown in Figure~\ref{fig:NDacceptance}. Inefficiency at very low lepton energy is due to events being misreconstructed as neutral current, which can also be seen in Figure~\ref{fig:NDacceptance}. For high energy, forward muons, the inefficiency is only due to events near the edge of the fiducial volume where the muon happens to miss the \dword{mpd}. At high transverse momentum, muons begin to exit the side of the \dword{lar} active volume, except when they happen to go along the 7 m axis. The acceptance is sensitive to the modeling of muons in the detector. An uncertainty is estimated based on the change in the acceptance as a function of muon kinematics. This uncertainty can be constrained with the \dword{mpd} by comparing the muon spectrum in \dword{cc} interactions between the liquid and gaseous argon targets. The acceptance in the \dword{mpd} is expected to be nearly 4$\pi$ due to the excellent tracking and lack of scattering in the detector. Since the target nucleus is the same, and the two detectors are exposed to the same flux, the ratio between the two detectors is dominated by the \dword{lar} acceptance. Given the rate in the \dword{mpd}, the expected constraint is at the level of $\sim$0.5\% in the peak and $\sim$3\% in the tail.

Inefficiency at high hadronic energy is due to the veto on more than 30 MeV deposited in the outer 30 cm collar of the active volume. Rejected events are typically poorly reconstructed due to low containment, and the acceptance is expected to decrease at high hadronic energy. Similar to the muon reconstruction, this acceptance is sensitive to detector modeling, and an uncertainty is evaluated based on the change in the acceptance as a function of true hadronic energy. This is more difficult to constrain with the \dword{mpd} because of the uncertain mapping between true and visible hadronic energy in the \dword{lar}.

