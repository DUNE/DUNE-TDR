\section{Appendix: Near Detector and Uncertainties}\label{sec:nu-osc-12}\label{sec:physics-lbnosc-ND-app}

The near detector concept is described in Section~\ref{sec:nu-osc-06}. For the oscillation sensitivity analysis presented in this document, the pressure vessel is assumed to be 3 cm titanium. The ECal tiles are $10 \times 10 \times 5$~mm, with a 2 mm copper absorber. The inner-most 10 ECal layers are inside the pressure vessel, giving excellent angular resolution to photon-induced showers, which generally do not convert in the gas. An additional 20 ECal layers are positioned outside the pressure vessel to contain energy from these showers. The magnet is a solenoid with an inner radius of 320 cm and total length along the cylinder axis of 780 cm. The yoke is cut out in the upstream barrel to minimize the passive material between the two TPC detectors.

