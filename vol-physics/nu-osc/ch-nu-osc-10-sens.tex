\section{Sensitivities}
%{\it Assigned to:} {\bf Mayly Sanchez} with contributions from Elizabeth Worcester. 
\label{sec:physics-lbnosc-results}
\subsection{Mass Hierarchy}
\label{sec:physics-lbnosc-mh}

The \kmadj{1300} baseline establishes one of DUNE's key strengths: sensitivity to the matter effect. This effect leads to a large asymmetry in the $\nu_\mu\to \nu_e$ versus $\bar{\nu}_\mu \to \bar{\nu}_e$ oscillation probabilities, the sign of which depends on the mass hierarchy (MH).  At 1300~km this asymmetry is approximately $\pm 40\%$ in the region of the peak flux; this is larger than the maximal possible CP-violating asymmetry associated with \deltacp, meaning that both the MH and \deltacp can be determined
unambiguously with high confidence within the same experiment using the beam neutrinos.  DUNE's goal is to determine the MH with a significance of at least $\sqrt{\overline{\Delta\chi^{2}}} = 5$ for all \deltacp values using beam neutrinos.  Concurrent analysis of the corresponding atmospheric-neutrino samples will improve the precision with which the MH is resolved.

Figure~\ref{fig:mh_nominal} shows the significance with which the MH can be determined as a function of the value of \deltacp, for an exposure of \SI{300}~\ktMWyr, which corresponds to seven years of data (3.5 years in neutrino mode plus 3.5 years in antineutrino mode) with a 40-kt detector and a 1.07-MW 80-GeV beam \todo{Establish baseline plot configuration}.  

\fixme{Update figures/interpret results} Update all figures using chosen framework(s) with/without osc and syst uncertainties using new plot format:
\begin{itemize}
\item CDR Figure 3.7: vs dcp
\item CDR Figure 3.8: vs exposure
\item CDR Figures 3.9-3.11: vs dcp w t23, t13, dm31 variations
\item CDR Figure 3.12: Statistical fluctuations (vs dcp)
\end{itemize}

Studies have indicated that special attention must be paid to the statistical interpretation of MH sensitivities~\cite{Qian:2012zn,Blennow:2013oma}.
In general, if an experiment is repeated many times, a distribution of $\Delta\chi^2$
values will appear due to statistical fluctuations.
It is usually assumed that the $\Delta \chi^2$ metric follows the expected chi-squared
function for one degree of freedom, which has a mean of
$\overline{\Delta\chi^2}$ and can be interpreted using a Gaussian
distribution with a standard deviation of
$\sqrt{|\overline{\Delta\chi^2}|}$.
In assessing the MH sensitivity of future experiments, it is common practice to generate
a simulated data set (for an assumed true MH) that does not include statistical fluctuations. 
In this typical case, $\overline{\Delta\chi^2}$ is reported as the expected sensitivity, 
where $\overline{\Delta\chi^2}$ is representative of the mean value of $\Delta\chi^2$ that 
would be obtained in an ensemble of experiments for a particular true MH.  
With the exception of Figure~\ref{fig:mhstats}, the sensitivity plots
in this document have been generated using this method.
However, studies in~\cite{Qian:2012zn,Blennow:2013oma}
show that, in the case of the mass hierarchy
determination, the $\Delta \chi^2$ metric {\em does not} follow the expected chi-squared
function for one degree of freedom.  Rather, these studies show that
when the observed counts in the experiment are large enough,
the distribution of $\Delta\chi^2$ used here approximately follows
a Gaussian distribution with a
mean and standard deviation of $\overline{\Delta\chi^2}$ and
$2\sqrt{|\overline{\Delta\chi^2}|}$, respectively. Because the distribution is atypical, the interpretation of 
test statistic values in terms of confidence intervals is different than in the standard case. The effect of statistical fluctuations in the MH measurement is shown
in Figure~\ref{fig:mhstats}.



\subsection{CP-Symmetry Violation}
\label{sec:physics-lbnosc-cpv}

In the particular parameterization of the PMNS matrix shown in
Equation~\ref{eqn:pmns}, the middle factor labeled ``II'' describes
the mixing between the $\nu_1$ and $\nu_3$ mass states, and depends on
the CP-violating phase \deltacp.  With the recent measurement of
$\theta_{13}$, it is now known that the minimal conditions required
for measuring \deltacp in the three-flavor framework have been met;
all three mixing angles are nonzero, and there are two distinct mass
splittings.  In the approximation for the electron neutrino appearance
probability given in Equation~\ref{eqn:appprob}, expanding the middle
term results in the presence of CP-odd terms (dependent on $\sin
\mdeltacp$) that have opposite signs in $\nu_{\mu} \rightarrow \nu_e$
and $\bar{\nu}_{\mu} \rightarrow \bar{\nu}_e$ oscillations.
For $\mdeltacp \neq 0$ or $\pi$, these terms introduce an asymmetry in
neutrino versus antineutrino oscillations. The magnitude of the
CP-violating terms in the oscillation depends most directly on the
size of the Jarlskog Invariant~\cite{Jarlskog:1985cw}, a function that
was introduced to provide a measure of CP violation independent of the
mixing-matrix parameterization. In terms of the parameterization
presented in Equation~\ref{eqn:pmns}, the Jarlskog Invariant is:
%
\begin{equation}
J_{CP}^{\rm PMNS} \equiv \frac{1}{8} \sin 2 \theta_{12} \sin 2 \theta_{13}
\sin 2 \theta_{23} \cos \theta_{13} \sin \mdeltacp.
\end{equation}
The relatively large values of the mixing angles in the lepton sector imply that
leptonic CP-violation effects may be quite large ---  
depending on the value of the phase \deltacp, which is currently unknown. 
Experimentally, it is unconstrained at the 3$\sigma$ level by the global fit~\cite{Gonzalez-Garcia:2014bfa}.
Given the current best-fit values of the mixing angles~\cite{Gonzalez-Garcia:2014bfa} and assuming normal hierarchy,
\begin{equation}
J_{CP}^{\rm PMNS} \approx 0.03 \sin \mdeltacp.
\end{equation}
This is in sharp contrast to the very small mixing in the quark sector,  
which leads to a very small value of the corresponding quark-sector
Jarlskog Invariant~\cite{Beringer:1900zz},
\begin{equation}
J_{CP}^{\rm CKM} \approx 3 \times 10^{-5},
\end{equation}
despite the large value of $\delta^{\rm CKM}_{CP}\approx70^{\circ}$.

The variation in the $\nu_\mu \rightarrow
\nu_e$ oscillation probability (Equation~\ref{eqn:appprob}) with the value of \deltacp
indicates that it is experimentally possible to measure the value of
\deltacp at a fixed baseline using only the observed shape of the
$\nu_\mu \rightarrow \nu_e$ {\em or} the 
$\bar{\nu}_\mu \rightarrow \bar{\nu}_e$
appearance signal measured over an energy range that encompasses at
least one full oscillation interval. A measurement of the value of
$\mdeltacp \neq 0 \ {\rm or} \ \pi$, assuming that neutrino mixing follows the three-flavor model, would imply CP violation.  

The CP asymmetry,
$\mathcal{A}_{CP}$, is defined as 
\begin{equation}
\label{eqn:cp-asymm}
 \mathcal{A}_{CP} = \frac{P(\nu_\mu \rightarrow \nu_e) -
  P(\bar{\nu}_\mu \rightarrow \bar{\nu}_e)}{P(\nu_\mu \rightarrow
  \nu_e) + P(\bar{\nu}_\mu \rightarrow \bar{\nu}_e)}.
\end{equation}
In the three-flavor model the asymmetry can be approximated to leading
order in $\Delta m_{21}^2$ as~\cite{Marciano:2006uc}:
\begin{equation}
\mathcal{A}_{CP} \sim \frac{\cos \theta_{23} \sin 2 \theta_{12}
  {\sin \mdeltacp}}{\sin \theta_{23} \sin \theta_{13}}
\left(\frac{\Delta m^2_{21} L}{ 4 E_{\nu}}\right) + {\rm matter
  \ effects}
\label{eqn:cpasym}
\end{equation}
Regardless of the measured value obtained for \deltacp, the explicit
observation of the asymmetry $\mathcal{A}_{CP}$ in $\nu_{\mu}
\rightarrow \nu_e$ and $\bar{\nu}_{\mu} \rightarrow
\bar{\nu}_e$ oscillations is sought to directly demonstrate the
leptonic CP-violation effect.  A measurement of \deltacp that is
inconsistent with the measurement of $\mathcal{A}_{CP}$ according to
Equation~\ref{eqn:cpasym} could be evidence of physics beyond the
standard three-flavor model.  Furthermore, for long-baseline
experiments such as DUNE where the neutrino beam propagates through
the Earth's mantle, the leptonic CP-violation effects must be
disentangled from the matter effects, discussed in
Section~\ref{sec:physics-lbnosc-mh}.

Figure~\ref{fig:cpv_nominal} shows the significance with which the CP
violation ($\mdeltacp \neq 0 \ {\rm or} \ \pi$) can be determined as a
function of the value of \deltacp for an exposure of \SI{300}~\ktMWyr,
which corresponds to seven years of data (3.5 years in neutrino mode
plus 3.5 years in antineutrino mode) with a 40-kt detector and a
1.07-MW 80-GeV beam.  Figure~\ref{fig:cpv_exposure} shows the significance
with which CP violation can be determined for 25\%, 50\% or 75\% of \deltacp
values as a function of exposure.
Table~\ref{tab:cpv_requiredexposure} lists the minimum exposure
required to determine CP violation with a significance of 5$\sigma$
for 50\% of \deltacp values or 3$\sigma$ for 75\% of \deltacp values
for both the CDR reference beam design and the optimized beam design.
The CP-violation sensitivity as a function of \deltacp as shown in
Figure~\ref{fig:cpv_nominal} has a characteristic double peak
structure because the significance of a CP-violation measurement
necessarily drops to zero where there is no CP violation: at the
CP-conserving values of $-\pi,~0,~{\rm and}~\pi$.  Therefore, unlike
the MH determination, it's not possible for any experiment to provide
100\% coverage in \deltacp for a CP-violation measurement because CP-violation effects vanish at certain values of \deltacp.


\fixme{Update figures/interpret results} Update all figures using chosen framework(s) with/without osc and syst uncertainties using new plot format:
\begin{itemize}
	\item CDR Figure 3.13: vs dcp
	\item CDR Table 3.7: Minimum exposure for CP violation at $3\sigma$ for 75\% of dcp, and $5\sigma$ for 50\% of dcp
	\item CDR Figure 3.14: vs exposure
	\item CDR Figures 3.15-3.17: vs dcp, w t23, t13, dm31 variations
\end{itemize}

\subsection{Precision Oscillation Parameter Measurements}
\label{sec:physics-lbnosc-prec}

In addition to the discovery potential for neutrino mass hierarchy and CP-violation, 
DUNE will improve the precision on key parameters that govern neutrino oscillations, including:
\begin{itemize}
 \item $\sin^2\theta_{23}$ and the octant of $\theta_{23}$
 \item \deltacp
 \item $\sin^22\theta_{13}$
 \item \dm{31}
\end{itemize}

Higher-precision measurements of the known oscillation parameters improves sensitivity to physics beyond the three-flavor oscillation model, particularly when compared to independent measurements by other experiments, including reactor measurements of $\theta_{13}$ and
measurements with atmospheric neutrinos. 

The most precise measurement of $\sin^2\theta_{23}$ to date comes from T2K, $\sin^2\theta_{23} = 0.514^{+0.055}_{-0.056}$ (normal hierarchy) and $\sin^2\theta_{23} = 0.511~\pm~0.055$ (inverted hierarchy)~\cite{Abe:2015awa} \todo{Update to latest value}.  This corresponds to a value of $\theta_{23}$ near 45\mbox{$^{\circ}$}, but leaves an ambiguity as to whether the value of $\theta_{23}$ is in the lower octant (less than 45\mbox{$^{\circ}$}), the upper octant (greater than 45\mbox{$^{\circ}$}), or exactly 45\mbox{$^{\circ}$}.  The value of $\sin^2 \theta_{23}$ from the global fit reported by~\cite{Gonzalez-Garcia:2014bfa} is $\sin ^2 \theta_{23} = 0.452
^{+0.052} _{-0.028} (1 \sigma)$ for normal hierarchy (NH), but the distribution of the $\chi^2$ from the global fit has another local minimum -- particularly if the MH is inverted -- at $\sin^2 \theta_{23} = 0.579 ^{+0.025} _{-0.037} (1 \sigma)$ \todo{Update to latest value}. A \emph{maximal} mixing value of $\sin^2 \theta_{23} =0.5$ is therefore still allowed by the data and the octant is still largely undetermined.  A value of
$\theta_{23}$ exactly equal to 45\mbox{$^{\circ}$} would indicate that $\nu_{\mu}$ and $\nu_{\tau}$ have equal contributions from $\nu_3$, which could be evidence for a previously unknown symmetry.  It is therefore important experimentally to determine the value of $\sin ^2
\theta_{23}$ with sufficient precision to determine the octant of $\theta_{23}$.  The measurement of $\nu_\mu \rightarrow \nu_\mu$ oscillations is sensitive to $\sin ^2 2 \theta_{23}$, whereas the measurement of $\nu_\mu \rightarrow \nu_e$ oscillations is sensitive to $\sin^2 \theta_{23}$.  A combination of both $\nu_e$ appearance and $\nu_\mu$ disappearance measurements can probe both maximal mixing and
the $\theta_{23}$ octant.  The $\Delta\chi^2$ metric is defined as:

\begin{eqnarray}
\Delta\chi^2_{octant} & = & |\chi^2_{\theta_{23}^{test}>45^\circ} - \chi^2_{\theta_{23}^{test}<45^\circ}|, \\ \nonumber
\end{eqnarray}
where the value of $\theta_{23}$ in the \emph{wrong} octant is constrained 
only to have a value within the \emph{wrong} octant (i.e., it is not required
to have the same value of $\sin^22\theta_{23}$ as the true value).
Figure~\ref{fig:octant} shows the sensitivity to determining the octant as a function of $\theta_{23}$.  Figure~\ref{fig:res_th23} shows the resolution of $\sin^2\theta_{23}$ as a function of exposure, assuming the true value is $\sin^2\theta_{23} = 0.45$ from the current global fit.

As mentioned in Section~\ref{sec:physics-lbnosc-cpv}, DUNE will seek not only to demonstrate explicit CP violation by observing a difference in the neutrino and antineutrino oscillation probabilities, but also to measure the value of the parameter \deltacp. Figure~\ref{fig:res_cp} shows the resolution of \deltacp as a function of exposure for a CP-conserving value (\deltacp = 0) and the value
that gives the maximum CP violation for normal MH (\deltacp =
90\mbox{$^{\circ}$}).  A minimum exposure of approximately
\SI{111}~\ktMWyr{} \todo{Update value} is required to measure \deltacp with a resolution of 10\mbox{$^{\circ}$} for
a true value $\mdeltacp = 0$.

The rich oscillation structure that can be observed by DUNE and the excellent particle identification capability of the detector will enable precision measurement  in a single experiment of all the mixing parameters governing $\nu_1$-$\nu_3$ and $\nu_2$-$\nu_3$ mixing. Theoretical models probing quark-lepton universality predict specific values of the mixing angles and the relations between them. The mixing angle $\theta_{13}$ is
expected to be measured accurately in reactor experiments by the end of the decade with a precision that will be limited by systematic uncertainties. 
The combined statistical and systematic uncertainty on the value of \sinstt{23}  from the Daya Bay reactor neutrino experiment, which has the lowest systematic error, is currently $\sim6$\% (\sinstt{13} $= 0.084\pm0.005$),
with a projected uncertainty of $\sim$3\% by 2017~\cite{Zhang:2015fya} \todo{Update to latest}.
While the constraint on $\theta_{13}$ from the reactor experiments will be important in the
early stages of DUNE for determining CP violation, measuring
\deltacp and determining the $\theta_{23}$ octant, 
DUNE itself will eventually be able to measure
$\theta_{13}$ independently with a similar precision to that expected from the reactor experiments. 
Whereas the reactor experiments measure $\theta_{13}$ using $\bar{\nu}_e$ disappearance, DUNE will measure it through $\nu_e$ and $\bar{\nu}_e$ appearance, thus providing an independent constraint on
the three-flavor mixing matrix.   Figure~\ref{fig:res_th13} shows the resolution of \sinstt{13} as a function of exposure, assuming the true value is \sinstt{13}$ = 0.085$ from the current global fit.

DUNE can also significantly improve the
resolution on the larger mass splitting beyond the precision of current experiments.  The current best-fit value for 
\dm{32} from MINOS is $|\Delta m^2_{32}| = (2.34\pm0.09)\times10^{-3}$~eV$^2$ (normal hierarchy) and $|\Delta m^2_{32}| = (2.37^{+0.11}_{-0.07})\times10^{-3}$~eV$^2$ (inverted hierarchy)~\cite{Sousa:2015bxa}, with comparable precision achieved by both Daya Bay and T2K \todo{Update to latest}. The precision on \dm{31} will ultimately depend on tight control of energy-scale systematic errors.  Figure~\ref{fig:res_dm2} shows the expected resolution of \dm{31} as a function of exposure, assuming the true value is \dm{31} = $2.457\times10^{-3}$~eV$^2$ from the current global fit \todo{update to latest}.


\fixme{Update figures/interpret results} Update all figures using chosen framework(s) with/without osc and syst uncertainties using new plot format:
\begin{itemize}
	\item CDR Figure 3.18: t23 octant sensitivity
	\item CDR Figures 3.19-3.22: t23, dcp, t13, dm31 resolutions vs exposure
\end{itemize}
