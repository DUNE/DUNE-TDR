\section{Overview and Theoretical Context}
%{\it Assigned to:} {\bf Mayly Sanchez} 

\label{sec:physics-lbnosc-context}

% {\bf [Note: Copied from Section 2.2 of the LBNE Science Document.]}  
The Standard Model of particle physics presents a remarkably accurate
description of the elementary particles and their
interactions. However, its limitations pose deeper questions about
Nature. With the discovery of the Higgs boson at CERN, the Standard
Model would be ``complete'' except for the discovery of neutrino
mixing, which indicated neutrinos had a very small but nonzero
mass. In the Standard Model, the simple Higgs mechanism is responsible
for both quark and charged lepton masses, quark mixing and
charge-parity (CP) violation. However, the small size of neutrino
masses and their relatively large mixing bears little resemblance to
quark masses and mixing, suggesting that different physics -- and
possibly different mass scales -- in the two sectors may be present,
thus motivating precision study of mixing and CP violation in the
lepton sector of the Standard Model. 
%\fixme{thus motivating?}

%\fixme{Wrote a better opening paragraph using text from the Sci. Opp. - MB}

DUNE plans to pursue a detailed study of neutrino mixing, resolve the
neutrino mass ordering, and search for CP violation in the lepton
sector by studying the oscillation patterns of
high-intensity \numu and \anumu %$\nu_\mu$ and $\bar{\nu}_\mu$
beams measured over a long baseline.  Neutrino oscillation arises from
mixing between the flavor 
(\nue, \numu, \nutau) and mass $(\nu_1,\, \nu_2,\, \nu_3)$ eigenstates
of neutrinos.
% corresponding to the weak and gravitational interactions, respectively. 
% This three-flavor-mixing
% scenario can be described by a rotation between the weak-interaction
% eigenstate basis $(\nu_e,\, \nu_\mu,\, \nu_\tau)$ and the basis of
% states of definite mass $(\nu_1,\, \nu_2,\, \nu_3)$.  
In direct correspondence with mixing in the quark sector, the transformations
between basis states is expressed in the form of a complex unitary
matrix, known as the \textit{PMNS mixing matrix}: 

\begin{equation}
\left(\begin{array}{ccc} \nu_e \\ \nu_\mu \\ \nu_\tau \end{array} \right)= 
\underbrace{
  \left(\begin{array}{ccc}
      U_{e 1} &  U_{e 2} & U_{e 3} \\ 
      U_{\mu1} &  U_{\mu2} & U_{\mu 3} \\ 
      U_{\tau 1} &  U_{\tau 2} & U_{\tau 3} 
    \end{array} \right)
}_{U_{\rm PMNS}} \left(\begin{array}{ccc} \nu_1 \\ \nu_2 \\ \nu_3 \end{array} \right).
\label{eqn:pmns0}
\end{equation}
The PMNS matrix in full generality depends on just three mixing angles
and a CP-violating phase\footnote{In the case of Majorana neutrinos, there are two additional CP phases, but they are unobservable in the oscillation processes.}.  The mixing angles and phase are designated
as $(\theta_{12},\, \theta_{23},\, \theta_{13})$ and
\deltacp.
This matrix can be parameterized as the product of three
two-flavor mixing matrices as follows~\cite{Schechter:1980gr}, where $c_{\alpha \beta}=\cos \theta_{\alpha \beta}$ and $s_{\alpha
 \beta}=\sin \theta_{\alpha \beta}$:

\begin{equation}
U_{\rm PMNS} = 
  \underbrace{
    \left( \begin{array}{ccc}
        1 & 0 & 0 \\ 
        0 & c_{23} & s_{23} \\ 
        0 & -s_{23} & c_{23}
    \end{array} \right)
  }_{\rm I}
\underbrace{
  \left( \begin{array}{ccc}
        c_{13} & 0  & e^{-i\mdeltacp} s_{13} \\ 
         0 & 1 & 0 \\ 
        e^{-i\mdeltacp} s_{13} & 0 & c_{13}
  \end{array} \right)   
  }_{\rm II}
\underbrace{
 \left( \begin{array}{ccc}
      c_{12} & s_{12} & 0 \\ 
      -s_{12} & c_{12} & 0 \\ 
      0 & 0 & 1
  \end{array} \right)
}_{\rm III}.
\label{eqn:pmns}
\end{equation}

The parameters of the PMNS
matrix determine the probability amplitudes of the neutrino
oscillation phenomena that arise from mixing.  The frequency of neutrino oscillation 
depends on the difference in the squares of the neutrino
masses, $\Delta m^{2}_{ij} \equiv m^{2}_{i} - m^{2}_{j}$; a set of three
neutrino mass states implies two independent mass-squared differences
(the ``solar'' mass splitting, $\Delta m^{2}_{21}$, and the ``atmospheric'' mass splitting, 
$\Delta m^{2}_{31}$), where $\Delta m^{2}_{31} = \Delta m^{2}_{32} + \Delta m^{2}_{21}$. The ordering of the
mass states is known as the \emph{neutrino mass ordering} or \emph{neutrino mass hierarchy}. An ordering of
$m_1 < m_2 < m_3$ is known as the \emph{normal ordering} since it matches
the mass ordering of the charged leptons in the Standard Model, whereas an ordering of $m_3 < m_1 < m_2$
is referred to as the \emph{inverted ordering}.

The entire complement of neutrino experiments to date has measured
five of the mixing parameters~\cite{Esteban:2018azc,deSalas:2017kay,Capozzi:2017yic}: the three angles $\theta_{12}$,
$\theta_{23}$, and $\theta_{13}$, and the two mass differences
$\Delta m^{2}_{21}$ and $\Delta m^{2}_{31}$. The sign of $\Delta
m^{2}_{21}$ is fixed by convention, but the neutrino mass ordering (i.e.: the sign of $\Delta m^{2}_{31}$) is unknown.
The values of $\theta_{12}$ and $\theta_{23}$ are large, while 
$\theta_{13}$ is smaller. The value of \deltacp is not well known, though global fits to neutrino oscillation data are beginning to provide some information on its value.
The absolute values of the entries of the PMNS matrix, which
contains information on the strength of flavor-changing weak decays in
the lepton sector, can be expressed in approximate form as
\begin{equation}
|U_{\rm PMNS}|\sim \left(\begin{array}{ccc} 0.8 & 0.5 & 0.1 \\ 0.5 & 0.6 & 0.7 \\ 0.3 & 0.6 & 0.7\end{array} \right),
\label{eq:pmnsmatrix}
\end{equation}
using values for the mixing angles given in Table~\ref{tab:oscpar_nufit}. 
While the three-flavor-mixing scenario for neutrinos is now well
established, the mixing parameters are not known to the same precision 
as are those in the
corresponding quark sector, and several important quantities, including
the value of \deltacp and the sign of the large mass splitting, are
still undetermined. 

The relationships between the values of the parameters in the neutrino
and quark sectors suggest that mixing in the two sectors is
qualitatively different. Illustrating this difference, the value of
the entries of the CKM quark-mixing matrix (analogous to the PMNS matrix for
neutrinos, and thus indicative of the strength of flavor-changing weak
decays in the quark sector) can be expressed in approximate form as
\begin{equation}
|V_{\rm CKM}|\sim \left(\begin{array}{ccc} 1 & 0.2 & 0.004\\ 0.2 & 1 & 0.04 \\ 0.008 & 0.04 & 1\end{array} \right),
\label{eq:ckmmatrix}
\end{equation}
for comparison to the entries of the PMNS matrix given in Equation~\ref{eq:pmnsmatrix}.
As discussed in \cite{King:2014nza}, the question of why the quark mixing angles are
smaller than the lepton mixing angles is an important part of the %``flavor problem.'' AH 5/9
flavor pattern question. 

To quote the discussion in~\cite{deGouvea:2013onf}, ``while the CKM
matrix is almost proportional to the identity matrix plus
hierarchically ordered off-diagonal elements, the PMNS matrix is far
from diagonal and, with the possible exception of the $U_{e3}$
element, all elements are ${\cal O}(1)$.''
%A special role is played here by the fact that the magnitude of the smaller of the lepton mixing angles is similar to the larger of the quark mixing parameters, namely the Cabibbo angle~\cite{Boucenna:2012xb}.  Anne 3/9
It is important here to note that the smaller of the lepton
mixing angles is of similar magnitude to the larger of the quark mixing parameters, namely the Cabibbo angle~\cite{Boucenna:2012xb}.
One theoretical method often used to address this question involves the use of non-Abelian discrete
subgroups of $SU(3)$ as flavor symmetries; the popularity of this method %comes partially 
is due in part from
the fact that these symmetries can give rise to the nearly \emph{tri-bi-maximal}\footnote{Tri-bi-maximal mixing refers to a form of the neutrino mixing matrix with effective bimaximal mixing of $\nu_\mu$ and $\nu_\tau$
at the atmospheric scale ($L/E \sim$ \SI{500}{\km / \GeV}) and effective trimaximal
mixing for $\nu_e$ with $\nu_\mu$ and $\nu_\tau$ 
at the solar scale ($L/E \sim$ \SI{15000}{\km / \GeV})~\cite{Harrison:2002er}.} 
structure of the PMNS matrix.
Whether employing these flavor symmetries or other methods,
any theoretical principle that attempts to describe the fundamental
symmetries implied by the observed organization of quark and neutrino
mixing --- such as those proposed in unification models --- leads to
testable predictions such as sum rules between CKM and PMNS
parameters~\cite{King:2014nza,deGouvea:2013onf,Mohapatra:2005wg,Albright:2006cw}.
Data on the patterns of neutrino mixing 
are already proving crucial in the quest for a 
relationship between quarks and leptons and their seemingly arbitrary generation
structure.  

Clearly much work remains in order to complete the standard three-flavor 
mixing picture, particularly 
with regard to $\theta_{23}$ (is it less than, greater than, or equal
to $45^\circ$?), mass ordering (normal or inverted?) 
and \deltacp.
Additionally, there is 
great value in obtaining a set of measurements for multiple parameters 
\emph{from a single experiment}, so that correlations and systematic 
uncertainties can be handled properly.  Such an experiment will also be 
well positioned to extensively test the standard picture of three-flavor mixing.  
DUNE is designed to be this experiment.