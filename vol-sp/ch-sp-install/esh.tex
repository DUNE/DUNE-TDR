\section{Environmental, Safety, and Health (ES\&H)}
\label{sec:fdsp-tc-safety}

 \tcchesh of the \dword{dune} \dword{tdr} outlines the requirements and regulations that \dword{dune} work must comply with, whether (1) at \dword{fnal}, (2) in areas  leased by \dword{fnal} or the \dword{doe}, (3) in leased space at \dword{surf}, or (4) at collaborating institutions.
 
%%%%%%%%%%%%%%%%%%%%%%%%%%%%%%%%%
\subsection{Documentation Approval Process}


\dword{dune} implements an engineering review and approval process for all required documentation, including structural calculations, assembly drawings, load tests, \dwords{ha}, and procedural documents for a comprehensive set of identified individual tasks. 
As for \dword{pdsp}, all these documents are stored in \dword{edms}. 
For the larger operations and systems like \dword{tpc} component factories, the \dword{dss}, cleanroom, and assembly infrastructure, \dword{dune} safety also reviews the documentation then visits the site to conduct
 an \dword{orr}, which includes a demonstration of the final operations. The \dwords{orr} are listed in project schedule. 
 
 Structural calculations, assembly drawings and proper documentation of  load tests, hazard analyses, and procedures for various items and activities will require review and approval before operational readiness is granted. 

 
%%%%%%%%%%%%%%%%%%%%%%%%%%%%%%%%%
\subsection{Support and Responsibilities}

The \dword{esh} coordinator for each shift, who will report to the \dword{dune} project \dword{esh} manager, has overall \dword{esh} oversight responsibility for the \dword{dune} activities at the  \dword{sdwf} and on the \dword{surf} site. 
This person coordinates any \dword{esh} activities and facilitates the resolution of any issues that are subject to the requirements of the \dword{doe} Workers Safety and Health Program, Title 10, Code Federal Regulations (CRF) Part 851 (10 CFR 851) (see Volume~\volnumbertc{}).  The on-site \dword{esh} coordinator facilitates training and runs weekly safety meetings. This  person is also responsible for managing \dword{esh}-related  documentation, including training records, \dword{ha} documents, weekly safety reports, records on materials-handling equipment, near-miss and accident reports, and equipment inspections. 

If the \dword{esh} is absent, the shift supervisor acts in this capacity.

All workers have work stop authority in support of a safe working environment. 

%%%%%%%%%%%%%%%%%%%%%%%%%%%%%%%%%
\subsection{Safety Program}

The on-site \dword{esh} coordinators will guide the \dword{fd} installation safety program, using the following:

\begin{enumerate}
\item	the \dword{feshm};
\item the \dword{dune} Installation \dword{esh} Plan, which includes the fire evacuation plan, fire safety plan, lockdown plans, and the site plan;
\item	work planning and controls documentation which includes both hazard analysis and procedures; 
\item	Safety Data Sheets (SDS); 
\item	the respiratory plan, as required for chemical or \dword{odh} hazards; and 
\item	the training program, which covers required certifications and  training records.
\end{enumerate}


During the installation setup phase, as new equipment is being installed and tested, new employees and collaborators will be trained to access the facility and use the equipment. At the end of this phase, we will require two shifts per day.


The \coldbox and cryogenics system will not be 
tested during the trial assembly work at \dword{ashriver}. 
While the new \coldbox design is very similar to \dword{pdsp}'s, it will be operated under \dword{doe} and \dword{feshm} regulations.  Procedures for operating the \coldbox will be written according to the established requirements.

The \dword{dune} installation team
 will develop an  \dword{esh} plan for detector  installation that defines  
the \dword{esh} requirements and responsibilities for personnel during  assembly, installation, and construction of equipment at \dword{surf}. It will cover at least the following areas:

{Work Planning and \dword{ha}:} The goal of the work planning and \dword{ha} process is to initiate thought about the hazards associated with work activities and plan how to perform the work. Work planning ensures the scope of the job is understood, appropriate materials and tools are available, all hazards are identified, mitigation efforts are established, and all affected employees understand what is expected of them. 
The work planning and \dword{ha} program is documented in Chapter 2060 in the \dword{feshm}.

The shift supervisor and the \dword{esh} coordinator  will lead a work planning meeting at the start of each shift  to (1) coordinate the work activities, (2) notify the workers of potential safety issues, constraints, and hazard mitigations, (3) ensure that employees have the necessary \dword{esh} training and \dword{ppe}, and (4) answer any questions.

{Access and training:}  All \dword{dune} workers requiring access to the \dword{surf} site must (1) register through the \dword{fnal} Users Office to receive the necessary user training and a \dword{fnal} identification number, and (2) they must apply for a \dword{surf} identification badge. 
The workers will be required to complete \dword{surf} surface and underground orientation classes. Workers accessing the underground must also complete 4850L and 4910L specific unescorted access training, and obtain a \dword{tap} for each trip to the underground area; this is required as part of \dword{surf}'s Site Access Control Program. 
A properly trained guide will be stationed on all working levels. 

{\dword{ppe}:} 
The host laboratory is responsible for supplying appropriate \dword{ppe} to all workers. 

{\dword{ert}:} The \dword{sdsta} will maintain an emergency response incident command system and an \dword{ert}.  The guides on each underground level will be trained as first responders to help in a medical emergency.
  
  Guides: The shift supervisor and lead workers will be trained as guides.
  
  {House cleaning:} All workers are responsible for keeping a clean organized work area. This is particularly important underground. Flammable items must be in proper storage cabinets, and items like empty shipping crates and boxes must be removed and 
transported back to the surface to make space.


{Equipment operation:} All overhead cranes, gantry cranes, fork lifts, motorized equipment, e.g., trains and carts, will be operated only by trained  operators. 
Other equipment, e.g., scissor lifts, pallet jacks, hand tools, and shop equipment, will be operated only by people trained
and certified for the particular piece of equipment. All installation equipment will be electrically powered.
