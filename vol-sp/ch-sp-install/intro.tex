%\chapter{Detector Installation}
%\label{ch:sp-tc}
% \fdth{}s
% \cite{bib:docdb8255}

This chapter covers all the work and infrastructure required to install the \dword{sp} detector module. 
 
We first provide a reminder of the scale of the task, beginning with the two facts that drive all others: A \dword{dune} \dword{fd} module is enormous, with outer cryostat dimensions  of (L W H) 
$62\times 19\times 18$ m$^{3}$; and every piece of a \dword{detmodule} must descend 
\SI{1500}{m} down the Ross shaft to the 4850-foot level of \dword{surf} and be transported to a detector cavern.


The \dword{spmod}'s 
150 \dwords{apa}, each $6.0$ m high and $2.3$ m wide, and  weighing $600$ kg with $3500$ strung sense and shielding wires, must be taken down the shaft as special ``slung loads'' and moved to the area just outside the \dword{dune} cryostat. 
The \dword{apa}s are moved into a pre-prepared \SI{30 x 19}{m} clean room (a portion of which is \SI{17}{m} high), where they are first be outfitted  with \dword{pd} units and passed through a series of qualification tests.
Here two \dword{apa}s are linked into a vertical \SI{12}{m} high double unit and connected to readout electronics. 
They receive a cold-test in place, then move into the cryostat to be connected at the proper location on the previously installed \dword{dss} and get cabled up with the \fdth{}s. 
In parallel, the \dwords{fc} that define the \dword{tpc} active volume must be installed with all their \dword{hv} connections, along with  elements of the \dword{cisc} and calibration systems.

After twelve months of detector component installation, which follows twelve months of detector infrastructure installation, the cryostat closes (with the last installation steps occurring in a confined space accessed through a narrow human access port). 
Following leak checks, final electrical connection tests, and installation of the neutron calibration source, the process of filling the cryostat with $17,000,000$ kg of \dword{lar} begins.

The installation requires meticulous planning and execution of thousands of tasks by well trained teams of technicians, riggers, and detector specialists. 
High-level requirements for these tasks are spelled out in Table~\ref{tab:specs:just:SP-TC} and the text that follows it. 
In all the planning and future work, the pre-eminent requirement in the installation process is safety.
\dword{dune}'s goal is zero accidents resulting in personal injury, damage to detector components, or harm to the environment.


% This file is generated, any edits may be lost.

\begin{longtable}{p{0.14\textwidth}p{0.13\textwidth}p{0.18\textwidth}p{0.22\textwidth}p{0.20\textwidth}}
\caption{Specifications for SP-TC \fixmehl{ref \texttt{tab:spec:SP-TC}}} \\
  \rowcolor{dunesky}
       Label & Description  & Specification \newline (Goal) & Rationale & Validation \\  \colhline

   
  \newtag{SP-TC-1}{ spec:logistics-material-handling }  & Compliance with the SURF Material Handling Specification for all material transported underground  &  SURF Material Handling Specification &  Loads must fit in the shaft be lifted safely. &  Visual and documentation check \\ \colhline
     % 1
   
  \newtag{SP-TC-2}{ spec:logistics-shipping-coord }  & Coordination of shipments with CMGC; DUNE to schedule use of Ross Shaft  &  2 wk notice to CMGC &  Both DUNE and CMGC need to use Ross Shaft &  Deliveries will be rejected \\ \colhline
     % 2
   
  \newtag{SP-TC-3}{ spec:logistics-materials-buffer }  & Maintain materials buffer at logistics facility in SD   &  $>1$ month &  Prevent schedule delays in case of shipping or customs delays &  Documentatation and progress reporting \\ \colhline
     % 3
   
  \newtag{SP-TC-4}{ spec:apa-storage-sd }  & APA stroage at logistics facility in SD  &  700 m$^2$ &  Store APAs during lag between production and installation &  Agree upon space needs \\ \colhline
     % 4
   
  \newtag{SP-TC-5}{ spec:cleanroom-specification }  & Installation cleanroom Specificaiton  &  ISO 8 &  Reduce dust (contains U/Th) to prevent induced radiological background in detector &  Monitor air purity \\ \colhline
     % 5
   
  \newtag{SP-TC-6}{ spec:cleanroom-uv-filters }  & UV filter in ITF and installation cleanrooms for PDS sensor protection  &  na &  Prevent damage to PD coatings  &  Visual or spectrographic inspection \\ \colhline
     % 6


\label{tab:specs:just:SP-TC}
\end{longtable}


\fixme{Jim needs to fix the specification for TC 5and 6 (Anne asks: still needed? 4/24}
\fixme{Might want to not use dwords in the spec table so that they get written out in the text instead - JPO, SDSD, surf, APA. Anne}

Installation of the \dword{spmod} presents a multitude of hazards that includes  manipulation of heavy loads in the tight spaces at the 4850 level and in the \dword{detmodule},  working at considerable heights above the floor, repeated utilization of large volumes of cryogens, multiple tests with \dword{hv}, commissioning of a class IV laser system, and deployment of a high-activity neutron source. Mitigation of these hazards begins with the strong professional on-site \dword{esh} teams of the \dword{sdsd} and \dword{surf}.
All installation team members, both at the surface and underground, will undergo rigorous formal safety training that will be updated at daily intervals. Any team member can stop work at any time for safety purposes. Further details of the overall \dword{dune} safety plan are provided in   
 Volume~\volnumbertc~\voltitletc, Section~\ref{sec:far_site_safety} of this \dword{tdr}.  In addition, each section of this chapter provides further details on the evolving safety plan for installation. This plan has been informed by the successful safety experience of \dword{surf} with other underground experiments (e.g., \dword{lux}, \dword{mjdemo}, \dword{lz}), \dword{dune} members in executing projects at other underground locations (e.g., \dword{minos} at Soudan, Minnesota, USA), at other locations remote from major international laboratories (e.g., \dword{dayabay}, China and \dword{nova} Far Detector (Ash River, Minnesota, USA), and at the home laboratories of both \dword{fnal} and \dword{cern}.


% risk table values for subsystem SP-FD-JPO
\begin{longtable}{p{0.15\textwidth}p{0.13\textwidth}p{0.13\textwidth}p{0.28\textwidth}p{0.06\textwidth}p{0.06\textwidth}p{0.06\textwidth}} 
\caption{Specification for SP-FD-JPO \fixmehl{ref \texttt{tab:specs:SP-FD-JPO}}} \\
\rowcolor{dunesky}
ID & Risk & Label & Mitigation & Prob ability & Cost Impact & Sched ule Impact \\  \colhline
RT-JPO-001 & Personnel injury & jpo-person-injury & Follow established safety plans. & M & L & H \\  \colhline
RT-JPO-002 & Shipping delays & jpo-shipping-delay & Plan one month buffer to store  materials locally. Provide logistics manual. & H & L & L \\  \colhline
RT-JPO-003 & Missing components cause delays & jpo-missing-components & Use detailed inventory system to verify availability of  necessary components.  & H & L & L \\  \colhline
RT-JPO-004 & Import, export, visa issues  & jpo-import-visa & Dedicated \dword{fnal} \dword{sdsd}division will expedite import/export and visa-related issues. & H & M & M \\  \colhline
RT-JPO-005 & Lack of available labor  & jpo-labor-avail & Hire early and use Ash River setup to train \dword{jpo} crew. & L & L & L \\  \colhline
RT-JPO-006 & Parts do not fit together & jpo-cannot-assemble & Generate \threed model, create interface drawings, and prototype detector assembly. & H & L & L \\  \colhline
RT-JPO-007 & Cryostat damage & jpo-cryostat-damage & Use cryostat false floor and temporary protection. & L & L & M \\  \colhline
RT-JPO-008 & Weather closes SURF & jpo-weather-delay & Plan for \dword{surf} weather closures & H & L & L \\  \colhline
RT-JPO-009 & Detector failure during \cooldown & jpo-cooldown-failure & Cold test individual components then cold test \dword{apa} assemblies immediately before installation. & L & H & H \\  \colhline

\label{tab:risks:SP-FD-JPO}
\end{longtable}
\fixme{Anne needs to fix the risk table caption}


As part of the \dword{dune} design process the detector components and the \dword{tpc} have been prototyped at various stages. \dword{pdsp}, which was assembled from full-scale %dune 
components has recently been completed and has taken data. 
This process has been extremely important in planning the \dword{spmod} %far detector 
installation. A detailed list of lessons learned from \dword{pdsp} construction and installation was compiled\cite{bib:docdb8255}. 
These lessons %learned 
and other experience from the team planning the installation was used to develop a list of \textquotedblleft risks \textquotedblright for the %\dword{dune} 
\dword{spmod} installation. 


A list of installation risks was generated based on the \dword{pdsp} experience and other experience from the installation team, and mitigation strategies have been formulated. 
The highest-impact risks -- those requiring a mitigation strategy -- are listed in Table~\ref{tab:risks:SP-FD-JPO}. 
These risks and all the lessons learned from \dword{pdsp} will be factored into the detailed installation plan.

This chapter is divided into three main sections. 
The first section describes how material will be delivered to the South Dakota region and forwarded to the Ross Headframe on the \dword{surf} site. 
The second section describes the infrastructure needed to install and operate the detector. This includes a cleanroom and its contents, and also electronics racks, cable trays, storage facilities and machining facilities. 
The third section describes the installation process itself, which is divided into three phases: the \dword{cuc} setup phase, the installation setup phase, and the detector installation phase. These are summarized in Section~\ref{sec:sp-iic-sched}.
