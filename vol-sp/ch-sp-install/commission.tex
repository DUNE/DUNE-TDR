\section{Detector Commissioning}
\label{sec:fdsp-tc-commiss}


After the \dword{spmod} is installed in the cryostat and the \dword{tco} is closed, much work remains before it can be operated. In parallel, to the final work inside the cryostat described above the \lar pumps are installed at the ends of the cryostat and final connections are made to the recirculation plant. 

Before the cooldown starts a series of tests are performed to verify that the detector is operating nominally. 

\begin{enumerate}

    \item A pedestal and \dword{rms} characterization of all \dword{ce} channels will verify that all \dword{apa} \dword{fe} boards are responding and no dead channel or new noise sources arose following the \dword{tco} closing.
    
    \item A noise scan of all \dword{pd} channels is performed as a last check. % before sealing. 

    \item Each \dword{apa} wireplane will be checked to verify it is isolated from the \dword{apa} frame and properly connected to its \dword{hv} power supply through the following steps:
    
\begin{itemize}

    \item The \dword{shv} connector of each wire plane bias channel will be unplugged at the power supply, and both the resistance and capacitance between inner conductor and ground will be measured. 
    The resistance should show that the wireplane is electrically isolated from the ground, while the capacitance value should match that of the cold \dword{hv} cable and the capacitance of the circuit on the \dword{apa} top frame.

    \item \SI{50}{V} is applied to each wireplane and the current drawn checked against the expected value.
    
    \item Nominal voltages are applied to each wireplane, and the  drawn current is checked against the expected value. 
    
\end{itemize}

    \item A low \dword{hv} (i.e., \SIrange{1}{2}{kV}) is applied to the cathode, and the drawn current is checked against the expected value to ensure the integrity of the \dword{hv} line.

\end{enumerate}

Next, everything is leak tested, and the cryogenics plant brought into operation. The system first purges the air inside the cryostat  by injecting pure \dword{gar} at the bottom  at a rate that fills the cryostat volume uniformly but faster than the diffusion rate. This ``piston purge'' process produces a column of \dword{gar}  that rises through the volume and pushes out the air through the \dword{gar} purge lines and the \dword{gar} venting lines.  When the piston purge is complete, misting nozzles inject a liquid-gas mix into the cryostat that cools the detector components at a controlled rate. 


Once the detector is cold, the filling process begins. \dword{lar} stored at the surface  
is vaporized, brought down the shaft in gaseous form, and re-condensed underground. The \lar then flows through filters to remove any H$_2$O and O$_2$ before entering the cryostat. Given the volume of the cryostat and the limited cooling power for recondensing, \num{12} months will be required to fill the first \dword{detmodule}. The detector readout electronics will be  monitoring the status of the detector. 



A number of the following tests (and likely others) will  take place during the \cooldown and filling phase: 

\begin{enumerate}


    \item Each \dword{apa} wireplane isolation and proper connection to its \dword{hv} power supply will be checked at regular time intervals as was done before sealing the cryostat.
    
    \item \SIrange{1}{2}{kV} will be held on the cathode, and the current drawn will be  monitored constantly to observe the trend in temperature of the total resistance.
    
    \item \dword{ce} noise figures (pedestal, \dword{rms}) will be measured at regular intervals and their trends with temperature recorded.
    
    \item \dword{pd} system noise (pedestal, \dword{rms}) will be measured at regular intervals and its trend with temperature recorded.
    
     \item Values of the temperature sensors deployed in several parts of the cryostat will be monitored constantly to watch the progress of the \cooldown phase and to relate the temperature to the behavior of the other \dword{spmod} subsystems. 
     
\end{enumerate}

Regular monitoring of \dword{ce} and \dword{pd} noise, as well as checks of wire plane isolation and proper connections to the bias supply system will continue throughout the filling, recording noise variations as a function of the progressively reduced temperature. In addition,

\begin{enumerate}

    \item As each purity monitor is submerged in liquid, it will be turned on every eight hours to check \dword{lar} purity. 
    
    \item As soon as top \dwords{gp} are submerged, \dword{hv} on the cathode will be raised up to \SI{10}{V} to check that the current drawn by the system agrees with expectations.

\end{enumerate}

Once the \dword{detmodule} is full, the drift \dword{hv} will be carefully ramped up following these steps:

\begin{enumerate}

    \item Evaluate need for a filter regeneration before starting any operation;

    \item Once filter regeneration is completed (if needed), examine the \dword{lar} surface 
    using cameras to verify that the surface is flat, with no bubbles or turbulence;
    
    \item Start \dword{lar} recirculation while monitoring the \dword{lar} surface 
    again to see if activating the recirculation system introduced any turbulence into the liquid;
    
    \item Wait one day after beginning  recirculation to stabilize the \dword{lar} flow inside the \dword{detmodule}, then start the \dword{hv} ramp up.
    
\end{enumerate}

Cathode voltage is raised in steps over three days. 
On the first day, cathode voltage is first raised to \SI{60}{kV}, then to \SI{90}{kV} after waiting two hours, and finally to \SI{120}{kV} after waiting another two hours, and then left at this value overnight.
On the second day, cathode voltage is first raised to \SI{140}{kV}, then to \SI{160}{kV} after waiting four hours, and then left at this value overnight. 
On the third day, cathode voltage is first raised to \SI{170}{kV} and then to the nominal operating voltage of \SI{180}{kV} after waiting four hours. 
During each \dword{hv} ramp up, all \dword{ce} current draws are monitored, and the procedure is stopped if any of the current draws go out of the allowed range. 
During each waiting period, regular \dword{daq} runs monitor \dword{ce} and \dword{pd} noise and response, while cathode \dword{hv} and current draw stability are constantly monitored.

In \dword{pdsp}, this process took three days, after which the system was  ready for data-taking. With a detector twenty times larger, the process will take longer, but the turn on time should still be relatively short. 










