%\section{Quality Control}
%\label{sec:sp-calib-qc}

%\fixme{This is a copy of text we sent to Jim Stewart for the integration chapter. This is basic at this time, but we expect to refine it as we develop the ProtoDUNE plans. The general theme should be clear-- test and retest to ensure hardware is sensible before placed in the detector. }

The manufacturer and the institutions in charge of devices will conduct a series of tests to ensure the equipment can perform its intended function as part of \dword{qc}. \dword{qc} also includes post-fabrication tests and tests run after shipping and installation. The overall strategy for the calibration devices is to test the systems for correct and safe operation in dedicated test stands, then at \dword{protodune}, then as appropriate near \dword{surf} at SDSMT, and finally underground. Electronics and racks associated to each full system will be tested before transporting them underground.

\begin{itemize}
    \item {\bf Laser System:} The first %critical 
    important test is  design validation in \dword{pdsp}. For assembly and operation of the laser and feedthrough interface, this will be carried out on a mock-up flange for each of the full hardware sets (periscope, feedthrough, laser, power supply, and electronics). All operational parts (UV laser, red alignment laser, trigger photodiode, attenuator, diaphragm, movement motors, and encoders) will be tested for functionality before being transported underground.
    \item {\bf Pulsed Neutron Source System:} The first test will be safe operation of the system in a member institution radiation-safe facility. Then, the system will be validated at \dword{protodune}. The same procedure %(operation at MSU) , then transportation to SURF and underground, 
    will be carried out for any subsequent devices before the devices are transported to \dword{surf} and underground. System operation will be tested with shielding assembled to confirm safe operating conditions and sufficient neutron yields using an external dosimeter as well as with the installed neutron monitor. The entire system, once assembled, can be brought down the Ross shaft.
    %\item {\bf Radioactive Source Deployment System:}
    %A mechanical test of the Double Chooz fish-line deployment system with an \dword{lar} mock-up column will be done in the high bay laboratory at South Dakota School of Mines and Technology. The ultimate test of the system will be done at \dword{protodune}. Safety checks will also be done for the source and for appropriate storage on the surface and underground. 
    %\item {\bf Power supply and racks:}  Electronics and racks associated to each full system will be tested before transporting them underground.
\end{itemize}
