

The energy and position reconstruction requirements for physics measurements lead to requirements on the necessary precision of the laser %calibration 
\efield measurement, its spatial coverage and granularity. The next sections discuss the rationale behind each requirement, which we take as the \dword{dune} specification.
%, with ALARA (or AHARA for the coverage) as goal.

\paragraph{\efield precision:}


In the \dword{lbl} and high-energy range, \physchlbl of this \dword{tdr}
%the \dword{dune} physics \dword{tdr} 
states that the calibration information must provide approximately \num{1} to \SI{2}{\%} understanding of normalization, energy scale and resolution, and position resolution within the detector.
Because a smaller \efield leads to higher electron-ion recombination and therefore a lower collected charge, distortions of the \efield can introduce
%are one of the possible causes of an 
energy scale bias. To connect this
%that requirement 
to a specification for the necessary precision of the \efield measurement, we note that, via recombination studies~\cite{bib:mooney2018}, we expect a \SI{1}{\%} distortion on \efield to lead to a \SI{0.3}{\%} bias on collected charge.
Because other effects will contribute to the lepton energy scale uncertainty budget, we consider a goal for the 
%calibration 
laser system to measure the \efield to a precision of $\sim$\SI{1}{\%} so that its effect on the collected charge is well below \SI{1}{\%}.
This is also motivated by consistency with the high level DUNE specification on field uniformity throughout the volume due to component alignment and \dword{hv} system, that 
%was 
is set at \SI{1}{\%}.
Together with two other high-level \dword{dune} specifications, the \dword{apa} wire spacing (\SI{4.7}{\mm}) and the front end peaking time (\SI{1}{\micro\s}), the effect of this \efield precision requirement on engineering parameters of the calibration laser system is discussed further %ahead, 
in Section~\ref{sec:sp-calib-sys-las-ion-meas}.

\paragraph{\efield measurement coverage:}

In practice, measuring the \efield  throughout the whole volume of the \dword{tpc} will be difficult, so we must establish a goal for the coverage and granularity of the measurement. 
Until a detailed study of the propagation of the coverage and granularity into a resolution metric is available, a rough estimate of the necessary coverage can be made as follows.

Assuming \SI{4}{\%} as the maximum \efield distortion %resulting 
that is %expectable 
expected from a compounding of multiple possible effects in the \dword{dune} \dword{fd} %as stated in the physics volume of the \dword{tdr},
as described in the previous section,
%(\cite{Abi:2018dnh}, page~4-53), 
we can then ask what would be the maximum acceptable size of the spatial region uncovered by the calibration system, if a distortion of that magnitude (systematically biased in the same direction) were present in that region. Our criterion of acceptability is to keep the overall \efield distortion, averaged over the whole detector, at the \SI{1}{\%} level. 
%, so then that 
To meet this requirement, the aforementioned spatial region should be no larger than \SI{25}{\%} of the total fiducial volume. Therefore, we aim to have a coverage of \SI{75}{\%} or more.

In addition, we need to consider that the method used to estimate \efield distortions is based on obtaining position displacement maps~\cite{bib:uBlaser2019}, and that the comparison between the reconstructed and true direction of a single track does not %univocally  %unequivocally
unambiguously determine a specific displacement map. Having tracks coming from different origins crossing in the same position is a direct way to eliminate that ambiguity, since the displacement vector is given simply by the vector connecting the intersections of the two reconstructed and the two true tracks. A joint iterative analysis of several close-by tracks is the default method for all other positions, but the system design should allow for the maximum possible number of positions %where there can be 
for crossing tracks from different beams.

\paragraph{\efield measurement granularity:}

The Volume~\volnumberphysics~(\voltitlephysics) of this \dword{tdr} states that a \dword{fv} uncertainty of \SI{1}{\%} is required. 
This translates to a position uncertainty of \SI{1.5}{\cm} in each coordinate (see \dword{tdr} \spchapa). 
In the $y$ and $z$ coordinates, position uncertainty is given mainly by the \dword{apa} wire pitch, and since this is about \SI{4.7}{\mm}, the requirement is met. In the drift ($x$) direction, the position is calculated from timing, and considering the electronics peaking time of \fepeaktime, the uncertainty should be even smaller.

The position uncertainty, however, also depends on the \efield, via the drift velocity. Because the position distortions accumulate over the drift path of the electron, it is not enough to specify an uncertainty on the field. We must accompany it by specifying the size of the spatial region of that distortion. For example, a \SI{10}{\%} distortion would not be relevant if it was confined to a \SI{2}{\cm} region and if the rest of the drift region was at nominal field.
%So 
Therefore, what matters is the product of [size of region] $\times$ [distortion]. Moreover, one can distinguish distortions into two types:
\begin{enumerate}
\item Those affecting the magnitude of the field. Then the effect on the drift velocity $v$ is also a change of magnitude. According to the function provided in \cite{Walkowiak:2000wf}, close to \SI{500}{\V\per\cm}, the variation of the velocity with the field is such that a \SI{4}{\%} variation in field $E$ leads to a \SI{1.5}{\%} variation in $v$.
\item Those affecting the direction of the field. Nominally, the field $E$ should be along $x$, so $E = E_L$ (the longitudinal component). If we consider that the distortions introduce a new transverse component $E_T$, in this case, this translates directly into the same effect in the drift velocity, which gains a $v_T$ component, $v_T=v_L  E_T/E_L $, i.e., a \SI{4}{\%} transverse distortion on the field leads to a \SI{4}{\%} transverse distortion on the drift velocity.
\end{enumerate}

Thus, a \SI{1.5}{\cm} shift comes about from a constant \SI{1.5}{\%} distortion in the velocity field over a region of \SI{1}{\m}. In terms of \efield, that could be from a \SI{1.5}{\%} distortion in $E_T$ over \SI{1}{\m} or a \SI{4}{\%} distortion in $E_L$ over the same distance.

%From ref.~\cite{Abi:2018dnh}, page~4-53, 
\efield distortions can be caused by space-charge effects due to accumulation of positive ions caused by \Ar39 decays (cosmic ray rate is low in \dword{fd}), or detector defects, such as \dword{cpa} misalignments (Figure~\ref{fig:efield_cpa_distortions_boyu2017}), \dword{fc} resistor failures (Figure~\ref{fig:efield_resistorfailure_mooney2019}), resistivity non-uniformities, etc.
%~\cite{Abi:2018dnh}. 
These effects added in quadrature can be as high as \SI{4}{\%}. 
%From ref. ~\cite{bib:mooney2018}, 
The space charge effects due to \Ar39~\cite{bib:mooney2018} can be approximately \SI{0.1}{\%} for the \dlong{sp} (\dshort{sp}), and \SI{1}{\%} for the \dshort{dp} (\dlong{dp}), so in practice these levels of 
%that kind of distortion 
distortions must cover several meters to be relevant.
Other effects due to \dword{cpa} or \dword{fc} imperfections can be higher because of space charge, but they are also much more localized. If we assume there are no foreseeable effects that would distort the field more than \SI{4}{\%}, and considering the worst case scenario (transverse distortions), then the smallest region that would produce a \SI{1.5}{\cm} shift is \SI{1.5}{\cm}/\num{0.04}~=~\SI{37.5}{\cm}. This provides a target for the granularity of the measurement of the \efield distortions in $x$ to be smaller than approximately \SI{30}{\cm}, with, of course, a larger region if the distortions are smaller. Given the above considerations, then a voxel size of \num{10}$\times$\num{10}$\times$\SI{10}{\cubic\cm} appears to be enough to measure the \efield with the granularity needed for a good position reconstruction precision. In fact, because the effects that can likely cause bigger \efield distortions are problems or alignments in the \dword{cpa} (or \dword{apa}) or in the \dword{fc}, it is conceivable to have different size voxels for different regions, saving the highest granularity of the probing for the walls/edges of the drift volume.

%% This file is generated, any edits may be lost.

\begin{longtable}{p{0.14\textwidth}p{0.13\textwidth}p{0.18\textwidth}p{0.22\textwidth}p{0.20\textwidth}}
\caption{Specifications for SP-CALIB \fixmehl{ref \texttt{tab:spec:SP-CALIB}}} \\
  \rowcolor{dunesky}
       Label & Description  & Specification \newline (Goal) & Rationale & Validation \\  \colhline

   \newtag{SP-FD-1}{ spec:min-drift-field }  & Minimum drift field  &  $>$\,\SI{250}{ V/cm} \newline ( $>\,\SI{500}{ V/cm}$ ) &  Lessens impacts of $e^-$-Ar recombination, $e^-$ lifetime, $e^-$ diffusion and space charge. &  ProtoDUNE \\ \colhline
    
   
  \newtag{SP-FD-2}{ spec:system-noise }  & System noise  &  $<\,\SI{1000}\,e^-$ &  Provides $>$5:1 S/N on induction planes for  pattern recognition and two-track separation. &  ProtoDUNE and simulation \\ \colhline
    
   
  \newtag{SP-FD-3}{ spec:light-yield }  & Light yield  &  $>\,\SI{20}{PE/MeV}$ (avg), $>\,\SI{0.5}{PE/MeV}$ (min) &  Gives PDS energy resolution comparable that of the TPC for 5-7 MeV SN $\nu$s, and allows tagging of $>\,\SI{99}{\%}$ of nucleon decay backgrounds with light at all points in detector. &  Supernova and nucleon decay events in the FD with full simulation and reconstruction. \\ \colhline
    
    \\ \rowcolor{dunesky} \newtag{SP-FD-4}{ spec:time-resolution-pds } & Name: Time resolution \\
    Description & The time resolution of the photon detection system shall be less than 1 microsecond in order to assign a unique event time.   \\  \colhline
    Specification (Goal) &  $<\,\SI{1}{\micro\second}$  ( $<\,\SI{100}{\nano\second}$ ) \\   \colhline
    Rationale &   Enables \SI{1}{mm} position resolution for \SI{10}{MeV} SNB candidate events for instantaneous rate $<\,\SI{1}{m^{-3}ms^{-1}}$.  \\ \colhline
    Validation &   \\
   \colhline

   \newtag{SP-FD-5}{ spec:lar-purity }  & Liquid argon purity  &  $<$\,\SI{100}{ppt} \newline ($<\,\SI{30}{ppt}$) &  Provides $>$5:1 S/N on induction planes for  pattern recognition and two-track separation. &  Purity monitors and cosmic ray tracks \\ \colhline
    
    \\ \rowcolor{dunesky} \newtag{SP-FD-7}{ spec:misalignment-field-uniformity } & Name: Drift field uniformity due to component alignment \\
    Description & Misalignments of the various TPC components shall not introduce drift-field nonuniformities beyond those specified in the HVS requirements.   \\  \colhline
    Specification &  $<\,1\,$\% throughout volume \\   \colhline
    Rationale &   Maintains APA, CPA,  FC orientation and shape.  \\ \colhline
    Validation & ProtoDUNE  \\
   \colhline

    
   
  \newtag{SP-FD-9}{ spec:apa-wire-spacing }  & APA wire spacing  &  \SI{4.669}{mm} for U,V; \SI{4.790}{mm} for X,G &  Enables 100\% efficient MIP detection, \SI{1.5}{cm} $yz$ vertex resolution. &  Simulation \\ \colhline
    
    \\ \rowcolor{dunesky} \newtag{SP-FD-11}{ spec:hvs-field-uniformity } & Name: Drift field uniformity due to HVS \\
    Description & Design of TPC cathode and FC components shall ensure uniform field.  Production tolerances shall be set so as to maintain flatness of component surfaces and, by extension, the shape of the drift field volume.   \\  \colhline
    Specification &  $<\,\SI{1}{\%}$ throughout volume \\   \colhline
    Rationale &   High reconstruction efficiency.  \\ \colhline
    Validation & ProtoDUNE and simulation  \\
   \colhline

    
   \newtag{SP-FD-13}{ spec:fe-peak-time }  & Front-end peaking time  &  \SI{1}{\micro\second} \newline ( Adjustable so as to see saturation in less than \SI{10}{\%} of beam-produced events ) &  Vertex resolution; optimized for \SI{5}{mm} wire spacing. &  ProtoDUNE and simulation \\ \colhline
    
   
  \newtag{SP-FD-16}{ spec:det-dead-time }  & Detector dead time  &  $<\,\SI{0.5}{\%}$ &  Meet physics goals in timely fashion. &  ProtoDUNE \\ \colhline
    
   
  \newtag{SP-FD-22}{ spec:data-rate-to-tape }  & Data rate to tape  &  $<\,\SI{30}{PB/year}$ &  Cost.  Bandwidth. &  ProtoDUNE \\ \colhline
    
   
  \newtag{SP-FD-23}{ spec:sn-trigger }  & Supernova trigger  &  $>\,\SI{90}{\%}$ efficiency for SNB within \SI{100}{kpc} &  $>\,$90\% efficiency for SNB within 100 kpc &  Simulation and bench tests \\ \colhline
    
    \\ \rowcolor{dunesky} \newtag{SP-FD-24}{ spec:local-e-fields } & Name: Local electric fields \\
    Description & The integrated detector design shall minimize potential pathways for HV discharges.   \\  \colhline
    Specification &  $<\,\SI{30}{kV/cm}$ \\   \colhline
    Rationale &   Maximize live time; maintain high S/N.  \\ \colhline
    Validation & ProtoDUNE  \\
   \colhline

   
  \newtag{SP-FD-25}{ spec:non-fe-noise }  & Non-FE noise contributions  &  $<<\,\SI{1000}{enc} $ &  High S/N for high reconstruction efficiency. &  Engineering calculation and ProtoDUNE \\ \colhline
    
    \\ \rowcolor{dunesky} \newtag{SP-FD-26}{ spec:lar-impurity-contrib } & Name: LAr impurity contributions from components \\
    Description & Contributions to LAr contamination from detector components, through outgassing or other processes, shall remain << 30 ppt so as to avoid significantly increasing the nominal level of contamination.   \\  \colhline
    Specification &  $<<\,\SI{30}{ppt} $ \\   \colhline
    Rationale &   Maintain HV operating range for high live time fraction.  \\ \colhline
    Validation & ProtoDUNE  \\
   \colhline

    \\ \rowcolor{dunesky} \newtag{SP-FD-27}{ spec:radiopurity } & Name: Introduced radioactivity \\
    Description & Introduced radioactivity shall be less than that from 39Ar.   \\  \colhline
    Specification &  less than that from $^{39}$Ar \\   \colhline
    Rationale &   Maintain low radiological backgrounds for SNB searches.  \\ \colhline
    Validation & ProtoDUNE and assays during construction  \\
   \colhline


   \newtag{SP-CALIB-1}{ spec:efield-calib-precision }  & Ionization laser electric field measurement precision  &  \SI{1}{\%} \newline ( $<$\SI{1}{\%} ) &  Electric field affects energy and position measurements. &  ProtoDUNE and external experiments. \\ \colhline
    
   \newtag{SP-CALIB-2}{ spec:efield-calib-coverage }  & Ionization laser \efield measurement coverage  &  $>\,\SI{75}{\%}$ \newline ( \SI{100}{\%} ) &  Allowable size of the uncovered detector regions is set by the highest reasonably expected field distortions, 4%. &  ProtoDUNE \\ \colhline
    
   \newtag{SP-CALIB-3}{ spec:efield-calib-granularity }  & Ionization laser \efield measurement  granularity  &  $<\,\SI{30\times30\times30}{\centi\meter}$ \newline ( $<\,\SI{10\times10\times10}{\centi\meter}$ ) &  Minimum measurable region is set by the maximum expected distortion and position reconstruction requirements. &  ProtoDUNE \\ \colhline
    
   \newtag{SP-CALIB-4}{ spec:laser-position-precision }  & Laser beam position precision  &  $~\SI{0.5}{\milli\radian}$ \newline ($<\SI{0.5}{\milli\radian}$) &  The necessary spatial precision does not need to be smaller than the APA wire gap. &  ProtoDUNE \\ \colhline
    
   \newtag{SP-CALIB-5}{ spec:neutron-source-coverage }  & Neutron source coverage  &  $>$\SI{75}{\%} \newline ( \SI{100}{\%} ) &  The coverage of the pulsed neutron system depends on the energy resolution requirements at low energy. &  Simulations \\ \colhline
    
   \newtag{SP-CALIB-6}{ spec:data-volume-laser }  & Ionization laser data volume per year (per 10 kt)  &  $>\SI{184}{TB/yr/10 kt}$ \newline ($>\SI{368}{TB/yr/10 kt}$) &  The laser data volume must allow the needed coverage and granularity. &  ProtoDUNE and simulations \\ \colhline
    
   \newtag{SP-CALIB-7}{ spec:data-volume-pns }  & Neutron source DAQ rate per year (per 10 kton)  &  $>$\SI{84}{TB/yr/10 kton} \newline ( $>$\SI{168}{TB/yr/10 kton} ) &  The coverage of the pulsed neutron system depends on the energy resolution requirements at low energy. &  Simulations \\ \colhline
    
   
  \newtag{SP-CALIB-8}{ spec:rate-gammas-source }  & Rate of 9 MeV capture gamma events in the proposed radioactive source  &  $<\,\SI{1}{\kilo\hertz}$ &  The source rate must be such that there is no more than one event per drift time. &  Lab tests \\ \colhline
    


\label{tab:specs:SP-CALIB}
\end{longtable} ALREADY PULLED IN IN THE OVERVIEW SECTION (ANNE 4/18)

\begin{comment}
\begin{dunetable}
[Calibration Requirements]
{p{0.5\textwidth}p{0.15\textwidth}p{0.15\textwidth}}
{tab:calibreq}
{Calibration Specifications and Goals}   
Requirement & Specification & Goal \\ \toprowrule
\efield measurement precision & < 1\% & ALARA \\ \colhline
\efield measurement coverage & > 75\% & AHARA \\ \colhline
\efield measurement granularity & < 30x30x30 cm & ALARA \\ 
\end{dunetable}
\end{comment}

