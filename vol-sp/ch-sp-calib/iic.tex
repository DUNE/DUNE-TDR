%\section{Installation, Integration and Commissioning}
%\label{sec:sp-calib-iic}

%\fixme{This is a copy of text we sent to Jim Stewart for the integration chapter. We need guidance for how this chapter and that chapter need to reference each other.}
% Text from here: https://docs.google.com/document/d/1-rqOJOFtTlNPj1myXHtz3eNL75u1iZE5yfizwMTFsu4/edit

%\fixme{SG: I am done reviewing and did restructing; KM: Only one comment below to remove a sentence. JM: check if you are okay, more details sent in email.}

\subsection{Ionization Laser System} 
Only the laser system (alternative design) has components that need to installed inside the cryostat via the \dword{tco}. The pulsed neutron source and radioactive source deployment systems are installed only using the cryostat roof ports.

\textit{Inside \dword{tco}:} A long horizontal track system is to be installed outside the end-wall field cage, directly below the corresponding calibration ports, and suspended by them. The system farthest away from the \dword{tco} must be installed before TPC (FC/APA/CPA) installation begins. This installation requires the simultaneous installation of the corresponding periscopes, from the calibration ports, so that the two systems can be properly connected. The relevant QC is essentially alignment test.

In addition, the alternative laser positioning system has sets of photo-diodes that can either be pre-mounted on the HV system bottom ground plane, or simply on a tray close to the cryostat membrane. The only step that needs to be done inside the \dword{tco} is connecting the cabling to available flange (work is still underway to understand how to route cables and which flanges to use).

\textit{Outside \dword{tco}:}  Alignment of the visible and UV (Class 4) lasers, that requires special safety precautions, has to be carried out once for each periscope/laser system, prior to the installation of further TPC components. For that reason, the laser boxes need to be installed on their location on the cryostat roof as soon at that area becomes accessible. 

The periscopes on the top of the TPC towards the center of the cryostat can be installed after the relevant structural elements (e.g. field cage), these proceed in sequence with the assembly of other components (furthest from \dword{tco} is assembled first) and alignments can be done as elements are installed with the alignment laser system. 

%\fixme{KM: not confident in this sentence} It may be possible to do this special alignment operation for all lasers at roughly the same time, to minimize any possible disruptions. 

A support beam structure closest to the \dword{tco} temporarily blocks the calibration ports, this is removed after the last TPC component. After that, the final calibration components can be installed, including the periscopes on the \dword{tco} end wall and the horizontal track closest to the \dword{tco} would be the last items to be installed. 

\subsection{Laser Positioning System}
The laser positioning system has to be integrated with the HV system during installation underground. Two components (baseline design: mirror clusters, and alternative design: diodes) would require interface with the HV and field cage structural systems, discussed below.

The baseline consists of a set of about 40 mirror clusters - a plastic piece holding 4 to 6 small mirrors (5 mm diameter), each at a different angle - to which the ionization laser will point in order to obtain an absolute pointing reference. These clusters will be attached to the bottom field cage profiles facing into the TPC. 
%These cross bars must contain small alignment slots, matching the cluster pieces, in order for us to know the exact position of each cluster. 
This attachment/assembly of the mirror clusters on corresponding the FC profiles will be done during FC assembly underground.

An alternative design, that can be done in addition to the mirror clusters, which, following on the mini-CAPTAIN experience, is based on a set of diodes that fire when the laser beam hits them. Since the laser shoots from above, and the diodes need to be in a low voltage region, the plan is to attach them below the bottom FC, facing upwards. They can either be pre-mounted on the HV system bottom ground plane, or simply on a tray close to the cryostat membrane
For the pointing measurement, the beams will pass through the FC electrodes and hit the diodes below. At least 20 of these diode clusters would be installed, and this assembly needs to be done underground at the FC/HV assembly point (in case the ground planes are used).

\subsection{Photoelectron Laser System} 
A large number of photoelctric targets (about 4000) need to be fixed to the cathode. Experience from other experiments indicates that targets can be glued to the cathode surface. This process should take place after cathode assembly, but prior to installation in the cryostat. 
The second important aspect is a high precision survey of the photoelectric target locations, once the cathode plane assemblies are in place, which is necessary for the absolute calibration of the electric field with photoelectron laser. 
The third aspect is installation of quartz optical fibers on the anode plane, needed for illumination of the photoelectric targets with NdYag laser light. Fiber tips must be properly fastened and oriented for effective illumination and fiber bundle routing will bring bundles to the outside of the cryostat where NdYag laser injection points will be located. 
%\todo{SG: need text for this.}

\subsection{Pulsed Neutron Source System} 
The pulsed neutron source will be installed after the human access ports are closed as it sits above the cryostat. The installation of the system is expected to take place in two stages. At the first stage, the assembly of the system would be independent of the liquid argon TPC installation. The whole system will be installed on the ground outside the cryostat at a dedicated radiation safe facility. Once assembled, the neutron source will be lifted by a crane and integrated to the cryostat structure. Final QC testing for the system will be operating the source and measuring the flux with integrated monitor and dosimeter.

\subsection{Radioactive Source System}
The Radioactive source guide system is installed as the first element before TPC elements for the end wall furthest from the \dword{tco}, and as the last system (concurrent and coordinated with the alternative laser system) once the TPC is installed before \dword{tco} closing. The Radioactive source deployment system is installed at the top of the cryostat and can be installed when DUNE becomes operational.% Space near the radioactive source ports is important to operate the system. 

\subsection{Other installation requirements} 
For all calibration subsystems, space on mezzanine surrounding each calibration port is important in order to power and operate the calibration systems. They can be installed following the associated in-cryostat installation of the components (e.g. in the case of laser, after the periscopes are installed).

\begin{comment}
\subsection{Installation}
Only the laser system alternative design has components that need to installed inside the cryostat via the TCO. The pulsed neutron source and radioactive source deployment systems are installed only using the cryostat roof ports.

Laser, inside TCO: A long horizontal track system is to be installed outside the end-wall field cage, directly below the corresponding calibration ports, and suspended by them. The system farthest away from the TCO must be installed before TPC (FC/APA/CPA) installation begins. This installation requires the simultaneous installation of the corresponding periscopes, from the calibration ports, so that the two systems can be properly connected. The relevant QC is essentially alignment test.

In addition, the alternative laser positioning system has sets of photo-diodes pre-mounted on the HV system bottom ground planes. The only step that needs to be done inside the TCO is connecting the cabling to available flange (still working out how to route cables and which flange to use).



Laser, outside TCO: The periscopes on the top of the TPC in the center can be installed after the relevant structural elements (e.g. field cage), these proceed in sequence with the assembly of other components (furthest from TCO is assembled first) and alignments can be done as elements are installed with the alignment laser system. Once for each periscope/laser system, prior to the installation of further TPC components, we will need to clear the cavern to align the UV (Class 4) and visible lasers this will need special safety precautions. It may be possible to do this special alignment operation for all lasers at roughly the same time, to minimize the disruption.  

A support beam structure closest to the TCO temporarily blocks the calibration ports, this is removed after the last TPC component. After that, the final calibration components can be installed, including the the periscopes on the TCO endwall and the horizontal track closest to the TCO would be the last items to be installed. 



Power supply and racks:  Space on mezzanine close to each calibration port is important in order to power and operate the calibration systems (laser and PNS). They can be installed following the associated periscope installation.

Radioactive Source Deployment System: The RSDS guide system can be installed as the first element before TPC elements for the endwall furthest from the TCO, and the last system (concurrent and coordinated with the alternative laser system).
The RSDS is installed at the top of the cryostat and can be installed when DUNE is working.
\end{comment}

%\subsection{Commissioning}

%Once the detector is completely full, the laser system and pulsed neutron source can be tested.
