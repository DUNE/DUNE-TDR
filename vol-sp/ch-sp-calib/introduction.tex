

%%%%%%%%%%%%%%%%%%%%%%%%%%%%
%\fixme{SG:A paragraph length preamble 0 goes before Introduction as suggested by Tim; SG to edit in}

A detailed understanding of the overall detector response is essential for achieving \dword{dune} physics goals. The precision with which each calibration parameter must be measured is spanned by the requirements on the %JM
systematic uncertainties for the \dword{lbl} %, low-energy (\dword{snb}), and other 
and \dword{snb} physics programs at \dword{dune}. The calibration program must generally provide measurements at the few-percent-or-better 
%few-percent %JM; SG: this was a language edit from editors
level stably across an enormous volume and over a long period and provide sufficient redundancy. A detailed description of the calibration strategy for the \dword{dune} \dword{fd} is provided in \physchtools of this \dword{tdr}; here, we provide a brief summary.

The %current 
calibration strategy for the \dword{dune} \dword{fd} uses existing sources of particles, external measurements, and dedicated external calibration hardware systems. Existing calibration sources for \dword{dune} include beam or atmospheric neutrino-induced samples, cosmic rays, argon isotopes, and instrumentation devices such as \dword{lar} purity and temperature monitors. Dedicated calibration hardware systems consist of laser  and neutron source deployment systems.  External measurements by \dword{protodune2} and \dword{sbn} experiments  will validate techniques, tools, and the design of systems applicable to the \dword{dune} calibration program. These sources and systems provide measurements of the detector response model parameters, or provide tests of the response model itself. Calibration measurements can also provide corrections to data, data-driven efficiencies, systematics, and particle responses.

%Chapter~4 of the Physics volume of the \dword{tdr} provides a more detailed description of the calibration strategy for the \dword{dune} \dword{fd}. 
%using existing sources of particles (e.g., cosmic ray muons), external measurements (e.g., \dword{protodune}), monitors (e.g., purity monitors), and dedicated calibration hardware systems. 

This chapter focuses on describing the dedicated calibration hardware systems to be deployed for the \dword{dune} \dword{spmod} that provide necessary information beyond the reach of external measurements and existing sources and monitors. These include an ionization laser system, a \phel laser system, and a \dlong{pns} system. The possibility of deploying a radioactive source system is also currently being explored. The responsibility of the calibration hardware systems falls under the joint \dword{sp} and \dword{dp} calibration consortium, which was formed in November 2018.


Section~\ref{sec:sp-calib-overview} discusses general aspects driving the calibration program: scope, requirements and data taking strategy.
The baseline calibration hardware designs are described in Section~\ref{sec:sp-calib-systems} and respective subsections. 

Section~\ref{sec:sp-calib-sys-las-ion} describes the baseline design for the ionization laser system that provides an independent, fine-grained measurement of the electric field throughout the detector, which is an essential parameter that affects the spatial and energy resolution of physics signals. 
Volume~\volnumberphysics~(\voltitlephysics) of this \dword{tdr}
%The DUNE \dword{cdr}~\cite{Acciarri:2015uup}
assumes that the \dword{fv} is known to the \SI{1}{\%} level. Through measurements of the spatial distortions and drift velocity map, the laser calibration system mainly helps define the detector \dword{fv}, thus allowing for the correct prediction of the \dword{fd} spectra. The laser system also offers many secondary uses such as alignment checks, stability monitoring, and diagnosing detector performance issues. 
Possible electron lifetime measurements are under study. 
With the goal of knowing precisely the direction of the laser beam tracks, an independent 
\dword{lbls}
%positioning system 
is also planned, and is described in Section~\ref{sec:sp-calib-sys-las-loc}.
Alternative designs for the ionization laser system that may improve the physics capability and/or reduce overall cost are also under development and are described in Appendix~\ref{sec:sp-calib-laser-alter}. %{sec:lasertopfcpen}
Section~\ref{sec:sp-calib-sys-las-pe} describes the \phel laser system that can be used to rapidly diagnose electronics or \dword{tpc} response issues along with many other useful measurements such as integrated field across drift, drift velocity, and electronics gain. 

Section~\ref{sec:sp-calib-sys-pns} describes the baseline design for the \dword{pns} system, which provides a triggered, well defined, energy deposition from neutron capture in Ar detectable throughout the detector volume. Neutron capture is an important component of signal processes for \dword{snb} and \dword{lbl} physics, enabling direct testing of the detector response spatially and temporally for the low-energy program and the efficiency of the detector in reconstructing the low-energy spectra. A spatially fine-grained measurement of electron lifetime is also planned with this source.
The proposed \dword{rsds} described in the Appendix~\ref{sec:sp-calib-sys-rsds}, 
%radioactive source system
%which 
is in many ways complementary to the \dword{pns}
%pulsed neutron source 
system, and can provide at known locations inside the detector a source of gamma rays in the same energy range of \dword{snb} and solar neutrino physics. %But the 
The \dword{rsds} is the only calibration system that could probe the detection capability for single isolated solar neutrino events and study how well radiological backgrounds can be suppressed. In contrast, the \dword{pns} is externally triggered and does not provide such a well defined source location for gamma rays inside the detector. On the other hand, the \dword{pns} can probe the uniformity of the full detector, while the \dword{rsds} could only scan the ends of the detector. %A possible  complementary \dword{rsds} system is described in the Appendix~\ref{sec:sp-calib-sys-rsds}. 


For all the calibration hardware systems, the goal is to deploy prototype designs and validate them at \dword{protodune2} during the post long shutdown 2 (LS2) running  at \dword{cern}. The validation plan for calibration systems at \dword{protodune2} and other experiments is described in Section~\ref{sec:sp-calib-val}. 

Section~\ref{sec:sp-calib-intfc} describes interfaces calibration has with other \dword{dune} consortia, especially  with \dword{daq} which are described in more detail in Section~\ref{sec:sp-calib-daqreq}. 

Sections~\ref{sec:sp-calib-const} and \ref{sec:sp-calib-org-manag} conclude the chapter with descriptions of the aspects related to construction and installation of the systems, as well as organizational aspects, including schedule and milestones, discussed in Section~\ref{sec:sp-calib-sched}.

