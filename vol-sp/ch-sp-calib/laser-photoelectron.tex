\subsection{Photoelectron Laser System}
\label{sec:sp-calib-sys-las-pe}
%\subsubsection{Physics Motivation}
%\label{sec:sp-calib-sys-las-pe-phys} 
Well localized electron sources represent excellent calibration tool for study of the electron transport in the \dword{lartpc}, identification of the inhomogeneities in the TPC \efield in all directions, and precise determination of the electron drift velocity. Verification and calibration of the \efield distortions play an important role in particle vertex reconstruction and identification and affects the associated systematic errors, leading to increased rate of mis-identification and poorer energy reconstruction. A photoelectron laser can provide well localized electron sources on the cathode at predetermined locations leading to improved characterization of the \efield, and consequent reduction of detector instrumentation systematic errors. The photoeletron laser system, being a simpler system operationally compared to the ionization laser system, can be used as a wake-up system to quickly diagnose if the detector is alive. This is especially important due to the low cosmic ray environment in the detector underground. 

\subsubsection{Design}
\label{sec:sp-calib-sys-las-pe-des}

In order to produce localized clouds of electrons using a photoelectric effect, small aluminum discs or thin discs with evaporated gold film, will be used as targets. As stated in reference~\cite{Li:2016ods}, gold film can be just \SI{22}{\nano\m} thick. Several photoelectric strips will complement the circular targets to calibrate the rate of transverse diffusion in \dword{lar}. Based on the experience from T2K and BNL \dword{lar} test-stands~\cite{Li:2016ods}, \SIrange{8}{10}{\milli\m} diameter targets are sufficient. Targets will be placed on the cathode and distance between the dots will be determined based on the calibration needs and simulation outcomes. Nominally, dot spacing should be \SI{1}{\m} with photocathode strips every \SI{5}{\m}. The photoelectric dots and strips layout will be further refined based on the calibration requirements and performance simulation results. It will be essential to conduct a survey of the photocathode disc locations on the cathode  after installation and prior to detector closing. In this way, the absolute spatial calibration of the electric field can be achieved. 
At \SI{266}{\nano\m} Nd:Yag quadrupled wavelength, the single photon energy of \SI{4.66}{\eV} is sufficient to generate photoelectrons from aluminum, but not from gold, that requires two-photon absorption and therefore higher intensities.
%At S\I{266}{\nano\m} Nd:Yag quadrupled wavelength, the photon energy of\SI{4.66}{\eV} is sufficient to generate photoelectrons from both aluminum and gold. 
While aluminum has a lower associated cost, gold film surface is easier to protect from contamination. 
%A couple of hundred electrons 
%A couple of thousand electrons 
A few thousand electrons are expected per spill from each dot. The laser beam will be coming from the anode injection points, used as sources, guided to injection points via cryogenic optical fibers with defocusing elements on the other end. 

If aluminum is chosen, then single photon absorption is enough, and a lower laser intensity is required, opening up the 
%Much lower energy required for photoelectric laser, opens up the
possibility for a rather efficient calibration of each drift volume. Namely, laser pulse can be distributed to two drift volumes at the time in order, while illuminating the entire cathode assembly. Since the photoelectron clouds from different dots are very well localized, calibration of the \efield distortion in the entire drift volume can be done with a single laser trigger, if the light is distributed to all injection fibers for one drift volume. 

The photoelectron system will use the same lasers used for argon ionization. Stability of the laser pulses will be monitored  with  power meter. Dielectric mirrors will guide the laser light to injection points, but fraction of the light will be transmitted instead of reflected to the power meter behind the mirror. 

Laser will also send forced trigger signal to the DAQ based on the photodiode that will be triggered on the fraction of the light passing through the dielectric mirror. Special mirrors reflective to \SI{266}{\nano\m} light will be utilized. 

The photoelectron system will require the following tasks to complete the design: 
\begin{itemize}
    \item test the mounting of the targets on the cathode plane assembly; 
    \item survey of the dots position to the required level of precision; and  
    \item study thickness of the target and photoelectron yield as a function of target choice, laser power and attenuation of the laser light in the optical fibers.
\end{itemize}
%The first thing that needs to be tested is the mounting of the targets on the cathode plane assembly. In addition, survey of the dots position to the required level of precision is needed. Thickness of the target and photoelectron yield as a function of target choice, laser power and attenuation of the laser light in the optical fibers.

\subsubsection{Measurement Program}
\label{sec:sp-calib-sys-las-pe-meas}

Photoelectron systems have been used in other experiments to diagnose electronics issues by using the known time period between triggered laser signal and read out times, and to perform rapid checks of the readout of the TPC itself. The reciprocal of the \efield (integrated along the drift direction) is also measured.
