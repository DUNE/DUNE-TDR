
Interfaces between calibration and other consortia have been identified and appropriate documents have been developed. %We plan to have first full drafts of the interface documents by June 2019. 
The documents are currently maintained in the \dword{dune} document database (DocDB). %\fixme{This has not been defined in either the common glossary or in the glossary for this document.}
%\todo{documents being prepared now}
A brief summary is provided in this section. Table~\ref{tab:fdgen-calib-interfaces} lists the interfaces and corresponding DocDB document numbers. 
The main systems calibration has interfaces with are \dword{hv}, \dword{pds}, and \dword{daq}, and the important issues that must be considered are listed below.

\begin{description}
    \item[HV] Evaluate the effect of the calibration hardware, especially the laser system periscopes, on the \efield. This needs to be done for the alternative design described in the appendix, in which the \dword{fc} is penetrated, but should also be done in the baseline design, for which there is no penetration of the \dword{fc}. 
    %\fixme{This is not clear. FC penetrations for laser? is that a second evaluation that must be made with the effect of the incident laser beam being a third evaluation? Basically, the previous phrase must be a complete sentence with full content.} 
    Evaluate the effect of the incident laser beam on the \dword{cpa} material (Kapton); Integrate the hardware of the %alternative 
    photoelectron laser system (targets) and the \dword{lbls} (diodes) within the \dword{hv} system components. Ensure that the radioactive source deployment is in a safe field region and cannot do mechanical harm to the \dword{fc}.
    \item[PDS] Evaluate long term effects of laser light, even if just diffuse or reflected, on the scintillating components (\dword{tpb} plates) of the \dword{pds}; establish a laser run plan to avoid direct hits; evaluate the effect of laser light on alternative \dword{pds} ideas, such as having reflectors on the \dwords{cpa}; validate light response model and triggering for low energy signals. 
    %\fixme{Note how PDS is phrased. The other two items should be phrased similarly. Begin with a verb (evaluate, establish, evaluate, validate). Then follow each item in the list with a semicolon as in this section.}
    \item[DAQ] Evaluate \dword{daq} constraints on the total volume of calibration data that can be acquired; develop strategies to maximize the efficiency of data taking with data reduction methods; study how to implement a way for the calibration systems to receive trigger signals from \dword{daq} to maximize supernova live time. More details on this are presented in Section~\ref{sec:sp-calib-daqreq}.
    %\item[Computing] Evaluate any additional semi-offline processing needs. Coordinate needs for calibration databases. 
\end{description}

Integrating and installing calibration devices will interfere with other devices, requiring coordination with the appropriate consortia as needed. Similarly, calibration will have significant interfaces at several levels with cryostat and facilities in coordinating resources for assembly, integration, installation, and commissioning (e.g., networking, cabling, safety). Rack space distribution and interaction between calibration and systems from other consortia will be managed by \dword{tc} in consultation with those consortia.

\begin{dunetable}
[Calibration system interface links]
{p{0.14\textwidth}p{0.40\textwidth}p{0.14\textwidth}}
{tab:fdgen-calib-interfaces}
{Calibration Consortium Interface Links.}   
\small
Interfacing System & Description & Reference \\ \toprowrule
\dword{hv}	&
effect of calibration hardware (laser and radioactive source) on \efield and field cage; laser light effect on \dword{cpa} materials, field cage penetrations; attachment of positioning targets to HV supports 
& \citedocdb{14005} 
\\ \colhline
\dword{pds}	& 
effect of laser light on \dword{pds}, reflectors on the \dword{cpa}s (if any); validation of light response and triggering for low energy signals 
& \citedocdb{14008}
\\ \colhline
\dword{daq}	& 
DAQ constraint on total volume of the calibration data; receiving triggers from DAQ
& \citedocdb{14011}  
\\ \colhline
\dword{cisc} &
multi-functional \dword{cisc}/calibration ports; space sharing around ports; fluid flow validation; slow controls and monitoring for calibration quantities 
& \citedocdb{7072} 
\\ \colhline
TPC Electronics	         &  
Noise, electronics calibration
& \citedocdb{14014}  
\\ \colhline
\dword{apa}	&
\dword{apa} alignment studies using laser and impact on calibrations
& \citedocdb{14002} 
\\ \colhline
Physics	&
tools to study impact of calibrations on physics
& \citedocdb{14017}  
\\ \colhline
Software \& Computing	  &
Calibration database design and maintenance
& \citedocdb{14020} 
\\ \colhline
TC Facility              &   
Significant interfaces at multiple levels   
& \citedocdb{14023}   \\ \colhline
TC Installation     	  &     
Significant interfaces at multiple levels
& \citedocdb{14026}    \\ 
%TC Integration Facility    &    
%Significant interfaces at multiple levels
%& ---   \\ 
\end{dunetable}

%\fixme{sample from HV - use as template; see Sec 3.5 of guidance doc}

%\begin{dunetable}
%[High Voltage System Interface Links]
%{p{0.4\textwidth}p{0.2\textwidth}}
%{tab:HVinterfaces}
%{High Voltage System Interface Links }   
%Interfacing System & Linked Reference \\ \toprowrule
%CISC & \href{http://docs.dunescience.org/cgi-bin/ShowDocument?docid=6787&version=1}{DocDB 6787} \cite{bib:docdb6787} \\ \colhline
%DP CRP & \cite{bib:docdb6754} \\ \colhline
%... & \cite{} \\ \colhline
%(last item)& \cite{} \\
%\end{dunetable}