%\section{Safety}
%\label{sec:sp-calib-safe}

%We consider here only personal risk to humans, that apply in the prototyping phase, including ProtoDUNE deployment, and also during integration and commissioning at the DUNE far detector site. Risks of damaging the systems and/or other DUNE detector components are discussed in Section~\ref{sec:sp-calib-risks}.

This section discusses risks to personnel safety. Detector safety and risks involving damage to detector components are discussed in Section~\ref{sec:sp-calib-risks}.

Human safety is of critical importance during all phases of the calibration work, including R\&D, laboratory testing, prototyping (including \dword{pdsp} deployment), and integration and commissioning at the \dword{dune} \dword{fd} site. \dword{dune} \dword{esh} personnel review and approve the work planning for all phases of work as part of the initial design review, as well as before implementation. All documentation of component cleaning, assembly, testing, installation, and operation will include hazard analysis and work planning documentation and will be reviewed appropriately before production begins. In addition, in the case of planned \dword{protodune2} tests, the consortium will interface with \dword{cern} safety system to ensure all requirements are met.


Several areas are of particular importance to calibration are
\begin{itemize}
\item {\bf Underground laboratory safety:} All personnel working underground or in other installation facilities must follow appropriate safety training and be provided with the required \dword{ppe}. Risks associated with installing and operating the calibration devices include, among others, working at heights, confined space access, falling objects during overhead operations, and electrical hazards. Appropriate safety procedures will include aerial lift and fall protection training for working at heights. For falling objects, the corresponding safety procedures, including hard hats (brim facing down) and a well restricted safety area, will be part of the safety plan. More details on \dword{ppe} are provided in \dword{tdr} \tcchesh.
%Chapter~10 of \dword{tdr} Volume~\volnumbertc~(\voltitletc).  

\item {\bf Laser safety:} The laser system requires operating a class IV laser~\cite{FNAL:Class4Lasers,CERN:Class4Lasers}. This requires an interlock on the laser box enclosure for normal operation, with only trained and authorized personnel present in the cavern for the one-time alignment of the laser upon installation in the feedthroughs. The trained personnel will be required to wear appropriate laser protective eye wear. A standard operating procedure will be required for the laser which will be reviewed and approved by the \dword{fnal} laser safety officer. 

\item {\bf Radiation safety for \dword{pns}:} A $DD$ neutron generator will be used as a calibration device. The design of safety systems for this system include key control, interlock, moderator, and shielding. Lithium-polyethylene (\SI{7.5}{\%}) is chosen to be the material for the neutron shield which is rich in hydrogen. The gammas from neutron capture on hydrogen in the shielding material could cause potential radiation hazards. The design of the radiation safety systems (custom shielding and moderator) will be designed to meet \dword{fnal} Radiological Control Manual (FRCM) safety requirements and will be reviewed and approved by \dword{fnal} radiological control organization. Material safety data sheets will be submitted to the \dword{dune} \dword{esh} to understand other safety hazards such as fire. Before beginning any operations at \dword{pdsp}, the entire system will be assembled in a neutron shielded room and tested to confirm no leaking of neutrons will occur. The system will also have a neutron monitor that can provide an interlock. 


\item {\bf High voltage safety:} Some of the calibration devices will use high voltage. Fabrication and testing plans will show compliance with local \dword{hv} safety requirements at any institution or laboratory that conducts testing or operation, and this compliance will be reviewed as part of the design process.

\item {\bf Hazardous chemicals:} Hazardous chemicals (e.g., epoxy compounds used to attach components of the system) and cleaning compounds will be documented at the consortium management level, with a material safety data sheet as well as approved handling and disposal plans in place.

\item {\bf Liquid and gaseous cryogens:} Cryogens (e.g., liquid nitrogen and \dword{lar}) will most likely be used in testing of calibration devices. Full hazard analysis plans will be in place at the consortium management level for full module or module component testing that involves cryogens. These safety plans will be reviewed appropriately by \dword{dune} \dword{esh} personnel before and during production.

\end{itemize}

