%\section{Risks}


%Tables~\ref{tab:fdgen-calib-risks1} to \ref{tab:fdgen-calib-risks3}
Table~\ref{tab:risks:SP-FD-CAL} lists the possible risks identified by the calibration consortium along with corresponding mitigation strategies and impact on probability, cost, and schedule post-mitigation.
%The table list ?? risks at low and medium level. 
%A more detailed list of risks with additional descriptions is currently under development. 
The table shows all risks are medium or low level, mitigated with necessary steps and precautions. More discussion on each risk is provided below. 
\begin{itemize}
    \item \textit{Risk 1:} The \dword{pdsp} design tests being inadequate for the \dword{fd} is an important one because this requires early validation from \dword{protodune} data so we can perform R\&D of alternate designs and/or improvements on a reasonable time scale. In addition, the calibration ports will be designed to be multipurpose to enable deployment of new systems if they are developed. Therefore, in general, the calibration systems mitigate risk to the experiment as the systems sit above the cryostat and/or use multipurpose ports and may be removed.
    \item \textit{Risk 2:} This is a medium-level risk where the elements of the calibration system fail engineering requirements, such as laser beam divergence and precision of the mechanical system, in which case the as-built system will not meet the physics requirements. The mitigation strategy for this involves testing the same designs envisioned for the \dword{fd} in dedicated lab tests and \dword{pdsp2}, to identify any issues and address them. The pre-installation \dword{qc} will also allow us to reject parts that do not meet requirements. 
    \item \textit{Risk 3:} If the ionization laser beam directly hits the elements of the \dword{pds} system for an extended time, the scintillation efficiency might be degraded. The mirror movement controller of the laser system must avoid the beam directly hitting the \dword{pds}. An automated system will block or turn off the laser beam in case of saturation at one of the \dword{pds} channels. The laser electrical system must allow the later implementation of a hardware interlock if that is found to be necessary.
    \item \textit{Risk 4:} This is a low level risk, where the laser beam location system fails; this would reduce the precision of the \efield measurement but will not prevent the measurement from being made. Pre-fill \dword{qc} will be carried out to minimize this risk. Additionally, redundancy will be built into the system, with alternative targets, including some passive ones. A possible alternative way to obtain an absolute measurement is to use reflections off of the aluminum \dword{fc} profiles, with a very slow angular scan.
    \item \textit{Risk 5:} This risk relates to the laser beam misalignment. If the laser beam becomes misaligned with the mirror sequence, then that specific ionization laser module becomes unusable for calibration. To mitigate this, the ionization laser system includes a visible (red) laser specifically for the purpose of alignment. If the misalignment is not just with the warm mirrors, but also with the cold ones, cryostat cameras might be needed to check arrival of red light to the \dword{tpc}.
    \item \textit{Risk 6:} If the effective attenuation length of \SI{57}{keV} neutrons in \dword{lar} turns out to be significantly smaller than \SI{30}{m}, then \dword{pns} system will not cover the whole detector, or additional modules will be needed. This will be resolved in the next year by a measurement at the Los Alamos National Lab (LANL); the \dword{protodune} run will also provide a full end-to-end demonstration.
    \item \textit{Risk 7:} If the neutron flux from the $DD$ generator of the \dword{pns} system is enough to activate the moderator and cryostat insulation, then a new source of radiological backgrounds might be created. This can be mitigated by neutron activation studies of insulation material, and \dword{protodune} testing at neutron flux intensities and durations well above the run plan, as well as simulation studies done in collaboration with the \dword{dune} Background Task Force.
    \item \textit{Risk 8:} If the neutron yield from the $DD$ generator is not high enough to provide sufficient neutron captures inside the \dword{tpc}, then either the neutron calibration cannot be done or a higher flux generator must be obtained, or additional sources must be used. Investigation is being done on both commercially available and  custom $DD$ generators. Additionally, operating the $DD$ generator with wider  pulse is under consideration, which would require the \dword{pds} to provide the neutron capture time t$_{0}$. Another possibility is to carry out dedicated runs at higher pulse rate and, to ensure that the \dword{daq} can handle it, one would acquire only the data from the \dword{apa}s farthest from the source. All of this will be tested in the \dword{protodune2} run. Placing the neutron source closer to the \dword{tpc} may increase the neutron yield by a factor of 6. An alternative design (Figure~\ref{fig:PNS_Two_Designs}) with neutron source inside the calibration feedthrough ports (centrally located on the cryostat) is being studied. This compact neutron source would be light enough to be moved across different feedthroughs and will provide additional coverage.
    \item \textit{Risk Opportunity 9:} The ionization laser system assumes that the laser beams will be sufficiently narrow for a measurement up to \SI{20}{m} distances. However, as the Rayleigh scattering is of the order \SI{40}{m}, it is possible the laser may travel further than \SI{20}{m}. This may reduce the number of lasers needed and therefore the overall cost. The maximum laser distance will be assessed in \dword{protodune2}.
\end{itemize}

%\fixme{Anne just added this generated risk table. 4/18 4:53 pm CDT}

% risk table values for subsystem SP-FD-CAL
\begin{longtable}{p{0.18\textwidth}p{0.20\textwidth}p{0.32\textwidth}p{0.02\textwidth}p{0.02\textwidth}p{0.02\textwidth}} 
\caption{Risks for SP-FD-CAL \fixmehl{ref \texttt{tab:risks:SP-FD-CAL}}} \\
\rowcolor{dunesky}
ID & Risk & Mitigation & P & C & S  \\  \colhline
RT-SP-CAL-01 & Inadequate baseline design & Early detection allows R\&D of alternative designs accommodated through multipurpose ports & L & M & M \\  \colhline
RT-SP-CAL-02 & Inadequate engineering or production quality & Dedicated small scale tests and full prototyping at ProtoDUNE; pre-installation QC & L & M & M \\  \colhline
RT-SP-CAL-03 & Laser impact on PDS & Mirror movement control to avoid direct hits; turn laser off in case of PDS saturation & L & L & L \\  \colhline
RT-SP-CAL-04 & Laser positioning system stops working & QC at installation time, redundancy in available targets, including passive, alternative methods & L & L & L \\  \colhline
RT-SP-CAL-05 & Laser beam misaligned & Additional (visible) laser for alignment purposes & M & L & L \\  \colhline
RT-SP-CAL-06 & The neutron anti-resonance is much less pronounced & Dedicated measurements at LANL and test at ProtoDUNE & L & L & L \\  \colhline
RT-SP-CAL-07 & Neutron activation of the moderator and cryostat & Neutron activation studies and simulations & L & L & L \\  \colhline
RT-SP-CAL-08 & Neutron yield not high enough & Simulations and tests at ProtoDUNE & L & M & M \\  \colhline
RT-SP-CAL-09 & Neutrons do not reach detector center & Alternative, movable design and simulations & L & L & L \\  \colhline

\label{tab:risks:SP-FD-CAL}
\end{longtable}

