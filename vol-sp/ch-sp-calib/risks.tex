%\section{Risks}
%\label{sec:sp-calib-risks}
%\fixme{New risk format as of late  March. I will send Excel file template to you. Anne}
%\fixme{Table 1.8: Excel sheet with risks is created and sent to Anne who will generate the automated risk table for the TDR. To be replaced.}

%Tables~\ref{tab:fdgen-calib-risks1} to \ref{tab:fdgen-calib-risks3}
Table~\ref{tab:risks:SP-FD-CAL} lists the possible risks identified by the calibration consortium along with corresponding mitigation strategies and impact on probability, cost, and schedule post-mitigation.
%The table list ?? risks at low and medium level. 
%A more detailed list of risks with additional descriptions is currently under development. 
The table shows all risks are medium or low level, mitigated with necessary steps and precautions. More discussion on each risk is provided below. 
\begin{itemize}
    \item \textit{Risk 1:} The \dword{pdsp} design tests being inadequate for the \dword{fd} is an important one because this requires early validation from \dword{protodune} data so we can perform R\&D of alternate designs and/or improvements on a reasonable time scale. In addition, the calibration ports will be designed to be multipurpose to enable deployment of new systems if they are developed. Therefore, in general, the calibration systems mitigate risk to the experiment as the systems sit above the cryostat and/or use multipurpose ports and may be removed.
    \item \textit{Risk 2:} This is a medium-level risk where the elements of the calibration system fail engineering requirements, such as laser beam divergence and precision of the mechanical system, in which case the as-built system will not meet the physics requirements. The mitigation strategy for this involves testing the same designs envisioned for the \dword{fd} in dedicated lab tests and \dword{pdsp2}, to identify any issues and address them. The pre-installation \dword{qc} will also allow us to reject parts that do not meet requirements. 
    \item \textit{Risk 3:} If the ionization laser beam directly hits the elements of the \dword{pds} system for an extended time, the scintillation efficiency might be degraded. The mirror movement controller of the laser system must avoid the beam directly hitting the \dword{pds}. An automated system will block or turn off the laser beam in case of saturation at one of the \dword{pds} channels. The laser electrical system must allow the later implementation of a hardware interlock if that is found to be necessary.
    \item \textit{Risk 4:} This is a low level risk, where the laser beam location system fails; this would reduce the precision of the \efield measurement but will not prevent the measurement from being made. Pre-fill \dword{qc} will be carried out to minimize this risk. Additionally, redundancy will be built into the system, with alternative targets, including some passive ones. A possible alternative way to obtain an absolute measurement is to use reflections off of the aluminum \dword{fc} profiles, with a very slow angular scan.
    \item \textit{Risk 5:} This risk relates to the laser beam misalignment. If the laser beam becomes misaligned with the mirror sequence, then that specific ionization laser module becomes unusable for calibration. To mitigate this, the ionization laser system includes a visible (red) laser specifically for the purpose of alignment. If the misalignment is not just with the warm mirrors, but also with the cold ones, cryostat cameras might be needed to check arrival of red light to the \dword{tpc}.
    \item \textit{Risk 6:} If the effective attenuation length of \SI{57}{keV} neutrons in \dword{lar} turns out to be significantly smaller than \SI{30}{m}, then \dword{pns} system will not cover the whole detector, or additional modules will be needed. This will be resolved in the next year by a measurement at the Los Alamos National Lab (LANL); the \dword{protodune} run will also provide a full end-to-end demonstration.
    \item \textit{Risk 7:} If the neutron flux from the $DD$ generator of the \dword{pns} system is enough to activate the moderator and cryostat insulation, then a new source of radiological backgrounds might be created. This can be mitigated by neutron activation studies of insulation material, and \dword{protodune} testing at neutron flux intensities and durations well above the run plan, as well as simulation studies done in collaboration with the \dword{dune} Background Task Force.
    \item \textit{Risk 8:} If the neutron yield from the $DD$ generator is not high enough to provide sufficient neutron captures inside the \dword{tpc}, then either the neutron calibration cannot be done or a higher flux generator must be obtained, or additional sources must be used. Investigation is being done on both commercially available and  custom $DD$ generators. Additionally, operating the $DD$ generator with wider  pulse is under consideration, which would require the \dword{pds} to provide the neutron capture time t$_{0}$. Another possibility is to carry out dedicated runs at higher pulse rate and, to ensure that the \dword{daq} can handle it, one would acquire only the data from the \dwords{apa} farthest from the source. All of this will be tested in the \dword{protodune2} run. Placing the neutron source closer to the \dword{tpc} may increase the neutron yield by a factor of 6. An alternative design (Figure~\ref{fig:PNS_Two_Designs}) with neutron source inside the calibration feedthrough ports (centrally located on the cryostat) is being studied. This compact neutron source would be light enough to be moved across different feedthroughs and will provide additional coverage.
    \item \textit{Risk Opportunity 9:} The ionization laser system assumes that the laser beams will be sufficiently narrow for a measurement up to \SI{20}{m} distances. However, as the Rayleigh scattering is of the order \SI{40}{m}, it is possible the laser may travel further than \SI{20}{m}. This may reduce the number of lasers needed and therefore the overall cost. The maximum laser distance will be assessed in \dword{protodune2}.
\end{itemize}

%\fixme{Anne just added this generated risk table. 4/18 4:53 pm CDT}

% risk table values for subsystem SP-FD-CAL
\begin{longtable}{p{0.18\textwidth}p{0.20\textwidth}p{0.32\textwidth}p{0.02\textwidth}p{0.02\textwidth}p{0.02\textwidth}} 
\caption{Risks for SP-FD-CAL \fixmehl{ref \texttt{tab:risks:SP-FD-CAL}}} \\
\rowcolor{dunesky}
ID & Risk & Mitigation & P & C & S  \\  \colhline
RT-SP-CAL-01 & Inadequate baseline design & Early detection allows R\&D of alternative designs accommodated through multipurpose ports & L & M & M \\  \colhline
RT-SP-CAL-02 & Inadequate engineering or production quality & Dedicated small scale tests and full prototyping at ProtoDUNE; pre-installation QC & L & M & M \\  \colhline
RT-SP-CAL-03 & Laser impact on PDS & Mirror movement control to avoid direct hits; turn laser off in case of PDS saturation & L & L & L \\  \colhline
RT-SP-CAL-04 & Laser positioning system stops working & QC at installation time, redundancy in available targets, including passive, alternative methods & L & L & L \\  \colhline
RT-SP-CAL-05 & Laser beam misaligned & Additional (visible) laser for alignment purposes & M & L & L \\  \colhline
RT-SP-CAL-06 & The neutron anti-resonance is much less pronounced & Dedicated measurements at LANL and test at ProtoDUNE & L & L & L \\  \colhline
RT-SP-CAL-07 & Neutron activation of the moderator and cryostat & Neutron activation studies and simulations & L & L & L \\  \colhline
RT-SP-CAL-08 & Neutron yield not high enough & Simulations and tests at ProtoDUNE & L & M & M \\  \colhline
RT-SP-CAL-09 & Neutrons do not reach detector center & Alternative, movable design and simulations & L & L & L \\  \colhline

\label{tab:risks:SP-FD-CAL}
\end{longtable}

\begin{comment}
\begin{dunetable}
[Calibration risks1]
{p{0.03\linewidth}p{0.4\linewidth}p{0.05\linewidth}p{0.4\linewidth}}
{tab:fdgen-calib-risks1}
{Possible risk scenarios for the laser systems along with mitigation strategies. The level of risk is indicated by letters ``H'', ``M'', and ``L'' corresponding to high, medium and low level risks.}   
No. & Risk  & Risk Level & Mitigation Strategy  \\ \toprowrule
1 & The baseline design for calibration devices is not adequate for the \dword{fd}. & M & This should be detected early on such that R\&D on alternative designs can proceed on a reasonable time scale. In addition, the cryostat ports have been designed to be multipurpose to enable deployment of new systems if they are developed. \\ \colhline

2 & Elements of the calibration system fail engineering requirements, such as laser beam divergence, and precision of the mechanical system. & L & Although not under identical conditions/apparatus, the same designs envisioned for the FD will be tested in dedicated lab tests and \dword{pdsp}, to identify these issues and mitigate them.
\\ \colhline

3 & The laser light can impact the \dword{pds} system. & M & The mirror movement controller must avoid the beam directly hitting the \dword{pds}. The laser electrical system must allow the latter implementation of a hardware interlock if that is found to be necessary. \\ \colhline

4 & The laser beam location system stops working. & L & Pre-fill QC will be carried out. A possible alternative way to obtain an absolute measurement is to use reflections off of the aluminum FC  profiles, with a very slow angular scan.\\ \colhline

5 & The laser beam becomes misaligned with the mirror sequence. & M & The system includes a visible (red) laser specifically for the purpose of alignment. If the misalignment is not just with the warm mirrors, but also with the cold ones, cryostat cameras might be needed to check arrival of red light to the TPC.\\ 

%While DD Generators produce neutrons with relatively modest fluxes and most materials do not have significant activation (which is why they are typically not used for activation studies), it is prudent to have actual measurements of the activation of materials in the vicinity of the PNS to be able to predict accurately the long-term activation. We propose to use the UC Berkeley DD Generator facility in the Advanced Technology and Innovation Laboratory (ATIL) to exposure cryostat materials to many orders of DD flux (2.45 MeV) than they will see from the PNS over the lifetime of DUNE. ATIL will let us use their facility for a small charge, and results will be used to ensure no long-term significant activation will occur. 

%The scattering length at the $^{40}Ar$ $57\; keV$ anti-resonance has been theoretically calculated to be $1400\; m$, but since argon is 0.0629\% $^{38}Ar$ and 0.3336\% $^{36}Ar$ with scattering lengths of $542\; m$ and $33\; m$ respectively, the overall scattering length of $30 \;m$ does not depend significantly on the exact depth of the anti-resonance. Nevertheless, it is desirable to verify the overall scattering length with a measurement at a dedicated scattering facility such as LANSCE. LANSCE has a neutron Time-Of-Flight (TOF) beam with good resolution in the $10-100 \; keV$ range and so a simple transmission experiment using a liquid argon cylindrical target of diameter $5\;cm$ and length $100-200\; cm$ should be more than sufficient to measure the scattering cross-section in the region of interest.  

%Such an experiment will be proposed to LANSCE in March 2019 to run in early Fall 2019. Costs will be minimal - with only the need to provide a LAr target with a small $~2\; cm$ thin window on both ends, plus perhaps a small halo counter to reject double scatters and a collimated neutron TOF detector (LANL may be able to provide this). While desirable to do, this is not critical.

\end{dunetable}
\end{comment}
%%%%%%%%%%%%%%%%%%%%%%%

\begin{comment}

\begin{dunetable}
[Calibration risks2]
{p{0.03\linewidth}p{0.4\linewidth}p{0.05\linewidth}p{0.4\linewidth}}
{tab:fdgen-calib-risks2}
{Possible risk scenarios for the pulsed neutron source system along with mitigation strategies. The level of risk is indicated by letters ``H'', ``M'', and ``L'' corresponding to high, medium and low level risks.}   
No. & Risk  & Risk Level & Mitigation Strategy  \\ \toprowrule

6 & The effective attenuation length of 57 keV neutrons in \dword{lar} turns out to be significantly smaller than 30 m. & M & A measurement of the transmission at this energy is being proposed at Los Alamos prior to the ProtoDUNE run. The ProtoDUNE run will also provide demonstration. \\ \colhline

7 & The neutron flux from the $DD$ generator could activate the moderator and cryostat insulation. & L & Neutron activation studies of insulation material, and ProtoDUNE testing at neutron flux intensities and durations well above the run plan, as well as simulation studies done in collaboration
with Background Task Force. \\ \colhline

8 & The neutron yield from $DD$ generator is not high enough to provide sufficient neutron captures inside the TPC. & M & Investigation is being done on both commercially available and lab research $DD$ generators; Placing the neutron source closer to the liquid argon TPC may increase the neutron yield by a factor of 6; Operating the $DD$ generator with wider  pulse is under consideration, which would require the photodetector system to provide the neutron capture time t$_{0}$. All of this will be tested in the ProtoDUNE-SP-II run. \\ \colhline

9 & Neutrons produced by the Pulsed Neutron Sources placed at the human access ports at the cryostat corners may not reach the center of the cryostat. & L & An alternative design (Design B in Figure~\ref{fig:PNS_source_design}) with neutron source inside the calibration feedthrough ports (centrally located on the cryostat) is being studied. This small format neutron source would be light enough to be moved across different feedthrough ports, providing complementary coverage to the neutron sources at the human access port locations. \\ 

\end{dunetable}
\end{comment}


\begin{comment}
\begin{dunetable}
[Calibration risks3]
{p{0.03\linewidth}p{0.4\linewidth}p{0.05\linewidth}p{0.4\linewidth}}
{tab:fdgen-calib-risks3}
{Possible risk scenarios for the radioactive source system along with mitigation strategies. The level of risk is indicated by letters ``H'', ``M'', and ``L'' corresponding to high, medium and low level risks.}   
No. & Risk  & Risk Level & Mitigation Strategy  \\ \toprowrule

10 & The deployed radioactive source can potentially swing into detector elements if not controlled or if large currents exist in the \dword{lar} & M & Guide-wires mitigate this risk.\\ \colhline

11 & Radioactivity could leak into the detector during a deployment. & L & Rigorous source certification under large pressure and cryogenic temperatures mitigates this risk.\\ \colhline

12 & The source could get stuck or lost in the detector. & L & Fish-line an order of magnitude stronger than needed to hold the weight, round edges of the moderator and a torque limit of the stepper motor will mitigate this risk.\\ \colhline

13 & Oxygen and nitrogen could get into the \dword{lar} in case the purge-box has a small leak. & M & Leak checks before deployments, purge-box in under-pressure inside w/r to the detector, will mitigate this risk.\\ \colhline

14 & Light could couple into the detector. & M &
Light-tight purge-box, internally equipped with an infra-red camera for visual checks will mitigate this risk.\\ \colhline

15 & The source activity can activate the cryostat insulation. & L & Detailed simulations/activation measurements can say what is a tolerable activity and the source activity can be chosen to be below that. \\ 

\end{dunetable}
\end{comment}

