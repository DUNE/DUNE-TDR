
%%%%%%%%%%%%%%%%%%%%%%%%%%%%%%%%%%%%%%%%%%%%%
%\section{Institutional Responsibilities}
 

Calibrations will be a joint effort for \single and \dual. Design validation, testing, calibration, and performance of calibration devices will be evaluated using \dword{protodune} data.

Following the conceptual funding model for the consortium, various responsibilities have been distributed across institutions within the consortium. 
Table~\ref{tab:calib-inst-resp} shows the current institutional responsibilities for primary calibration subsystems. 
For physics and simulations studies and validation with \dword{protodune}, a number of institutions are interested. 

\begin{dunetable}
[Institutional responsibility for calibrations]
{p{0.25\textwidth}p{0.65\textwidth}}
{tab:calib-inst-resp}
{Institutional responsibilities in the Calibration Consortium}  Subsystem & Institutional Responsibility \\ \toprowrule
Ionization Laser System & Bern, LIP, LANL, Hawaii \\ \colhline 
\dlong{lbls} & Hawaii, LIP \\ \colhline 
Photoelectron Laser System & LANL, Hawaii \\ \colhline
Pulsed Neutron Source System & BU, CSU, UC Davis, Iowa, LIP, MSU, LANL, SDSMT \\ \colhline
Proposed Radioactive Source System & SDSMT \\ \colhline
Physics \& Simulation & BU, CSU, Hawaii, LANL, LIP, MSU, SDSMT, UC Davis, Pittsburgh, Iowa \\  
\end{dunetable}