\section{Environmental, Safety, and Health (ES\&H)}
\label{sec:fdsp-tc-safety}

 Volume \volnumbertc{}, Section~\ref{vl:tc-ESH} of this \dword{tdr} outlines the requirements and regulations that \dword{dune} work must comply with, whether (1) in areas  owned or leased by the host lab, \dword{fnal}, or the \dword{doe}, (2) at collaborating institutions, or (3)  in unleased space at the \dword{fd} site, \dword{surf}. 

\subsection{Documentation Approval Process}

\dword{dune} implements an engineering review and approval process for all required documentation, including structural calculations, assembly drawings, load tests, \dwords{ha}, and procedural documents for a comprehensive set of identified individual tasks.  For the larger operations and systems like \dword{tpc} component factories, the \dword{dss}, cleanroom, and assembly infrastructure, a joint safety committee also reviews the documentation then visits the site to conduct
 an \dword{orr}. The committee watches the full operation before signing off on the documentation.
 
\subsection{\dword{esh} Support and Responsibilities}

The \dword{dune} Project \dword{esh} Coordinator has overall \dword{esh} oversight responsibility for the \dword{dune} Project, as discussed in \volnumbertc{} \fixme{add section}. 

\fixme{who is?} responsible for managing \dword{esh}-related  documentation including training records, weekly safety reports, near-miss and accident reports, and equipment inspection.

All employees have work stop authority in support of  a safe working environment. 

%\subsubsection{Logistics Organization}
\fixme{this next pgraph should just be in Steve's TC volume}

The \dword{dune} Project \dword{esh} Coordinator has overall \dword{esh} oversight responsibility for the \dword{dune} Project.  This person coordinates any \dword{esh} activities and facilitates the resolution of any issues that are subject to the requirements of the \dword{doe} Workers Safety and Health Program, Title 10, Code Federal Regulations (CRF) Part 851 (10 CFR 851), and that involve different project stakeholders. These requirements are promulgated through the Fermilab Director's Policy Manual \fixme{ref} and Fermilab \dword{esh} manual (FESHM\cite{feshm}), which aligns with the \dword{surf} \dword{esh} manual.  

\subsection{Logistics Safety}

\fixme{should cover: transport,  warehouse (receiving and shipping, QC), ross headframe (receiving)}

Using the \dword{nova} Far Detector Laboratory as a guideline for remote facilities, several other key documents will guide the \dword{dune} logistics safety program, as follows:
\fixme{these seem like they apply also to installation, etc., not just logistics}

\begin{enumerate}
\item	Fire Safety and Building Emergency Evacuation Plan, which includes the fire evacuation plan, fire safety plan, lockdown plans, and the site plan;
\item	\dword{ha} document, which describes all typical hazards and their mediation procedures; 
\item	Safety Data Sheets (SDS), 
\item	Respiratory Plan, as required for chemical or ODH hazards, and 
\item	Training Program, which covers required certifications and  training records.
\end{enumerate}

The current Technical Coordination Facilities Management Plan \fixme{ref} specifies a safety officer for the \dword{sdwf} \fixme{and surf areas?}. This safety officer facilitates training, writes hazard analysis documents, runs weekly safety meetings, and keeps documentation records on materials-handling equipment and personnel training.

%%%%%%%%%%%%%%%%%%
%%%%%%%%%%%%%%%%%%%%%%%%%%%%
\subsection{Installation Setup Phase Safety}
\label{sec:fdsp-tc-infr-safety}

\fixme{DSS, cryostat roof, cuc?, cable trays, elec mezzanine, cleanrooom, coldboxes inside of cryostat, cryostat floor, crossing tubes, pipe distribution networks, lifts, lighting, apa assembly towers, cranes, qc,}

During the installation setup phase, as new equipment is being installed and tested, new employees and collaborators will be trained, and larger teams from the consortia, \dword{sdsd},  and contractors will need access to the underground facilities.  Given the maximum of 140 \dword{fte} underground at any given time, we will move from one to two shifts per day at this point. 

Structural calculations, assembly drawings and proper documentation of  load tests, hazard analyses, and procedures for various items and activities will require review and approval before operational readiness is granted. 

Unlike most items, the \coldbox and cryogenics system will not be fully tested during the trial assembly work at Ash River. 
While the new \coldbox design is very similar to \dword{pdsp}'s, some new requirements will be in effect. \fixme{why? what changed?}  Procedures for operating the \coldbox will be written according to the established requirements.

During this phase, \dword{lbnf} will be completing the cold structure on \dword{detmodule} \#1 and beginning the warm structure on  \dword{detmodule} \#2. Once \cooldown begins on module \# 1, unlike for \dword{pdsp}, it will be safe for workers to remain in the cleanroom.  \fixme{again, why?}

  

%%%%%%%%%%%%%%%%%%%%%%%%%%%%
\subsection{Installation Phase Safety}
\label{sec:fdsp-tc-inst-safety}

\fixme{should this text be in Steve's vol?}
Fermilab and \dword{dune} are committed to supporting the health and safety of staff, the community, and the environment in research and operations, as stated in the \dword{lbnf}/\dword{dune} Integrated Safety Management Plan\cite{bib:docdb291}. The safety and health program complies with applicable standards and local, state, and federal legal requirements through Fermilab's Work Smart Set of Standards \fixme{ref } and the contract between \dword{fra} and the \dword{doe}. \dword{fnal} and the \dword{sdsd} have the host laboratory responsibilities for \dword{lbnf} and \dword{dune} operations at \dword{surf} in Lead, South Dakota.
The Fermilab facilities are further subject to the requirements of the \dword{doe} Workers Safety and Health Program 10 CFR 851\cite{doe-10cfr851}. These requirements are promulgated through the Fermilab Directors Policy Manual, and the \dword{feshm}, which align with the \dword{surf} \dword{esh} manual.

While \dword{esh} is  a host laboratory (\dword{fnal}'s \dword{sdsd}) responsibility, a  \dword{gsc} will evaluate applicable codes and standards including international code equivalency for the design, assembly, and installation of the \dword{dune} \dwords{detmodule}. The \dword{gsc} is  a team of engineering and \dword{esh} experts from within \dword{lbnf} and \dword{dune} organizations.  These requirements will be adopted by \dword{dune}, \dword{jpo}, and \dword{lbnf} organizations and used to develop manufacturing, assembly, and installation processes and procedures. 
\fixme{End part that sounds like it should be in Steve's volume}

\dword{dune} will develop an %Installation 
\dword{esh} plan for \dword{detmodule} installation to define %a specific set of 
the \dword{esh} requirements and responsibilities for personnel during  assembly, installation, and construction of equipment \fixme{construction of eqp sounds like setup phase - aren't these things required at any phase?} at \dword{surf}. % and \dword{itf}. 

{\bf Access and training:}  All \dword{dune} workers requiring access to the \dword{surf} site must (1) register through the \dword{fnal} Users Office to receive the necessary user training and a \dword{fnal} identification number, and (2) they must apply for a \dword{surf} identification badge. 
All \dword{dune} workers will be required to complete \dword{surf} Surface and Underground Orientation classes. Workers accessing the underground must also complete 4850L and 4910L specific unescorted access training, and obtain a \dword{tap} for each trip to the underground area; this is required as part of \dword{surf}'s Site Access Control Program. 
A properly trained guide will be stationed on all working levels. 

{\bf \dword{ppe}:} 
The host laboratory (\dword{fnal}'s \dword{sdsd}) is responsible for supplying appropriate \dword{ppe} to all workers. 

{\bf \dword{em} program} The \dword{sdsta} will maintain an Emergency Response incident command system and an \dword{ert}.  The guides on each underground level will be trained as first responders to help in a medical emergency.
  
  
  {\bf House cleaning:} All workers are responsible for keeping a clean organized work area. This is particularly important underground. Flammable items must be in proper storage cabinets, and items like empty shipping crates and boxes must be removed and 
transported back to the surface to make space.


{\bf Equipment operation:} All overhead cranes, gantry cranes, fork lifts, motorized equipment trains/carts will be operated only by trained  operators. 
Other equipment, e.g., scissor lifts, pallet jacks, hand tools, and shop equipment, may only be operated by people trained
and certified for the particular piece of equipment.

\fixme{the requ at collab inst should be in steve's}
{\bf Requirements at collaborating %laboratories and 
institutions:} All work performed at collaborating institutions will be completed in accordance with the %collaborating 
institution's \dword{esh} policies and programs. 
Equipment and operating procedures provided by the %collaborating 
institution must conform to the \dword{dune} Project \dword{ieshp}. %\dword{esh} and Integrated Safety Management policies and procedures. 
The %collaborating 
institution's \dword{esh} department is responsible for providing \dword{esh}  oversight for all work activities carried out in their facilities. % in collaborating institution facilities. 
\dword{lbnf} and \dword{dune} personnel will also follow the \dword{esh} manual and procedures of the collaborating institutions.


{\bf Work Planning and \dword{ha}:} The goal of the work planning and \dword{ha} process is to initiate thought about the hazards associated with work activities and plan how to perform the work. Work planning ensures the scope of the job is understood, appropriate materials and tools are available, all hazards are identified, mitigation efforts are established, and all affected employees understand what is expected of them. 
The Work Planning and Hazard Analysis program is documented in Chapters 2060 in the \dword{feshm}.

The shift supervisor and a local \dword{esh} coordinator  will lead a work planning meeting at the start of each shift  to (1) coordinate the work activities, (2) notify the workers of potential safety issues, constraints, and hazard mitigations, (3) ensure that employees have the necessary \dword{esh} training and \dword{ppe}, and (4) answer any questions.

 

  %%%%%%%%%%%%%%%%





   




