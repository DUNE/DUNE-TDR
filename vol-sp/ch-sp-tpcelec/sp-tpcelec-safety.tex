\section{Safety}
\label{sec:fdsp-tpcelec-safety}

Personnel safety during construction, testing, integration,
and installation of the \dword{ce} components for the \dword{dune}
\dword{sp} \dword{fd} is crucial for the success
of the project. The members of \dword{ce} consortium will
respect the safety rules of the institutions where the work is
performed, which may be one of the national laboratories, \dword{itf}, \dword{surf},
or one of the universities participating in the project. A
preliminary analysis of the risks involved in the design,
construction, integration, and installation of the detector
components provided by the \dword{ce} consortium has been
performed using the approach discussed in the Technical
Coordination part of this \dword{tdr} (Volume 7 Chapter 10). 
%Commented out because it breaks the automatic build system %(\refch{tc}{vl:tc-ESH}).


%%%%%%%%%%%%%%%%%%%%%%%%%%%%%%%%%%%
\subsection{Personnel Safety During Construction}
\label{sec:fdsp-tpcelec-safety-personnel}

The main risks for consortium personnel are exposure to
the \lntwo used for cooling down components during testing (risk HA-8
in Table 10.1 in Volume 7);
%in Table~\refvol{tc}{tab:hazards}), %Commented out because it breaks the automatic build system
oxygen deficiency hazard, possibly caused by leaks
of either \lntwo or \dword{lar} from test setups (risk HA-8);
electrical shocks (risk HA-6); and falls from heights (risk HA-1). The leadership of the
\dword{ce} consortium will work with the \dword{lbnf}/\dword{dune}
\dword{esh} manager and other relevant responsible personnel at the
various institutions to ensure all the members of the
consortium receive the appropriate training for the work they
are performing and that all preventive measures to minimize
the risk of accidents are in place. Where appropriate,
we will try to adopt the strictest standard and requirements among
those of different institutions. Hazard analyses will be performed,
and the level of \dword{ppe} will be determined
appropriately for each task. \Dword{ppe} includes 
appropriate gloves for handling \lntwo dewars, fall
protection equipment for work at heights, and steel toed shoes and
hard hats for integration work with the \dword{apa}. Oxygen
monitors should be used for areas with large concentrations of
cryogenic gases.

\dword{esh} plans for the activities to be performed in various
locations, including all universities, national laboratories,
the \dword{itf} and \dword{surf}, will be part of the 
various reviews (Preliminary Design, Engineering Design, Production 
Readiness, Production Progress) that will take place during the construction of the detector.

%%%%%%%%%%%%%%%%%%%%%%%%%%%%%%%%%%%
\subsection{Detector Safety during Construction}
\label{sec:fdsp-tpcelec-safety-detcon}


In addition to personnel safety during detector
construction, including all testing, 
integration, installation, and commissioning, we have also
considered how to protect the detector
components and minimize any chances of damaging
them during handling. We identified two main risks 
to the safety of the detector during construction and one risk during
operation. The most important risk during construction is damage 
induced by \dword{esd} in the 
electronic components. The second risk is mechanical damage to 
parts during transport and handling. For operation risks, we
must consider the risk of damage to the electronics 
caused by accumulated dust inside the components
installed on the top of the detector. In this Section, 
we discuss these three risks and ways to minimize their possible 
effect. In the following Section, we will discuss how to prevent
damage during operation to the \dword{ce} components 
by using the interlocks of the detector safety system.

\Dwords{esd} can damage any of the electronics
components mounted on the \dwords{femb}, \dwords{wiec},
the bias voltage supplies, or the power supplies. If the
the damage occurs early in construction, 
the outcome is a reduction
of the yield for some of components, which must be
addressed by keeping a sufficient number of spares on hand to prevent
schedule delays associated with procuring new parts. \dword{esd}
damage on the \dwords{femb} after the \dword{apa}s have been
installed inside the cryostat could result in a permanent
reduction of the fraction of operating channels in the
detector. Even if most components, including the custom 
\dwords{asic} designed for use in the \dwords{femb}, contain 
some level of protection  against \dword{esd}, this kind of damage cannot be discounted, and appropriate 
preventive measures must be taken during  
assembly, testing, installation, and shipping of all the detector 
components provided by the \dword{ce} consortium. These 
measures include using appropriate \dword{esd}-safe packing materials, 
appropriate clothing and gloves, wearing conducting wrist or foot straps 
to prevent charges from accumulating 
on workers' bodies, anti-static mats to conduct harmful electric 
charges away from the work area, and humidity control. All laboratories with detector components provided by the \dword{ce} consortium will implement these
measures,
including \dword{itf} and \dword{surf}. 

Additional measures
include using custom made terminations for 
power, control, and read-out cold cables when these
are being routed through the \dword{apa} frames or through the
cryostat penetrations. Storage cabinets where \dwords{asic} and
\dwords{femb} are stored should have \dword{esd} mats
on the shelves and humidity control. Most importantly, all personnel must be trained to take the appropriate preventive measures. We 
will require that all the personnel working on the \dword{ce} 
consortium components take a training class originally developed 
at \dword{fermilab} for handling the charge-coupled devices of the Dark Energy Survey (DES) experiment 
(the material for this training class can also be used at remote 
sites). The scientists in charge of the \dword{ce} activities at each site involved in the project will be responsible for monitoring the training of personnel at universities, 
\dword{itf}, and \dword{surf}.

Most of the damage to detector components happens during 
transport among the various sites where assembly, testing, integration,
and installation take place. When appropriate, measures to prevent
\dword{esd} damage must be taken also for shipments. Appropriate 
packaging will be used to ensure that parts are not damaged
during transport. We will perform tests on receiving  
\dwords{femb} as well as integration tests for cold cables as
part of the quality control process discussed in 
Section~\ref{sec:fdsp-tpcelec-integration-qc} to ensure the 
full functionality of these parts, which are very hard to replace 
after detector integration and installation. For  
components on the top of the cryostat that
can be replaced if damaged during transport, we will 
perform integration tests after installation.

In addition to damage during shipping, we must also consider the
possibility of damage caused by handling of the detector parts.
Additional precautions are being considered for operations where
the risk of damage to \dword{ce} detector components is
high. For testing \dwords{asic} and \dwords{femb} 
in \lntwo, this has resulted in developing the
\dword{cts} discussed in Section~\ref{sec:fdsp-tpcelec-qa-facilities-additional}
to prevent condensation on components after they are extracted from
the \lntwo at the end of a test. For the cold cables, this 
includes modifying the size of the tubes used for the \dword{apa} frames,
adding a conduit inside the frames, and placing a mesh around 
the cables. Special tooling will be designed for arranging the
cold cables on the spools used when cables are routed through
the \dword{apa} frames. Similarly, tooling will be developed to 
support the cold cables while they are being routed through the 
cryostat penetrations. All cables and fibers will
be installed in cable trays on top of the cryostat, and for the fibers, additional protection in the form of sleeves or tubes 
may also be used. 

To ensure that the \dword{dune} detector will be operational for a long
time, we also will attempt to minimize 
damage that could happen to detector components inside the 
experimental cavern, which can come from two sources: incorrect
operation of the detector and environmental conditions. We
will discuss the first in the next Section. Once the cryostat is filled with \dword{lar}, the environmental
conditions inside the cryostat are extremely stable. Experience from 
previous experiments using electronics inside 
\dword{lar} indicates that, apart from initial problems, little loss of read-out channels occurred over long periods. 

Therefore, the main concern would be electronics 
installed on top of the cryostat. There, the main problem
is dust accumulation on detector components. Over the long
term, dust could damage the cooling fans used
in the \dwords{wiec} and the \dword{ce} racks, break down 
on the surface of diodes used in bias voltage supplies,
and, if the dust contains any amount of salt and if the air has sufficient humidity, dendrites may grow that could create 
shorts between traces on a printed circuit board. While the
experimental cavern should be a very dry environment,
protections should still be in place to prevent water from 
dripping on the \dwords{wiec} and on the racks containing the
\dword{ce} supplies. HEPA filters will be added to the
air supply used to cool the \dwords{wiec} and the \dword{ce} power and bias voltage supplies, thus minimizing
the accumulation of dust. The air humidity in the cavern 
will be controlled to prevent condensation.

%%%%%%%%%%%%%%%%%%%%%%%%%%%%%%%%%%%
\subsection{Detector Safety during Operation}
\label{sec:fdsp-tpcelec-safety-detops}

In this Section, we discuss where we will use
the detector safety system described in the Technical
Coordination part of this \dword{tdr} (Volume 7 Chapter 6). %Commented out because it breaks the automatic build system %(\refch{tc}{vl:tc-facility})
to avoid unsafe conditions for the \dword{ce} detector 
during operations. Hardware interlocks will be put in place
in test setups, at both the \dword{itf} and \dword{surf}, to prevent operating or even powering up
detector components 
unless conditions are safe
both for the detector and for personnel. Interlocks will be
used on all low voltage power and on bias voltage 
supplies, including inputs from environmental monitors both
inside and outside the cryostat. Examples of these interlocks include turning off the power to the \dwords{wiec}
if the corresponding cooling fans are not operational or
if the temperature inside the crates exceeds a pre-set value.
Similar interlocks will be used for low voltage power
and bias voltage supplies in \dword{ce} racks.
Interlocks may be needed connecting the value of the 
bias voltage on the \dword{fc} termination electrodes to the
high voltage applied on the \dword{tpc} cathode. Interlocks will turn 
off transmitters on the \dwords{wiec} if the readout fibers'
bundles are cut. One problem we must address is 
the connection between the \dword{plc} used by the detector 
safety system and the \dwords{wiec} to avoid introducing noise 
inside the detector. We can easily decouple the environmental 
sensors required by the detector safety system inside the 
\dwords{wiec} by following the appropriate grounding rules. 
The connection used to provide the enable/disable signals 
from the \dword{plc} to the \dwords{wiec} will require optical 
fibers to avoid possible ground loops. Interlocks connected
to the detector safety system will also be used during tests 
of the \dword{apa}s in the cold boxes at the \dword{itf} and 
at \dword{surf}. The \dword{cts} has its own interlock system to
prevent condensation from forming on the \dwords{femb} once
they have warmed to room temperature. We cannot exclude the possibility that for some of the smaller test stands, we will 
have to rely on software interlocks for detector safety,
but they should be kept to a minimum, and no software
interlock should be used at the \dword{itf} or \dword{surf}.
