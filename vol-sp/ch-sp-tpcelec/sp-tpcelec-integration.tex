\section{Integration, Installation, and Commissioning}
\label{sec:fdsp-tpcelec-integration}

\metainfo{The material contained in this section could be moved entirely 
into the Technical Coordination chapter of the Single Phase volume. 
Jim Stewart seems to prefer this. The CE consortium would like to 
retain the testing aspects into the CE chapter. We agree that there 
may be a few paragraphs from this section that need to be moved into 
the TC chapter. Since the TC chapter is going to be provided to the
LBNC at a later date, for the moment we prefer to keep all the text
here to facilitate the February 28 review by the LBNC.}

Chapter~\ref{ch:sp-tc} provides a complete discussion of the plans for 
integrating, installing, and commissioning the detector.
Here, we briefly discuss the responsibilities of the \dword{ce}
consortium for the activities taking place at the
\dword{itf} and at \dword{surf}, with emphasis on
the \dword{qc} process. We also discuss the timeline
and the resources for the integration and installation activities.
Finally, we conclude with a discussion of the commissioning
of the \dword{ce} detector components that take place while
the cryostat is being filled and immediately after. 

%%%%%%%%%%%%%%%%%%%%%%%%%%%%%%%%%%%
\subsection{Timeline and Resources}
\label{sec:fdsp-tpcelec-integration-timeline}

The current \dword{ce} consortium plan is to receive all detector 
components at the logistic facility in South Dakota, store them temporarily, 
and, if needed, change the packaging before transporting the components to 
\dword{surf}. In most cases, this will entail simply dividing a shipment 
received at the logistic facility into the number of units required for a 
specific installation job at \dword{surf}. The \dwords{femb} are transported 
from the logistic facility to the \dword{itf} where they undergo
a quick reception test to ensure that the boards and \dwords{asic} 
have not been damaged in any way during transport from the \dword{femb} 
\dword{qc} sites. The \dwords{femb} are then installed on the \dword{apa}s,
followed by a second check of the functionality of 
the boards and of the corresponding \dwords{asic} as well as a check of 
the connection to the \dword{apa} wires. 

All other integration takes place at \dword{surf} in the clean room
in front of the detector cryostat. After a pair of \dword{apa}s are 
connected and moved inside the clean room, the \dword{ce} cables
for the bottom \dword{apa} are routed through the \dword{apa} frames.
The cables are then connected to the \dwords{femb}, and the bundles
of cables are placed in the trays on the top of the \dword{apa} pair.
At this point, the pair of \dword{apa}s is moved into one of the cold
boxes, and the cables are connected to a patch panel inside the cold box
to save the time that would be required for routing the cables through the cryostat
penetration of the cold box and connecting them to the end flange.
The \dword{ce} electronics is then tested at both room temperature
and at a temperature close to that of \lntwo, much like
what was done for the \dword{apa}s installed in the \dword{pdsp} detector.

Later, the pair of \dword{apa}s is moved to its final position 
inside the cryostat. The \dword{ce}
and \dword{pds} cables are routed through the cryostat penetration and
connected to the corresponding warm flanges, and final leak tests are performed
on the cryostat penetration. At this point, the \dword{wiec} is attached
to the warm flange and all the cables and fibers required to provide 
power and control signals to the \dword{ce} and for data
readout are connected. This permits additional testing with the full 
\dword{daq} readout chain and the final power and controls distribution
system. Once initial tests are completed successfully, more \dword{apa}s
can be installed, and the \dword{apa}s can remain accessible until the \dword{fc} are
deployed.

This installation sequence assumes that all the \dword{ce} detector 
components required for readout of a pair of \dword{apa}s  on top of the cryostat 
are installed before \dword{apa}s are inserted into the cryostat. This 
includes the \dwords{wiec} with their boards, the power supplies in 
the racks, and all cables and fibers required to distribute power and
control signals as well as for detector readout. Installation should occur at least two weeks before
the \dword{apa}s are inserted into the cryostat to allow time for some checks. 

One exception
is installing the cryostat penetrations with the warm flanges
for both the \dword{ce} and the \dword{pds}. The cryostat
penetrations should be installed as soon as possible, at the latest
simultaneous to installing the detector support structure
inside the cryostat. This ensures the cryostat is almost 
completely sealed to minimize the amount of dust 
entering the cryostat. During the routing of the \dword{ce} and
\dword{pds} cables through the cryostat penetrations, dust entering the cryostat will be minimized by having a small
over-pressure inside the cryostat and by isolating each penetration
from the cavern using a tent mounted over 
the work area.

We expect that three months at most will be needed to commission
all the test systems at the \dword{itf}, and the current planning
assumes that eighteen months are required to integrate the \dword{ce}
and the \dword{pds} with two work stations. This translates to
a rate of one pair of \dword{apa}s per week. Given that \dword{apa}
production will start well in advance of the \dwords{asic} fabrication
and the assembly of \dwords{femb}, initial integration tests will 
be performed with \dword{femb} prototypes, and the overall schedule
for integration will be driven by the availability of the \dword{ce}
components. The work performed by the \dword{ce}
consortium at the \dword{itf} requires 2 FTEs/week for installing and testing
the \dwords{femb} on the \dword{apa}s as well as  
\dword{femb} reception tests. We expect these shifts to be covered 
by scientific personnel from the \dword{ce} consortium. In addition, 
we plan to have one engineer and one technician available at the 
\dword{itf} as support for the consortium, including 
preparing components for shipment to \dword{surf}.
% The integration of the \dwords{femb} on the \dword{apa}s for a
% second Single Phase TPC \dword{fd} would proceed after the
% completion of the first detector and would require an additional
% 26 months. This period of time is longer than the corresponding
% one for the first detector and it is driven by the availability
% of the \dword{apa}s that take longer to build compared to the
% \dword{ce} components. The personnel requirements will be
% similar to the ones for the first detector.

The schedule of activities at \dword{surf} is designed so all 
\dword{apa}s can be installed in the cryostat on a timescale of seven
months, proceeding at a rate of one row of six \dword{apa}s per week and
allowing for a ramp-up period at the beginning of the process. This
requires that personnel from the \dword{ce} consortium be available
for two shifts per day at \dword{surf} at all times, including weekends. A
total of 15 FTEs/week will be needed to install and test the \dword{apa}s. An additional 3 FTEs/week are
required to install the other detector components on
top of the cryostat, which must proceed in parallel, but slightly
ahead of the schedule for integrating and installing the
\dword{apa}s. 

The installation of the cryostat penetrations, which must be done before all these activities, requires 4 FTEs working full time
for one month. Most of these FTEs should be scientific personnel from
the \dword{ce} consortium, and as with the \dword{itf}, we plan
to have one engineer and one technician available at \dword{surf} to support
integration and installation. To install
the cryostat penetrations, a team of 8 FTEs is required, split over
two shifts per day, for a period of one month. Installing
the cryostat penetration should take place at the latest while the 
detector support structure is installed, but it could be
scheduled much earlier, possibly starting as soon
as the welding of the cold membrane is completed in part of the cryostat.

% The activities at \surf for a second Single Phase TPC \dword{fd}
% will have a similar timeline and similar personnel requirements as
% for the first detector.

%%%%%%%%%%%%%%%%%%%%%%%%%%%%%%%%%%%
\subsection{Quality Control}
\label{sec:fdsp-tpcelec-integration-qc}

Many of the activities of the \dword{ce}
consortium at the \dword{itf} and at \dword{surf} have the goal of 
ensuring that the detector will be fully functional once the cryostat
is filled with \dword{lar}. All the detector components provided
by the \dword{ce} consortium that arrive at the \dword{itf}
and at \dword{surf} have gone through a qualification process to ensure
that they are fully functional and that they meet the \dword{dune} 
specifications. Additional tests and checks are performed at the
\dword{itf} and at \dword{surf} to ensure that the components have not
been damaged during the transport or during the installation itself,
and most importantly that all the parts are properly connected.

\dwords{femb} are tested multiple times during this process. 
The \dwords{femb} first undergo a test after they are received and then a test after 
installation on the \dword{apa}s at the \dword{itf}. Further 
tests are performed at \dword{surf} before and after the 
\dword{apa}s are installed in the cryostat, using the final cables to connect the \dwords{femb} and the detector flanges. 
Results of these tests at
the \dwords{itf} and \dword{surf} are compared with the results of the
tests performed during the qualification of \dwords{asic} and
\dwords{femb} to detect possible deviations that could signal 
damage in the boards or problems in the connections. All test 
results will be stored in the same database system used for
results obtained during the qualification of components.

Both the reception test at the \dword{itf} and the test after 
installation on the \dword{apa}s are performed at room temperature 
by connecting up to four \dwords{femb} to a \dword{wib} that is 
connected directly to a laptop computer for readout over 1~Gbps
ethernet, with power provided by a portable 12~V supply. For 
the reception test, the \dwords{femb} are attached to a capacitive 
load to simulate the presence of wires, which allows connectivity 
tests, measurement of the baseline, and \rms of the noise for 
each channel. Dead channels are identified using the calibration 
pulse internal to the \dword{fe} \dword{asic} as well as the measured
noise level relative to that associated with the temporary capacitive load.
Overall, the reception test and the test performed after attaching the
\dwords{femb} to the \dword{apa}s each require approximately half an hour per
motherboard, including  the time for connecting and disconnecting test cables.
The \dword{ce} consortium plans to have a \dword{cts} available
at the \dword{itf} to perform checks at \lntwo temperature
of \dwords{femb} that fail the quality control procedures at \dword{surf},
and eventually for sample checks on the \dwords{femb} as they are received
at \dword{itf}. Having a cold box at the \dword{itf}
will allow a fraction of the \dword{apa}s to be tested at
\lntwo temperature after they are integrated with the \dword{pds}
and \dwords{femb} before the test at 
\dword{surf}. This additional cold test on the surface can
detect possible problems with the integration of \dword{apa}s, \dwords{femb},
and \dword{pds} components before cold
box tests in the \dword{surf} cavern.

At \dword{surf}, an initial test of the readout is performed at room temperature
once the pair of \dword{apa}s is in the cold box, to ensure the final cables are
properly connected to the FEMBs. This test is made using elements of the final DAQ system.
Fast Fourier transforms of
the noise measurements made in the closed cold box will be inspected for indications
of coherent noise. All \dword{fe} gain and shaping time settings will be exercised,
and the gain will be measured using the integrated pulser circuit in the \dword{fe}
\dword{asic} and/or the \dword{wib}. The connectivity and noise measurements, as well
as the check for dead channels, are repeated later after the \dword{apa}s pair cools
down to a temperature close to that of \lntwo in the cold box. The bias voltage
connections and the \dword{pds} are also checked at this time.

Results of all these tests will be compared with results obtained 
in earlier \dword{qc} tests.  If problems are found, it will be possible 
to fix them by re-seating cables or replacing individual \dwords{femb}.
Noise levels are also monitored during the cool-down and warm-up 
operations of the cold boxes. These tests also ensure that the power,
control, and readout cables are properly connected
on the \dword{femb} side and that this connection will withstand temperature 
cycles. This addresses one problem observed during integration,
installation, and commissioning of the \dword{protodune}\dword{sp} 
detector. While the connection between the cables and the \dword{femb}
has been redesigned to minimize problems seen in \dword{protodune},
repeating these tests during integration
and installation of the detector is important because a single connection problem would
result in the loss of one entire \dword{femb}. In addition, the tests 
performed in the cold box at \dword{surf} demonstrate that the power, control, and
readout cables for the bottom \dword{apa}s are not damaged when they are routed 
through the \dword{apa} frames. Additional measurements of the noise
level inside the cryostat will be performed regularly by closing 
the \dword{tco} temporarily with an RF shield electrically connected 
to the cryostat steel. 

All readout tests are repeated after the \dword{apa}s are put
in their final position inside the cryostat and after the power, control, and
readout cables are connected to the warm flange attached to the cryostat
penetration. At this point, the connection between the cables and the flange
is validated, and the entire power, control, and readout chain, including the
final \dword{daq} back-end used during normal operations, are exercised. The
installation plan for the detector components inside the cryostat (\dword{apa}s,
\dwords{cpa}, and deployment of the \dwords{fc}) allow minor repairs on some \dwords{femb} without extracting
the \dword{apa}s from the cryostat. The testing of all detector components
will continue throughout the installation of all the elements of the \dword{tpc}, 
until the cryostat is ready to be filled with \dword{lar}.  When the 
\dword{apa}s are in their final position, replacing \dwords{femb} 
or cold cables will be more difficult and may require extracting the \dword{apa}s 
from the cryostat. This operation will be performed only if major problems occur with the \dwords{femb}.

In addition to measurements that show the \dword{apa}
readout is working as planned, the \dword{ce} consortium will also
test the bias voltage system together with the \dword{apa}
and \dword{cpa} consortia. These tests should show that
the cables that provide the bias voltage to the \dword{apa} wires, the
\dword{fc} termination electrodes, and the electron diverters are connected
properly with no short circuits. %For safety reasons, these
%tests will be performed with limited voltage (50--100 V) while the access
%to the inside of the cryostat (or to the area close to the \dword{apa} being
%tested) is limited.
These tests will be performed as soon as an \dword{apa}
is in its final position after connecting the bias voltage cables to the 
\dword{shv} boards on the \dword{apa}, using a resistive load for the 
\dword{fc} termination electrodes and the electron diverters. This ensures
the continuity of the bias voltage distribution system from the bias voltage
supplies to the \dword{apa}s. The test must be repeated for the \dword{fc}
termination electrodes and the electron diverters after 
the \dword{fc} are deployed.

Additional tests will be performed on the other detector components
provided by the \dword{ce} consortium before they are inserted into the
\dword{apa}s. After the cryostat penetrations are put in place, leak checks
will be performed by spraying helium inside the cryostat penetration using 
a leak detector outside the detector. These tests will be repeated 
after all cables have been routed through the cryostat penetration.
As soon as the bias voltage and the power supplies are installed on the detector
mezzanine and cables are put in place between the corresponding racks and
cryostat penetrations, tests will be performed to ensure that the proper 
power and bias voltage can be delivered to the \dwords{wiec} before these components
are installed. Even before connecting the \dwords{wiec} to the warm flanges,
tests will be performed to ensure that they can be properly powered up, controlled,
and readout by the \dword{daq} back-end. Tests will be performed on the
readout fiber plant to ensure that all fiber connections are functional
and properly mapped. Additional tests will be performed on the slow control
system and on the detector safety system several times during the
installation of the detector. These tests will take place before the
corresponding \dword{apa}s are installed, after their installation, after
all the corresponding cables and fibers are connected, and finally
during all the integrated tests that take place before
the \dword{tco} is closed and the cryostat is filled. Negative results in any
of these tests will halt integration, installation, and
commissioning activities. The results will then be used in reviews that must
take place before the closure of the \dword{tco} is authorized, the \dword{lar} filling operation begins, and the detector is
commissioned after the cryostat has been filled.

%%%%%%%%%%%%%%%%%%%%%%%%%%%%%%%%%%%
\subsection{Internal Calibration and Initial Commissioning}
\label{sec:fdsp-tpcelec-integration-calib}

While the cryostat is being closed (and any time there is welding 
on the cryostat), the electronics should be turned off and all 
cables between the detector racks, including the low voltage
and bias voltage, fans, and heater power should be disconnected 
from the \dwords{wiec}. Once the cryostat is closed, the baseline 
and noise of all channels should be measured. Dead electronics channels 
should be identified by measuring the response of all channels to 
the internal electronics calibration pulser at a nominal setting, 
such as $\pm\SI{600}{mV}$, which distinguishes between induction 
and collection channels. The noise should be measured with the wire bias 
voltages fully enabled on the G, U, and X planes of the \dword{apa}s. 
It should also be measured with the cathode high voltage on at a very 
low value, e.g. \SI{50}{V}. The non-responsive channels, identified
as having very low noise, and the channels that have noise that 
significantly exceeds the average should be flagged and recorded.
Sources of noise that exceeds the expectations should be identified
and, if possible, fixed. Any warm electronics components with
issues should be replaced with spares.

% While the cryostat is being filled with gaseous argon, the electronics 
% should be powered off. 
Once the cryostat is filled with gaseous
argon, the baseline and noise of all the channels will be measured
again, and any new non-responsive channels in the electronics 
should be identified by injecting $\pm\SI{600}{mV}$ with the 
internal calibration pulser. As the cryostat is cooled down, the 
temperature at the electronics and the noise of all channels should 
be monitored periodically. Any new non-responsive channels should 
be flagged and excess noise sources that are exposed as the 
electronics cool should be identified, and if possible, fixed.

Once the electronics is fully submerged in \dword{lar}, a full 
set of electronics diagnostic tests should be run, including: 
baseline and noise measurement, and a full gain calibration on 
all channels with the internal calibration pulser at settings 
up to the saturation of the \dword{fe} inputs. The shaping time 
should be measured on all channels by injecting the $\pm\SI{600}{mV}$
internal pulser at each of the four settings and fitting the 
pulse shape. Any new non-responsive channels during the pulser 
runs should be flagged. Any new disconnected channels should be 
flagged and excess noise sources should be identified. These tests 
can be performed on the electronics installed on the bottom
\dword{apa}s even while the corresponding wires are in 
the gaseous argon.
