%%%%%%%%%%%%%%%%%%%%%%%%%%%%%%%%%%%
\subsection{Remaining design and prototyping tasks}
\label{sec:fdsp-tpcelec-overview-remaining}

The two \dword{protodune} detectors have been built with multiple
goals, one of which was the demonstration that the specifications
for DUNE could be met with a design that would only require a simple
scale up of the detector size. The data collected with \dword{pdsp}
in Fall 2018 has demonstrated that noise levels well below the target
of one thousand electrons can be achieved in \lar, validating the
detector system design approach planned for DUNE. Furthermore, the amount
of coherent noise observed in \dword{pdsp} in untreated raw data
after a few days of operation is at the level of what was observed
in \dword{microboone} after a significant level of analysis and
filtering. The residual coherent noise on the \dwords{femb} is in
a frequency range where it can be more easily filtered out compared 
to previous \lar detectors.

Despite the success of \dword{pdsp} there are multiple areas where
additional design and prototyping work is required before the
beginning of the DUNE detector construction. These design changes
can be ordered depending on the risk and the amount of engineering
that is required. At one end of the spectrum is the change of the
design of the cryostat penetration, that in \dword{pdsp} housed one
\dword{ce} and one \dword{pds} flange in a tee-shape. For DUNE
there will be two \dword{ce} flanges in addition to the \dword{pds}
flange, arranged in the shape of a cross. The increase of the number
of flanges may require some structural reinforcement and additional
finite element analysis simulations to estimate the proper flow of
Ar to avoid any back-diffusion of oxygen into the cryostat in case
of leaks on the flanges and to ensure that the temperature gradient
in the Argon is acceptable. In cases like this it can be easily
argued that the amount of engineering required to finalize the
detector design for DUNE is minimal compared to the work already
done for past \lar detectors, starting with \microboone and
other prototypes, and finishing with \dword{pdsp}. At the opposite
side of the spectrum is the \dword{asic}s' design, with the
development of \dword{coldata} and of a completely new \dword{adc}
that is being designed to address the shortcomings of the BNL-designed
P1-\dword{adc}, non-linearities and stuck bits. In Table~\ref{tab:SPCE:designstatus}
we describe our estimate of the amount of work required to complete
the design and prototyping of the \dword{ce} detector components for
DUNE. Later in this Section we analyze the risk that the design
and prototyping of these components could affect the DUNE detector
construction schedule.

\begin{dunetable}
[Status of the design of the different CE detector components]
{p{0.30\textwidth}p{0.15\textwidth}p{0.40\textwidth}}
{tab:SPCE:disorientates}
{Status of the design of the different CE detector components and expected
amount of engineering and prototyping required prior to construction.}
Component & Status & Expected work \\ \toprowrule
\dword{larasic} & Advanced & Fix minor issues in the design, port differential output from \dword{coldadc} design \\ \colhline
Commercial ADC & Complete & None \\ \colhline
\dword{coldadc} & \multicolumn{2}{l}{See text for details} \\ \colhline
\dword{coldata} & \multicolumn{2}{l}{See text for details} \\ \colhline
\dword{cryo} & \multicolumn{2}{l}{See text for details} \\ \colhline
\dword{femb} & Advanced & Experience with multiple prototypes, final design will follow the \dword{asic} selection \\ \colhline
Cold cables & Very advanced & Minor modifications, additional vendor qualification \\ \colhline
Cryostat penetrations & Advanced & Add \dword{ce} flange for bottom \dword{apa} \\ \colhline
\dword{wiec} & Very advanced & Add HEPA filter and hardware interlock system \\ \colhline
\dword{wib} & Advanced & Update design to use cheaper FPGA, modify \dword{femb} power \\ \colhline
\dword{ptc} & Very advanced & Add interface to interlock system \\ \colhline
Power supplies & Very advanced & Investigate possible additional vendors, rack arrangement \\ \colhline
Warm cables & Very advanced & Finalize cable layout, identify vendors \\ \colhline
Readout and control fiber plant & Very advanced & Finalize plant layout \\ \colhline
\end{dunetable}

The area that requires most work is that of the \dword{asic}s that are mounted
on the \dwords{femb}. The FE \dword{asic} has already gone through eight design
iterations, the last three directly targeted for DUNE, and has already been used, 
in one of its versions, for \dword{microboone} and for \dword{pdsp} reaching the 
noise levels specified for DUNE. At least one additional design iteration is
required to address the issues observed during the \dword{pdsp} operations and
to implement a single ended to differential converter to improve the interface
with the newly developed \dword{coldadc}. To ensure the success of the next 
design iteration we are investing in the development of appropriate transistor
models for operation in \lar, such that the saturation effect observed in 
\dword{pdsp} can be properly addressed first in simulation and then with improvements 
in design. It should be noted that so far approximate models, originally developed for 
the same \SI{180}{nm} technology, but with different design rules, have been used for 
the \dword{larasic} development, and therefore it should not be a surprise
that the FE \dword{asic} may have limitations in some corner of the phase space.
The circuitry for the single ended to differential converter has already been 
developed in the \SI{65}{nm} technology and needs to be ported to the \SI{180}{nm}
technology used for \dword{larasic}. We consider that appropriate measures have been
put in place to minimize the risk associated with the need of a further
prototyping iteration. Nevertheless, in Section~\ref{sec:fdsp-tpcelec-risks-design}
we consider a generic risk for a delay in the availability of \dword{asic}s and
argue that this delay would not have an impact on the beginning of DUNE operations.

The plan for DUNE foresees the usage of two additional custom design \dword{asic}s
for the ADC and for the data serialization, and the possibility of using a single 
\dword{asic} that combines these functionalities with that of the FE. It should
be noticed that a solution based on the use of a commercial off-the-shelf ADC 
and of an FPGA for the data serialization has been demonstrated to work by the 
SBND Collaboration. This solution at this point is considered as a fall-back
solution for DUNE. Further verification work would be required to certify that
the transceivers in the FPGA will meet the requirements of DUNE for reliability
in \lar. The custom solutions for the \dword{asic}s are being developed to
simplify the \dword{femb} assembly and reduce the power dissipated by the
electronics in the \lar, which has consequences for the cryogenics and for
the cross section of the cold cables and for the feedthroughs.

The preferred solution for the ADC and the data serialization is based on two new 
\dword{asic}s, \dword{coldadc} and \dword{coldata}.
The first iteration of the \dword{coldadc} has been submitted for fabrication
at the end of October 2018, with the delivery of chips expected for January 2019.
Test boards are being developed and results from the initial validation of the
chip functionality are expected for Spring 2019. Even if this is the first version
of the chip, we estimate that the design is already close to an advanced stage
for the following reasons:
\begin{itemize}
\item{An ADC with the same architecture has already been implemented by the
lead engineer of the BNL-FNAL-LBNL team that has designed the new \dword{coldadc}.
The new design is required to ensure operation in \lar, and the appropriate
models for the \SI{65}{nm} technology have been used, as discussed in
Section~\ref{fdsp-tpcelec-design-femb-adc}.}
\item{The interfaces between the core of the ADC and \dword{larasic} on 
the input side and \dword{coldata} on the output side are entirely new.
These parts of the design have gone through an extensive verification
program, including SPICE (Spectre) simulations of analog blocks,
mixed-mode (Verilog AMS) simulations of the core ADC, and
UVM (SystemVerilog) verification of all digital logic. In addition, there are multiple 
input options that would allow the validation of the design of the core 
of the ADC, in the case that one of them does not function as designed.}
\item{For some blocks of the ADC (input buffers, voltage references) the
design includes multiple redundant options that can be selected during 
operation.}
\end{itemize}
If the tests of the new \dword{coldadc} in Spring 2019 are successful, the
estimate of the status of the design should be changed at least to the 
``Advanced'' status. One additional design iteration is planned before
the production, and as in the case of \dword{larasic}, plans can be made for a
further iteration without impacting the beginning of DUNE operations.

We are also considering an alternative solution for the readout, where
the three \dword{asic}s are replaced with a single one, the \dword{cryo}
chip that is going through a development that is proceeding with a 
timeline similar to that of \dword{coldadc}. Also in this case the
chips from the first submission at expected to be delivered in the
second half of January 2019. They will then undergo standalone tests
and later system tests on a timescale similar to that of \dword{coldadc}.
The \dword{cryo} chip is more complicated than \dword{coldadc},
since it implements the functionality of three \dword{asic}s into a
single one, and therefore its current design maturity status is
lower than that of \dword{coldadc}. Some of the design blocks of
\dword{cryo} are at a more advanced level: for example the control
and data transmission blocks have been reused from other \dword{asic}
designs from the SLAC group, and the implementation of the FE amplifier
is very similar to that of \dword{larasic}, except for the signal 
shaping. Depending on the results of the tests the status of the
\dword{cryo} design will evolve in a way similar to that of \dword{coldadc}.

The first complete prototype of \dword{coldata} will be submitted in 
February/March 2019, with the delivery of chips expected for June 2019.
The design of this first complete prototype builds on the success of
the first partial prototype (CDP1) that was fabricated and tested in 2018.
This first prototype included the control registers inside the 
\dword{asic}, the I2C interface required to program and read-out
the status of these registers, the SPI interface required to
configure \dword{larasic}, the phase-lock loop required to generate
clocks internal to the chip, the data serializer, and a current-mode
line driver. The transition to the first complete prototype that
will be submitted in February/March 2019 and tested in Summer 2019
requires the addition of two interfaces to the \dword{coldadc} (UART
and I2C), an additional I2C interface to allow programming two
\dword{coldata} \dword{asic}s on the same \dword{femb}, and the
addition of a line-driver with pre-emphasis required for operation
with long cables (up to \SI{25}{m}) in \lar. The same design and
validation methodology used for the CDP1 prototype and the \dword{coldadc}
is being used. This should guarantee that after the tests done in
Summer 2019 \dword{coldata} will reach the ``Advanced'' design 
status. 

There have already been multiple iterations of \dwords{femb} that
have been fabricated and tested and used for data taking in 
\dword{microboone} and in \dword{pdsp}. SBND is starting the
production of \dwords{femb} based on the commercial ADC and
FPGA solution. The design of the \dword{femb} needs to be adapted
for the different \dword{asic} solutions that are being considered
for DUNE. This development is already ongoing, as system tests 
where the \dwords{femb} are connected to an \dword{apa} are part
of the qualification tests. The design status for the \dword{femb}
is already at the ``Advanced'' level, and it will reach the 
``Very advanced'' level at the time of the \dword{asic}
selection. At that point only minor modifications may be
required. 

The only other \dword{ce} detector components that do not yet
reach the ``Very advanced'' level are the cryostat penetrations, as
discussed above, and the \dword{wib}, where small
design changes will be done prior to production to use a more
modern and cheaper FPGA. Additional changes will be required to
the power distribution scheme, since the number of power lines
and the corresponding voltages will be reduced compared to
\dword{pdsp}. The transition to a more modern FPGA will allow 
more extensive data monitoring inside the \dword{wib}, but may
also require developing new software and porting the firmware
from one family of FPGAs to another. 

For all other detector components the estimate of the design
maturity is considered ``Very advanced'' based on the experience
gained with commissioning and operation of \dword{pdsp}. The 
cold signal cables will be modified to reduce the number of
connections and to address the issues observed with the connector
on the \dword{femb}. The design of the \dword{wiec} needs to
be modified to include HEPA filters to minimize the possible
damage from dust during the lifetime of the experiment at SURF.
The \dword{ptc} is going to be modified to add an interface to
the hardware interlocks of the detector safety system. For
cables and fibers on the top of the cryostat the only work that
remains to be done is the design of the actual cable plant, 
which will then fix the length of the cables. The arrangement
of power supplies in the racks on top of the cryostat is the
only other remaining design task. For many components the
qualification of additional vendors could also be considered
as part of value engineering, to reduce the risks of vendor
lock-in and to minimize costs. 


