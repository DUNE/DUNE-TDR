\section{Risks}
\label{sec:fdsp-tpcelec-risks}

In this section, we discuss the risks that could be encountered during design
and construction of the \dword{dune} \dword{sp} detector, as well as the
risks that can be encountered later during commissioning and operation.
For every risk, we will describe the mitigating actions 
being put in place even now at the design stage of the experiment, and the 
possible responses that we will take should a specific risk be realized. 
Table~\ref{tab:risks:SP-FD-TPCELEC} contains a list of all risks that we are 
considering. For each risk we assess a probability for a risk to be
realized (P), and  cost (C) and schedule (S) impacts, after the 
mitigation activities discussed in the text are put in place. It should
be noted that in the case of poor lifetime of the %\dword{tpc} electronics detector
components installed inside the cryostat, there is no cost or schedule
impact, as they cannot be accessed and replaced.
All these risks are discussed in detail in the remainder of this
section.


% risk table values for subsystem SP-FD-TPCELEC
\begin{footnotesize}
%\begin{longtable}{p{0.18\textwidth}p{0.20\textwidth}p{0.32\textwidth}p{0.02\textwidth}p{0.02\textwidth}p{0.02\textwidth}}
\begin{longtable}{P{0.18\textwidth}P{0.20\textwidth}P{0.32\textwidth}P{0.02\textwidth}P{0.02\textwidth}P{0.02\textwidth}} 
\caption[Risks for SP-FD-TPCELEC]{Risks for SP-FD-TPCELEC (P=probability, C=cost, S=schedule) More information at \dword{riskprob}. \fixmehl{ref \texttt{tab:risks:SP-FD-TPCELEC}}} \\
\rowcolor{dunesky}
ID & Risk & Mitigation & P & C & S  \\  \colhline
RT-SP-TPC-001 & Cold ASIC(s) not meeting specifications & Multiple designs, use of appropriate design rules for operation in LAr & H  & M & L \\  \colhline
RT-SP-TPC-002 & Delay in the availability of ASICs and FEMBs & Increase pool of spares for long lead items, multiple QC sites for ASICs, appropriate measures against ESD, monitoring of yields & M & L & L \\  \colhline
RT-SP-TPC-003 & Damage to the FEMBs / cold cables during or after integration with the APAs & Redesign of the FEMB/cable connection, use of CE boxes, ESD protections, early integration tests & M & L & L \\  \colhline
RT-SP-TPC-004 & Cold cables cannot be run through the APAs frames & Redesign of APA frames, integration tests at Ash River and at CERN, further reduction of cable plant & L & L & L \\  \colhline
RT-SP-TPC-005 & Delay and/or damage to the TPC electronics components on the top of the cryostat & Sufficient spares, early production and installation, ESD protection measures & L & L & L \\  \colhline
RT-SP-TPC-006 & Interfaces between TPC electronics and other consortia not adequately defined & Early integration tests, second run of ProtoDUNE-SP with pre-production components & M & L & L \\  \colhline
RT-SP-TPC-007 & Insufficient number of spares & Early start of production, close monitoring of usage of components, larger stocks of components with long lead times & M & L & L \\  \colhline
RT-SP-TPC-008 & Loss of key personnel & Distributed development of ASICs, increase involved of university groups, training of younger personnel & H & L & M \\  \colhline
RT-SP-TPC-009 & Excessive noise observed during detector commissioning & Enforce grounding rules, early integration tests, second run of ProtoDUNE-SP with pre-production components, cold box testing at SURF & L & L & M \\  \colhline
RT-SP-TPC-010 & Lifetime of components in the LAr & Design rules for cryogenic operation of ASICs, measurement of lifetime of components, reliability studies & L & n/a & n/a \\  \colhline
RT-SP-TPC-011 & Lifetime of components on the top of the cryostat & Use of filters on power supplies, stockpiling of components that may become obsolete, design rules to minimize parts that need to be redesigned / refabricated & L & M & L \\  \colhline

\label{tab:risks:SP-FD-TPCELEC}
\end{longtable}
\end{footnotesize}

%%%%%%%%%%%%%%%%%%%%%%%%%%%%%%%%%%%
\subsection{Design and Construction Risks}
\label{sec:fdsp-tpcelec-risks-design}

Despite the successful operation of \dword{pdsp}, we cannot build and operate
the \dword{dune} \dword{sp} detector using the same \dwords{asic}, so we have
undertaken to improve the design of the \dword{fe} amplifier, redesign
the \dword{adc}, and replace the \dword{fpga} on the \dword{femb} with a dedicated data
serialization and transmission \dword{asic} (\dword{coldata}). The project schedule has
sufficient time for a second submission of all these \dwords{asic} in the current 
development cycle. We  nevertheless must consider the risk (RT-SP-TPC-001
in Table~\ref{tab:risks:SP-FD-TPCELEC}) that, even after the second iteration, 
we may not have a set of \dwords{asic} that meets all the \dword{dune} requirements
(including the lifetime requirements). 

To reduce this possibility, we are pursuing the development of the \dword{cryo} \dword{asic}, and we
will also perform system tests of the \dword{cots} solution for the \dword{adc} that is planned 
for use in the \dword{sbnd} experiment. Should this risk become reality, an
additional development cycle would be required, which would delay the availability
of \dwords{femb} for approximately 12 months. Based on the current schedule, this would
not be a problem for the detector integration and installation.
More importantly, such delay would require
that, during the first part of the \dword{apa} production, a sufficient number
of \dwords{femb} with non-final electronics be available for integration tests
on the \dword{apa}s. These boards would then have to be replaced later
by the final boards.

The second risk (RT-SP-TPC-002) is a general delay in the availability of the \dwords{asic}
and/or \dwords{femb}, which would then not be available for integration
on the \dword{apa}s. This risk has several possible triggers: a lower fabrication yield 
than expected for \dwords{asic} and \dwords{femb};
significant downtime at one of the \dword{qc} sites; or losses during 
handling and transport, in addition to the issues with the design already 
covered in the previous risk. Our planning includes several ways to mitigate the risk. By
procuring spares for the \dwords{asic} and the discrete components to be mounted on the
\dwords{femb}, we can continue integration with the \dword{apa}s;
we are also splitting the \dword{qc} process among various sites. We will also emphasize,
as discussed in Section~\ref{sec:fdsp-tpcelec-safety-detcon}, the use of appropriate
measures that minimize any \dword{esd} damage. We will then monitor the use
of spare components and, if needed, fabricate additional parts. Appropriate
monitoring of the production yield and spares should minimize
delays in \dword{apa} integration. As in the case of the first risk, should this
second risk become reality, we do not expect to have delays in the installation of the 
detector inside the cryostat. 
 
The third risk (RT-SP-TPC-003) is the possibility of damage to the 
\dwords{femb} during or after integration with the \dword{apa}s. This could
happen while the \dwords{femb} are being installed or, more likely, during the
installation of the cold cables, something that has already taken place during
the installation of \dword{pdsp}. This damage could also happen when the \dword{apa}s are
moved into the cryostat or during the final cabling of the detector. The damage
to the \dwords{femb} and cold cables could be either mechanical damage during the
handling or \dword{esd} damage. We are redesigning the connection between the \dwords{femb}
and the cold cables to minimize the first possibility and putting procedures in place
to minimize the second. The risk of mechanical damage
is further reduced by housing the \dwords{femb} inside the \dword{ce} boxes and by having 
appropriate strain relief on the cold cables. To minimize the possibility of
\dword{esd} damage, we will follow all appropriate procedures, and in addition,
we will use plugs on all the cold cables while they are routed through the
\dword{apa} frames and cryostat penetrations, to avoid injecting charge on
the \dwords{femb} that could cause \dword{esd} damage. Finally, we are planning for
extensive testing of the \dword{ce} several times during integration
and installation that would allow us to replace the \dwords{femb}
or the cold cables if necessary. This includes significant time for testing the entire readout 
chain after the \dword{apa}s are placed in their final position inside the cryostat,
when repairs are still possible.

We still consider risk (RT-SP-TPC-004) a possibility: that the cold cables cannot
be routed through the frames of the \dword{apa}s. In that case, the cables for the
\dwords{femb} attached to the bottom \dword{apa} would have to be routed along the
walls of the cryostat, requiring a significant redesign of the entire detector. 
To minimize this risk, we have significantly redesigned the \dword{apa} frame 
to use larger tubes, and many studies have been performed in recent months.
These studies are based on the assumption that there can be a small reduction in
the cable plant size compared to \dword{pdsp}. The probability of this risk
being realized has been significantly reduced following the cable insertion
tests performed at Ash River using a stacked pair of \dword{apa}s, discussed
in Sections~\ref{sec:fdsp-apa-qa-prototyping} and~\ref{sec:fdsp-tpcelec-design-ft}.
This risk has not yet been retired, since the expected reduction of the cable plant 
needs to be demonstrated with the design and test of new \dwords{femb}. Should
this risk be realized, because we do not achieve the expected reduction of the 
cable plant of each \dword{femb}, we will consider further reductions of the cable 
plant. For example, the control signals could be shared between multiple 
\dwords{femb}, instead of the considering the option of routing the 
cables along the walls of the cryostat.

The next risk (RT-SP-TPC-005) involves delays in the availability of or
damage to the \dword{tpc} electronics components installed on top of the
cryostat. As discussed in Section~\ref{sec:fdsp-tpcelec-integration-timeline},
we plan to have all \dword{tpc} electronics detector components on top of the cryostat, those 
required to power, control, and readout one pair of \dword{apa}s installed
and available before inserting the \dword{apa}s into the cryostat.
This allows extensive testing of \dword{apa}s to mitigate
the risk of damage to the readout chain. To mitigate the risk associated with
delays in installing the \dword{tpc} electronics components on top of the cryostat,
we plan to have sufficient spares and to use appropriate \dword{esd} 
prevention measures. If only a subset of all components is available, cables 
and fibers on the top of the cryostat would have to be re-routed to allow  
integrating and installing the \dword{apa}s to continue without delays, 
and tests will have to be repeated when all the components become available 
and are installed. The worst possible consequence is a delay in closing 
the cryostat and beginning operation. 

Another risk (RT-SP-TPC-006) is that incompatibilities 
between various components of the \dword{dune} \dword{fd} go undetected until
these components are integrated during prototyping, during integration at 
the \dword{surf}, or during installation. These incompatibilities could 
result in reworking or redesigning some of the components and therefore in
delays of the project. This risk clearly
diminishes as long as integration tests, including mock-ups and prototypes,
come early in design and construction. The schedule for the 
design and construction of the \dword{tpc} electronics detector components foresees many
integration tests to reduce this risk as much as possible. These tests include 
integration tests with components provided by the \dword{apa}, \dword{pds}, 
and \dword{daq} consortia, as well as integration tests of cable routing in 
the cable trays and through the cryostat penetrations. The second run of 
\dword{pdsp}, using pre-production detector components, will further help
mitigating this risk. At that point any design change or any deviation from
established procedures will need to go through very extensive vetting to
avoid the introduction of new incompatibilities between detector components.

Another issue that can arise during the detector integration and installation
is a delay caused by the excessive usage of spare detector parts (RT-SP-TPC-007).
For \dwords{asic} and \dwords{femb} this kind of risk has already been considered
(RT-SP-TPC-002). For the other \dword{tpc} electronics detector components the
risk should be considered separately, since the other components are needed 
at an earlier time relative and have completely different fabrication and
testing schedules. Some of the actions required to mitigate this risk are
similar, like early start of the production, careful monitoring of yields
during the \dword{qc} process, larger number of spares in the case of
long lead items. 

The final risk we consider for the construction of the \dword{tpc} electronics
detector components is the loss of key personnel (RT-SP-TPC-008). The number of
scientists and engineers that have become involved with the \dword{tpc}
electronics has significantly increased since the construction of \dword{pdsp},
and in some sense this has already contributed to reducing significantly the
probability and possible impacts of this risk. In some areas, like \dword{asic}
design, the addition of large teams of engineers involved in the design of
the new \dwords{asic} means that the probability of this risk is now negligible.
There are areas where the experience from the construction and operation
of \dword{pdsp} resides with a few expert scientists and engineers, and areas
where only single engineer is responsible for the design of a set of detector
components. The mitigation of this risk involves enlarging the team(s) that
are responsible for the design and prototyping of the detector components.
This has already been done for the \dwords{asic} and plans are already 
in place to involve university groups in the design of the \dwords{femb}
and of the \dword{wiec}. Succession plans for the consortium leadership
need to be put in place, by training younger personnel. 

%%%%%%%%%%%%%%%%%%%%%%%%%%%%%%%%%%%
\subsection{Risks during Commissioning}
\label{sec:fdsp-tpcelec-risks-commissioning}

The biggest risk during the commissioning phase
(RT-SP-TPC009 in Table~\ref{tab:risks:SP-FD-TPCELEC}) is excessive noise 
caused by some detector component not respecting the \dword{dune} grounding
rules. This risk was realized at least twice during the 
integration and commissioning of the \dword{pdsp} detector. During the 
integration of the first \dword{apa}, a source of noise was discovered 
in the electronics used for the readout of the photon  detector, which 
required a simple fix on all the boards. Later a large noise source was 
discovered in the temperature monitors. The overall noise in the \dword{pdsp} 
was reduced compared to previous \dword{lar} experiments or prototypes, 
like $\mu$Boone or the 35t prototype. Even if some unresolved source of noise 
is still apparent in the \dword{pdsp} data, this should not preclude using the
data collected for calibration and for physics analyses. Further studies are
planned for 2019 and 2020 to investigate the remaining sources of noise.

The main problem in going from the \dword{pdsp} to \dword{dune} is one of scale.
Even if the detector design addresses all possible noise sources, the simple
fact that the detector is 25 times larger and has a correspondingly larger
number of cryostat penetrations requires much more attention to detail
during installation and commissioning. Observations of excessive 
noise in \dword{dune} would result in a delay in commissioning and  
data taking until the source of the noise is found and mitigated. To minimize excessive 
electronic noise, we plan to enforce the grounding  rules throughout the 
design phase, based on the lessons learned from the operation of the 
\dword{pdsp} detector. We also plan to perform  integrated tests to discover 
possible problems as early as possible. This includes system tests at the \dword{iceberg} test stand at \dword{fnal} for each generation 
of the \dwords{femb} and photon detectors. We plan to perform noise 
measurements in the cold boxes at \dword{surf}, and later 
during the insertion of the \dword{apa}s inside the cryostat before the 
\dword{tco} closure, and repeat the measurements before the \dword{lar} filling. 
We expect that the extensive testing will allow a quick transition to 
detector operations, first with cosmics and later with beam, as soon as the 
cryostat has been completely filled. 

%%%%%%%%%%%%%%%%%%%%%%%%%%%%%%%%%%%
\subsection{Risks during Operation}
\label{sec:fdsp-tpcelec-risks-operations}

The expectation for the \dword{dune} detector is that data taking will continue
for at least two decades. Assuming that after commissioning, the
detector operates as designed, two risks must be considered.
These are related to the lifetime of the \dword{ce} components
installed inside (RT-SP-TPC-010 in Table~\ref{tab:risks:SP-FD-TPCELEC}) and on top (RT-SP-TPC-011)
of the cryostat. The components inside the cryostat are not replaceable,
and therefore any malfunction of a detector component will result in
a loss of sensitive volume. The components on top of the cryostat
(with the exception of the flange at the transition from the cold to
the warm volume) can be replaced, and as long as we have sufficient
spares this will not result in any loss beyond the amount 
of time required for replacing the component. The risk of losing components installed inside the cryostat has been considered
from the earliest stage of the design of \dwords{asic}. As discussed
in Section~\ref{sec:fdsp-tpcelec-qa-reliability}, we have formed a 
reliability committee to ensure that all appropriate measures
are considered in the design and that all tests are performed
as part of our \dword{qa} process to ensure that the lifetime of components is
such that we will see only minimal losses of sensitivity in the 
detector during operations. 

As discussed in Section~\ref{sec:fdsp-tpcelec-safety-detops}, we
are taking measures (like adding air filters to the
\dwords{wiec} and bias voltage and low voltage power supplies) to
minimize damage from environmental conditions to the 
detector components on top of the cryostat. We have discussed in
Section~\ref{sec:fdsp-tpcelec-production-spares} our plan for 
spare detector components. We cannot exclude the possibility that we will not have enough
spares, which we plan to build during construction, for the lifetime of the experiment, and that during 
operations, we will need to fabricate new boards or procure new
supplies. One possible issue related to this is the 
continued availability of certain components, in particular
\dwords{fpga} and optical transmitters and receivers, which may 
become obsolete and no longer be available when we need to fabricate new
parts. While it will always be 
possible to design new boards using more modern components, we
wish to keep the maintenance costs for the detector to a minimum,
and this may involve following the technology evolution and stockpiling components that may become obsolete
and/or hard to procure. We are also considering placing the
\dwords{fpga} and the optical components on mezzanine cards
to minimize the redesign and procurement costs should these 
components become unavailable during the lifetime of the experiment.
