\section{Risks}
\label{sec:fdsp-tpcelec-risks}

%%%%%%%%%%%%%%%%%%%%%%%%%%%%%%%%%%%
\subsection{Design and Construction Risks}
\label{sec:fdsp-tpcelec-risks-design}

In this Section we discuss the risks that could be encountered during the design
and the construction of the \dword{dune} \dword{sp} detector. In the two following
sections we will then discuss the risks that could be met during commissioning and
operation. For every risk we will be describe the mitigating actions that are
being put in place already at the stage of designing the experiment, and the 
possible responses that we will take should a specific risk be realized. 
Table~\ref{tab:SPCE:risks} contains a list of all the risk that we are 
considering and that we are discussing in detail in the following.

\fixme{Ensure that the list of risks covers the risks that are derived
from not implementing the lessons learned from ProtoDUNE.}

\begin{dunetable}[Single Phase \dword{tpc} Cold Electronics Risks]
{p{0.15\textwidth}p{0.75\textwidth}}
{tab:SPCE:risks}
{Summary of Single Phase \dword{tpc} Cold Electronics Risks}
ID & Risk \\ \toprowrule
1 & Cold \dword{asic}(s) not meeting specifications \\ \colhline
2 & Delay in the availability of \dwords{asic} and \dwords{femb} \\ \colhline
3 & Damage to the \dwords{femb} / cold cables during or after integration with the \dword{apa}s \\ \colhline
4 & Cold cables cannot be routed through the \dword{apa}s' frames \\ \colhline
5 & Delay and or damage to the \dword{ce} components on the top of the cryostat \\ \colhline
6 & Interfaces between \dword{ce} and other consortia not adequately identified \\ \colhline
7 & Excessive noise observed during detector commissioning \\ \colhline
8 & Lifetime of components in the \dword{lar} \\ \colhline
9 & Lifetime of components on the top of the cryostat \\ \colhline
\end{dunetable}

Despite the successful operation of \dword{pdsp} we cannot build and operate
the \dword{dune} \dword{sp} detector using the same \dwords{asic} and we have
undertaken a project of improving the design of the \dword{fe} amplifier, redesigning
the \dword{adc}, and replacing the \dword{fpga} on the \dword{femb} with a dedicated data
serialization and transmission \dword{asic} (\dword{coldata}). The project schedule provides
sufficient time for a second submission of all these \dwords{asic} in addition
to the current development cycle. We have nevertheless consider the risk (ID 1
in Table~\ref{tab:SPCE:risks}) that even after the second iteration 
we may not have a set of \dwords{asic} that meets all the \dword{dune} requirements
(including the lifetime requirements). To reduce the probability of this 
risk we are pursuing also the development of the \dword{cryo} \dword{asic}, and we
will also perform system tests of the \dword{cots} solution for the \dword{adc} that is being
planned for use in the SBND experiment. Should this risk become realized an
additional development cycle would be required, which would delay the availability
of \dwords{femb} for about 12 months. Based on the current schedule this would
not be a problem for the detector installation, but would require extra 
person-power at the \dword{itf} to partially recuperate the delay in the
installation of the \dwords{femb} on the \dword{apa}s. It would also require
that during the first part of the \dword{apa} production a sufficient number
of \dwords{femb} with non-final electronics is available for integration tests
on the \dword{apa}s. These boards would then have to be replaced at a later
time using the final ones.

The second risk (ID 2) that we consider is a general delay in the availability of the \dwords{asic}
and/or \dwords{femb} which would then not be available for the integration
on the \dword{apa}s. There could be multiple triggers for this risk: a lower yield 
compared to our planning assumptions during the \dwords{asic} and \dwords{femb} 
fabrication or during the QC process, down-times at one of the QC sites, or losses during 
handling and transport, in addition to the issues with the design that are already 
covered by the previous risk. There are multiple risk mitigations that have been
put in place in our planning. This is done in part through the procurement of
spares for the \dwords{asic} and the discrete components to be mounted on the
\dwords{femb}, which would allow us to continue the integration with the \dword{apa}s,
and in part by splitting the QC process among various sites. We will also emphasize,
as discussed in Section~\ref{sec:fdsp-tpcelec-safety-detcon}, the use of appropriate
measure to minimize the occurrence of \dword{esd} damage. We will then monitor the usage
of spare components and if needed fabricate additional parts. With appropriate
monitoring of the production yield and of the spares we should be able to minimize
any delay in the \dword{apa}s' integration. As for the previous risk, we
do not expect to have delays in the installation of the detector inside the cryostat.
 
The third risk (ID 3) that we consider is the possibility of damage to the 
\dwords{femb} during of after the integration with the \dword{apa}s. This could
take place while the \dwords{femb} are being installed or most likely during the
installation of the cold cables (this is actually a risk that was realized during
the installation of \dword{pdsp}). It could also happen when the \dword{apa}s are
moved into the cryostat or during the final cabling of the detector. The damage
to the \dwords{femb} and cold cables could be of two types: mechanical damage during the
handling and \dword{esd} damage. We are redesigning the connection between the \dwords{femb}
and the cold cables to minimize the first risk and putting procedures in place
to minimize the second one. A further reduction of the risk of mechanical damage
is achieved by housing the \dwords{femb} inside the \dword{ce} boxes and by having 
appropriate strain reliefs on the cold cables. To minimize the probability of
\dword{esd} damage we will be following all the appropriate procedures and in addition
we will be using plugs on all the cold cables while they are being routed through the
\dword{apa}s' frames and the cryostat penetrations, to avoid injecting charge on
the \dwords{femb} that could cause \dword{esd} damage. Finally we are planning for
extensive testing of the Cold Electronics at multiple steps during the integration
and installation that would allow for the possibility of replacing the \dwords{femb}
or the cold cables. This includes also significant time for testing the entire readout 
chain after the \dword{apa}s are placed in their final position inside the cryostat,
when repairs are still possible.

We still consider a risk (ID 4) for the possibility that the cold cables cannot
be routed through the frames of the \dword{apa}s. In that case the cables for the
\dwords{femb} attached to the bottom \dword{apa} would have to be routed on the
walls of the cryostat, requiring a significant redesign of the entire detector. 
To minimize this risk there has been a significant redesign of the \dword{apa} frame, 
to use larger tubes, and many studies have been performed in recent months. 
These studies are based on the assumption that there can be a small reduction in
the cable plant size compared to \dword{pdsp}. We expect to retire this risk
in the Spring of 2019 once final cable routing tests are performed on
a stacked pair of \dword{apa}s and once we finalize the design of the \dword{ce} cable plant. Should
this risk be realized, because these tests somehow fail or because we do not 
achieve the expected reduction of the cable plant of each \dword{femb}, we will 
consider further reductions of the cable plant, where control signals are not
transmitted individually to the \dwords{femb}, before considering the option of 
routing the cables on the walls of the cryostat.

The next risk (ID 5) that we consider is that of delays in the availability or
damage to the cold electronics components that are installed on the top of the
cryostat. As discussed in Section~\ref{sec:fdsp-tpcelec-integration-timeline}
we plan to have all the \dword{ce} detector components on the top of the cryostat 
required to power, control, and read-out one pair of \dword{apa}s installed
and available prior to the insertion of the \dword{apa}s inside the cryostat.
This allows for extensive testing of \dword{apa}s, which is required to mitigate
the risk of damage to the read-out chain. To mitigate the risk associated with
delays in the installation of \dword{ce} components on top of the cryostat
we plan to have a sufficient number of spares, and to use appropriate \dword{esd} 
prevention measures. If only a subset of all the components is available, cables 
and fibers on the top of the cryostat would have to be re-routed to allow the 
integration and installation of the \dword{apa}s to continue without delays, 
and tests will have to be repeated when all the components become available 
and are installed. The worse possible consequence is a delay in the closure of 
the cryostat and the beginning of operations. 

The final risk (ID 6) that we consider is that caused by incompatibilities 
between various components of the Far Detector that can go undetected until
these components are integrated during prototyping, during integration at 
the \dword{itf}, or during installation. These incompatibilities could 
result in reworking or redesigning some of the components and therefore in
delays of the project. The probability of this risk being realized clearly
diminishes as long as integration tests, including using mock-ups and prototypes,
take place early in the design and construction. The schedule for the 
design and construction of the \dword{ce} detector components foresees a lost of
these integration tests aimed at reducing as much as possible the occurrence
of this risk. This includes also integration tests with the components provided
by the\dword{apa}, \dword{pds}, and \dword{daq} consortia, as well as integration
tests including the routing of cables in the cable trays and through the cryostat
penetrations. 

%%%%%%%%%%%%%%%%%%%%%%%%%%%%%%%%%%%
\subsection{Risks During Commissioning}
\label{sec:fdsp-tpcelec-risks-commissioning}

The biggest risk that could be realized during the commissioning phase
(ID 7 in Table~\ref{tab:SPCE:risks}) is the observation of excessive noise 
caused by some detector component not respecting the \dword{dune} grounding
rules. In some sense this risk was realized at least twice during the 
integration and commissioning of the \dword{pdsp} detector. During the 
integration of the first \dword{apa} a source of noise was discovered 
in the electronics used for the read-out of the photon  detector, which 
required a simple fix on all the boards. Later a large noise source was 
observed with the temperature monitors. The overall noise in the \dword{pdsp} 
is reduced compared to previous \dword{lar} experiments or prototypes, 
like $\mu$Boone or the 35t prototype. Even if some unresolved source of noise 
are still visible in the \dword{pdsp} data, this should not preclude the 
possibility of using the data collected for calibration purposes and for 
physics analyses. Further studies are planned after the end of the \dword{pdsp}
data taking in 2018 to investigate the remaining sources of noise.

The main problem in going from the \dword{pdsp} to DUNE is one of scale.
Even if the detector design addresses all possible noise sources, the simple
fact that the detector is 25 times larger and has a correspondingly larger
number of cryostat penetrations requires a much larger attention to details
during the installation and the commissioning. The observation of excessive 
noise in \dword{dune} would result in a delay of the commissioning and of 
data taking until the source of the noise is found and remedial actions 
are taken. In order to minimize the probability of observing excessive 
electronic noise we plan to enforce the grounding  rules throughout the 
design phase, based on the lessons learned from the operation of the 
\dword{pdsp} detector. We also plan to perform  integrated tests to discover 
possible problems as early as possible. This includes the system tests that 
we plan to perform at the \dword{iceberg} test stand at Fermilab for each generation 
of the \dwords{femb} and of the photon detectors. We plan to perform noise 
measurements in the cold boxes at \dword{itf} and at \surf, and later 
during the insertion of the \dword{apa}s inside the cryostat, before the 
\dword{tco} closure, and repeat the measurements prior to the \dword{lar} filling. 
We expect that the extensive testing will allow us a quick transition to 
detector operations, first with cosmics and later with beam, as soon as the 
cryostat has been completely filled. 

%%%%%%%%%%%%%%%%%%%%%%%%%%%%%%%%%%%
\subsection{Risks During Operations}
\label{sec:fdsp-tpcelec-risks-operations}

The expectation for the DUNE detector is that data taking will continue
for at least two decades. Assuming that after the end of commissioning the
detector operates as designed, there are two risks to be considered.
These are related to the lifetime of the Cold Electronics components
installed inside (ID 8 in Table~\ref{tab:SPCE:risks}) and on top (ID 9)
of the cryostat. The components inside the cryostat are not replaceable,
and therefore any malfunctionment of a detector component will result in
a loss of sensitive volume. The components on the top of the cryostat
(with the exception of the flange at the transition from the cold to
the warm volume) can be replaced and as long as there is a sufficient
number of spares this will not result in any loss beyond the amount 
of time required for the replacement operation. The risk of loss for
the components installed inside the cryostat has been taken into consideration
already at the earliest stage of the design of \dwords{asic}. As discussed
in Section~\ref{sec:fdsp-tpcelec-qa-reliability} we have formed a 
reliability committee to ensure that all the appropriate measures
are considered in the design and that all the tests are performed
as part of our QA process to ensure that the lifetime of components is
such that there will be only minimal losses of sensitivity in the 
detector during operations. 

As discussed in Section~\ref{sec:fdsp-tpcelec-safety-detops} we
are taking measures (like adding HEPA filters to the
\dwords{wiec} and bias voltage and low voltage power supplies) to
minimize the damage from the environmental conditions to the 
detector components on top of the cryostat. We have discussed in
Section~\ref{sec:fdsp-tpcelec-production-spares} our plan for 
spare detector components. We cannot exclude that the number of
spares which we plan to build during the construction will not be
sufficient for the lifetime of the experiment, and that during 
operations we will need to fabricate new boards or procure new
supplies. One of the possible issues related to this is the 
continued availability of certain components, in particular
\dwords{fpga} and optical transmitters and receivers, that may 
become obsolete and may no longer be available at the time new
parts would have to be fabricated. While it will always be 
possible to design new boards using more modern components, we
wish to keep the maintenance costs for the detector to a minimum,
and this may involve following the technology evolution and at
a certain point stocking components that may becoming obsolete
and/or hard to procure. We are also considering placing the
\dwords{fpga} and the optical components on mezzanine cards
to minimize the redesign and procurement costs should these 
components become unavailable during the lifetime of the experiment.
