%%%%%%%%%%%%%%%%%%%%%%%%%%%%%%%%%%%%%%%%%%%%%%%%%%%%%%%%%%%%%%%%%%%%
\section{Organization and Management}
\label{sec:fdsp-pd-org}
%\metainfo{\color{red}\bf  Content: Segreto/Warner}

The \dword{sp} \dword{pd} consortium benefits from the contributions of many institutions and facilities in Europe and North and South America.  Table~\ref{tab:sp-pds-institutes-i}
%and \ref{tab:sp-pds-institutes-ii} 
lists the member institutions. 

%%%%%%%%%%%%%%%%%%%%%%%%%%%%%%%%%%%
%\subsection{Consortium Organization}
\label{sec:fdsp-pd-org-consortium}

\begin{longtable}
{ll}
\caption{PD System Consortium Institutions}\\ \colhline
\rowcolor{dunetablecolor} Member Institute  &  Country       \\  \toprowrule
Federal University of ABC & Brazil \\ \colhline
State University of Feira de Santana & Brazil \\ \colhline
Federal University of Alfenas Po\c{c}os de Caldas & Brazil \\ \colhline
Centro Brasileiro de Pesquisas F\'isicas & Brazil \\ \colhline
Federal University of Goi\'as & Brazil \\ \colhline
Brazilian Synchotron Light Laboratory LNLS/CNPEM & Brazil \\ \colhline
University of Campinas & Brazil \\ \colhline
CTI Renato Archer & Brazil \\ \colhline
Federal Technological University of Paran\'a & Brazil \\ \colhline
Universidad del Atlantico & Colombia \\ \colhline
Universidad Sergia Ablada & Colombia \\ \colhline
University Antonio Nari\~{n}o & Colombia \\ \colhline
Institute of Physics CAS & Czech Republic \\ \colhline
Czech Technical University in Prague & Czech Republic \\ \colhline
Universidad Nacional de Assuncion & Paraguay \\ \colhline
Pontificia Universidad Catilica Per\'{u} & Per\'{u} \\ \colhline
Universidad Nacional de Ingineria & Per\'{u} \\ \colhline
University of Warwick & UK \\ \colhline
University of Sussex & UK \\ \colhline
University of Manchester & UK \\ \colhline
Edinburgh University & UK \\ \colhline
Argonne National Laboratory & USA \\\colhline
Brookhaven National Laboratory & USA \\ \colhline
California Institute of Technology & USA \\ \colhline
Colorado State University   &  USA  \\ \colhline
\dword{fnal}    &   USA    \\ \colhline
Duke University & USA \\ \colhline
Idaho State University & USA \\ \colhline
Indiana University & USA \\ \colhline
University of Iowa & USA \\ \colhline
Louisiana State University & USA \\ \colhline
Massachusetts Institute of Technology & USA \\ \colhline
University of Michigan & USA \\ \colhline
Northern Illinois University & USA \\ \colhline
South Dakota School of Mines and Technology & USA \\ \colhline
Syracuse University & USA \\ \colhline
University of Bologna and INFN & Italy \\ \colhline
University of Milano Bicocca and INFN & Italy \\ \colhline
University of Genova and INFN & Italy \\ \colhline
University of Catania and INFN & Italy \\ \colhline
Laboratori Nazionali del Sud & Italy \\ \colhline
University of Lecce and INFN & Italy \\ \colhline
INFN Milano & Italy \\ \colhline
INFN Padova & Italy \\  \colhline
\label{tab:sp-pds-institutes-i}
\end{longtable}

The \single \dword{pds} consortium follows the typical organizational structure of DUNE consortia:
\begin{itemize}
\item A consortium lead provides overall leadership for the effort, and attends meetings of the DUNE Executive and Technical Boards.
\item A technical lead provides technical support to the consortium lead, attends the technical board and other project meetings, oversees the project schedule and \dword{wbs}, and oversees the operation of the project working groups.  
\item A project management board, composed of the project leads from the participating countries, the consortium leadership team, and a few ad-hoc members, which maintains tight communication between the countries participating in the consortium construction activity.
%In the case of the \dword{pds}, the technical lead is supported by a deputy technical lead.
\end{itemize}


Below the leadership, the consortium is divided up into six working groups, each led by two or three working group conveners (see Table~\ref{tbl:pds-wgs}).  Each working group is charged with one primary area of responsibility within the consortium, and the conveners report directly to the technical lead regarding those responsibilities.

\begin{dunetable}[\dshort{pd} Working Groups]
{ll}
{tbl:pds-wgs}
{PD working groups and responsibilities}
Working Group			 & Responsibilities\\ \toprowrule
Light Collector WG & Mechanical design, materials selection for PD modules\\ \colhline
Photosensors WG & Selection, validation, procuring of photosensors, cold active ganging\\ \colhline
Readout electronics WG & Warm electronics, cable harness, DAQ interface\\ \colhline
Integration and Installation WG & Internal (inter-WG) and external (inter-consortia) interfaces\\ \colhline
Physics and Simulation WG & Physics and simulations studies to determine \dword{pd} specifications\\ \colhline
ProtoDUNE Analysis WG & Validation of PD system in \dshort{pdsp} and \dshort{pdsp2}\\
\end{dunetable}

The working group conveners are appointed by the \dword{pds} Consortium Lead and Technical Lead; the structure may evolve as the consortium matures and additional needs are identified.

\subsection{High-Level Schedule}
\label{sec:fdsp-pd-org-cs}

Table \ref{tab:Xsched} lists key milestones in the design, validation construction and installation of the single-phase photon detector system.  These milestones include external milestones indicating linkages to the main DUNE schedule (highlighted in color in the table), as well as internal milestones such as design validation and technical reviews.

In general the flow of the schedule commences with a 60\% design review based on module performance testing at \dword{unicamp} and at \dword{iceberg}, and integration testing at Ash River.  Additional similar design validation follows, leading to a \dword{fdr}.  Following the \dword{fdr}, 30 modules and required electronics, cabling and \dword{pd} monitoring system components for \dword{pdsp2} will be built, installed and validated during a second ProtoDUNE run at \dword{cern}.  Once the data from this test have undergone initial analysis, production readiness reviews will be conducted and module fabrication will begin.

Some parts of the \dword{pds} system, such as the support rails and electrical connectors required in mid-2020 for \dword{apa} assembly, and photosensors and filter plates which have a long procurement cycle, will require an abbreviated design review process as detailed in the narrative earlier in this document and shown in the milestone table.


\begin{longtable}
{p{0.75\textwidth}p{0.25\textwidth}}
\caption{PD System Consortium schedule}\\ \colhline
\rowcolor{dunetablecolor}Milestone & Date   \\ \toprowrule
60 percent design validation testing complete & February 2020    \\ \colhline
60 percent design review & March 2020    \\ \colhline
Final design review for PD rails, cables, connectors & April 2020\\ \colhline
\dword{prr} for PD rails, cables. connectors & May 2020\\ \colhline
Fabrication of PD rails, cables, connectors begins & May 2020\\ \colhline
Final design validation testing complete & June 2020    \\ \colhline
Down selection to two photosensor candidates & June 2020\\ \colhline
Final design review for remaining PD components & July 2020\\ \colhline
Start of module 0 component production for \dword{pdsp2} & March 2021\\ \colhline
End of module 0 component production for \dword{pdsp2} & June 2021\\ \colhline
\rowcolor{dunepeach} Start of \dword{pdsp}-II installation& \startpduneiispinstall      \\ \colhline
\dword{prr} for photosensors & July 2021\\ \colhline
Begin procurement of production photosensors  & July 2021\\ \colhline
End of module 0 installation for \dword{pdsp2} & August 2021\\ \colhline
Begin procurement of filter plates  & October 2021\\ \colhline
\dword{pdsp2} initial results available & December 2021\\ \colhline
\dword{prr} for remaining photon detector components & January 2022\\ \colhline
Begin fabrication/procurement of remaining module components & January 2022\\ \colhline
Begin assembly of PD monitoring system  & January 2022\\ \colhline
\rowcolor{dunepeach} Start of \dword{pddp}-II installation & \startpduneiidpinstall      \\ \colhline
Begin assembly of front-end electronics modules  & March 2022\\ \colhline
\rowcolor{dunepeach}\dshort{sdwf} available& \sdlwavailable      \\ \colhline
Begin assembly of X-ARAPUCA modules  & July 2022\\ \colhline
\rowcolor{dunepeach}Beneficial occupancy of cavern 1 and \dword{cuc}& \cucbenocc      \\ \colhline
Initial batch (80 PD modules) assembled  & March 2023\\ \colhline
\rowcolor{dunepeach} \dword{cuc} counting room accessible& \accesscuccountrm      \\ \colhline
Initial batch (80 PD modules) arrive at US PD Reception Facility  & June 2023\\ \colhline
Second batch (160 PD modules) assembled  & July 2023\\ \colhline
Initial batch (80 PD modules) arrive at \dshort{sdwf}  & September 2023\\ \colhline
Second batch (160 PD modules) arrive at US PD Reception Facility  & October 2023\\ \colhline
%(\#1 TPC) First 560 modules fabricated   & October 2023\\ \colhline
\dword{pd} monitoring system at \dshort{sdwf}   & October 2023\\ \colhline
Third batch (320 PD modules) assembled  & November 2023\\ \colhline
Second batch (160 PD modules) arrive at \dshort{sdwf}  & December 2023\\ \colhline
\rowcolor{dunepeach}Top of \dword{detmodule} \#1 cryostat accessible& \accesstopfirstcryo      \\ \colhline
Third batch (320 PD modules) arrive at US PD Reception Facility  & January 2024\\ \colhline
%(\#1 TPC) First 560 modules at \dshort{sdwf}   & April 2024\\ \colhline
%(\#1 TPC) 
Front end electronics modules at \dshort{sdwf}   & February 2024\\ \colhline
Fourth batch (320 PD modules) assembled  & February 2024\\ \colhline
Third batch (320 PD modules) arrive at \dshort{sdwf}  & April 2024\\ \colhline
Fourth batch (320 PD modules) arrive at US PD Reception Facility  & May 2024 \\ \colhline
Fifth batch (320 PD modules) assembled  & June 2024\\ \colhline
\rowcolor{dunepeach}Start of \dword{detmodule} \#1 TPC installation& \startfirsttpcinstall      \\ \colhline
Fourth batch (320 PD modules) arrive at \dshort{sdwf}  & August 2024\\ \colhline
Fifth batch (320 PD modules) arrive at US PD Reception Facility  & September 2024 \\ \colhline
Final batch (300 PD modules) assembled  & December 2024\\ \colhline
Fifth batch (320 PD modules) arrive at \dshort{sdwf}  & December 2024\\ \colhline
Final batch (300 PD modules) arrive at US PD Reception Facility  & February 2025 \\ \colhline
Final batch (300 PD modules) arrive at \dshort{sdwf}  & April 2025\\ \colhline
%(\#1 TPC) Final PD modules at \dshort{sdwf}   & April 2025\\ \colhline
\rowcolor{dunepeach}End of \dword{detmodule} \#1 TPC installation& \firsttpcinstallend      \\ \colhline
\rowcolor{dunepeach}Top of \dword{detmodule} \#2 accessible& \accesstopsecondcryo      \\ \colhline
 \rowcolor{dunepeach}Start of \dword{detmodule} \#2 TPC installation& \startsecondtpcinstall      \\ \colhline
\rowcolor{dunepeach}End of \dword{detmodule} \#2 TPC installation& \secondtpcinstallend      \\ 
\label{tab:Xsched}
\end{longtable}

\subsection{High-Level Cost Narrative}

In the fall of 2018 we completed an initial cost estimate for fabrication of \dword{pd} modules for one \SI{10}{kt} \dword{dune} module, and updated the estimate extensively in March/April of 2019.  The estimates are based on \dword{pdsp} costs, modified as necessary for an \dword{xarapu} design.  Vendor quotations or vendor estimates are used for all the major components.  The biggest uncertainties in fabrication costs center around the photosensor fabrication, which 
constitute approximately half the total PD system cost.  We have estimates from Hamamatsu for photosensors which would reduce this line by nearly a factor of two, significantly reducing the system cost.  We also have preliminary indications that similar cost savings may also be available from using \dword{fbk} photosensors.  As noted earlier in this \dword{tdr}, a major focus of our remaining development work is focused on realizing these potential savings.

The dichroic filter procurement and coating represent the other major cost driver for the project.  The costing for the filter plates is based on initial contacts with a Brazilian filter firm.  Initial samples of filter substrates have been received at \dword{unicamp} and have been successfully coated and tested through multiple cryogenic cycles with no indication of failure. Extensive additional validation of the Brazilian filters  will occur during late 2019 as part of the SBND module fabrication.

These filter plates are significantly cheaper than the filters manufactured by Omega Inc. that were tested in our earlier validation studies.  Until these tests are complete, the filter plates remain a significant cost and schedule risk.

Extensive use of design-for-fabrication techniques throughout the module development phase, as well as multiple rounds of prototype development, have allowed us to minimize the component cost for the remaining components.  In-house fabrication and assembly using university shop facilities and student labor for assembly (particularly at \dword{unicamp}) have also reduced costs.

Modification of an existing and well understood readout electronics system has very significantly reduced initial cost estimates for that portion of the system.
