%%%%%%%%%%%%%%%%%%%%%%%%%%%%%%%%%%%%%%%%%%%%%%%%%%%%%%%%%%%%%%%
\section{System Interfaces}
\label{sec:fdsp-pd-intfc}
%\metainfo{\color{blue} Content: Onel, Kemp}
%\metainfo{(Length: \dword{tdr}=10 pages, TP=2 pages)}

%\fixme{??Include an image of each interface in appropriate sections.}

%>> Revision: Ernesto Kemp, Yasar Onel & David Warner Apr/12/2019 
%>> Revision: Ernesto Kemp, Yasar Onel & David Warner Jan/08/2019 
%>> Revision: Ernesto Kemp, Yasar Onel, David Warner Nov/23/2018 
%>> Revision: Ernesto Kemp & Norm Buchanan Mar/15/2018 
%>> Start: Ernesto Kemp Feb/10/2018 
\subsection{Overview}
Table~\ref{tbl:SPPDinterfaces} contains a summary and brief description of all the interfaces between the \dword{spmod} \dword{pds} consortium and other consortia, working groups and task forces, with references to the current version of the interface documents describing those interfaces.  
%At this stage, the interface documentation is not complete, and many interfaces are defined at the conceptual level. 
Drawings of the mechanical interfaces and diagrams of the electrical interfaces are 
%still under development. 
included in the interface documents as appropriate.
It is expected that further refinements of the interface documents will take place prior to the final \dword{prr} for the detector. The interface documents specify the responsibility of different consortia or groups during all phases of the experiment including design and prototyping, integration,  installation, and  commissioning.



Additional details describing the interface between the \dword{spmod} \dword{pds} and the other consortia, task forces (TF) and subsystems are given below:



\begin{dunetable}
[Single Phase PD system interface links]
{p{0.25\textwidth}p{0.5\textwidth}l}
{tbl:SPPDinterfaces}
{Single Phase PD system interface links.}
Interfacing System & Description & Linked Reference \\ \toprowrule
Detector Subsystems & &\\ \colhline
\dword{apa} & Mechanical support for PDs, PD installation slots, PD cabling support, access slots & \citedocdb{6667} \\ \colhline
\dword{ce} & Electrical signal interference, grounding, cable routing, cryostat flange, installation and testing & \citedocdb{6718} \\ \colhline
\dword{hv} & Mounting of \dword{pd} monitoring system, possible reflector foil support, electrical discharge or corona effect light contamination & \citedocdb{6721} \\ \colhline
\dword{daq} & Data format, data timing, trigger information, timing and synchronization & \citedocdb{6727} \\ \colhline
\dword{cisc} & Rack layout, flange heaters, power supply selection, power and signal cable selection, monitoring cameras and camera lighting, purity monitor lighting, controls and data monitoring & \citedocdb{6730} \\ \colhline
Technical Coordination & &\\ \colhline
Facility interfaces & Cable trays inside the cryostat, cryostat penetrations, rack layout and power distribution on the detector mezzanine, cable and fiber trays on top of the cryostat & \citedocdb{6970} \\ \colhline
Installation interfaces & Sequence of integration and installation activities at SURF, equipment required for \dword{pd} consortium activities, environmental controls in the cryostat during installation, post-installation testing  & \citedocdb{6997} \\ \colhline
 
%Integration and Test Facility (ITF) interfaces & Material handling and testing activities prior to integration, environmental requirements for \dword{pd}, cold box for APA tests,  & \citedocdb{7024} \\ \colhline
Calibration task force interfaces & Interface of \dword{spmod}\ \dword{dp} monitoring system into calibration system. & \citedocdb{7051} \\ \colhline
%Physics interfaces & TBD & \citedocdb{7078} \\ \colhline
Physics, Software and Computing interfaces & Covers interfaces between the \dword{pd} group and the joint computing task force, including specifications required for physics, data handling, and computing and storage requirements. & \citedocdb{7105} \\
\end{dunetable}
%\fixme{Table 1.6 Interface links has two TBDs (physics and S/w)}

%%%%%%%%%%%%%%%%%%%%%%%%%%%%%%%%%%%
\subsection{Anode Plane Assembly}

The interface with the Anode Plane Assembly (APA) represents the most significant mechanical interface for the PD system. Interfaces with the \dword{apa} are involved in meeting specifications SP-PDS-2, SP-PDS-7, SP-PDS-8, SP-PDS-9, SP-PDS-10, SP-PDS-11, and SP-PDS-12 (see table \ref{tab:specs:SP-PDS}).  The interface document will be written to monitoring of these specifications.

%\fixme{Added PD specifications related to APA interface  DWW: Done}

The APA provides:

\begin{itemize}

\item mechanical support and alignment for the \dword{pd} modules, including access slots through the side of the frame for insertion of modules after the \dword{apa}s are wrapped in wire;

\item mounting support for the PD electrical connections between the \dword{pd} modules and the cable harness mounted inside the \dword{apa} frame;

\item mechanical support and strain relief for PD cables located inside the completed \dword{apa} frame; and

\item provision to connect the \dword{pd} cables from the lower \dword{apa} to the upper \dword{apa} in an assembled \dword{apa} stack, and to connect the the 
cables from the top of the \dword{apa} stack to the cryostat flange.

\end{itemize}

Work on the 2-\dword{apa} connection and inspection in the assembly area in the underground assembly area will be performed by the \dword{apa} group. Work on cabling prior to installation is performed by \dword{pds} and \dword{ce} groups under supervision of the \dword{apa} group. Once the \dwords{apa} are moved inside the cryostat, the \dword{pds} and \dword{ce} consortia will be responsible for the routing of the cables in the trays hanging from the top of the cryostat. 

Careful interface control will be required to ensure a successful assembly, which will be guided by the interface control document between the \dword{pd} and \dword{apa} consortia.  




%%%%%%%%%%%%%%%%%%%%%%%%%%%%%%%%%%%
\subsection{TPC Cold Electronics}
\label{sec:fdsp-pd-intfc-ce}



Interfaces with the TPC cold electronics are involved in meeting specifications SP-FD-2, SP-PDS-8, and SP-PDS-10 (see Table~\ref{tab:specs:SP-PDS}).  The interface between the \dword{pd} and \dlong{ce} systems primarily consists of:

\begin{itemize}
    \item ensuring no electrical cross-talk between \dword{pd} and \dword{ce} electronics and cabling harnesses systems;
    \item ensuring there be no electrical contact between the \dword{pds} and \dword{ce} components except for sharing a common reference voltage point (ground) at the \fdth{}s;
    \item developing a common cable routing plan allowing the systems to share a common cable tray system on top of the \dword{apa} frame and routing the cables to the cryostat flanges; and 



    \item managing the interface between the \dword{pd} and \dword{ce} flanges in the cryostat cabling tees.
  
  \end{itemize}  
The \dword{ce} and \dword{pd} use a common cable tray system but separate flanges for the cold-to-warm transition and each consortium is responsible for the design, procurement, testing, and installation, of their flange on the \fdth{}, together with \dword{lbnf}, which is responsible for the design of the cryostat. 
The installation of the racks on top of the cryostat is a responsibility of the facility, but the exact arrangement of the various crates inside the racks will be reached after common agreement between the \dword{ce}, \dword{pd}, \dword{cisc}, and possibly \dword{daq} consortia. The \dword{pd} and \dword{ce} consortia will retain all responsibilities for the selection, procurement, testing, and installation of their respective racks, unless for space and cost considerations an agreement is reached where common crates are used to house low voltage or high/bias voltage modules for both \dword{pd}S and \dword{ce}. 






%%%%%%%%%%%%%%%%%%%%%%%%%%%%%%%%%%%
%\subsection{Cathode Plane Assembly and High Voltage System: Reflector foils (light enhancement)}
\subsection{Cathode Plane Assembly and High Voltage System}
\label{sec:fdsp-pd-intfc-le}



%EK - Apr/15/19
Interfaces with the \dword{hv} system are involved in meeting specifications SP-PDS-6 and SP-PDS-9 (see Table~\ref{tab:specs:SP-PDS}).  The primary interface between the \dword{pd} and \dword{hv} systems is summarized as follows:
\begin{itemize}
    \item minimizing background light due to electrical discharge (corona effects);
    \item mounting the \dword{pd} monitoring system light diffusers to the \dword{cpa} faces; and 
    \item providing an optical fiber routing path and strain relief system to the cryostat calibration hatch.
\end{itemize} 

%\fixme{added item covering electrical discharge light  DWW:  Done}

This interface has strong overlap with the calibration consortium; those are described in more detail in Section~\ref{sec:fdsp-pd-intfc-calib}.



If the light reflector foil option were to be implemented, production of the FR4+resistive Kapton \dword{cpa} frames will be the responsibility of the \dword{hv} consortium, together with design of the structure for mounting the \dword{pd} reflector foils to the \dword{cpa} structure.  The \dword{hv} consortium will also provide mounting attachment points in the \dword{cpa} frame structure.
 The reflector foils themselves, \dword{tpb} coating of the \dword{wls} foils, and any required hardware for mounting the foils will be the responsibility of the \dword{pd} consortium, with the understanding that all designs and procedures will be approved by the \dword{hv} consortium. 



\subsection{Data Acquisition}
\label{sec:fdsp-pd-intfc-daq}


Interfaces with the \dword{daq} system are involved in meeting specifications SP-PDS-5, SP-PD-6 and SP-PDS-13 (see Table~\ref{tab:specs:SP-PDS}).  The \dword{pds} interfaces with the Data Acquisition (DAQ) system are described in \citedocdb{6727} and include:



\begin{itemize}

\item Data Physical Links: Data are passed from the \dword{pd} to the \dword{daq} on 25 optical links following the 1000Base-SX standard. The links run from the \dword{pd} readout system on the cryostat to the \dword{daq} system in the \dword{cuc}.

\item Data Format: Data are encoded using UDP/IP.  The data format consists of a header containing the word count, event time stamp, and channel ID, followed by the digitized waveform in \SI{80}{MHz} samples.
%EK - Apr/18/2019
The data format has also been specified to use compression (zero suppression) and custom communication protocol.

\item Data Timing: The data must contain enough information to identify the time at which it was taken.

\item Trigger Information: The \dword{pd} may provide summary information useful for data selection. If present, this will be passed to the \dword{daq} on the same physical links as the remaining data.

\item Timing and Synchronization: Clock and synchronization messages will be propagated from the \dword{daq} to the \dword{pd} using a backwards compatible development of the \dword{pdsp} timing system protocol~\cite{bib:docdb1651}. There will be at least one timing fiber available for each data link coming from the \dword{pds}. 

\item Power-on initialization and start-of-run setup:  The \dword{pd}S may require initialization and setup on power-on and start of run. Power on initialization should not require communication with the \dword{daq}. Start run/stop run and synchronization signals such as accelerator spill information will be passed by the timing system interface.

\end{itemize}

%EK - Apr/18/2019
The data format has already been determined. Nevertheless, there is the possibility to include additional summary information to the header which depends on the outcome of triggering studies which are on-going. This is a fairly minor potential modification which can be accommodated easily.



\subsection{Cryogenics Instrumentation and Slow Control}
\label{sec:fdsp-pd-intfc-xeon}

The primary interactions between the \dword{pd} and the \dword{cisc} include:

\begin{itemize}
    \item warm electronics rack controls, power supplies, rack safety equipment;
    \item warm cable and connector selection;
    \item cryogenic camera systems for detector monitoring, including lighting systems;
    \item purity monitor lighting requirements;
    \item cryostat flanges required for PD signal cable and monitoring systems; and
    \item \dword{pd} slow control (including bias voltage) and data monitoring.
\end{itemize}

Additional interaction may occur in the case that the xenon doping performance enhancement is selected for inclusion in the detector.  This system requires pre-mixing xenon gas and argon gas to introduce xenon doping into the \lar volume. 



Any required hardware for this enhancement will be the responsibility of the Photon Detection consortium, with the understanding that all designs and procedures will be pre-approved by the Cryogenics group. 

%This new detector component was not tested in the 2018 \dword{pdsp} operation, but may be proposed for future running. Intense R\&D will be required before deciding on its implementation.

\subsection{Facility, Integration and Installation Interfaces}

%EK - Apr/15/19
The interface document with the project Interface and Installation working group covers the interface of the \dword{pd} group with the technical coordination groups who oversee the integration of the \dword{pd} modules and electronics into the \dword{apa} and \dword{daq}. Interfaces with the facility, integration and installation group are involved in meeting specifications SP-PDS-1, SP-PDS-3, SP-PDS-4 and SP-PDS-5 (see Table~\ref{tab:specs:SP-PDS}).  The interfaces are distributed among the facility, integration and installation working groups, and primarily consist of:
\begin{itemize}
    \item electrical racks, cable trays, and cryostat penetrations, and power distribution on the mezzanine;
    \item storage for arriving \dword{pd} modules prior to their integration;
    %Facilities group
    \item planning of pre-integration tests of \dword{pd} components at the integration area and required equipment/tools;
    %Integration group
    \item sequence of integration and installation activities at SURF (including environmental controls);
    \item quality management testing of \dword{pd} modules during integration and installation;
    %Installation group
    \item equipment required for \dword{pd} consortium activities; and
    %Installation group
    \item environmental controls in the cryostat during installation, and post-installation testing.
\end{itemize}



The \dword{pd} consortium retains responsibility for providing quality management tooling and test plans at the integration area, as well as specialized labor and supervisory personnel for \dword{pd} module integration and installation. Distribution of these responsibilities are covered in the Integration \dword{pd} consortium interface document.


The installation is described in detail in 
Volume~\volnumbertc{}, \voltitletc{}.
%Chapter~\ref{ch:sp-tc}.

%\fixme{\color{red} Add correct links to integration and installation chapter -- Anne will you do this? I don't know the chapter number}


%%%%%%%%%%%%%%%%%%%%%%%%%%%%%%%%%%%

%%%%%%%%%%%%%%%%%%%%%%%%%%%%%%%%%%%
\subsection{Calibration and Monitoring}
\label{sec:fdsp-pd-intfc-calib}

%\todo{Content: Kemp/Onel/Djurcic}



% EK, April/15
This subsection concentrates on the description of the interface between the SP-\dword{pd} and Calibration/Monitoring Task Force (CTF).
%, since there are components of the system planned to be installed with the \dword{hv} system Cathode, and through \dword{fc} strips and \dword{fc} \dword{gp}. It is proposed that the SP PDS gain and timing calibration system, a pulsed UV-light system, also be used for PD monitoring purposes during both commissioning and experimental operation.
%======================================
 Main interface items are:
\begin{itemize}
    \item cold components: light sources (diffusers and fibers) placed on the cathode planes to illuminate the detectors;
    \item warm components: a controlled pulsed-UV source and warm optics; and 
    %These warm components interface calibration and monitoring  with the \dword{cisc} and \dword{daq} subsystems, and are described in corresponding documents (\citedocdb{6730} and \citedocdb{6727} respectively).
    \item the optical \fdth, used to \dword{pd} bring monitoring system fiberoptics through the calibration and monitoring flange.  Note that the flange itself is a shared interface between the \dword{pd}, the calibration task force, and the \dword{cisc}.
\end{itemize}


Hardware components required for \dword{pd} monitoring and calibration systems will be designed and fabricated by the \dword{sp}-\dword{pds} consortium. 

%EK - Apr/15
Cold components (diffusers and fibers) interface with \dword{hv} and are described in a separate interface document (\citedocdb{6721}). Warm components interface calibration and monitoring  with the \dword{cisc} and \dword{daq} subsystems, and are described in corresponding documents (\citedocdb{6730} and \citedocdb{6727} respectively).
%==========================================


%EK - Apr/15/19
A joint development effort with HV/\dword{cpa} groups will define the optimization of materials and location of the photon diffusers, fiber routes, connectors location and also the installation procedure of the diffusers and fiber. The feed-through ports/locations and fiber routing along DSS will be determined jointly by SP-PDS and Cryostat/DSS group. The calibration TF and PD system group will share rack spaces. Multi-purpose ports are planned to be shared between various groups, calibration devices such as lasers and cameras will make use of them. CTF and SP-PDS will define the ports for deployment. An interlock system to avoid turning on light sources when the PD system is in operation will be provided.



\subsection{Physics, Software and Computing}

%\fixme{ DWW:  Can someone please address this?}
%\fixme{\color{blue}EK: still working on this section. Need to cut down - Jan/11}

Interfaces with physics, software and computing are involved in meeting specifications SP-FD-3, SP-FD-4, SP-PDS-2, SP-PDS-5, SP-PDS-6, SP-PDS-14, SP-PDS-15, and SP-PDS-16 (see table \ref{tab:specs:SP-PDS}). The physics topics covered by the %Supernova Burst/Low Energy Physics Working Group and the Nucleon Decay/High Energy Physics Working Group 
\dword{snb}/low energy and \dword{ndk}/\dword{hep} working groups are the most closely connected to the \single \dword{pds}. The connection stems from the need for self-triggering for DUNE non-beam physics addressed by these two groups. 
%In turn, self-triggering will likely need photon information.
However, there are connections to all physics working groups involving far detector observables, as scintillation light information will improve event reconstruction/classification beyond what is achievable by TPC information only. 



Below is a summary of interfaces between the \dune \single \dword{pds} and \dword{fd} and \dword{pdsp} Simulation/Reconstruction groups:

\begin{itemize}
    \item generating photon libraries, and the tools for doing so;
    \item simulating and evaluating performance of physics events;
    \item \single \dword{pds} reconstruction performance studies;
    \item algorithms for matching flashes to \dword{tpc} tracks; and
    \item analyzing the light produced by various species of charged particles.
\end{itemize}

It is critical that the performance specifications for the \dword{pds} meet the needs of the physics and reconstruction teams, both in terms of detector performance and background (including false triggers from radiologicals and light contamination from cryostat light leaks, HV system corona discharges, and calibration system effects such as purity monitors, laser flashers and cameras.  These interfaces will be captured here.
% EK - Jul/04/19
Light contamination of any nature must be studied quantitatively so that the impact on error budget due to misclassification of events can be calculated. Quantitative indicators should be established using \dword{pdsp} data, which should also provide the basis for identification algorithms of spurious signals caused by light leakage. 




The SP \dword{pds} shares interfaces with the DUNE core computing systems, primarily with databases. The two databases that will have direct interfaces with the SP \dword{pds} are the hardware/QC and calibration databases. All the off-line calibration values will be stored in the DUNE calibration database. Additionally, the system will interface with the DUNE hardware database. During all stages of production/procurement and QC evaluations of \dword{pds} components, as well as integration and installation of the system, tracking of the hardware and test results will be stored in the DUNE hardware/QC database. The SP \dword{pds} consortium will work with the database group to ensure that all schema, applications, and procedures for the database interfaces are developed. As components of the system will originate at multiple institutions, well defined procedures and management will be required to ensure that all data is archived in the DUNE hardware/QC database. 


%>> Start: Ernesto Kemp Feb/10/2018 
%>> Revision: Ernesto Kemp & Norm Buchanan Mar/15/2018 
%>> Revision: Ernesto Kemp, Yasar Onel & David Warner Nov/23/2018 
%>> Revision: Ernesto Kemp, Yasar Onel & David Warner Jan/08/2019 
%>> Revision: Ernesto Kemp, Yasar Onel & David Warner Apr/12/2019
