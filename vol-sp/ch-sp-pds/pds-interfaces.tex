%%%%%%%%%%%%%%%%%%%%%%%%%%%%%%%%%%%%%%%%%%%%%%%%%%%%%%%%%%%%%%%
\section{System Interfaces}
\label{sec:fdsp-pd-intfc}
%\metainfo{\color{blue} Content: Onel, Kemp}
%\metainfo{(Length: \dword{tdr}=10 pages, TP=2 pages)}

%\fixme{??Include an image of each interface in appropriate sections.}

%>> Revision: Ernesto Kemp, Yasar Onel & David Warner Jan/08/2019 <<<<<<<<<<<<<
%>> Revision: Ernesto Kemp, Yasar Onel, David Warner Nov/23/2018 >>>>>>>>>>>>>
%>> Revision: Ernesto Kemp & Norm Buchanan Mar/15/2018 >>>>>>>>>>>>>
%>> Start: Ernesto Kemp Feb/10/2018 >>>>>>>>>>>>>  
\subsection{Overview}
Table~\ref{tbl:SPPDinterfaces} contains a summary and brief description of all the interfaces between the \dword{spmod} \dword{pds} consortium and other consortia, working groups and task forces, with references to the current version of the interface documents describing those interfaces.  At this stage,  the interface documention is  not complete, and many interfaces are defined at the conceptual level. Drawings of the mechanical interfaces and diagrams of the electrical interfaces are still under development. It is expected that further refinements of the interface documents will take place in the remainder of 2019 and in the first half of 2020 prior to the engineering design reviews of the detector. All the interface documents specify the responsibility of different consortia or group during all the phases of the experiment including during the design and prototyping, the integration, the installation, and the commissioning.

Additional details describing the interface between the \dword{spmod} \dword{pds} and %several 
the other consortia, task forces (TF) and subsystems are given below:



\begin{dunetable}
[Single Phase Photon Detector Interface Links]
{p{0.25\textwidth}p{0.5\textwidth}l}
{tbl:SPPDinterfaces}
{Single Phase Photon Detector Interface Links }
Interfacing System & Description & Linked Reference \\ \toprowrule
Detector Subsystems & &\\ \colhline
\dword{apa} & Mechanical support for PDs, PD installation slots, PD cabling support, access slots & \citedocdb{6667} \\ \colhline
\dword{ce} & Electrical signal interference, grounding, cable routing, cryostat flange, installation and testing & \citedocdb{6718} \\ \colhline
\dword{hv} & Mounting of \dword{pd} monitoring system, possible reflector foil support, electrical discharge or corona effect light contamination & \citedocdb{6721} \\ \colhline
\dword{daq} & Data format, data timing, trigger information, timing and synchronization & \citedocdb{6727} \\ \colhline
\dword{cisc} & Rack layout, flange heaters, power supply selection, power and signal cable selection, monitoring cameras and camera lighting, purity monitor lighting, controls and data monitoring & \citedocdb{6730} \\ \colhline
Technical Coordination & &\\ \colhline
Facility interfaces & Cable trays inside the cryostat, cryostat penetrations, rack layout and power distribution on the detector mezzanine, cable and fiber trays on top of the cryostat & \citedocdb{6970} \\ \colhline
Installation interfaces & Sequence of integration and installation activities at SURF, equipment required for \dword{pd} consortium activities, environmental controls in the cryostat during installation, post-installation testing  & \citedocdb{6997} \\ \colhline
%\dword{itf} 
Integration and Test Facility (ITF) interfaces & Material handling and testing activities prior to integration, environmental requirements for \dword{pd}, cold box for APA tests,  & \citedocdb{7024} \\ \colhline
Calibration task force interfaces & Interface of \dword{spmod}\dword{dp} monitoring system into calibration system. & \citedocdb{7051} \\ \colhline
Physics interfaces & TBD & \citedocdb{7078} \\ \colhline
Software and Computing interfaces & TBD & \citedocdb{7105} \\
\end{dunetable}


%%%%%%%%%%%%%%%%%%%%%%%%%%%%%%%%%%%
\subsection{Anode Plane Assembly}

The interface with the Anode Plane Assembly represents the most significant mechanical interface for the PD system.  The anode plane assembly provides:

\begin{itemize}

\item mechanical support for the \dword{pd} modules, including access slots through the side of the frame for insertion of modules after the \dword{apa}s are wrapped in wire,

\item mounting support for the PD electrical connections between the \dword{pd} modules and the cable harness mounted inside the \dword{apa} frame,

\item mechanical support and strain relief for PD cables located inside the completed \dword{apa} frame, and

\item provision to connect the \dword{pd} cables from the lower \dword{apa} to the upper \dword{apa} in an assembled \dword{apa} stack, and to connect the the "Long haul" cables from the top of the \dword{apa} stack to the cryostat flange.

\end{itemize}

Careful interface control will be required to ensure a successful assembly, which will be guided by the interface control document between the \dword{pd} and \dword{apa} consortia.





%%%%%%%%%%%%%%%%%%%%%%%%%%%%%%%%%%%
\subsection{TPC Cold Electronics (CE)}
\label{sec:fdsp-pd-intfc-ce}



The interface between the \dword{pd} and \dword{ce} systems primarily consist of:

\begin{itemize}
    \item ensuring no electrical cross-talk between \dword{pd} and \dword{ce} electronics and cabling harnesses systems,
    \item ensuring there be no electrical contact between the \dword{pds} and \dword{ce} components except for sharing a common reference voltage point (ground) at the \fdth{}s,  
    \item developing a common cable routing plan allowing the systems to share a common cable tray system on top of the \dword{apa} frame and routing the cables to the cryostat flanges, and 
    \item managing the interface between the \dword{pd} and \dword{ce} flanges in the cryostat cabling tees.
  \end{itemize}  
In the current design \dword{ce} and \dword{pd} use separate flanges for the cold-to-warm transition and each consortium is responsible for the design, procurement, testing, and installation, of their flange on the \fdth{}, together with LBNF, who is responsible for the design of the cryostat. 
The installation of the racks on top of the cryostat is a responsibility of the facility, but the exact arrangement of the various crates inside the racks will be reached after common agreement between the \dword{ce}, \dword{pd}, \dword{cisc}, and possibly \dword{daq} consortia. The \dword{pd} and \dword{ce} consortia will retain all responsibilities for the selection, procurement, testing, and installation of their respective racks, unless for space and cost considerations an agreement is reached where common crates are used to house low voltage or high/bias voltage modules for both \dword{pd}S and \dword{ce}. 


\subsection{Cathode Plane Assembly and High Voltage System}
\label{sec:fdsp-pd-intfc-le}


The primary interface between the \dword{hv} system will consist in mounting the \dword{pd} monitoring system light diffusers to the \dword{cpa} faces, and providing an optical fiber routing path and strain relief system to the cryostat calibration hatch.



In the case that the light reflector foil option were to be implemented, production of the FR4+resistive Kapton \dword{cpa} frames will be the responsibility of the \dword{hv} consortium.
%are the responsibility of the \dword{hv} consortium. Production and \dword{tpb} coating of the \dword{wls} foils will be the responsibility of the Photon Detection consortium. 
The \dword{tpb} coating of the \dword{wls} foils and the fixing procedure for applying the \dword{wls} foils onto the \dword{cpa} frames and any required hardware will be the responsibility of the Photon Detection consortium, with the understanding that all designs and procedures will be pre-approved by the \dword{hv} consortium. 



\subsection{Data Acquisition (DAQ)}
\label{sec:fdsp-pd-intfc-daq}



%This section describes the \dword{pd} interfaces and related requirements with the \dword{daq} system described in Section~\ref{ch:fdsp-daq}. Here we list the main system interface areas.
The \dword{pds} interfaces with the DAQ system are described in \citedocdb{6727} and include:

%\{Is there a \dword{daq} chapter section on \dword{pd} to refer to?} - in the DAQ volume, found reference to DocDB 6727 for the PDS interface

\begin{itemize}

\item Data Physical Links: Data are passed from the \dword{pd} to the \dword{daq} on optical links conforming to an IEEE Ethernet standard. The links run from the \dword{pd} readout system on the cryostat to the \dword{daq} system in the Central Utilities Cavern (\dword{cuc}).

\item Data Format: Data are encoded using a data format based on UDP/IP. The data format is derived from the one used by the Dual Phase TPC readout. Details will be finalized by the time of the \dword{daq} \dword{tdr}.

\item Data Timing: The data must contain enough information to identify the time  at which it was taken.

\item Trigger Information: The \dword{pd} may provide summary information useful for data selection. If present, this will be passed to the \dword{daq} on the same physical links as the remaining data.

\item Timing and Synchronization: Clock and synchronization messages will be propagated from the \dword{daq} to the \dword{pd} using a backwards compatible development of the \dword{pdsp} Timing System protocol (DUNE docdb-1651). There will be at least one timing fiber available for each data links coming from the \dword{pd}S. Power-on initialization and Start of Run setup:  The \dword{pd}S may require initialization and setup on power-on and start of run. Power on initialization should not require communication with the \dword{daq}. Start run/stop run and synchronization signals such as accelerator spill information will be passed by the timing system interface.

\end{itemize}



\subsection{Cryogenics Instrumentation and Slow Control (CISC)}
\label{sec:fdsp-pd-intfc-xeon}

The primary interactions between the \dword{pd} and the \dword{cisc} include:

\begin{itemize}
    \item warm electronics rack controls, power supplies, rack safety equipment;
    \item warm cable and connector selection;
    \item cryogenic camera systems for detector monitoring, including lighting systems
    \item Purity monitor lighting requirements; and
    \item \dword{pd} slow control (including bias voltage) and data monitoring.
\end{itemize}

Additional interaction may occur in the case that the Xenon doping performance enhancement is selected for inclusion in the detector.  This system requires pre-mixing xenon gas and argon gas to introduce xenon doping into the \lar volume. 


Any required hardware for this enhancement will be the responsibility of the Photon Detection consortium, with the understanding that all designs and procedures will be pre-approved by the Cryogenics group. 

%This new detector component was not tested in the 2018 \dword{pdsp} operation, but may be proposed for future running. Intense R\&D will be required before deciding on its implementation.

\subsection{Facility, Integration and Installation Interfaces}

These interface documents cover the interface of the \dword{pd} group with the technical coordination groups.  They are covered in great detail in Chapter~\ref{ch:sp-tc}.


Interfaces with the facilities group cover electrical racks, cable trays, and cryostat penetrations, and power distribution on the mezzanine.

Integration interfaces cover the interactions at the \dword{itf}, including \dword{pd} module reception and materials handling, pre-installation quality management testing, integration into the \dword{apa} frames, and post-installation testing.  Also covered are environmental requirements for the assembly area.

Installation interfaces include sequence of integration and installation activities at \surf, equipment required for \dword{pd} consortium activities, environmental controls in the cryostat during installation, and post-installation testing.

%%%%%%%%%%%%%%%%%%%%%%%%%%%%%%%%%%%

%%%%%%%%%%%%%%%%%%%%%%%%%%%%%%%%%%%
\subsection{Calibration and Monitoring}
\label{sec:fdsp-pd-intfc-calib}



This subsection defines the internal calibration system for the \single \dword{pds}. 
%It may be interfaced to a calibration consortium later.

It is proposed that the \single \dword{pds} gain and timing calibration system, a pulsed UV-light system, also be used for \dword{pd} monitoring purposes during both commissioning and experimental operation. The hardware consists of warm and cold components.  


By placing diffusers designed to play a role of light sources on the cathode planes, the photon detectors embedded in the anode planes can be illuminated.
Cold components (diffusers and fibers) interface with \dword{hv} and are described in a separate interface document (\citedocdb{6721}). 
%{DUNE docdb-6721 n} 
Warm components include controlled pulsed-UV source and warm optics. These warm components interface calibration and monitoring  with the \dword{cisc} and \dword{daq} subsystems, and are described in corresponding documents (\citedocdb{6730} and \citedocdb{6727} respectively).
Optical \fdth is a cryostat interface. 

Hardware components will be designed and fabricated by SP-\dword{pd}S. Other aspects of hardware interfaces are described in the following. The CTF and \dword{pd}S groups might share rack spaces that needs to be coordinated between both groups. There will not be dedicated ports for all calibration devices. Therefore, multipurpose ports are planned to be shared between various groups. CTF and SP-\dword{pd} will define ports for deployment. It is possible that SP-\dword{pd} might use Detector Support Structure (DSS) ports or TPC signal ports for routing fibers. The CTF in coordination with other groups will provide a scheme for interlock mechanism of operating various calibration devices (e.g. laser, radioactive sources) that will not be damaging to the \dword{pd}. 



In the next steps there will be continued interface discussion with Cryostat/DSS group to identify feed-through ports/locations and to define fiber routes along DSS. The calibration TF and PDS group will share rack spaces. There won't be dedicated ports for all calibration devices e.g. laser and cameras will share ports. 
Therefore, multi-purpose ports are planned to be shared between various groups. CTF and SP-PDS will define ports for deployment. It is possible that SP-PDS might use Detector Support Structure (DSS) ports or TPC signal ports for routing fibers. CTF will be responsible for interlocks needed with other calibration devices 
like laser, radioactive sources etc. SP-PDS will not provide the interlock mechanism, but within interface between SP-PDS and CTF a system that will avoid turning  on other light sources when PDS is in operation will be defined.
It is also planned to continue work with HV/\dword{cpa} groups to identify optimal materials and locations of photon diffusers at \dword{cpa}, to define fiber routes and connector locations  at \dword{cpa}, and to define diffuser/fiber installation procedures.

The following additional steps will be completed before fabrication and installation of the photon-detector calibration and monitoring systems proceeds: add feedthroughs to cryostat drawings, diffuser/fiber to \dword{cpa}/HV drawings; diffuser material selection, prototyping, production and QA/QC testing; fiber selection, prototyping, production and QA/QC testing; identify, test and procure VUV light sources; productions and testing of calibration modules; and define integration and installation of system components.


\subsection{Physics, Software and Computing}

%\fixme{ DWW:  Can someone please address this?}
\fixme{EK: still working on this section. Need to cut down - Jan/11}

The physics topics covered by the Supernova Burst/Low Energy Physics Working Group and the Nucleon Decay/High Energy Physics Working Group are the most closely connected to the \single \dword{pds}. The connection stems from the need for self-triggering for DUNE non-beam physics addressed by these two groups. 
%In turn, self-triggering will likely need photon information.
However there are connections to all Physics Working Groups involving far detector observables, as scintillation light information may improve event reconstruction/classification beyond what is achievable by TPC information only. 

%***************************
\begin{comment}
The Physics Working Groups are responsible to provide input on high-level physics requirements to achieve physics goals. Physics quantities of relevance include:

\begin{itemize}
    \item Trigger efficiencies and FD uptimes for relevant non-beam events
    \item Reconstruction and selection efficiencies for relevant non-beam neutrino interaction channels and nucleon decay modes
    \item Requirements on relevant backgrounds: radiologicals, cosmogenics, atmospheric neutrinos
    \item Resolution requirements for neutrino interactions: energy, time and direction
%    \item Neutrino energy resolution
%    \item Neutrino interaction time resolution
%    \item Neutrino direction angular resolution
    \item Neutrino interaction channel and flavor classification 
\end{itemize}

The interface between the DUNE \single \dword{pds} and Software Working Group in practice means the FD and ProtoDUNE Simulation/Reconstruction groups. Below are the various items where there is an overlap of responsibility. Where one group has clear responsibility is marked, otherwise the responsibility is shared between the groups for the time being. 

Interfaces among all three groups:

\begin{itemize}
    \item Generating photon libraries, and the tools for doing so
%    \item Maintaining and improving simulation tools and reconstruction algorithms
%    \item Maintaining and improving general analysis tools
    \item Maintaining and improving simulation tools, reconstruction algorithms and general analysis tools
    \item Maintain and improve the simulated geometry
    \item Choosing and implementing optical and detector properties to simulate (Consortium)
\end{itemize}

Interfaces between the consortium and the FD Simulation/Reconstruction group:
\begin{itemize}
    \item Developing simulation solutions for large detectors (FD Sim/Reco)
    \item Simulating and evaluating performance on nucleon decay events (Consortium)
    \item Simulating and evaluating performance on supernova events (Consortium)
    \item Evaluating the potentials of alternative design options (Consortium)
\end{itemize}

Moreover, the Physics and Software Working Groups from the \single \dword{pds} have the joint responsibility to provide the required simulation studies to understand whether the \single \dword{pds} technical design meets the physics goals.

Simulation studies of relevance include:
\begin{itemize}
    \item Expected light yields in the \single \dword{pds} system
    \item \single \dword{pds} reconstruction performance studies
    \item \single \dword{pds} technical design optimization studies
\end{itemize}
    
Interfaces between the consortium and the ProtoDUNE Simulation/Reconstruction group:

\begin{itemize}
    \item Algorithms for matching flashes to cosmic ray tracks
    \item Analyzing the light produced by cosmic ray tracks to evaluate photon detector performance (Consortium)
    \item Using the light to tag and exclude cosmic ray tracks from beam particle analysis (ProtoDUNE)
    \item Analyzing the light produced by various species of charged particles
\end{itemize}

\end{comment}
%***************************




%>> Start: Ernesto Kemp Feb/10/2018 <<<<<<<<<<<<<
%>> Revision: Ernesto Kemp & Norm Buchanan Mar/15/2018 <<<<<<<<<<<<<
%>> Revision: Ernesto Kemp, Yasar Onel & David Warner Nov/23/2018 <<<<<<<<<<<<<
%>> Revision: Ernesto Kemp, Yasar Onel & David Warner Jan/08/2019 <<<<<<<<<<<<<
