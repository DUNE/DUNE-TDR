%%%%%%%%%%%%%%%%%%%%%%%%%%%%%%%%%%%%%%%%%%%%%%%%%%%%%%%%%%%%%%%
\section{System Interfaces}
\label{sec:fdsp-pd-intfc}
%\metainfo{\color{blue} Content: Onel, Kemp}
%\metainfo{(Length: \dword{tdr}=10 pages, TP=2 pages)}

%\fixme{Include an image of each interface in appropriate sections.}

%>> Revision: Ernesto Kemp, Yasar Onel, David Warner Nov/23/2018 >>>>>>>>>>>>>
%>> Revision: Ernesto Kemp & Norm Buchanan Mar/15/2018 >>>>>>>>>>>>>
%>> Start: Ernesto Kemp Feb/10/2018 >>>>>>>>>>>>>  

This section describes the interface between the \dword{spmod} \dword{pds} and several other consortia, task forces (TF) and subsystems listed below:

%\begin{enumerate}
\begin{itemize}
\item \dword{apa} (see Chapter~\ref{ch:fdsp-apa}),
\item{\fdth{}s} (see Chapter~??? where are they - cryo?),
\item TPC \dword{ce} (see Chapter~\ref{ch:fdsp-tpcelec}), 
\item{\dword{cpa} / \dword{hv} System -- if the coated-reflector foils option is implemented} (see Chapter~\ref{ch:fdsp-hv}),
\item{Cryogenics System -- if the xenon-doping option is implemented} (see Chapter~????), 
\item \dword{daq} (see Chapter~\ref{ch:fdsp-daq}),
\item Calibration and Monitoring (see Chapter~\ref{ch:fdsp-calib}).
\end{itemize}
%\end{enumerate}

\fixme{Anne: please check that the labels for the other chapters I have used are correct. }
%
The contents of this section are focused on what is needed to complete the design, fabrication, installation of the related subsystems, and are organized by the elements of the scope of each subsystem at the interface between them.


%%%%%%%%%%%%%%%%%%%%%%%%%%%%%%%%%%%
\subsection{Anode Plane Assembly}
\label{sec:fdsp-pd-intfc-apa}

%The Technical Proposal may include any suggestions or requests from the Photon Detector Consortium for changes to the \dword{apa} design. Any such changes need to be proposed as early as possible to ensure consideration within the \dword{apa} Consortium's schedule. Some examples include:
%\begin{itemize}
%\item Additional light collector bays
%\item Alternate outer \dword{apa} designs
%\end{itemize}


%The \dword{pd}S is integrated in the \dword{apa} frame to form a single unit for the detection of both ionization charge and scintillation light.

%\textbf{Hardware:}
%The hardware interface between \dword{apa} and \dword{pd}S has two main components:
%\begin{enumerate}
%\item Mechanical: a) supports for the \dword{pd}S detectors; b) access slots for installation of the detectors, if, as in the present baseline design, the detectors will be installed after the wire winding is completed; c) access slots for the cabling of the \dword{pd}S detectors; d) routing of the \dword{pd}S cables inside the side beams of the \dword{apa} frame.
%\item Electrical: grounding scheme and electrical insulation, to be defined together with the \dword{ce} consortium, given the \dword{ce} strict requirements on noise.
%\end{enumerate}

%\textbf{Design:}
%Since the design of the \dword{pd}S detectors is still under development, the \dword{apa} consortium will evaluate different solutions proposed by the \dword{pd}S consortium that may require modifications of the structure of the \dword{apa} frame. The evaluation phase is expected to be concluded by spring 2018. The \dword{pd}S consortium will provide detailed engineering drawings of the detectors and specifications on the size and number of cables, and of any connectors required. 
%If the \dword{pd}S detectors will be installed after the wire winding is completed, as in the present baseline design, the \dword{apa} consortium will evaluate possible variations of the baseline geometry, as suggested by the \dword{pd}S consortium: larger slot dimensions and increased number of slots.
%If photon detectors are installed in the \dword{apa} frame prior to the winding, the \dword{pd}S consortium is responsible for designing the supports inside the \dword{apa} frame and providing a protection for the detectors against UV light. The \dword{apa} consortium will evaluate the proposed design.
%The \dword{apa} consortium, together with the \dword{ce} and \dword{pd}S consortia, will revisit the requirements on the mesh pitch and wire size, and set specifications for the mesh procurement, compatibly with the availability on the market.
%The \dword{apa} consortium will evaluate the feasibility of routing the \dword{pd}S detector cables inside the side beams of the \dword{apa} frame, considering also the cabling needs of the Cold Electronics consortium. This may require the evaluation of side beams of larger dimensions.

%\textbf{Production:}
%The \dword{apa} Consortium fabricates all \dword{apa}-\dword{pd}S mechanical interface hardware.
%The \dword{pd}S consortium fabricates all \dword{pd}S detectors and electronic boards, and will provide all cables and connectors.

%\textbf{Testing:}
%The interface hardware and the interconnect procedures between \dword{apa} and \dword{pd}S, including cabling, are validated at one or more integration/installation trials at an integration test facility. The \dword{apa} and \dword{pd}S consortia will be responsible for the procurement of their respective hardware and delivery to the integration test facility. Experts from both groups work with the installation team to perform trial fit and cabling before the final design is completed.  Electrical test will be performed with a \dword{apa}/\dword{pd}S/\dword{ce} vertical slice test if available.

%\textbf{Commissioning: }
%The \dword{pd}S consortium will provide procedures and personnel for the commissioning of the \dword{pd}S components.
%The \dword{apa} consortium will be responsible for applying the bias voltages on the wire planes. 

The \dword{pd} is integrated in the \dword{apa} frame to form a single unit for the detection of both ionization charge and scintillation light. 
The hardware interface between \dword{apa} and \dword{pds} is both mechanical and electrical:

\begin{itemize}

\item Mechanical: (1) supports for the \dword{pds} detectors; (2) access slots for installation of the detectors;
% if, as in the present baseline design, the detectors will be installed after the wire winding is completed; 
 (3) access slots for the cabling of the \dword{pd} detectors; 
 %d) routing of the \dword{pds} cables inside the side beams of the \dword{apa} frame.
 (4) \dword{pd} cables inside the \dword{apa} frames will be preassembled prior to wrapping the \dword{apa} frames. Cable installation will occur at \dword{apa} site.
 
\item Electrical: (1) grounding scheme and electrical insulation, to be defined together with the \dword{ce} consortium, given the \dword{ce} strict requirements on noise; (2) \dword{pd} cabling operation at SURF: 2.i) connecting the upper to the lower \dword{apa} in a stack, 2.ii) cabling from the \dword{apa} top to cryostat flange, 2.iii) cabling from the flange to \Dword{fe} electronics, 2.iv) cabling from the \Dword{fe} electronics to \dword{daq}.

\end{itemize} 


%%%%%%%%%%%%%%%%%%%%%%%%%%%%%%%%%%%
\subsection{Feedthroughs}
\label{sec:fdsp-pd-intfc-feed}

Several \dword{pd} \dword{sipm} signals are summed together into a single readout channel. A long multi-conductor cable with four twisted pairs read out the \dword{pd} module. 
%Since there is no \dword{ce} associated with the \dword{pd}S, the un-amplified 
Analog signals from the \dwords{sipm} are transmitted directly by cables to the appropriate flanges to outside the cryostat. 
All cold cables originating from the inside the cryostat connect to the outside warm electronics through PCB board feedthroughs installed in the signal flanges that are distributed along the cryostat roof.

All technical specifications for the feedthroughs should be provided by the photon detector group. 


%%%%%%%%%%%%%%%%%%%%%%%%%%%%%%%%%%%
\subsection{TPC Cold Electronics}
\label{sec:fdsp-pd-intfc-ce}

%\hspace{0.5cm}\textbf{Hardware: }
%The hardware interfaces between the \dword{ce} and \dword{pd}S occur on the chimneys and in the racks mounted on the top of the cryostat which house low and high voltage power supplies for \dword{pd}S, low and bias voltage power supplies for \dword{ce}, as well as equipment for the Slow Control and Cryo Instrumentation (SC in the following), and possibly \dword{daq} consortia. There should be no electrical contact between the \dword{pd}S and \dword{ce} components except for sharing a common reference voltage point (ground) at the chimneys. An additional indirect hardware interface takes place inside the cryostat where the \dword{ce} and \dword{pd}S components are both installed on the \dword{apa} (responsibility of the single phase far detector \dword{apa} consortium, \dword{apa} in the following), with cables for \dword{ce} and \dword{pd}S that may be physically located in the same space in the \dword{apa} frame, and where the cables and fibers for \dword{ce} and \dword{pd}S may share the same trays on the top of the cryostat (these trays are the responsibility of the facility and the installation of cables and fibers will follow procedures to be agreed upon in consultation with the underground installation team, \dword{uit} in the following).

%\textbf{Chimneys: }in the current design \dword{ce} and \dword{pd}S use separate flanges for the cold/warm transition and each consortium is responsible for the design, procurement, testing, and installation, of their flange on the chimney, together with the LBN facility that is responsible for the design of the cryostat.
%Racks on top of the cryostat: the installation of the racks on top of the cryostat is a responsibility of the facility, but the exact arrangement of the various crates inside the racks will be reached after common agreement between the \dword{ce}, \dword{pd}S, SC, and possibly \dword{daq} consortia. The \dword{pd}S and \dword{ce} consortia will retain all responsibilities for the selection, procurement, testing, and installation of their respective racks, unless for space and cost considerations an agreement is reached where common crates are used to house low voltage or high/bias voltage modules for both \dword{pd}S and \dword{ce}. Even if both \dword{ce} and \dword{pd}S plan to use floating power supplies, the consequences of such a choice on possible cross-talk between the systems needs to be studied. 

%\textbf{Electrical contacts between \dword{pd}S and \dword{ce} components:} there should be no electrical contact between the \dword{ce} and \dword{pd}S components, neither inside nor outside the cryostat, with the exception of the use of the same reference voltage (grounding) on the chimneys, with each of the \dword{ce} and \dword{pd}S has a separate connection to the detector ground (the cryostat).

%\textbf{Software:} there are no direct interfaces between the \dword{ce} and \dword{pd}S systems. 

%\textbf{Test stands and integration facilities: }various test stands and integration facilities will be developed. In all cases the \dword{ce} and \dword{pd}S consortia will be responsible for the procurement, installation, and initial commissioning of their respective hardware in these common test stands. The main purpose of these test stands is study the possibility that one system may induce noise on the other, and the measures to be taken to minimize this cross-talk. For these purposes, it is desirable to repeat noise measurements whenever new, modified detector components are available for one or the other consortium. This requires that the \dword{ce} and \dword{pd}S consortia agree on a common set of tests to be performed and that the \dword{ce} consortium can operate the \dword{pd}S detectors within a pre-determined range of operating parameters, and vice versa, without the need of providing personnel from the \dword{pd}S consortium when the \dword{ce} consortium is performing tests or vice versa. Procedures should be set in place to decide the time allocation to tests of the components of one or the other consortium.

%\textbf{Installation:} the installation of the cables for the \dword{ce} and \dword{pd}S requires coordination with the \dword{apa} consortium and the \dword{uit}. This applies both for the routing of the \dword{ce} and \dword{pd}S cables through the \dword{apa} frames while the \dwords{apa} are hanging in the staging area ({\it toaster}) and later, after the \dword{apa} has been moved inside the cryostat, for the routing of the cables in the trays hanging from the top of the cryostat. The \dword{ce} consortium will retain the responsibility for the \dword{ce} cables, and similarly for the \dword{pd}S consortium, but the possibility that the \dword{ce} and \dword{pd}S cables may need to be routed together through the \dword{apa} frames, may require that the two installation teams cooperate for this task.

%\hspace{0.5cm}
The hardware interfaces between the \dword{ce} (see Chapter~\ref{ch:fdsp-tpcelec}) and \dword{pd} occur in the \fdth{}s and the racks mounted on the top of the cryostat which house low and high voltage power supplies for \dword{pd}, low and bias voltage power supplies for \dword{ce}, as well as equipment for the \dword{cisc}, and possibly \dword{daq} consortia. There should be no electrical contact between the \dword{pd}S and \dword{ce} components except for sharing a common reference voltage point (ground) at the \fdth{}s. An additional indirect hardware interface takes place inside the cryostat where the \dword{ce} and \dword{pd} components are both installed on the \dword{apa} (responsibility of the single phase far detector \dword{apa} consortium, \dword{apa} in the following), with cables for \dword{ce} and \dword{pd} that may be physically located in the same space in the \dword{apa} frame, and where the cables and fibers for \dword{ce} and \dword{pd} may share the same trays on the top of the cryostat. These trays are the responsibility of the facility and the installation of cables and fibers will follow procedures to be agreed upon in consultation with the underground installation team, \dword{uit} in the following.

In the current design \dword{ce} and \dword{pd} use separate flanges for the cold-to-warm transition and each consortium is responsible for the design, procurement, testing, and installation, of their flange on the \fdth{}, together with LBNF, who is responsible for the design of the cryostat. 
The installation of the racks on top of the cryostat is a responsibility of the facility, but the exact arrangement of the various crates inside the racks will be reached after common agreement between the \dword{ce}, \dword{pd}, \dword{cisc}, and possibly \dword{daq} consortia. The \dword{pd} and \dword{ce} consortia will retain all responsibilities for the selection, procurement, testing, and installation of their respective racks, unless for space and cost considerations an agreement is reached where common crates are used to house low voltage or high/bias voltage modules for both \dword{pd}S and \dword{ce}. Even if both \dword{ce} and \dword{pd} plan to use floating power supplies, the consequences of such a choice on possible cross-talk between the systems needs to be studied. 

Various test stands and integration facilities will be developed. In all cases the \dword{ce} and \dword{pd} consortia will be responsible for the procurement, installation, and initial commissioning of their respective hardware in these common test stands. The main purpose of these test stands is study the possibility that one system may induce noise on the other, and the measures to be taken to minimize this cross-talk. For these purposes, it is desirable to repeat noise measurements whenever new, modified detector components are available for one or the other consortium. This requires that the \dword{ce} and \dword{pd} consortia agree on a common set of tests to be performed and that the \dword{ce} consortium can operate the \dword{pd}S detectors within a pre-determined range of operating parameters, and vice versa, without the need of providing personnel from the \dword{pd}S consortium when the \dword{ce} consortium is performing tests or vice versa. Procedures should be set in place to decide the time allocation to tests of the components of one or the other consortium.


%%%%%%%%%%%%%%%%%%%%%%%%%%%%%%%%%%%
%\subsection{Cathode Plane Assembly and High Voltage System: Reflector foils (light enhancement)}
\subsection{Cathode Plane Assembly and High Voltage System}
\label{sec:fdsp-pd-intfc-le}

%This section describes the interface between a light collection boosting system of the SP-\dword{pd}S  under investigation and the \dword{hv} system. These systems interact in the case that the photon detection system includes wavelength-shifting reflector foils mounted on the \dword{cpa} (see Section~\ref{sec:fdsp-pd-enh}).

%\subsubsection{Hardware: }
%The purpose of installing the wavelength-shifting (\dword{wls}) foils is to allow enhanced detection of light from events near to the cathode plane of the detector. The \dword{wls} foils consist of a wavelength shifting material (likely tetraphenyl butadiene - TPB) coated on a reflective backing material. The foils would be mounted on the surface of the cathode plane array (\dword{cpa}) in order to enhance light collection from events occurring nearer to the \dword{cpa}, and thus greatly enhancing the spatial uniformity of the light collection system as detected at the \dword{apa} mounted light sensors. The foils may be laminated on top of the resistive kapton surface of the \dword{cpa} frames, with the option of using metal fasteners/tacks that would also serve to define the field lines. Production of the FR4+resistive kapton \dword{cpa} frames are the responsibility of the \dword{hv} consortium. Production and TPB coating of the \dword{wls} foils will be the responsibility of the Photon Detection consortium. The fixing procedure for applying the \dword{wls} foils onto the \dword{cpa} frames and any required hardware will be the responsibility of the Photon Detection consortium, with the understanding that all designs and procedures will be pre-approved by the \dword{hv} consortium. The assembly procedure of the \dword{cpa}/FC module could become more complex due to the presence of delicate \dword{wls} foils. This new detector component has not been implemented in protoDUNE, so potential performance and stability degradation effects (due, for example, to ion accumulation at the \dword{cpa} surface) will not be tested.  Intense R\&D will be required before deciding on its implementation.

%\textbf{R\&D studies: }It will be the responsibility of the \dword{hv} consortium to define testing requirements that will need to be carried out in order to verify that the proposed \dword{wls} foil installation will not degrade the performance of the \dword{cpa}. It will be the responsibility of the \dword{pd} consortium to carry out these tests.

%\textbf{Integration:} An integration test stand will likely be employed to verify the proper operation of the \dword{cpa} panels with the addition of \dword{wls} foils under high voltage conditions. Light performance (wavelength conversion and reflectivity efficiency) will also be verified. The \dword{hv} consortium will be responsible for \dword{hv} aspects of the test stand and the \dword{pd} consortium will be responsible for the light performance aspects.

%\textbf{Installation:} The wavelength shifting material (TPB) can degrade under certain environmental conditions, so its addition to the surface of the \dwords{cpa} will increase the handling requirements during installation. The \dword{pd} consortia will be responsible for defining the environmental and handling procedures that will ensure minimal degradation of the \dword{wls} foil performance. The \dword{pd} consortium will be responsible for the installation of the \dword{wls} foils onto the \dword{cpa} panels. The \dword{hv} consortium will be responsible for all other aspects of the installation of the \dword{cpa} panels.

%\textbf{Commissioning:} \dword{pd} and \dword{hv} consortia will provide staffing for commissioning the \dwords{cpa} in the cryostat in the following manner: Specialists from the \dword{pd}S Consortium will be responsible to fix the foils on the \dword{cpa} surface. The \dword{hv} Consortium has the oversight responsibility for this task.


The \dword{pd} and the \dword{hv} systems (see Chapter~\ref{ch:fdsp-hv}) interact in the case that the former includes wavelength-shifting reflector foils mounted on the cathode plane array (\dword{cpa}). An additional interface is addressed in Section~\ref{sec:fdsp-pd-intfc-calib}.

The purpose of installing the wavelength-shifting (\dword{wls}) foils is to allow enhanced detection of light from events near to the cathode plane of the detector. The \dword{wls} foils consist of a wavelength shifting material (such as TPB, PEN) coated on a reflective backing material. The foils would be mounted on the surface of the \dword{cpa} in order to enhance light collection from events occurring nearer to the \dword{cpa}, and thus greatly enhancing the spatial uniformity of the light collection system as detected at the \dword{apa} mounted light sensors. The foils may be laminated on top of the resistive Kapton surface of the \dword{cpa} frames, with the option of leaving small apertures in the foils to allow the field lines to reach the CPA. Metal fasteners or tacks could be used to attach the foils and would then also serve to define the field lines. 

Production of the FR4+resistive Kapton \dword{cpa} frames will be the responsibility of the \dword{hv} consortium.
%are the responsibility of the \dword{hv} consortium. Production and \dword{tpb} coating of the \dword{wls} foils will be the responsibility of the Photon Detection consortium. 
The \dword{tpb} coating of the \dword{wls} foils and the fixing procedure for applying the \dword{wls} foils onto the \dword{cpa} frames and any required hardware will be the responsibility of the Photon Detection consortium, with the understanding that all designs and procedures will be pre-approved by the \dword{hv} consortium. 

This new detector component is not being tested in \dword{pdsp}, however, it has been proposed to be tested in the second run of \dword{pdsp}. Its integration \dword{hv} system could imply performance and stability degradation (due for example to ion accumulation at the \dword{cpa} surface); the assembly procedure of the \dword{cpa}/FC module could become more complex due to the presence of delicate \dword{wls} foils. Intense R\&D will be required before deciding on its implementation.

An integration test stand will likely be employed to verify the proper operation of the \dword{cpa} panels with the addition of \dword{wls} foils under high voltage conditions. Light performance (wavelength conversion and reflectivity efficiency) will also be verified. The \dword{hv} consortium will be responsible for \dword{hv} aspects of the test stand and the \dword{pd} consortium will be responsible for the light performance aspects.


\subsection{Cryogenics System}
\label{sec:fdsp-pd-intfc-xeon}

The \dword{pd} and the Cryogenics systems interact in the case that the former includes pre-mixing xenon gas and argon gas to introduce xenon doping into the \lar volume. 

The purpose of xenon doping is to allow enhanced detection of light from events near to the cathode plane of the detector and to increase the light yield of the \dword{pds}. A location where the pre-mixing could occur has been identified in principle but no formal process has been initiated to get approval for a modification to the cryogenics system.

Any required hardware will be the responsibility of the Photon Detection consortium, with the understanding that all designs and procedures will be pre-approved by the Cryogenics group. 

This new detector component was not tested in the 2018 \dword{pdsp} operation, but may be proposed for future running. Intense R\&D will be required before deciding on its implementation.



%%%%%%%%%%%%%%%%%%%%%%%%%%%%%%%%%%%
\subsection{Data Acquisition}
\label{sec:fdsp-pd-intfc-daq}

%This section describes the Photon Detector interfaces and related requirements with the \dword{daq} system described in Section~\ref{ch:fdsp-daq}.

%\textbf{Data Physical Links: }Data are passed from the \dword{pd}S to the \dword{daq} on optical links conforming to an IEEE Ethernet standard. The links run from the \dword{pd}S readout system on the cryostat to the \dword{daq} system in the Central Utilities Cavern (\dword{cuc}).

%\textbf{Data Format:} Data are encoded using a data format based on UDP/IP. The data format is derived from the one used by the Dual Phase TPC readout. Details will be finalized by the time of the \dword{daq} \dword{tdr}.

%\textbf{Data Timing:} The data shall contain enough information to identify the time  at which it was taken.

%\textbf{Data Volume:} The \dword{daq} will have the capacity to receive up to 8~GBit/s of data from the \dword{pd}S per \dword{apa}.

%\textbf{Data Link Speed: }The \dword{pd}S data for each \dword{apa} may be transmitted either on multiple links following the 1000Base-SX standard or a single link following the 10GBase-SR standard. In either case the fibre will be chosen to give sufficient margin for the distance from the cryostat to the \dword{cuc}. Details will be finalized by the time of the \dword{daq} \dword{tdr}.

%\textbf{Trigger Information:} The \dword{pd}S may provide summary information useful for data selection. If present, this will be passed to the \dword{daq} on the same physical links as the remaining data.

%\textbf{Timing and Synchronization: }Clock and synchronization messages will be propagated from the \dword{daq} to the \dword{pd}S using a backwards compatible development of the protoDUNE Timing System protocol ( See Dune docdb-1651 ). There will be at least one timing fibre available for each data links coming from the \dword{pd}S. Power-on initialization and Start of Run setup:  The \dword{pd}S may require initialization and setup on power-on and start of run. Power on initialization should not require communication with the \dword{daq}. Start run/stop run and synchronization signals such as accelerator spill information will be passed by the timing system interface.

%\textbf{Local Monitoring:} The \dword{pd}S may require network connections for local monitoring and debugging. These are the responsibility of the \dword{pd}S.

%\textbf{Software:} There should be no software required for the \dword{pd}S to \dword{daq} interface. The definition of the data format should provide the required information. 

%\textbf{Interaction with other groups: }Related interface documents describe the interface between the \dword{ce} and LBNF, \dword{daq} and LBNF, \dword{daq} and Photon and both \dword{daq} and \dword{ce} with Technical Coordination. The cryostat penetrations including through-pipes, flanges, warm interface crates and feedthroughs and associated power and cooling are described in the LBNF/\dword{pd}S interface document.  The rack, computers, space in the \dword{cuc} and associated power and cooling are described in the LBNF/\dword{daq} interface document. Any cables associated with photon system data or communications are described in the \dword{daq}/Photon interface document. Any cable trays or conduits to hold the \dword{daq}/\dword{ce} cables are described in the LBNF/Technical Coordination interface documents and currently assumed to be the responsibility of Technical Coordination.

%\textbf{Integration:} Various integration facilities are likely to be employed, including vertical slice tests stands, \dword{pd}S test stands, \dword{daq} test stands and system integration/assembly sites. The \dword{daq} consortia will provide hardware and software for a “vertical slice test”. The \dword{pd}S consortia will provide \dword{pd}S emulators and \dword{pd}S readout hardware for \dword{daq} test stands. (The \dword{pd}S emulator and \dword{pd}S readout hardware may be the same physical object with different configuration ). Responsibility for supply and installation of \dword{daq}/\dword{pd}S cables in these tests will be defined by the time of the \dword{daq} \dword{tdr}.

%\textbf{Installation: }Responsibility for purchase of the \dword{daq}/\dword{ce} cables is assigned to the \dword{pd}S. The installation of the  \dword{daq}/\dword{ce} cables is assigned to the \dword{pd}S.


%This section describes the \dword{pd} interfaces and related requirements with the \dword{daq} system described in Section~\ref{ch:fdsp-daq}. Here we list the main system interface areas.
The main system interfaces with the DAQ system (see Chapter~\ref{ch:fdsp-daq})include:

%\fixme{Is there a \dword{daq} chapter section on \dword{pd} to refer to?}

\textbf{Data Physical Links: }Data are passed from the \dword{pd} to the \dword{daq} on optical links conforming to an IEEE Ethernet standard. The links run from the \dword{pd} readout system on the cryostat to the \dword{daq} system in the Central Utilities Cavern (\dword{cuc}).

\textbf{Data Format:} Data are encoded using a data format based on UDP/IP. The data format is derived from the one used by the Dual Phase TPC readout. Details will be finalized by the time of the \dword{daq} \dword{tdr}.

\textbf{Data Timing:} The data must contain enough information to identify the time  at which it was taken.

\textbf{Trigger Information:} The \dword{pd} may provide summary information useful for data selection. If present, this will be passed to the \dword{daq} on the same physical links as the remaining data.

\textbf{Timing and Synchronization: }Clock and synchronization messages will be propagated from the \dword{daq} to the \dword{pd} using a backwards compatible development of the \dword{pdsp} Timing System protocol (DUNE docdb-1651). There will be at least one timing fiber available for each data links coming from the \dword{pd}S. Power-on initialization and Start of Run setup:  The \dword{pd}S may require initialization and setup on power-on and start of run. Power on initialization should not require communication with the \dword{daq}. Start run/stop run and synchronization signals such as accelerator spill information will be passed by the timing system interface.

\textbf{Interaction with other groups: }Related interface documents describe the interface between the \dword{ce} and LBNF, \dword{daq} and LBNF, \dword{daq} and Photon and both \dword{daq} and \dword{ce} with Technical Coordination. The cryostat penetrations including through-pipes, flanges, warm interface crates and feedthroughs and associated power and cooling are described in the LBNF/\dword{pd}S interface document.  The rack, computers, space in the \dword{cuc} and associated power and cooling are described in the LBNF/\dword{daq} interface document. Any cables associated with photon system data or communications are described in the \dword{daq}/Photon interface document. Any cable trays or conduits to hold the \dword{daq}/\dword{ce} cables are described in the LBNF/Technical Coordination interface documents and currently assumed to be the responsibility of Technical Coordination.

\textbf{Integration:} Various integration facilities are likely to be employed, including vertical slice tests stands, \dword{pd}S test stands, \dword{daq} test stands and system integration/assembly sites. The \dword{daq} consortia will provide hardware and software for a vertical slice test. The \dword{pd} consortium will provide \dword{pd} emulators and \dword{pd} readout hardware for \dword{daq} test stands. (The \dword{pd} emulator and \dword{pd} readout hardware may be the same physical object with different configuration). Responsibility for supply and installation of \dword{daq}/\dword{pd} cables in these tests will be defined by the time of the \dword{daq} \dword{tdr}.



%%%%%%%%%%%%%%%%%%%%%%%%%%%%%%%%%%%
\subsection{Calibration and Monitoring}
\label{sec:fdsp-pd-intfc-calib}

%\todo{Content: Kemp/Onel/Djurcic}

%\fixme{this section describes the the C\&M interfaces - a new section is needed in the System Design and Production sections to describe the system itself}

%This subsection concentrates on the description of the interface between the SP-\dword{pd}S and Calibration/Monitoring Task Force (CTF), since there are components of the system planned to be installed with the \dword{hv} system Cathode, and through \dword{fc} strips and File Cage \dword{gp}.

%\textbf{Hardware:} The SP-\dword{pd}S has proposed the photon-detector gain and timing calibration system to be also used for SP-\dword{pd}S monitoring purposes during commissioning and experimental operation. A pulsed UV-light system is proposed to cross-calibrate and monitor the DUNE-SP photon detectors. The hardware consists of warm and cold components. By placing light sources and diffusers on the cathode planes designed to illuminate the anode planes the photon detectors embedded in the anode planes can be illuminated. Cold component (diffusers and fibers) interface with High-Voltage and will be described in a separate interface document. Warm components include controlled pulsed-UV source and warm optics. These warm components will interface CTF with Slow-Controls/\dword{daq} subsystems and will be described in corresponding documents. Optical feedthrough is the cryostat interface. Hardware components will be designed and fabricated by SP-\dword{pd}S. 

%Other aspects of hardware interfaces are described in the following. The CTF and \dword{pd}S groups might share rack spaces, which needs to be coordinated between both groups. There will not be dedicated ports for all calibration devices. Therefore, multi-purpose ports are planned to be shared between various groups. CTF and SP-\dword{pd}S will define ports for deployment. It is possible that SP-\dword{pd}S might use Detector Support Structure (DSS) ports or TPC signal ports for routing fibers. The CTF in coordination with other groups will provide a scheme for interlock mechanism of operating various calibration devices (e.g. Laser, radioactive sources) that will not be damaging to the \dword{pd}S. 

%The \dword{pd}S has proposed the photon-detector gain and timing calibration system. The system will be used for \dword{pd}S monitoring purposes during commissioning and for standard experimental operation. A pulsed UV-light system is proposed to cross-calibrate and monitor the DUNE-SP photon detectors. The hardware consists of warm and cold components. By placing light sources with diffusers on the cathode planes, the system is designed to illuminate the photon detectors embedded in the anode planes. The details are described in the DUNE Interface Document: SP-\dword{pd}S/CTF. Cold components of the calibration system (diffusers and fibers) interface with the \dword{hv} system. Diffusers are installed at \dword{cpa}, and therefore reside at the same \dword{cpa} potential. Quartz fibers are insulators used to transport light from optical feedthroughs (at the cryostat top) through Filed Cage \dword{gp}, and through Filed Cage strips to the \dword{cpa} top frame. These fibers are then optically connected to diffusers located at \dword{cpa} panels. Required fiber resistance is defined by \dword{hv} system requirements to ensure the cathode is protected from shorting out due to fiber conductivity. \dword{pd}S hardware components will be designed and fabricated by \dword{pd}S.

%\textbf{Firmware:} The firmware will enable UV-light system to interface to \dword{daq}/Slow-Controls to communicate start/stop of calibration run, and issue commands to define types (amplitude, timing, frequency) of calibration pulses. Protocols will be defined with \dword{daq}, but the firmware realization and testing will be responsibility of SP-\dword{pd}S. Timing and Synchronization: Clock and synchronization messages will be propagated from the \dword{daq} to the SP-\dword{pd}S calibration unit using a backwards compatible development of the protoDUNE Timing System protocol.
% No docDB reference (See Dune docdb-1651). 
%See also SP-\dword{pd}S to \dword{daq} interface definition.

%\textbf{Software:} The software interface between the groups consists of software needed to perform calibrations of the photon detection system and any simulations of the detector needed to develop calibration schemes. The calibration software which will analyze the photon detection input and calculate calibration quantities will be the responsibility of the SP-\dword{pd}S Consortium, with the guidance of the CTF. The CTF will be responsible for defining the quantities to be measured. The SP-\dword{pd}S Consortium will be responsible for providing a simulation model to test the calibration schemes. The CTF will provide the design and model of databases (DBs) to store the calibration information and these DBs will be filled out by the SP-\dword{pd}S Consortium.

%\textbf{Testing: }Some components of the system are being tested with protoDUNE. Additional tests will be managed between SP-\dword{pd}S and CTF if necessary, including a test stand with shared responsibility.

%\textbf{Integration: } Various integration facilities are likely to be employed, including vertical slice tests stands, cold electronics test stands, \dword{daq} test stands and system integration/assembly sites. The \dword{pd}S consortia will provide support for \dword{pd}S integration and operation. \dword{hv} system will verify \dword{hv} design and operation without discharges that could cause light emission observed by \dword{pd}S should this be a concern.

%\textbf{Installation: }Responsibility for fabrication/installation of \dword{pd}S Calibration components is assigned to \dword{pd}S. Responsibility for fabrication/installation of \dword{pd}S Calibration components is assigned to \dword{pd}S.

%\textbf{Commissioning:} \dword{pd}S-SP will provide staffing for commissioning of SP-\dword{pd}S calibration system in the cryostat. Cold \dword{pd}S-SP components need be installed with Cathode-Plane assemblies and \dword{fc} arrays, and possible with DSS. Warm SP-\dword{pd}S calibration components may be installed at the end and tested with \dword{daq}. Calibration scope and goals will be further defined within CTF. \dword{pd}S will provide staffing for commissioning of \dword{pd}S calibration system in the cryostat.


%This subsection concentrates on the description of the interface between the SP-\dword{pd} and Calibration/Monitoring Task Force (CTF), since there are components of the system planned to be installed with the \dword{hv} system Cathode, and through \dword{fc} strips and \dword{fc} \dword{gp}.


This subsection defines the internal calibration system for the \single \dword{pds}. 
%It may be interfaced to a calibration consortium later.

It is proposed that the \single \dword{pds} gain and timing calibration system, a pulsed UV-light system, also be used for \dword{pd} monitoring purposes during both commissioning and experimental operation. The hardware consists of warm and cold components.  
%The \single \dword{pds} has proposed that the \dword{pd} gain and timing calibration system also be used for \dword{pd} monitoring purposes during commissioning and experimental operation. A pulsed UV-light system is proposed to cross-calibrate and monitor the \dword{spmod} \dwords{pd}. The hardware consists of warm and cold components. 

%By placing light sources and diffusers on the cathode planes designed to illuminate the anode planes, the \dwords{pd} embedded in the \dwords{apa} can be illuminated. 
By placing diffusers designed to play a role of light sources on the cathode planes, the photon detectors embedded in the anode planes can be illuminated.
Cold components (diffusers and fibers) interface with \dword{hv} and are described in a separate interface document (DUNE docdb-6721). 
%\fixme{ref?}
%\fixme{DUNE docdb-6721 needs check} 
Warm components include controlled pulsed-UV source and warm optics. These warm components interface calibration and monitoring  with the \dword{cisc} and \dword{daq} subsystems, and are described in corresponding documents (DUNE docdb-6730 and 6727 respectively). %\fixme{ref?} 
Optical \fdth is a cryostat interface. 

Hardware components will be designed and fabricated by SP-\dword{pd}S. Other aspects of hardware interfaces are described in the following. The CTF and \dword{pd}S groups might share rack spaces that needs to be coordinated between both groups. There will not be dedicated ports for all calibration devices. Therefore, multi-purpose ports are planned to be shared between various groups. CTF and SP-\dword{pd} will define ports for deployment. It is possible that SP-\dword{pd} might use Detector Support Structure (DSS) ports or TPC signal ports for routing fibers. The CTF in coordination with other groups will provide a scheme for interlock mechanism of operating various calibration devices (e.g. laser, radioactive sources) that will not be damaging to the \dword{pd}. 

%The \dword{pd} has proposed the photon-detector gain and timing calibration system. The system will be used for \dword{pd} monitoring purposes during commissioning and for standard experimental operation. A pulsed UV-light system is proposed to cross-calibrate and monitor the DUNE-SP photon detectors. The hardware consists of warm and cold components. By placing light sources with diffusers on the cathode planes, the system is designed to illuminate the photon detectors embedded in the anode planes. The details are described in the DUNE Interface Document: SP-\dword{pd}S/CTF. Cold components of the calibration system (diffusers and fibers) interface with the \dword{hv} system. Diffusers are installed at \dword{cpa}, and therefore reside at the same \dword{cpa} potential. Quartz fibers are insulators used to transport light from optical feedthroughs (at the cryostat top) through \dword{fc} \dword{gp}, and through \dword{fc} strips to the \dword{cpa} top frame. These fibers are then optically connected to diffusers located at \dword{cpa} panels. Required fiber resistance is defined by \dword{hv} system requirements to ensure the cathode is protected from shorting out due to fiber conductivity. \dword{pd} hardware components will be designed and fabricated by \dword{pd}.


%>> Start: Ernesto Kemp Feb/10/2018 <<<<<<<<<<<<<
%>> Revision: Ernesto Kemp & Norm Buchanan Mar/15/2018 <<<<<<<<<<<<<
%>> Revision: Ernesto Kemp, Yasas Onel & David Warner Nov/23/2018 <<<<<<<<<<<<<





