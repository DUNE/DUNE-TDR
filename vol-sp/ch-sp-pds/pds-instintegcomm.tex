%%%%%%%%%%%%%%%%%%%%%%%%%%%%%%%%%%%%%%%%%%%%%%%%%%%%%%%%%%%%%%%
%\section{Installation, Integration and Commissioning}
\section{Transport and Handling}
\label{sec:fdsp-pd-install}
%\metainfo{Content: Onel, Kemp, Warner}

%>> Revision: Ernesto Kemp, Yasar Onel, David Warner Nov/23/2018 >>>>>>>>>>>>>

%
%\fixme{\color{blue} Yasar/Ernesto/Dave: Has the transfer of text to Jim Stewart's chapter been completed? More to be done in our chapter?}
%\fixme{\coor{red}Dave: This section changed to addressing only transport and handling.  All I and I discussion will move to Jim's chapter.  DWW  --- you write "will move", meaning that it hasn't yet? If you are done with this section except for the fixme below, just comment out this one.}

%=================================================

A storage facility near or at the \dword{fd} site (the \dword{sdwf}) will be established to allow storage of materials for detector assembly until needed.  Transport of assembled and tested PD modules, electronics, cabling, and monitoring hardware to the \dword{sdwf} is the responsibility of the \dword{pd} consortium.

%We have preliminary requirements for the environmental conditions.  Class 100,000 clean room, lighting specifications, and humidity requirements are met. 

%\fixme{Dave: Remove this fixme if you think it is no longer needed (RJW).  Ettore, Yasar, Ernesto:  I have added an intermediate testing stop in the US following shipping for PD modules before the are put into storage, since they may be in storage quite a while before testing}

Following assembly and quality management testing in Brazil the \dword{pd} modules  will be packaged and shipped to an intermediate testing facility in the US for post-shipping checkout. Following this, the modules will be stored in their shipping containers in the \dword{sdwf}.  Cables, readout electronics, and monitoring hardware will be shipped directly to the \dword{sdwf} and stored until needed underground for integration.

Packaging plans are informed by the \dword{pdsp} experience.  Each \dword{sp} module will be individually sealed into a light-tight anti-static plastic bag.  Bagged modules will be packaged in groups of ten modules (matching the need for a single \dword{apa} transported in a single shipping box), approximately \SI{20}{cm} $\times$ \SI{20}{cm} $\times$ \SI{250}{cm} long.  These shipping boxes will be gathered into larger crates to facilitate shipping.  The optimal number per shipment is being considered.

Documentation and tracking of all components and \dword{pd} modules will be required during the full logistics process. Well defined procedures are in place to ensure that all components/modules are tested and examined prior to, and after, shipping. Information coming from such testing and examinations will be stored in the \dword{dune} hardware database.  Each \dword{pd} module shipping bag will be labeled with a text and barcode label, referencing the unique ID number for the module contained, and allowing linkage to the hardware database upon unpacking prior to integration into the APAs underground.

Tests have been conducted and continue to validate environmental requirements for photon detector handling and shipping. The environmental condition specifications for lighting (SP-PDS-3 in Table~\ref{tab:specs:SP-PDS}), humidity (SP-PDS-4 in Table~\ref{tab:specs:SP-PDS}) and work area cleanliness (SP-PDS-1 in Table ~\ref{tab:specs:SP-PDS})
%in the specifications tables 
apply for surface and underground transport, storage and handling, and any exposure during installation and integration underground. 
%\fixme{Dave: There are hardwired Table reference here.}


%\fixme{let's find the right references after the rest of the mess is cleared up! Anne}

%\fixme{\color{blue}Dave: we need a link to the specifications table  Not sure how to do that right now.  SP-PDS-4, SP-PDS-4, 
%rjw - you can find the labels in the generated folder; look for req-SP-PDS-xxx, where xxx is the name of the specification in the spreadsheet e.g. req-SP-PDS-env-humidity.tex, e.g. the label for that table is tab:specs:SP-PDS, so this Table~\ref{tab:specs:SP-PDS} will reference the table that appears in the Design Specifications section.}
%\fixme{\color{blue}Dave:Apparently there is no class-100,000 clean room specification.  Do we want to add one? -- Up to you - rjw}

%Handling procedures that ensure environmental requirements are developed. This will include handling at all stages of component and system production and assembly, testing, shipping, and storage. It is likely that \dword{pd} modules and components will be stored for periods of time during production and prior to installation into the \dword{fd} cryostats. Appropriate storage facilities need to be constructed at locations where storage will take place. Shipping and storage containers need to be designed and produced. Given the large number of photon detector modules to be installed in the \dword{fd}, it will be cost effective to take advantage of reusable shipping containers.

Details of \dword{pd} integration into the \dword{apa} and installation into the cryostat, including quality management testing equipment, tests and documentation are included in \spchinstall{}, Detector Installation. 
%\fixme{DONE! Anne:  We need to reference the Single-Phase Technical Coordination chapter here.  Not sure how to do that:  Can you help?}



%  DWW>  Probably part of the integration section?  An Integration and Test Facility (ITF) will be constructed at a location to be decided by the collaboration/project for the integration of the \dword{pd}s into \dwords{apa}. Transportation to and from ITF should be carefully planned. The \dword{pd}S units will be shipped from the production area in quantities compatible with the \dword{apa} transport rates.
    
%Operations: The \dword{pd}S deliveries will be stored in temperature and humidity controlled storage area. Their mechanical status will be inspected.

%Transportation to SURF: The delivery to SURF will be such that the storage time before integration will be at most two weeks.


%%%%%%%%%%%%%%%%%%%%%%%%%%%%%%%%%%%
%\subsection{Integration with APA and Installation}
%\label{sec:fdsp-pd-install-pd-apa}

%\dword{pd} modules integration into the \dword{apa} frame will happen at the Integration facility. \dword{apa}s will be oriented horizontally for \dword{pd} modules integration. Experts from both groups will work with the installation team.  
%An electrical test with \dword{apa}/\dword{pd}S/\dword{ce} will be performed 
%at the integration facility 
%in the underground in a cold box, after the integration of \dword{pd}S and \dword{ce} on the \dword{apa} frame has been completed. During the cold test \dword{pd} modules will be operated for dark count measurements and LED illumination.

%The \dword{apa} consortium will be responsible for the transportation of the integrated \dword{apa} frames from the integration facility to the LBNF/SURF facility. 
%The \dword{uit} team, under supervision of the \dword{apa} group, will be responsible to move the equipment into the clean room. 
%Work on the 2-\dword{apa} connection and inspection underground, prior to installation in the cryostat, is performed by the \dword{apa} group.
%Work on cabling during this assembly process is performed by \dword{pd}S and \dword{ce} groups under supervision of the \dword{apa} group.
%Once the \dwords{apa} will be moved inside the cryostat, the \dword{pd}S and \dword{ce} consortia will be responsible for the routing of the cables in the trays hanging from the top of the cryostat. 

%Coldbox testing of integrated  \dword{apa}-\dword{ce}-\dword{pd} units will be done in underground prior to installation into the cryostat. To mitigate the imnplied risks, \dword{pd} modules will be criogenically tested at CSU prior shipping to ITF.

%%%%%%%%%%%%%%%%%%%%%%%%%%%%%%%%%%%
%\subsection{Installation into the Cryostat and Cabling}
%\label{sec:fdsp-pd-install-pd-cryo}

%The \dword{pd} modules are installed into the \dwords{apa}. There are ten \dword{pd}'s per \dword{apa}, inserted into alternating sides of the \dword{apa} frame, five from each direction. Once a \dword{pd} is inserted, it is attached mechanically to the \dword{apa} frame  and cabled up with a single power/readout cable. Following \dword{pd} installation cold electronics (\dword{ce}) units are installed at the top of the \dword{apa} frame. Cable connection will occur automatically with the modules integration. 

%After the \dword{apa} has been integrated with the \dword{pd}S and \dword{ce}, it will be moved to underground 
%via the rails in the clean room to the integrated 
%cold test stand 
%for testing 
%and then be moved into the cryostat. The two anode planes of the TPC will be assembled inside the cryostat, each of the fully tested \dwords{apa} mechanically linked together. Signal cables from the TPC readout and the \dword{pd} modules are routed up to the feedthrough flanges on the cryostat top side. The cables from each of the \dword{ce} and \dword{pd}'s on the \dword{apa} are then routed and connected to the final flanges on the cryostat.  These assembly, testing and final cabling steps will occur as part of APA ind CE installation (PD cables from the APAs to the flange will be connected as part of the CE cabling installation).  Specialized PD personnel will be provided as part of the installation effort, but the entire installation effort will be planned and directed by the integration and installation group.

%%%%%%%%%%%%%%%%%%%%%%%%%%%%%%%%%%%
%\subsection{Calibration and Monitoring}
%\label{sec:fdsp-pd-install-calib}
%\todo{Content: Djurcic}

%\fixme{\color{blue}Zelimir/Yasar/Ernesto:  Is the section complete?  --- Just the integration issues in this section.}

%Commissioning of the \dword{spmod} \dword{pds} will rely heavily on the readout electronics, \dword{daq}, and calibration and monitoring system.  Deployment and testing of the readout electronics separately from the in situ installation of photon detector modules in the \dword{apa} is important to establish their proper functioning before connection to the photon detectors or their flanges.  Careful checking at each step of the integration process will help to find unexpected problems early enough to be corrected before individual units are mounted into the larger systems (first in the \dword{apa} then after installation in the cryostat). 

%Once the electronics are read out out via the \dword{daq} system, it will be appropriate to add the \dword{pd} modules and continue commissioning of the installed system.  In order to be properly tested the \dword{pd} modules will have to be in the dark.  Making sure it is possible to make this check frequently enough to catch problems early is critical. This will have to be balanced with the needs of installation, as work progresses.  

%Once the basic operation of the readout system is established, the calibration and monitoring system will be of great use during the commissioning.  While the background signals from the warm photon detectors may make calibration difficult, the monitoring system will be able to flash UV light to excite the \dword{pd} modules.  These light signals can be used to determine that cabling is connected, and connected properly by looking at light from different UV emitters.  Once the detector is beginning to cool down, the operation of the calibration and monitoring system will become even more important as the monitoring of the individual channels should be a good indication of their proper operation, and again, the proper cabling and interface.

%>> Revision: Ernesto Kemp, Yasas Onel & David Warner Nov/23/2018 <<<<<<<<<<<<<

 
