%%%%%%%%%%%%%%%%%%%%%%%%%%%%%%%%%%%%%%%%%%%%%%%%%%%%%%%%%%%%%%%
\section{Transport and Handling}
\label{sec:fdsp-pd-install}
%\metainfo{Content: Onel, Kemp, Warner}

A storage facility near or at the \dword{fd} site (the \dword{sdwf}) will be established to allow storage of materials for detector assembly until needed.  Transport of assembled and tested PD modules, electronics, cabling, and monitoring hardware to the \dword{sdwf} is the responsibility of the \dword{pd} consortium.

Following assembly and quality management testing in Brazil, the \dword{pd} modules  will be packaged and shipped to an intermediate testing facility in the US for post-shipping checkout. Following this, the modules will be stored in their shipping containers in the \dword{sdwf}.  Cables, readout electronics, and monitoring hardware will be shipped directly to the \dword{sdwf} and stored until needed underground for integration.

Packaging plans are informed by the \dword{pdsp} experience.  Each \dword{sp} module will be individually sealed into a light-tight anti-static plastic bag.  Bagged modules will be packaged in groups of ten modules (matching the need for a single \dword{apa} transported in a single shipping box), approximately \SI{20}{cm} $\times$ \SI{20}{cm} $\times$ \SI{250}{cm} long.  These shipping boxes will be gathered into larger crates to facilitate shipping.  The optimal number per shipment is being considered.

Documentation and tracking of all components and \dword{pd} modules will be required during the full logistics process. Well defined procedures are in place to ensure that all components/modules are tested and examined prior to, and after, shipping. Information coming from such testing and examinations will be stored in the \dword{dune} hardware database.  Each \dword{pd} module shipping bag will be labeled with a text and barcode label, referencing the unique ID number for the module contained, and allowing linkage to the hardware database upon unpacking prior to integration into the \dword{apa}s underground.

Tests have been conducted and continue to validate environmental requirements for photon detector handling and shipping. The environmental condition specifications for lighting (SP-PDS-3 in Table~\ref{tab:specs:SP-PDS}), humidity (SP-PDS-4 in Table~\ref{tab:specs:SP-PDS}), and work area cleanliness (SP-PDS-1 in Table ~\ref{tab:specs:SP-PDS})
apply for surface and underground transport, storage and handling, and any exposure during installation and integration underground. 


Details of \dword{pd} integration into the \dword{apa} and installation into the cryostat, including quality management testing equipment, tests, and documentation are included in Chapter~\ref{ch:sp-install}. 

