\section{Risks}
\label{sec:fdsp-pd-risks}
%\metainfo{\color{blue} Content: Warner}
%New section 1/9/19; add same subsections as the TPC Electronics


In this Section we discuss the risks we identified during the design and prototyping the DUNE SP detector, particularly during the \dword{pdsp}. In the two following sections we will then discuss the risks that could be
met during commissioning and operation.  Table~\ref{tbl:SPPDrisks} contains a list of all the
risks that we are currently holding in the PD risk register.  The text below describes these risks, our current appraisal of the risk level, low (L), medium (M) or high (H), associated with the risks, and our proposed mitigation plans to allow us to eventually retire these risks.  The risk appraisal is a product of the probability of the occurrence of risk described and the schedule delay and cost associated with these risks.  

\begin{dunetable}
[SP PD System Risk Summary]
{p{0.15\textwidth}p{0.75\textwidth}}
{tbl:SPPDrisks}
{Single Phase Photon Detector Risk Summary}
ID & Risk                   \\ \toprowrule
1 & PD Modules don't collect enough light to meet system physics performance specifications \\ \colhline
2 & Timing performance of the PD electronics/SiPM/Cabling  does not meet the 1 us time resolution specification \\ \colhline
3 & Evolutions in the design of the photon detectors due to validation testing experience require modifications of the TPC elements at a late time \\ \colhline
4 & Cabling for PD and CE cannot be accommodated within the \dword{apa} frame or during the 2-APA assembly/installation procedure. \\ \colhline
5 & Experience with validation prototypes shows that the mechanical design of the PD is not adequate. \\ \colhline
6 & pTB WLS filter coating not sufficiently stable, contaminates \dword{lar}. \\ \colhline
7 & SiPMs fail due to multiple cold cycles or extended cryogen exposure. \\ \colhline
8 & SiPM active ganging cold amplifiers fail or degrade detector performance. \\ \colhline
9 & Previously undetected electro-mechanical interference discovered during integration\\ \colhline
10 & Design weaknesses manifest during module logistics-handling. \\ \colhline
11 & SiPM PD/CE Electrical Crosstalk \\ \colhline
12 & Lifetime of \dword{pd} components outside cryostat\\

\end{dunetable}

\subsection{Physics Performance Specification Risks}

Risk 1 in the table addresses the performance specification that the \dword{pd} system detect 0.5 pe/MeV of deposited energy.  The system may not reach this requirement.  Current design validation (\ref{sec:pds-validation}) provides firm indication that this specification will be met by the \dword{xarapu}, so a risk level of "L" is assigned to this risk.  Mitigation of this risk is being achieved by:  Allocating enough development resources to the PD to continue developing improved light collection modules, increasing the APA slot size to allow for larger modules, and possibly increasing the number of photosensors per \dword{xarapu} supercell.

Risk 2 addresses the performance specification that the PD system provide 1 us time resolution.  While the timing resolution specification has been met by the \dword{pdsp} SSP-based \dword{sarapu},  cost-saving modifications to the readout electronics may degrade the performance of the PD system below the 1 us requirement.  In addition, the combination of active and passive ganging of 48 photosensors could degrade timing performance.  Current design validation (\ref{sec:pds-validation}) provides firm indication that this specification will be met by the \dword{xarapu} and our baseline electronics, so a risk level of "L" is assigned to this risk.  Mitigation of this risk is being achieved by:  Allocating enough engineering resources to proceed rapidly with the design modifications of our reduced-cost baseline system, extensive testing of passive ganging prototypes, including parallel development of two design options for the active ganging circuit, and testing of timing performance in software simulation and multiple validation test stands.


\subsection{Design Risks}

Risk 3 addresses the interface of the \dword{apa} and \dword{pd} designs, and the possibility that in order to meet detector performance or reliability specifications the \dword{pd} design may evolve in a direction requiring modification of the \dword{apa}.  While our current design validation (\ref{sec:pds-validation}) provides firm indication that these specifications will be met by the \dword{xarapu}, we have not yet completed the validation process.  A risk level of "M" is assigned to this risk, due primarily to the significant potential costs (financial and schedule) associated with such a change following the TDR.  Mitigation of this risk involves close interaction between the \dword{apa} and \dword{pd} consortia, and assigning significant resources to \dword{pd} validation efforts.

Lessons learned during the \dword{pdsp} led to the re-design of the \dword{pd} cabling layout, moving the cables inside the APA frame where the will be unreachable following installation of the \dword{apa} wires.  Additionally, installation of the APAs into the cryostat will require making \dword{pd} cable connections between the upper and lower APAs underground.  Risk 3 addresses the concern that difficulties with these \dword{apa}/\dword{pd} interfaces will require changes to the cabling plan.  A risk level of "M" is assigned to this risk, due to limited current validation of the cable design.  Mitigation will consist of extensive validation tests, including full-scale integration tests at the Ashe River installation site.

Risk 5 concerns the possibility that continuing validation tests demonstrate that the \dword{pd} mechanical design is in some way not adequate to meet DUNE specifications.  The level associated with this risk is "L":  While validation is ongoing and the possibility or a required design change remains, the impact and cost of such a change is likely relatively low.  Mitigation includes continued design validation testing and sufficient engineering resources.

Risk 6 concerns the possibility that continuing validation tests demonstrate that the pTB coatings required on the dichroic filter plates are not sufficiently robust in cryogenic applications, and flake of dissolve off the surface and contaminate the \dword{lar}, possibly impacting electron lifetime of optical performance of the detector.  The level associated with this risk is "M":  Experience in \dword{pdsp} suggested that coatings of the filters is a delicate operation, and to possibility exists to produce unstable coatings.  Mitigation includes continued validation testing of coated filters and sufficient engineering resources.

One of the most significant lessons of the \dword{pdsp} for the \dword{pd} system was the failure of a significant number of photosensors during module assembly QC, due to an unannounced change in the manufacturer's photosensor packaging procedures.  Problems developed with initially reliable photosensors mid-way through fabrication, requiring rapid changes to the \dword{pd} design.  Risk 7 addresses the possibility of a re-occurrence of this or a similar problem.  This risk is assigned a level of "M", due to the difficulties of preventing changes in manufacturing processes outside our direct control.  Mitigation includes extensive QA testing prior to selecting the final photosensor candidate, careful coordination with photosensor vendor(s), and a rigerous QC testing procedures for photosensors.

Risk 8 addresses to possibility of a denigration in performance in \dword{pd} performance or outright failure due to the cold amplifiers required by the active ganging circuitry.  In order to reach the baseline design of 48 ganged photosensors per \dword{xarapu} supercell, a mix of active and passive ganging is required.  While initial validation testing is very promising, these circuits remain quite new, and a risk level of "M" has been assigned.  Mitigation of this risk involves additional validation testing in bench-top testing and in the \dword{iceberg} test stand.

\subsection{Risks During Integration}

Risk 9 addresses the possibility that a previously undetected flaw in the \dword{pd} module design or the integration plan with the \dwords{apa} manifests itself during the integration process.  Due to extensive validation testing of this interface which will have occured prior to integration, including full-scale integration tests at Ashe River, this risk is assigned a value of "L".  Steps taken to mitigate this risk include close coordination between the \dword{pd}, \dword{apa}, \dword{ce} and the integration task force coordinated by the project, including extensive full-scale testing at Ashe River and at other integration test sites.

Fabrication and testing of \dwords{pd} will occur in Brazil, and extensive logistics and handling will occur prior to the modules arriving at the \dword{itf}.  Risk 10 addresses the possibility that previously undetected weaknesses will be discovered in QC testing following receipt of the modules.  Due to extensive transportation testing before the beginning of mass production, a risk rating of "L" is assigned.  Mitigation of this risk includes careful design engineering, and testing of shipping and handling procedures.


\subsection{Risks During Installation/Commissioning/Operations}

The biggest risk that could be realized during the commissioning and operations phase is
the observation of excessive noise caused by failure to follow the DUNE
grounding rules.  Risk 11 addresses the possibility of discovering such a failure during installation QC testing or commissioning of the detector.  The observation of excessive noise in \dword{dune}
would result in a delay of the commissioning and of data taking until the source of the noise is
found and remedial actions are taken. In order to minimize the probability of observing excessive
electronic noise we plan to enforce the grounding rules throughout the design phase, based on
the lessons learned from the operation of the ProtoDUNE-SP detector.  In addition, testing at ICEBERG between \dword{pd} and all generations of \dword{ce} electronics should minimize this risk.  Consequently, this risk is assigned a value of "L".  

Risk 12 addresses PD maintenance during operation.   During operation, most \dword{pd} components are inaccessible due to being submerged in \dword{lar}.  However, Some components such as the warm readout electronics remain accessible. It is valuable to assign a risk to the need of their requiring spares beyond those planned for, or replacement due to a previously undetected flaw.  Due to extensive validation prior to installation, this risk is assigned a value of "L".  Mitigation steps include designing the warm systems to facilitate repair, and careful mean time between failure analysis to predict failure rates over the lifetime of the experiment and allow the procurement of sufficient spares.