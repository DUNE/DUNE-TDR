\section{Risks}
\label{sec:fdsp-pd-risks}
%\metainfo{\color{blue} Content: Warner}

Table~\ref{tab:risks:SP-FD-PD} contains a list of all the
risks that we are currently holding in the PD risk register.  Each line includes the official \dword{dune} risk register identification number, a description of the risk, the proposed mitigation for the risk, and finally three columns rating the post-mitigation (P)robability that the risk described comes to pass, the degree of (C)ost risk for that line, and the degree of (S)chedule risk.  Risk levels are defined as (L)ow (<10\% probability of occurring, <5\% cost impact, <2 month schedule impact), (M)edium (10 to 25\% probability of occurring, 5\% to 20\% cost impact, 2 to 6 month schedule impact), or (H)igh (>25\% probability of occurring, >20\% cost impact, >6 month schedule impact).  Most of these risks are reduced to a "Low" level following mitigation (as shown in the table), although several of them currently hold a higher risk levels (pre-mitigation), due to the early stage of development of the \dword{pd} system relative to other systems.  

In the following sections, we present a narrative description of each of the risks and the proposed mitigation.


% risk table values for subsystem SP-FD-PD
\begin{footnotesize}
%\begin{longtable}{p{0.18\textwidth}p{0.20\textwidth}p{0.32\textwidth}p{0.02\textwidth}p{0.02\textwidth}p{0.02\textwidth}}
\begin{longtable}{P{0.18\textwidth}P{0.20\textwidth}P{0.32\textwidth}P{0.02\textwidth}P{0.02\textwidth}P{0.02\textwidth}} 
\caption[Risks for SP-FD-PD]{Risks for SP-FD-PD (P=probability, C=cost, S=schedule) More information at \dword{riskprob}. \fixmehl{ref \texttt{tab:risks:SP-FD-PD}}} \\
\rowcolor{dunesky}
ID & Risk & Mitigation & P & C & S  \\  \colhline
RT-SP-PD -01 & Additional photosensors and engineering required to ensure PD modules collect enough light to meet system physics performance specifications. & Extensive validation of \dword{xarapu} design to demonstrate they meet specification. & L & M & L \\  \colhline
RT-SP-PD-02 & Improvements to active ganging/front end electronics required to meet the specified 1~$\mu$s time resolution. & Extensive validation of photosensor ganging/front end electronics design to demonstrate they meet specification. & L & L & L \\  \colhline
RT-SP-PD-03 & Evolutions in the design of the photon detectors due to validation testing experience require modifications of the TPC elements at a late time. & Extensive validation of \dword{xarapu} design to demonstrate they meet specification and control of PD/APA interface. & L & L & L \\  \colhline
RT-SP-PD-04 & Cabling for PD and CE within the \dword{apa} frame or during the 2-APA assembly/installation procedure require additional engineering/development/testing. & Validation of PD/APA/CE cable routing in prototypes at Ash River. & L & L & L \\  \colhline
RT-SP-PD-05 & Experience with validation prototypes shows that the mechanical design of the PD is not adequate to meet system specifications. & Early validation of \dword{xarapu} prototypes and system interfaces to catch problems ASAP. & L & L & L \\  \colhline
RT-SP-PD-06 & pTB WLS filter coating not sufficiently stable, contaminates \dword{lar}. & Mechanical acceleration of coating wear.  Long-term tests of coating stability. & L & L & L \\  \colhline
RT-SP-PD-07 & Photosensors fail due to multiple cold cycles or extended cryogen exposure. & Execute testing program for cryogenic operation of photosensors including mutiple cryogenic immersion cycles. & L & L & L \\  \colhline
RT-SP-PD-08 & SiPM active ganging cold amplifiers fail or degrade detector performance. & Validation testing if photosensor ganging in multiple test beds. & L & L & L \\  \colhline
RT-SP-PD-09 & Previously undetected electro-mechanical interference discovered during integration. & Validation of electromechanical designin Ash River tests and at \dword{pdsp2}. & L & L & L \\  \colhline
RT-SP-PD-10 & Design weaknesses manifest during module logistics-handling. & Validation of shipping packaging and handling prior to shipping.  Inspection of modules shipped to site immediately upon receipt. & L & L & L \\  \colhline
RT-SP-PD-11 & PD/CE signal crosstalk. & Validation in \dword{pdsp}, \dword{iceberg} and \dword{pdsp2}. & L & L & L \\  \colhline
RT-SP-PD-12 & Lifetime of \dword{pd} components outside cryostat. & Specification of environmental controls to mitigate detector aging. & L & L & L \\  \colhline

\label{tab:risks:SP-FD-PD}
\end{longtable}
\end{footnotesize}

\begin{comment}
\begin{dunetable}
[SP PD System Risk Summary]
{p{0.15\textwidth}p{0.75\textwidth}}
{tbl:SPPDrisks}
{Single Phase Photon Detector Risk Summary}
ID & Risk                   \\ \toprowrule
1 & PD Modules don't collect enough light to meet system physics performance specifications \\ \colhline
2 & Timing performance of the PD electronics/SiPM/Cabling  does not meet the 1 us time resolution specification \\ \colhline
3 & Evolution in the design of the photon detectors due to validation testing experience require modifications of the TPC elements at a late time \\ \colhline
4 & Cabling for PD and CE cannot be accommodated within the \dword{apa} frame or during the 2-APA assembly/installation procedure. \\ \colhline
5 & Experience with validation prototypes shows that the mechanical design of the PD is not adequate. \\ \colhline
6 & \dword{ptb} WLS filter coating not sufficiently stable, contaminates \dword{lar}. \\ \colhline
7 & SiPMs fail due to multiple cold cycles or extended cryogen exposure. \\ \colhline
8 & SiPM active ganging cold amplifiers fail or degrade detector performance. \\ \colhline
9 & Previously undetected electro-mechanical interference discovered during integration\\ \colhline
10 & Design weaknesses manifest during module logistics-handling. \\ \colhline
11 & SiPM PD/CE Electrical Crosstalk \\ \colhline
12 & Lifetime of \dword{pd} components outside cryostat\\

\end{dunetable}
\end{comment}

\subsection{Physics Performance Specification Risks}
\label{sec:pds-risks-text}

Risk RT-SP-PD-01 in the Table \ref{tab:risks:SP-FD-PD} addresses the performance specification that the \dword{pd} system detect 0.5 pe/MeV of deposited energy.  The system as designed may not reach this requirement during validation, necessitating additional engineering time and possibly additional system cost.  Current design validation (Section~\ref{sec:fdsp-pd-validation}) 
%\fixme{check ref --- rjw: Dave did you mean to  refer to the Validation section. } 
provides firm indication that this specification will be met by the \dword{xarapu}.  Mitigation of this risk is being achieved by allocating enough development resources to the PD to continue developing improved light collection modules; increasing the APA slot size to allow for larger modules; or increasing the number of photosensors per \dword{xarapu} supercell.  
The cost risk is rated M because photosensors are a significant cost driver for the project and increasing their number presents a significant medium level cost risk to the system.

Risk RT-SP-PD-02 addresses the performance specification that the PD system provide \SI{1}{$\mu$s} time resolution.  While the timing resolution specification has been met by the \dword{pdsp} SSP-based \dword{sarapu},  cost-saving modifications to the readout electronics could degrade the performance of the PD system below the \SI{1}{$\mu$s} requirement.  In addition, the combination of active and passive ganging of 48 photosensors could degrade timing performance.  Current design validation (Section~\ref{sec:fdsp-pd-validation}) provides firm indication that this specification will be met by the \dword{xarapu} and our baseline electronics, so a risk level of L is assigned to this risk.  Mitigation of this risk is being achieved by allocating enough engineering resources to proceed rapidly with the design modifications of our reduced-cost baseline system; extensive testing of passive ganging prototypes, including parallel development of two design options for the active ganging circuit; and testing of timing performance in software simulation and multiple validation test stands.


\subsection{Design Risks}

Risk RT-SP-PD-03 addresses the interface of the \dword{apa} and \dword{pd} designs, and the possibility that in order to meet detector performance or reliability specifications, the \dword{pd} design may evolve in a direction requiring modification of the \dword{apa}.  Our current design validation (Section~\ref{sec:fdsp-pd-validation}) provides firm indication that these specifications will be met by the \dword{xarapu}, but we have not yet completed the validation process.  While the design validation at this point is sufficient to reduce the overall risk to low following validation, this remains one of the principle risks we consider due primarily to the significant potential costs (financial and schedule) associated with such a change following the TDR.  Mitigation of this risk involves close interaction between the \dword{apa} and \dword{pd} consortia and assigning significant resources to \dword{pd} validation efforts.

Risk RT-SP-PD-04 covers the plans for running \dword{pd} cables within the \dword{apa} frames.  Lessons learned during the \dword{pdsp} led to the re-design of the \dword{pd} cabling layout, moving the cables inside the APA frame where they will be unreachable following installation of the \dword{apa} wires.  Additionally, installation of the \dword{apa}s into the cryostat will require making \dword{pd} cable connections between the upper and lower \dword{apa}s underground.  This risk addresses the concern that difficulties with these \dword{apa}/\dword{pd} interfaces will require changes to the cabling plan.  Mitigation consists of extensive validation tests, including full-scale integration tests at the Ash River installation site.

Risk RT-SP-PD-05 concerns the possibility that continuing validation tests demonstrate that the \dword{pd} mechanical design is in some way not adequate to meet DUNE specifications.  While validation is ongoing and the possibility or a required design change remains, the impact and cost of such a change is likely relatively low.  Mitigation includes continued design validation testing and sufficient engineering resources.

Risk RT-SP-PD-06 concerns the possibility that continuing validation tests demonstrate that the coatings required on the dichroic filter plates are not sufficiently robust in cryogenic applications and flake or dissolve off the surface and contaminate the \dword{lar}, possibly impacting electron lifetime or optical performance of the detector.  Experience in \dword{pdsp} suggested that coatings of the filters is a delicate operation, and the possibility exists to produce unstable coatings.  Mitigation includes continued validation testing of coated filters and sufficient engineering resources.  This is one of the more significant outstanding risks, due to the possibility of negatively impacting the performance of the \dword{tpc}.

Risk RT-SP-PD-07.  One of the most significant lessons of the \dword{pdsp} for the \dword{pd} system was the failure of a significant number of photosensors during module assembly QC due to an unannounced change in the manufacturer's photosensor packaging procedures.  Problems developed with initially reliable photosensors mid-way through fabrication, requiring rapid changes to the \dword{pd} design.  This risk addresses the possibility of a re-occurrence of this or a similar problem.  Mitigation includes (but is not limited to) extensive \dword{qa} testing prior to selecting the final photosensor candidate, careful coordination with photosensor vendor(s), and rigorous QC testing procedures (including tracking wafer fabrication and packaging batch information from the vendor) for photosensors.  We are in close contact with both candidate photosensor candidates to develop a QC/QA plan sufficient to address our concerns.

%\fixme{added text about wafer and packaging lot tracking to address Mont comment  DWW DONE}

Risk RT-SP-PD-08 addresses the possibility of a degradation in \dword{pd} performance or outright failure due to the cold amplifiers required by the active ganging circuitry.  In order to reach the baseline design of 48 ganged photosensors per \dword{xarapu} supercell, a mix of active and passive ganging is required.  While initial validation testing is very promising, these circuits remain quite new.  Mitigation of this risk involves additional validation testing in bench-top testing and in the \dword{iceberg} test stand.

\subsection{Risks During Integration}

Risk RT-SP-PD-09 addresses the possibility that a previously undetected flaw in the \dword{pd} module design or the integration plan with the \dwords{apa} manifests itself during the integration process.  Steps taken to mitigate this risk include close coordination between the \dword{pd}, \dword{apa}, \dword{ce}, and the integration task force coordinated by the project, including extensive full-scale testing at Ash River and at other integration test sites.

Risk RT-SP-PD-10 covers risks associated with integration into \dword{dune} detectors.  Fabrication and initial testing of \dwords{pd} will occur in Brazil, follow-on testing will occur at the US reception facility prior to storage at the \dword{sdwf}, and additional logistics and handling will occur prior to the modules arriving at the underground integration facility. This risk addresses the possibility that previously undetected weaknesses will be discovered in QC testing following receipt of the modules.  Mitigation of this risk includes careful design engineering and testing of shipping and handling procedures.


\subsection{Risks During Installation/Commissioning/Operations}

The biggest risk that could be realized during the commissioning and operations phase is
the observation of excessive noise caused by failure to follow the \dword{dune}
grounding rules.  Risk RT-SP-PD-11 addresses the possibility of discovering such a failure during installation \dword{qc} testing or commissioning of the detector.  The observation of excessive noise in \dword{dune}
would result in a delay of the commissioning and of data taking until the source of the noise is
found and remedial actions are taken. In order to minimize the probability of observing excessive
electronic noise, we plan to enforce the grounding rules throughout the design phase, based on
the lessons learned from the operation of the \dword{pdsp} detector.  In addition, testing at ICEBERG between \dword{pd} and all generations of \dword{ce} electronics will minimize this risk.  

Risk RT-SP-PD-12 addresses PD maintenance during operation.   During operation, most \dword{pd} components are inaccessible due to being submerged in \dword{lar}.  However, some components such as the warm readout electronics remain accessible. It is valuable to assign a risk to the need of their requiring spares beyond those planned for, or replacement due to a previously undetected flaw.  
Mitigation steps include two aspects: (1) designing the warm systems to facilitate repair, and (2) performing a careful mean time between failure analysis to predict failure rates over the lifetime of the experiment that will allow the procurement of sufficient spares in the production phase. 