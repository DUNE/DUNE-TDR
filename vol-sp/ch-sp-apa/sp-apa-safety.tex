%%%%%%%%%%%%%%%%%%%%%%%%%%%%%%%%%%%%%%%%%%%%%%%%%%%%%%%%%%%%%%%%%%%%
\section{Safety Considerations}
\label{sec:fdsp-apa-safety}

%Building on the experience of \dword{pdsp}, a full safety analysis will be performed and a set of safe work procedures developed for all stages of the fabrication process before starting \dword{dune} \dword{apa} production.  In the final design of the winding machine safety must be incorporated from the design stage and at all stages in the life of the machine: design, manufacture, installation, adjustment, operation, and maintenance.  Handling of the large, but delicate frames is a major challenge, and safe procedures will be developed for all phases of construction, including frame assembly, wiring, transport, and integration and installation in the cryostat.         

%At the \dword{apa} production sites, safety is ultimately the responsibility of the host institutions, and all local rules and regulations must be followed.  However, common job hazard analyses will be performed and documents prepared for many shared aspects of the tooling and activities.  In addition, safety will be an important element of production readiness reviews conducted for the project overall and for the production sites individually.   

%Safety procedures are being actively developed as the production procedures are developed and finalized.  More complete documentation will be added to this TDR in 2019.  

The project is committed to ensuring a safe work environment for workers at all institutions and facilities, from \dword{apa} fabrication to installation. The project utilizes the concept of an Integrated Safety Management System (ISMS) as an organized process whereby work is planned, performed, assessed, and systematically improved to promote the safe conduct of work. The LBNF/DUNE Integrated Environment, Safety and Health Management Plan \cite{bib:docdb291} %[DUNE-doc-291] 
contains details on LBNF/DUNE integrated safety management systems. This work planning and hazard analysis (HA) program utilizes detailed work plan documents, hazard analysis reports, equipment documentation, safety data sheets, personnel protection equipment (PPE), and job task training to minimize work place hazards. 

Prior to APA production, applying the experience of \dword{pdsp}, the project will coordinate with fabrication partner facilities to develop work planning documents, and equipment documentation, such as the Interlock Safety System for \dword{apa} winding machines to implement an automated protection against personnel touching the winding arm while the system is in operation. Additionally, the project will work with the local institutions' ESH coordinators to ensure that ESH requirements within the home institution's ESH Manual address the hazards of the work activities occurring at the facility. Common job hazard analyses may be shared across multiple fabrication facilities.   

Handling of the large but delicate frames is a challenge. Procedures committed to the safety of personnel and equipment will be developed for all phases of construction, including frame assembly, wiring, transport, and integration and installation in the cryostat. This documentation will continue to be developed through the Ash River trial assembly process, which maps out the step by step procedures and brings together the documentation needed for approving the work plan to be applied at the far site. 

As is Fermilab's practice, all personnel have the right to stop work for any safety issues.
