\chapter{Executive Summary}
\label{vl:tc-execsum}


The \dword{dune} \dword{tc} holds general responsibility for
facilities and functions related to the design, construction,
installation, and operation of the DUNE experiment as a whole. These
include all interfaces to \fnal as the host laboratory; \dword{lbnf},
\dword{surf} that houses and supports the \dword{dune} experiment This
includes interfaces to other surface facilities at or near \surf
operated for installation and operations activities; and to certain
other common test facilities operated for the benefit of
\dword{dune}. Led by the \dword{dune} \dword{tcoord}, \dword{dune}
\dword{tc} provides oversight and organizational support to the
consortia building the \dword{dune} detector.  This includes project
functions such as requirements, scheduling, risk management, value
engineering, interface to national project management and
reporting. The \dword{tc} organization establishes and maintains
standards and tools for all drawings and schematics, for grounding and
shielding, and for all DUNE subsystem-to-subsystem and
\dword{dune}-to-\dword{lbnf} interfaces.  \dword{tc} includes the
\dword{dune} safety manager and the \dword{dune} \dword{qa}
manager. \dword{dune} \dword{tc} manages the design, production and
operational readiness reviews, maintains the overall schedule and
monitors progress.

Chapters that follow lay out the structure of \dword{dune}
\dword{tc}, describe \dword{dune} facilities and detail
project-related functions and methodologies. At the stage of this
early draft to LBNC, \dword{dune} \dword{tc} functional examples are most fully
developed for the \dword{dsp} detector. The volume will be expanded to
encompass more functional examples associated with 
\dword{ddp} by the final draft.

