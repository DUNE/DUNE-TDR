\chapter{Project Functions}
\label{vl:tc-project}

[Steve]

As defined in the \dword{dune} Management Plan (DMP), the \dword{dune}
Technical Board (TB) generates and recommends technical decisions to the
collaboration executive board (EB).
It consists of all consortia scientific and technical leads. It meets
on a regular basis (approximately monthly) to review and resolve any
technical issues associated with the detector construction. It reports
through the EB to collaboration management. The \dword{dune} TB
is chaired by the technical coordinator. The
\dword{tc} engineering team also meets on a regular basis (approximately monthly)
to discuss more detailed technical issues. \Dword{tc} does not have
responsibility for financial issues; that will instead be referred to
the EB and Resource Coordinator (RC).

\Dword{tc} has several major project support tasks that need to be accomplished:
\begin{itemize}
\item Assure that each consortium has a well defined and complete
  scope, that the interfaces between the consortia are sufficiently
  well defined and that any remaining scope can be covered by \dword{tc}
  through \dword{comfund} or flagged as missing scope to the EB and RC. In
  other words, assure that the full detector scope is
  identified. Monitor the interfaces and consortia progress in
  delivering their scope.
\item Develop an overall project \dlong{ims}
  that includes reasonable production schedules, testing plans and a
  well developed installation schedule from each consortium. Monitor
  the \dword{ims} as well as the individual consortium schedules.
\item Ensure that appropriate engineering and safety standards are
  developed and agreed to by all key stakeholders and that these
  standards are conveyed to and understood by each
  consortium. Monitor the design and engineering work.
\item Ensure that all \dword{dune} requirements on \dword{lbnf} for
  conventional facilities, cryostat and cryogenics have been clearly
  defined and understood by each consortium. Negotiate scope
  boundaries with \dword{lbnf}. Monitor \dword{lbnf} progress on
  final conventional facility design, cryostat design and cryogenics
  design.
\item Ensure that all technical issues associated with scaling from
  \dword{protodune} have sufficient resources to converge on
  decisions that enable the detector to be fully integrated and
  installed.
\item Ensure that the integration and \dword{qc} processes for each
  consortium are fully developed and reviewed and that the
  requirements on an \dword{itf} are well defined.
\end{itemize}

\Dword{tc} is responsible for technical quality and schedule and is not
responsible for consortia funding or budgets.  \Dword{tc} will try to help
resolve any issue that it can, but will likely have to push all
financial issues to the TB, EB and RC for resolution.

\Dword{tc} maintains a web
page\footnote{\url{https://web.fnal.gov/collaboration/DUNE/DUNE\%20Project/\_layouts/15/start.aspx\#/}.}
with links to project documents. \Dword{tc} maintains repositories of
project documents and drawings. These include the \dword{wbs},
schedule, risk register, requirements, milestones, strategy, detector
models and drawings that define the \dword{dune} detector.

%%%%%%%%%%%%%%%%%%%%%%%%%%%%%%%%
\section{WBS}
\label{sec:fdsp-coord-wbs}

%%%%%%%%%%%%%%%%%%%%%%%%%%%%%%%%
\section{Cost}
\label{sec:fdsp-coord-cost}

\begin{itemize}
 \item cost model, how many SP/DP detectors
 \item spares, labor categories, ...
 \item summary of consortia costs
 \item TC cost details?
\end{itemize}

%%%%%%%%%%%%%%%%%%%%%%%%%%%%%%%%
\section{MOU}
\label{sec:fdsp-coord-mou}

%%%%%%%%%%%%%%%%%%%%%%%%%%%%%%%%
\section{Budget}
\label{sec:fdsp-coord-budget}

%%%%%%%%%%%%%%%%%%%%%%%%%%%%%%%%
\section{Schedule}
\label{sec:fdsp-coord-controls}

%%%%%%%%%%%%%%%%%%%%%%%%%%%%%%%%
\section{Risks}
\label{sec:fdsp-coord-risks}

\begin{itemize}
 \item Summary of high level risks
 \item Details of TC risks?
\end{itemize}

%%%%%%%%%%%%%%%%%%%%%%%%%%%%%%%%
\section{Requirements}
\label{sec:fdsp-coord-requirements}

\begin{itemize}
 \item Summary of consortia requirements?
 \item TC requirements (cleanliness, APA spacing, …)?
\end{itemize}

%%%%%%%%%%%%%%%%%%%%%%%%%%%%%%%%
\section{Value Engineering}
\label{sec:fdsp-coord-ve}

Principles and expectations for value engineering by consortia

%%%%%%%%%%%%%%%%%%%%%%%%%%%%%%%%
\section{Lessons Learned}
\label{sec:fdsp-coord-lessons}


%%%%%%%%%%%%%%%%%%%%%%%%%%%%%%%%
\section{Interface to National Projects}
\label{sec:fdsp-coord-national}


%%%%%%%%%%%%%%%%%%%%%%%%%%%%%%%%
\section{Reporting}
\label{sec:fdsp-coord-reporting}


%%%%%%%%%%%%%%%%%%%%%%%%%%%%%%%%
\section{Management of Schedule and Risks}
\label{sec:fdsp-coord-mgmt}


%%%%%%%%%%%%%%%%%%%%%%%%%%%%%%%%
\section{Design Process}
\label{sec:fdsp-coord-designprocess}

FNAL stds and international stds

%%%%%%%%%%%%%%%%%%%%%%%%%%%%%%%%
\section{Integration Facility}
\label{sec:fdsp-coord-itf}

Do we explain the concept of ITF here in general terms?

