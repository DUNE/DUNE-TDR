\chapter{Quality Assurance}
\label{vl:tc-QA}

\section{Overview}

DUNE Technical Coordination monitors the technical contributions from
the different agencies and provides the necessary centralized project
coordination functions. As part of the centralized project
coordination, this includes standardizing quality control/quality
assurance practices. One of the facets of standardizing these
practices is to assist the consortia in defining and implementing
quality assurance/quality control plans that maintain a uniform,
high-level standard across the entire detector construction
effort. The Quality Assurance effort will include participating in the
design, construction readiness and progress reviews as appropriate for
the DUNE detector subsystems.

\subsection{Purpose}

The primary objective of the LBNF/DUNE Quality Assurance (QA) program
is to implement quality in the construction of the LBNF facility and
DUNE experiment while providing protection of LBNF/DUNE personnel, the
public, and the environment. The QA Plan aligns the LBNF/DUNE QA
activities, which are spread around the world, with the Fermilab
Quality Assurance Manual. The Manual identifies the Fermilab
Integrated Quality Assurance Program features that serve as the basis
for the LBNF/DUNE QA Plan.

The LBNF/DUNE QA Plan outlines the QA requirements for all LBNF/DUNE
collaborators and subcontractors and describes how the requirements
shall be met. QA criteria can be satisfied using a graded
approach. This Plan is implemented by the development of quality
plans, procedures and guides by the Consortia to accommodate those
specific quality requirements.

\subsection{Scope}

The LBNF/DUNE Quality Plan provides Quality Assurance requirements
applicable to all Consortia, encompassing all activities performed
from research and development (R\&D) through fabrication, and component
commissioning. Consortia will be responsible for providing their
deliverables, whether subsystems, components or services in accordance
with applicable agreements. All parties will be responsible for
implementing a quality plan that meet the requirements of the
LBNF/DUNE Quality Plan. Oversight of the work of the Consortia will be
the responsibility of the DUNE Technical Coordinator and LBNF/DUNE QA
Manager.

\subsection{Graded Approach}

A key element of the LBNF/DUNE QA Plan is the concept of Graded
Approach; that is, applying a level of analysis, controls, and
documentation commensurate with the potential for an environmental,
safety, health or quality impact. The Graded Approach seeks to tailor
the kinds and extent of quality controls applied in the process of
fulfilling requirements. Application of the graded approach entails:
\begin{itemize}
 \item Identifying activities that present significant ES\&H and/or quality risk
 \item Defining the activity
 \item Evaluating risk and control choice
 \item Documenting and approving the application of the graded approach.
\end{itemize}

\section{Quality Assurance Program}

The LBNF/DUNE Systems Engineering teams maintain a LBNF/DUNE
Configuration Management Plan (CMP), which identifies the Projects’
Configuration Items Data List (CIDL) and Interface Control matrices
that provide the tier structure for the flow down of QA Plans, with
the LBNF/DUNE QA Plan as the top tier.

Specific QA plans shall be developed by the Consortia with the
assistance of the LBNF/DUNE Quality Assurance Manager (QAM) for
component or system quality assurance. Due to the limited scope of
work of some of the Consortia, they may elect to work under the
LBNF/DUNE QA Plan for their scope of work. In case of conflict between
sets of QA requirements, DUNE Project Management will provide
resolution.

With many institutions carrying responsibility for various aspects of
the Project, institutional QA Plans will be reviewed by the DUNE
Project to ensure compliance with the LBNF/DUNE QA Plan. Using a
graded approach, supplements to institutions’ existing plans will be
implemented for their DUNE scope of work, if necessary.

Overall QA supervision, including all activities described above, is
the responsibility of the DUNE Technical Coordinator.

\subsection{Responsibility for Project Management}

The DUNE Consortia Leaders manage their Projects and are responsible
for achieving performance goals. The LBNF/DUNE QAM is responsible for
ensuring that a quality system is established, implemented, and
maintained in accordance with requirements. The LBNF/DUNE QAM reports
to the DUNE Technical Coordinator and provides oversight and support
to the Consortia Leaders to ensure a consistent quality program.

The DUNE Consortia Leaders are responsible for quality within their
Project and report their Quality Assurance issues to the DUNE
Technical Coordinator and LBNF/DUNE QA Manager. DUNE Consortia Leaders
may designate Quality Assurance Representatives (QAR) within their
organization to perform some of their work defined in the LBNF/DUNE
Quality Plan. The DUNE Consortia Leader shall retain overall
responsibility for these activities even though they have designated a
QAR.

The DUNE-US Project Manager is appointed by the DUNE-US Project
Director and runs the DUNE-US Project Office. The DUNE-US Project
Manager, in coordination with the DUNE-US Project Director, is
responsible and accountable for the day-to-day management and
execution of the project. For the responsibilities of the DUNE
Technical Director that are defined in the LBNF/DUNE Quality Plan, the
DUNE-US Project Manager will retain the same responsibilities for
DUNE-US activities. The DUNE-US Project Manager is responsible for
coordinating the activities of the DUNE-US L2 Project Managers. The
DUNE-US L2 Project Managers will have the same responsibilities as the
DUNE Consortia Leaders within the LBNF/DUNE QA Plan.

\subsection{Levels of Authority and Interface}

The DUNE Management Plan, the LBNF/DUNE PMP, and the LBNF/DUNE Quality
Assurance Plan (QAP) define the responsibility, authority, and
interrelation of personnel who manage, perform, and verify work that
affects quality. The QAP defines the QA roles and responsibilities of
these management and working levels of the DUNE Project.

All Consortia Members are responsible for the quality of the work that
they do and for using guidance and assistance that is available. Each
has the authority to stop work and report adverse conditions that
affect quality of DUNE products to their respective DUNE Consortia
Leader and the LBNF/DUNE QA Manager. The Consortia Leader responsible
for DUNE components or systems is required to determine and document
their acceptance criteria. DUNE personnel at each level are
responsible for evaluation of quality through self-assessments;
however, independent quality assessments may also be requested by
Project management.  The LBNF/DUNE QA Manager is responsible for
development, implementation, assessment, and improvement of the QA
program.

The LBNF/DUNE QAM is also responsible for periodically reporting on
the performance of the quality system to the DUNE Technical
Coordinator for their review and as a basis for improvement of the
quality system. The DUNE Technical Coordinator may call for Quality
Assurance Program readiness assessments as the Project nears major
milestones. The DUNE Technical Coordinator, Consortia Leaders, and
LBNF/DUNE QA Manager are all responsible for providing the resources
needed to conduct the Project successfully, including those required
to manage, perform and verify work that affects quality.

\subsection{Quality Assurance Organization}

The LBNF/DUNE QA Manager may request personnel from the DUNE Project
Teams to act on behalf of the LBNF/DUNE QA Manager to perform quality
assurance functions, based on need, in accordance with the Graded
Approach described above. The requested personnel shall possess
qualifications or receive the appropriate training required to perform
these functions.

\section{Personnel Training and Qualification}

The DUNE Consortia Leaders are responsible for identifying the
resources to ensure that their team members are adequately trained and
qualified to perform their assigned work. Before allowing personnel to
work independently, they are responsible to ensure that their team
members have the necessary experience, knowledge, skills, and
abilities. Personnel qualifications are based on the following
factors:
\begin{itemize}
 \item previous experience, education, and training
 \item performance demonstrations or tests to verify previously acquired skills
 \item completion of training or qualification programs
 \item on-the-job training
\end{itemize}

All DUNE Consortia Leaders are responsible for ensuring that their
training and qualification requirements are fulfilled, including
periodic re-training to maintain proficiency and qualifications.


\section{Quality Improvement}

All DUNE Consortia members participate in quality improvement
activities that identify opportunities for improvement. They can
respond to the discovery of quality-related issues and follow up on
any required actions. This quality-improvement process requires that
any failures and non-conformances be identified and reported to the
appropriate Consortia Leader; and, that root causes be identified and
corrected. All Consortia members are encouraged to identify problems
or potential quality improvements and may do so without fear of
reprisal or recrimination. Items, services, and processes that do not
conform to specified requirements shall be identified and controlled
to prevent their unintended use. Inspection and test reports or
similar tools will be used to implement this requirement. Each
Consortia Leader is responsible for reporting non-conformances to the
LBNF/DUNE QA Manager and the LBNF/DUNE QA Manager will periodically
report these non-conformances to DUNE Project Management.

DUNE Consortia members will perform Root Cause Analysis and Corrective
and Preventive Actions for conditions that do not meet defined
requirements. Consortia Leaders may perform Root Cause analysis and
Corrective and Preventive Actions under their own procedures or
Fermilab procedures.  This problem identification, analysis and
resolution process for quality consists of the following steps:
\begin{enumerate}
 \item Identify problem
 \item Understand the process
 \item Grade the process and identify Root Cause Analysis (RCA) method
 \item Identify possible causes
 \item Collect and analyze data
 \item Communicate Lessons Learned and document RCA
 \item Implement Corrective and Preventative Action procedure
\end{enumerate}

\subsection{Lessons Learned}

To promote continuous improvement, the DUNE Project will develop a
lesson learned program based on the Fermilab Office of Project Support
Services’ Lessons Learned Program. This program provides a systematic
approach to identify and analyze relevant information for both good
and adverse work practices that can influence Project execution. Where
appropriate, improvement actions are taken to either promote the
repeated application of a positive lesson learned or prevent
recurrence of a negative lesson learned. Lessons learned shall be
gathered throughout the Project life cycle. As part of the transition
to operations a lessons learned report will be submitted.

In addition, the LBNF/DUNE QA Manager will periodically publish a best
practices and lessons learned report. Lessons learned from the DUNE
Project will be screened for applicability to other organizations. The
DUNE project will periodically check external lessons learned sources
for applicability to the DUNE Project. Sources of lessons learned
include the DOE Lessons Learned List Server, the Fermilab ESH\&Q
Lessons Learned Database, and DUNE team members who participate in
peer reviews of other Projects Reviews of the DUNE Project serve as
input to quality improvement.

\section{Documents and Records}

Engineering and scientific documents (including drawings) are prepared
by DUNE personnel to define the design, manufacture and
construction. Ultimately, before these documents are put into effect
they are reviewed and signed by the DUNE Consortia Leader or
designee. The DUNE Project Offices manage all documents under a
document control system as identified in the DUNE Configuration
Management Plan (CMP). The system to control document preparation,
approval, issuance to users, and revision is described in the
CMP. Consortia Leaders will use the graded approach described in this
plan to determine work in their scope that requires the LBNF/DUNE QA
Manager review and signature. Project documents that contain quality
requirements shall be reviewed by the LBNF/DUNE QA Manager.

Records are prepared and maintained to document how decisions are
made, for instance, decisions on how to arrive at a design, how to
record the processes followed to manufacture components, and the means
and methods of cost and schedule change control. LBNF/DUNE will follow
the guidelines for storing and maintaining records for the Project in
accordance with Fermilab Records
Management(http://ccd.fnal.gov/records). DUNE Technical Coordinator,
LBNF/DUNE QA Manager and Consortia Leaders are responsible for
identifying the information to be preserved. In addition to the
technical, cost, and schedule baseline and all changes to it, records
must be preserved as evidence that a decision was made or an action
taken, and to provide the justification for the decision or action.

\section{Work Processes}

DUNE team members are responsible for the quality of their work, and
Consortia leaders are responsible for procuring the resources and
support systems to enable their staff to complete their work with high
quality. All DUNE work will be performed using methods that promote
successful completion of tasks, conformance to DUNE requirements, and
compliance with the LBNF/DUNE Integrated ESH Plan. Work processes
consist of a series of actions planned and carried out by qualified
personnel using approved procedures, instructions and equipment, under
administrative, technical, and environmental controls, to achieve a
high-quality result.

\subsection{Fabrication Work Processes}

Fabrication work on the DUNE Projects shall be performed to
established technical standards and administrative controls using
approved instructions and procedures. Fabrication work processes with
QA inspections and tests shall be documented on Travelers that are
retained with the hardware item. Items, including consumables, shall
be identified and controlled to ensure their proper use and prevent
the use of incorrect, unaccepted, or unidentified items. The Consortia
will define a system of controls to ensure that items are handled,
stored, shipped, cleaned, and preserved to prevent them from
deteriorating, being damaged, or becoming lost. Equipment used for
process monitoring or data collection shall be calibrated and
maintained.

Work shall be performed safely, in a manner that ensures adequate
protection for employees, the public, and the environment. Consortia
members and the DUNE Technical Coordinator shall exercise a degree of
care commensurate with the work and the associated hazards. See the
LBNF/DUNE Integrated ES\&H Plan for more details on LBNF/DUNE
integrated safety management systems.

\subsection{Change-Controlled Work Processes}

Change-controlled work processes are those for which the DUNE Change
Control Boards (CCB) require that work, both design and fabrication is
tracked. The CCB assign a unique tracking number to identify those
design and fabrication items for which the associated change is
effective. The LBNF/DUNE Configuration Management Plan defines the
change control process in detail.

\section{Design}

The DUNE design process provides appropriate control of design inputs
and design products. The primary design inputs are the DUNE
scientific/engineering requirements (physics requirements, detector
requirements, specifications, drawings, engineering reports, etc.) and
Configuration Management documentation provided at the Systems
Engineering website.

The basis of the design process requires sound engineering judgment
and practices, adherence to scientific principles, and use of
applicable orders, codes and standards. This basis of the design
process naturally incorporates environment, health and safety
concerns.

\subsection{Design Process}

The LBNF/DUNE Systems Engineering website documentation defines the
scope of design work for any given scientific/engineering work
group. From these two sources, work groups will begin preliminary
design of DUNE by breaking their work down into sets of engineering
drawings, specifications and reports. This is the design output.

Throughout the design process, engineers and designers work with
Consortia Leaders and the LBNF/DUNE QA Manager to determine QA
inspection criteria of fabricated products and installations. Close
coordination must be made with DUNE scientists to assure the
engineering satisfies the scientific requirements of the
experiment. Configuration Management as documented in the LBNF/DUNE
Configuration Management Plan will be systematically implemented for
DUNE. Final Design work sets the final Quality Assurance parameters
for the parts, assemblies and installations. Design during Final
Design and production is confined to Change-Controlled changes, as
above; and, minor changes necessary to facilitate production, drawing
error correction, material substitutions and similar functional areas.

\subsection{Design Verification and Validation}

Design is verified and validated to an extent commensurate with its
importance to safety, complexity of design, degree of standardization,
state of the art, and similarity to proven design
approaches. Acceptable verification methods include but are not
limited to any one or combination of (1) design reviews, (2)
alternative calculations, and (3) prototype, qualification testing
and/or (4) comparison of the new design with a similar proven design
if available. Verification work shall be completed before approval and
implementation of the design.

Design reviews shall verify and validate that the following criteria
are met at the appropriate milestone:
\begin{itemize}
 \item Adherence to requirements
 \item Technical adequacy of the design
 \item Adequacy of work instructions
 \item Thoroughness of specifications
 \item Test results
 \item Adequacy of Technical Reports
 \item Adequacy of design calculations and drawings
 \item Reliability and maintainability
 \item Calibration program for measurement and test equipment
\end{itemize}

The DUNE Review Plan describes the design reviews recommended for its
Consortia.  Wherever the design method involves the use of computer
software to make engineering calculations or static dynamic models of
the structure, system, or component's functionality, the software must
be verified to demonstrate that the software produces valid
results. The verification needs to be documented in a formal Report of
Validation that is maintained in records that are accessible for
inspection. However, exemptions may be made for commercially available
software that is widely used and for codes with an extensive history
of refinement and use by multiple institutions. Exemptions affecting
systems or components shall be identified to the LBNF/DUNE Systems
Engineering team.

Critical software and firmware computer codes, especially those codes
that are involved in controlling DUNE Data acquisitions systems (DAQ),
shall also be subjected to reviews for verification and
validation. Some items to be considered during computer code review
are as follows:
\begin{itemize}
 \item Adequacy of code testing scheme
 \item Code release control and configuration management
 \item Output data verification against code configuration
 \item Verification that code meets applicable standards
 \item Verification of code compatibility to other systems that use the data
 \item Verification that code meets applicable hardware requirements
 \item Adequacy of code maintenance plans
 \item Adequacy of code and data backup systems
\end{itemize}

Validation ensures that any given design product conforms to DUNE
Science and Engineering Requirements on the Systems Engineering
website. In any review, validation of conformity to requirements
follows verification that the engineering design or computer code
meets all criteria. Engineering designs and computer codes shall be
validated, preferably before procurement, manufacture, or
construction; but no later than acceptance and use of the item; this
is to ensure the design or computer code:
\begin{itemize}
 \item Meets the DUNE requirements,
 \item Contains or makes reference to acceptance criteria, and
 \item Identifies all characteristics crucial to the safe and proper use of the equipment or system and its associated interfaces
\end{itemize}

Each inspection, test or review will feed the QA evaluation process,
which is a comparison of results with acceptance criteria to determine
acceptance or rejection. Rejection identifies the need for Quality
Improvement based on Section 4 of this document. In some cases, the
outcome of the Quality Improvement process may be to request change(s)
to the design requirements.

QA reporting formality escalates as the significance of the
inspection, test or review nonconformance increases. Higher levels of
management must be aware of and participate in the correction of the
most significant nonconformance. Section 4, Quality Improvement,
identifies the required course of action when nonconformance is
encountered.

\section{Procurement}

\subsection{Procurement Controls}

Procurement controls will be implemented to ensure that purchased
items and services meet DUNE requirements and comply with the
LBNF/DUNE Quality Assurance Plan.  The Consortia members requesting
procurement of items and services are responsible for providing all
documentation that adequately describes the item or service being
procured so that the Supplier can understand what is required for the
Consortias’ acceptance. Development of this documentation may be
achieved through the involvement of Consortia Leaders and established
review and approval systems. The following factors will be considered
for review and approval of this documentation:
\begin{itemize}
 \item Inclusion of technical performance requirements
 \item Identification of required codes and standards, laws and regulations
 \item Inclusion of acceptance criteria, including requirements for receiving inspection and/or source inspection
 \item DUNE requirements for vendor qualifications and certifications
 \item DUNE intention to perform acceptance sampling in lieu of full inspection and test item acceptance
\end{itemize}
NOTE: For Vendor Qualification and acceptance of purchased items or
material by Consortia members this may be performed under their own
institution requirements.

Previously accepted Suppliers shall be monitored to ensure that they
continue supplying acceptable items and services. Source surveillance
is the recommended method to ensure that items are free of damage and
that specified requirements are met. Supplier deliveries will be
verified against previously established acceptance criteria.

Unacceptable Supplier items or services shall be documented. Records
of Supplier performance, Inspection Test Records (ITR) and
contract-required submittals, are kept for future procurement
consideration.

Inspections shall be conducted to detect counterfeit and/or suspect
parts. For work funded by DOE, when counterfeit/suspect parts are
found, they will be identified, segregated, and disposed of in
accordance with the Fermilab Quality Assurance Manual Chapter 12020
Suspect/Counterfeit Items (S/CI) Program. DUNE Consortia may use their
own institution’s procedure for counterfeit/suspect parts.

\subsection{Inspection and Acceptance Testing}

Inspection and testing of electrical, mechanical and structural
components, associated services, and processes by Consortia members
shall be conducted using acceptance and performance criteria. ITR
forms, Travelers, and a Traveler database are the primary tools used
to organize this activity. Inspections will be conducted in accordance
with the Graded Approach.

Equipment used for all inspections and tests shall be calibrated and
maintained. Calibration will be controlled by a system or systems
making appropriate use of qualified calibration service
providers. Consortia Leaders shall ensure that equipment requiring
calibration have their calibration status identified on the item or
container, are traceable back to the calibration documentation and are
tracked to ensure the equipment is calibrated at the required
interval. The LBNF/DUNE QA Manager shall oversee and support the DUNE
calibration programs.

\section{Assessments}

\subsection{Management Assessments}

DUNE management at all levels shall regularly evaluate achievement of
personnel relative to performance requirements and shall appropriately
validate or update performance requirements and expectations to ensure
quality of products and processes. The management assessment process
shall periodically include an evaluation of the Consortia's products
and processes to determine whether the Project's missions are being
fulfilled. The results of management assessments that focus on means
of improving the quality of work performed shall be reported to the
appropriate responsible line or Project management level.

When performance does not meet established standards, management
shall, with the assistance of others with appropriate expertise,
determine the cause and initiate corrective action. QA representatives
may assist, lead, or facilitate cause investigations.

\subsection{Independent Assessments}

The LBNF/DUNE QA Manager will plan reviews as independent assessments
to assist the DUNE Technical Coordinator in identifying opportunities
for quality/performance-based improvement and to ensure compliance
with specified requirements. Independent assessments of the DUNE
Projects can be requested by DUNE management. Independent assessments
typically focus on quality or ESH\&Q management systems, self-
assessment programs, or other organizational functions identified by
management. The DUNE Project uses a formal process for assigning
responsibility in response to recommendations from independent
assessments. These recommendations are tracked to closure.

Personnel conducting independent assessments shall be technically
qualified and knowledgeable in the areas assessed. A qualified lead
assessor (auditor), who is a Subject Matter Expert (SME) in the
technical area of assessment, is required. The team may include other
SMEs to evaluate the adequacy and effectiveness of activities only if
they are not responsible for the work being assessed.

The Fermilab Directorate appoints an independent Long Baseline
Neutrino Committee (LBNC) to advise it and DUNE Management. The role
of this standing committee is described in the LBNF/DUNE PMP. The DOE
and other funding agencies perform external assessments that provide
an objective view of performance and thus contribute to the
independent assessment process. Since such assessments are not under
the control of DUNE, they are not necessarily considered a part of the
independent assessment criterion. However, DUNE management considers
external assessment results in determining the scope and schedule of
independent assessments.

