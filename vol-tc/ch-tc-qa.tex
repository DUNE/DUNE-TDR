\chapter{Quality Assurance}
\label{vl:tc-QA}

\section{Overview}

\dword{dune} \dword{tc} monitors technical contributions from
collaborating institutions and provides centralized project
coordination functions. One part of this project
coordination is standardizing quality control/quality
assurance practices. One facet of standardization
is to assist consortia in defining and implementing
quality assurance/quality control plans that maintain uniform,
high standards across the entire detector construction
effort. \fixme{based on comments from the review, I think we need one or two org charts. One shows the connections to all of the consortia QA representatives, with names. Does another show how QA fits in TC/JPO? with LBNF? maybe shows how it relates with the review team? does it have a role in drawing and/or production signoff? - how does it show up in change control? How will JPO sign off on accepting consortia deliverables.} The \dword{qa} effort will include participating in the
design, construction readiness and progress reviews as appropriate for
the \dword{dune} detector subsystems. \fixme{maybe refer here to success with ProtoDUNE-SP?}

\subsection{Purpose}

The primary objective of the \dword{lbnf}/\dword{dune}  \dword{qa} program
is to implement quality in the construction of the \dword{lbnf} facility and
\dword{dune} experiment while providing protection of \dword{lbnf}/\dword{dune} personnel, the
public and the environment. The \dword{qa} plan aligns \dword{lbnf}/\dword{dune}  \dword{qa}
activities, which are spread around the world, with the \fnal
Quality Assurance Manual. The Manual identifies the \fnal
Integrated Quality Assurance Program features that serve as the basis
for the \dword{lbnf}/\dword{dune} \dword{qa} plan.

The \dword{lbnf}/\dword{dune} \dword{qa} plan outlines the \dword{qa}
requirements for all \dword{lbnf}/\dword{dune} collaborators and
subcontractors and describes how the requirements shall be met. \Dword{qa}
criteria can be satisfied using a graded approach. This \dword{qa} plan is
implemented by the development of quality plans, procedures and guides
by the consortia to accommodate those specific quality requirements.

\subsection{Scope}

The \dword{lbnf}/\dword{dune} \dword{qa} plan provides  \Dword{qa} requirements
applicable to all consortia, encompassing all activities performed
from research and development (R\&D) through fabrication, and component
commissioning. Consortia will be responsible for providing their
deliverables, whether subsystems, components or services in accordance
with applicable agreements. All parties will be responsible for
implementing a quality plan that meet the requirements of the
\dword{lbnf}/\dword{dune} \dword{qa} plan. Oversight of the work of the consortia will be
the responsibility of the \dword{dune} \dword{tcoord} and \dword{lbnf}/\dword{dune} \dword{qa} manager. \fixme{Based on review comments, don't we need to describe how QA worked in ProtoDUNE and how we have learned lessons to improve it for DUNE?}

\subsection{Graded Approach}

A key element of the \dword{lbnf}/\dword{dune} \dword{qa} plan is the concept of Graded
Approach; that is, applying a level of analysis, controls and
documentation commensurate with the potential for an environmental,
safety, health or quality impact. The Graded Approach seeks to tailor
the kinds and extent of quality controls applied in the process of
fulfilling requirements. Application of the graded approach entails:
\begin{itemize}
 \item Identifying activities that present significant \dword{esh} and/or quality risk
 \item Defining the activity
 \item Evaluating risk and control choice
 \item Documenting and approving the application of the graded approach.
\end{itemize}

\section{Quality Assurance Program}

The \dword{lbnf}/\dword{dune} Systems Engineering teams maintain a \dword{lbnf}/\dword{dune}
\dword{cmp}, which identifies the project
Configuration Items Data List (CIDL) and Interface Control matrices
that provide the tier structure for the flow down of \dwords{qa} plan, with
the \dword{lbnf}/\dword{dune} \dword{qa} plan as the top tier.

Specific \dwords{qap} shall be developed by the consortia with the
assistance of the \dword{lbnf}/\dword{dune} \dword{qa} manager for
component or system quality assurance. Due to the limited scope of
work of some of the consortia, they may elect to work under the
\dword{lbnf}/\dword{dune} \dword{qa} plan for their scope of work. In case of conflict between
sets of  \dword{qa} requirements, \dword{dune} \dword{tc} will provide
resolution.

With many institutions carrying responsibility for various aspects of
the project, institutional \dwords{qa} plan will be reviewed by \dword{dune}
\dword{tc} to ensure compliance with the \dword{lbnf}/\dword{dune} \dword{qa} plan. Using a
graded approach, supplements to institutions existing plans will be
implemented for their \dword{dune} scope of work, if necessary.

Overall \dword{qa} supervision, including all activities described above, is
the responsibility of the \dword{dune} \dword{tcoord}.

\subsection{Responsibility for Project Management}

The \dword{dune} consortia leaders manage their projects and are
responsible for achieving performance goals. The
\dword{lbnf}/\dword{dune} \dword{qa} manager is responsible for
ensuring that a quality system is established, implemented and
maintained in accordance with requirements. The
\dword{lbnf}/\dword{dune} \dword{qa} manager reports to the
\dword{dune} \dword{tcoord} and provides oversight and support
to the consortia leaders to ensure a consistent quality program.

The \dword{dune} consortia leaders are responsible for quality within
their project and report Quality Assurance issues to the \dword{dune}
\dword{tcoord} and \dword{lbnf}/\dword{dune} \dword{qa}
manager. \dword{dune} consortia leaders may designate \dword{qa}
representatives within their organization to perform some of their
work defined in the \dword{lbnf}/\dword{dune} \dword{qa} plan. The
\dword{dune} consortia leader shall retain overall responsibility for
these activities even though they have designated a \dword{qa}
representative.

\subsection{Levels of Authority and Interface}

The \dword{dune} Management Plan, the \dword{lbnf}/\dword{dune} PMP
and the \dword{lbnf}/\dword{dune} \dword{qa} plan define the
responsibility, authority and interrelation of personnel who manage,
perform and verify work that affects quality. The \dword{qa} plan defines
the \dword{qa} roles and responsibilities of the \dword{dune} project.

All consortia members are responsible for the quality of the work that
they do and for using guidance and assistance that is available. Each
has the authority to stop work and report adverse conditions that
affect quality of \dword{dune} products to their respective \dword{dune} consortium
leader and the \dword{lbnf}/\dword{dune} \dword{qam}. The consortium leader responsible
for \dword{dune} components or systems is required to determine and document
their acceptance criteria. \dword{dune} personnel at each level are
responsible for evaluation of quality through self-assessments;
however, independent quality assessments may also be requested by
project management.  The \dword{lbnf}/\dword{dune} \dword{qam} is responsible for
development, implementation, assessment and improvement of the \dword{qa}
program.

The \dword{lbnf}/\dword{dune} \dword{qa} manager is responsible for
periodically reporting on the performance of the quality system to the
\dword{dune} \dword{tcoord} for review and as a basis for
improving the quality system. The \dword{dune} \dword{tcoord}
may call for \dword{qa} plan readiness assessments as the project nears
major milestones. The \dword{dune} \dword{tcoord}, consortia
leaders and \dword{lbnf}/\dword{dune} \dword{qa} manager are all responsible
for providing the resources needed to conduct the project
successfully, including those required to manage, perform and verify
work that affects quality.

\subsection{Quality Assurance Organization}

\dword{lbnf}/\dword{dune} \dword{qa} manager may request personnel
from the \dword{dune} project teams to act on behalf of the
\dword{lbnf}/\dword{dune} \dword{qa} manager to perform quality
assurance functions, based on need, in accordance with the Graded
Approach described above. The requested personnel shall possess
qualifications or receive the appropriate training required to perform
these functions.

\section{Personnel Training and Qualification}

The \dword{dune} consortia leaders are responsible for identifying the
resources to ensure that their team members are adequately trained and
qualified to perform their assigned work. Before allowing personnel to
work independently, they are responsible to ensure that their team
members have the necessary experience, knowledge, skills and
abilities. Personnel qualifications are based on the following
factors:
\begin{itemize}
 \item previous experience, education and training
 \item performance demonstrations or tests to verify previously acquired skills
 \item completion of training or qualification programs
 \item on-the-job training
\end{itemize}

All \dword{dune} consortia leaders are responsible for ensuring that their
training and qualification requirements are fulfilled, including
periodic re-training to maintain proficiency and qualifications.


\section{Quality Improvement}
\label{sec:quality_improvement}

All \dword{dune} consortia members participate in quality improvement
activities that identify opportunities for improvement. They can
respond to the discovery of quality-related issues and follow up on
any required actions. This quality-improvement process requires that
any failures and non-conformances be identified and reported to the
appropriate consortia leader; and, that root causes be identified and
corrected. All consortia members are encouraged to identify problems
or potential quality improvements and may do so without fear of
reprisal or recrimination. Items, services and processes that do not
conform to specified requirements shall be identified and controlled
to prevent their unintended use. Inspection and test reports or
similar tools will be used to implement this requirement. Each
consortia leader is responsible for reporting non-conformances to the
\dword{lbnf}/\dword{dune} \dword{qa} manager and the \dword{lbnf}/\dword{dune}
\dword{qa} manager will periodically report these non-conformances to
\dword{dune} \dword{tc}.

\dword{dune} consortia members will perform Root Cause Analysis and Corrective
and Preventive Actions for conditions that do not meet defined
requirements. Consortia leaders may perform Root Cause analysis and
Corrective and Preventive Actions under their own procedures or
\fnal procedures.  This problem identification, analysis and
resolution process for quality consists of the following steps:
\begin{enumerate}
 \item Identify problem
 \item Understand the process
 \item Grade the process and identify Root Cause Analysis (RCA) method
 \item Identify possible causes
 \item Collect and analyze data
 \item Communicate Lessons Learned and document RCA
 \item Implement Corrective and Preventative Action procedure
\end{enumerate}

\subsection{Lessons Learned}
\label{sec:lessons_learned}

To promote continuous improvement, \dword{dune} \dword{tc} will develop a
lesson learned program based on the \fnal Office of Project Support
Services Lessons Learned Program. This program provides a systematic
approach to identify and analyze relevant information for both good
and adverse work practices that can influence project execution. Where
appropriate, improvement actions are taken to either promote the
repeated application of a positive lesson learned or prevent
recurrence of a negative lesson learned. Lessons learned shall be
gathered throughout the project life cycle. As part of the transition
to operations a lessons learned report will be submitted.

In addition, the \dword{lbnf}/\dword{dune} \dword{qa} manager will
periodically publish a best practices and lessons learned
report. Lessons learned from the \dword{dune} project will be screened
for applicability to other organizations. The \dword{dune} project
will periodically check external lessons learned sources for
applicability to the \dword{dune} project. Sources of lessons learned
include the \dword{doe} Lessons Learned List Server, the \fnal ESH\&Q
Lessons Learned Database, and \dword{dune} team members who
participate in peer reviews of other projects. Reviews of the
\dword{dune} project serve as input to quality improvement.

\section{Documents and Records}

Engineering and scientific documents (including drawings) are prepared
by \dword{dune} personnel to define the design, manufacture and
construction. Ultimately, before these documents are put into effect
they are reviewed and signed by the \dword{dune} consortia leader or
designee. The \dword{dune} project manages all documents under the
document control systems \dword{edms} and \docdb as identified in the
\dword{dune} \dword{cmp}.  The system to control document preparation,
approval, issuance to users and revision is described in the
CMP. Consortia leaders will use the graded approach described in this
plan to determine work in their scope that requires the
\dword{lbnf}/\dword{dune} \dword{qa} manager review and
signature. Project documents that contain quality requirements shall
be reviewed by the \dword{lbnf}/\dword{dune} \dword{qa} manager.

Records are prepared and maintained to document how decisions are
made, for instance, decisions on how to arrive at a design, how to
record the processes followed to manufacture components and the means
and methods of cost and schedule change
control. \dword{lbnf}/\dword{dune} will follow the guidelines for
storing and maintaining records for the project in accordance with
\fnal Records Management(http://ccd.fnal.gov/records). The
\dword{dune} \dword{tcoord}, \dword{lbnf}/\dword{dune}
\dword{qa} manager and consortia leaders are responsible for
identifying the information to be preserved. In addition to the
technical, cost and schedule baseline and all changes to it, records
must be preserved as evidence that a decision was made or an action
taken and to provide the justification for the decision or action.

\section{Work Processes}

\dword{dune} team members are responsible for the quality of their work, and
consortia leaders are responsible for procuring the resources and
support systems to enable their staff to complete their work with high
quality. All \dword{dune} work will be performed using methods that promote
successful completion of tasks, conformance to \dword{dune} requirements and
compliance with the \dword{lbnf}/\dword{dune} \dword{ieshp}. Work processes
consist of a series of actions planned and carried out by qualified
personnel using approved procedures, instructions and equipment, under
administrative, technical and environmental controls, to achieve a
high-quality result.

\subsection{Fabrication Work Processes}

Fabrication work on the \dword{dune} project shall be performed to
established technical standards and administrative controls using
approved instructions and procedures. Fabrication work processes with
\dword{qa} inspections and tests shall be documented on Travelers that are
retained with the hardware item. Items, including consumables, shall
be identified and controlled to ensure their proper use and prevent
the use of incorrect, unaccepted or unidentified items. The consortia
will define a system of controls to ensure that items are handled,
stored, shipped, cleaned and preserved to prevent them from
deteriorating, being damaged or becoming lost. Equipment used for
process monitoring or data collection shall be calibrated and
maintained.

Work shall be performed safely, in a manner that ensures adequate
protection for employees, the public and the environment. Consortia
members and the \dword{dune} \dword{tcoord} shall exercise a degree of
care commensurate with the work and the associated hazards. See the
\dword{lbnf}/\dword{dune} \dword{ieshp} for more details on
\dword{lbnf}/\dword{dune} integrated safety management systems.

\subsection{Change-Controlled Work Processes}

Change-controlled work processes are those for which the \dword{dune}
Change Control Boards (CCB) require that work, both design and
fabrication is tracked. \fixme{Do we need to describe who is currently on the CCB (with names)? When it last met? How it worked in ProtoDUNE...how it will be better in DUNE?}
The CCB assign a unique tracking number to
identify those design and fabrication items for which the associated
change is effective. The \dword{lbnf}/\dword{dune} Configuration
Management Plan defines the change control process in detail.

\section{Design}

The \dword{dune} design process provides appropriate control of design
inputs and design products. The primary design inputs are the
\dword{dune} scientific/engineering requirements (physics
requirements, detector requirements, specifications, drawings,
engineering reports, etc.) as discussed in
Section~\ref{sec:fdsp-coord-requirements} and Configuration Management
documentation provided at the Systems Engineering
website\footnote{https://fermipoint.fnal.gov/project/LBNF/Project\%20Office/Systems\%20Engineering/SitePages/Configuration\%20Management.aspx}. \fixme{I wonder about this Configuration Management web site. It does not mention EDMS, which is where our  documents reside. it mentions ICDs, which are not discussed using that name anywhere in the TDR.}

The basis of the design process requires sound engineering judgment
and practices, adherence to scientific principles, and use of
applicable orders, codes and standards. This basis of the design
process naturally incorporates environment, health and safety
concerns.

\subsection{Design Process}

The \dword{lbnf}/\dword{dune} Systems Engineering website
documentation defines the scope of design work for any given
scientific/engineering work group. From these two sources, work groups
will begin preliminary design of \dword{dune} by breaking their work
down into sets of engineering drawings, specifications and
reports. This is the design output.

Throughout the design process, engineers and designers work with
consortia leaders and the \dword{lbnf}/\dword{dune} \dword{qa} manager to
determine \dword{qa} inspection criteria of fabricated products and
installations. Close coordination must be made with \dword{dune}
scientists to assure the engineering satisfies the scientific
requirements of the experiment. Configuration Management as documented
in the \dword{lbnf}/\dword{dune} \dword{cmp} will be
systematically implemented for \dword{dune}. Final Design work sets
the final Quality Assurance parameters for the parts, assemblies and
installations. Design during Final Design and production is confined
to Change-Controlled changes, as above; and, minor changes necessary
to facilitate production, drawing error correction, material
substitutions and similar functional areas.

\subsection{Design Verification and Validation}
\label{sec:verification}

Design is verified and validated to an extent commensurate with its
importance to safety, complexity of design, degree of standardization,
state of the art and similarity to proven design
approaches. Acceptable verification methods include but are not
limited to any one or combination of (1) design reviews, (2)
alternative calculations, and (3) prototype, qualification testing
and/or (4) comparison of the new design with a similar proven design
if available. Verification work shall be completed before approval and
implementation of the design.

Design reviews shall verify and validate that the following criteria
are met at the appropriate milestone:
\begin{itemize}
 \item Adherence to requirements
 \item Technical adequacy of the design
 \item Adequacy of work instructions
 \item Thoroughness of specifications
 \item Test results
 \item Adequacy of Technical Reports
 \item Adequacy of design calculations and drawings
 \item Reliability and maintainability
 \item Calibration program for measurement and test equipment
\end{itemize}

The \dword{dune} Review Plan as discussed in
Chapter~\ref{vl:tc-review} describes the design reviews recommended
for its consortia.  Wherever the design method involves the use of
computer software to make engineering calculations or static dynamic
models of the structure, system, or component's functionality, the
software must be verified to demonstrate that the software produces
valid results. The verification needs to be documented in a formal
Report of Validation that is maintained in records that are accessible
for inspection. However, exemptions may be made for commercially
available software that is widely used and for codes with an extensive
history of refinement and use by multiple institutions. Exemptions
affecting systems or components shall be identified to the
\dword{lbnf}/\dword{dune} Systems Engineering team.

Critical software and firmware computer codes, especially those codes
that are involved in controlling \dword{dune} data acquisitions systems (DAQ),
shall also be subjected to reviews for verification and
validation. Some items to be considered during computer code review
are as follows:
\begin{itemize}
 \item Adequacy of code testing scheme
 \item Code release control and configuration management
 \item Output data verification against code configuration
 \item Verification that code meets applicable standards
 \item Verification of code compatibility to other systems that use the data
 \item Verification that code meets applicable hardware requirements
 \item Adequacy of code maintenance plans
 \item Adequacy of code and data backup systems
\end{itemize}

Validation ensures that any given design product conforms to \dword{dune}
Science and Engineering Requirements on the Systems Engineering
website. In any review, validation of conformity to requirements
follows verification that the engineering design or computer code
meets all criteria. Engineering designs and computer codes shall be
validated, preferably before procurement, manufacture, or
construction; but no later than acceptance and use of the item; this
is to ensure the design or computer code:
\begin{itemize}
 \item Meets the \dword{dune} requirements
 \item Contains or makes reference to acceptance criteria
 \item Identifies all characteristics crucial to the safe and proper
   use of the equipment or system and its associated interfaces
\end{itemize}

Each inspection, test or review will feed the \dword{qa} evaluation process,
which is a comparison of results with acceptance criteria to determine
acceptance or rejection. Rejection identifies the need for Quality
Improvement based on Section~\ref{sec:quality_improvement}. In some cases, the
outcome of the Quality Improvement process may be to request change(s)
to the design requirements.

\dword{qa} reporting formality escalates as the significance of the
inspection, test or review nonconformance increases. Higher levels of
management must be aware of and participate in the correction of the
most significant nonconformance. Section 4, Quality Improvement,
identifies the required course of action when nonconformance is
encountered.

\section{Procurement}

\subsection{Procurement Controls}

Procurement controls will be implemented to ensure that purchased
items and services meet \dword{dune} requirements and comply with the
\dword{lbnf}/\dword{dune} \dword{qa} plan.  The consortia members requesting
procurement of items and services are responsible for providing all
documentation that adequately describes the item or service being
procured so that the supplier can understand what is required for
consortia acceptance. Development of this documentation may be
achieved through the involvement of consortia Leaders and established
review and approval systems. The following factors will be considered
for review and approval of this documentation:
\begin{itemize}
 \item Inclusion of technical performance requirements
 \item Identification of required codes and standards, laws and regulations
 \item Inclusion of acceptance criteria, including requirements for
   receiving inspection and/or source inspection
 \item \dword{dune} requirements for vendor qualifications and certifications
 \item \dword{dune} intention to perform acceptance sampling in lieu
   of full inspection and test item acceptance
\end{itemize}
NOTE: For Vendor Qualification and acceptance of purchased items or
material by consortia members this may be performed under their own
institution requirements.

Previously accepted Suppliers shall be monitored to ensure that they
continue supplying acceptable items and services. Source surveillance
is the recommended method to ensure that items are free of damage and
that specified requirements are met. Supplier deliveries will be
verified against previously established acceptance criteria.

Unacceptable supplier items or services shall be documented. Records
of supplier performance, Inspection Test Records (ITR) and
contract-required submittals, are kept for future procurement
consideration.

Inspections shall be conducted to detect counterfeit and/or suspect
parts. For work funded by \dword{doe}, when counterfeit/suspect parts are
found, they will be identified, segregated and disposed of in
accordance with the \fnal Quality Assurance Manual Chapter 12020
Suspect/Counterfeit Items (S/CI) Program. \dword{dune} consortia may use their
own institution’s procedure for counterfeit/suspect parts.

\subsection{Inspection and Acceptance Testing}

Inspection and testing of electrical, mechanical and structural
components, associated services, and processes by consortia members
shall be conducted using acceptance and performance criteria. ITR
forms, Travelers, and a Traveler database are the primary tools used
to organize this activity. Inspections will be conducted in accordance
with the Graded Approach.

Equipment used for all inspections and tests shall be calibrated and
maintained. Calibration will be controlled by a system or systems
making appropriate use of qualified calibration service
providers. Consortia Leaders shall ensure that equipment requiring
calibration have their calibration status identified on the item or
container, are traceable back to the calibration documentation and are
tracked to ensure the equipment is calibrated at the required
interval. The \dword{lbnf}/\dword{dune} \dword{qa} manager shall oversee and
support the \dword{dune} calibration programs.
\fixme{Should we say something in this chapter as to how we will accept consortia deliverables in South Dakota?}

\section{Assessments}

\subsection{Management Assessments}

\dword{dune} management at all levels shall regularly evaluate achievement of
personnel relative to performance requirements and shall appropriately
validate or update performance requirements and expectations to ensure
quality of products and processes. The management assessment process
shall periodically include an evaluation of the consortia products
and processes to determine whether the project's missions are being
fulfilled. The results of management assessments that focus on means
of improving the quality of work performed shall be reported to the
appropriate responsible line or project management level.

When performance does not meet established standards, management
shall, with the assistance of others with appropriate expertise,
determine the cause and initiate corrective action. \dword{qa} representatives
may assist, lead or facilitate cause investigations.

\subsection{Independent Assessments}

The \dword{lbnf}/\dword{dune} \dword{qa} manager will plan reviews as
independent assessments to assist the \dword{dune} \dword{tcoord}
in identifying opportunities for quality/performance-based
improvement and to ensure compliance with specified
requirements. Independent assessments of the \dword{dune} projects can
be requested by \dword{dune} management. Independent assessments
typically focus on quality or ESH\&Q management systems, self-
assessment programs, or other organizational functions identified by
management. The \dword{dune} project uses a formal process for
assigning responsibility in response to recommendations from
independent assessments. These recommendations are tracked to closure.

Personnel conducting independent assessments shall be technically
qualified and knowledgeable in the areas assessed. A qualified lead
assessor (auditor), who is a Subject Matter Expert (SME) in the
technical area of assessment, is required. The team may include other
SMEs to evaluate the adequacy and effectiveness of activities only if
they are not responsible for the work being assessed.

The \fnal Directorate appoints an independent Long Baseline Neutrino
Committee (LBNC) to advise it and \dword{dune} Management. The role of
this standing committee is described in the \dword{lbnf}/\dword{dune}
PMP. The \dword{doe} and other funding agencies perform external
assessments that provide an objective view of performance and thus
contribute to the independent assessment process. Since such
assessments are not under the control of \dword{dune}, they are not
necessarily considered a part of the independent assessment
criterion. However, \dword{dune} management considers external
assessment results in determining the scope and schedule of
independent assessments.

\section{ProtoDUNE to DUNE \dword{qa} Approach}

The approach to \dword{qa}/\dword{qc} for \dword{dune} is going to be very
similar to the activities and oversight that was performed for
\dword{protodune}.  For \dword{protodune}, the major \dword{qa}/\dword{qc}
activities included the review consortia \dword{tdr} \dword{qa}/\dword{qc}
Sections; assisting the consortia in development and review of \dword{qc}
Plans (Production and Installation), fabrication, inspection and test
procedures, installation plans and documentation; and the performance
of \dword{prr} at eleven consortia fabrication facilities.  There was
also \dword{qa} participation in the \dword{protodune} Design Reviews.
The \dword{prr} looked at the following criteria:
\begin{itemize}
  \item Final \dword{qa} plans for institutions not adopting the
    \dword{lbnf}/\dword{dune} \dword{qa} Plan;
  \item Final production drawings, specifications and manufacturing
    and test procedures;
  \item Final safety documents (i.e., Hazard Analysis documentation);
  \item Component \dword{qc} plan (i.e., travelers, test reports, software
    verification and validation documents, supplier documentation);
  \item Final procurement documents per institution practice,
  \item Completion and evaluation of prototypes, review of production
    process and \dword{qc} results
\end{itemize}
The reviews ensured the facilities were prepared for production and
any kinks in the processes had been identified and mitigations have
taken place. The positive outcome of these reviews was the amount of
equipment received at CERN with little to no damage.

\dword{ppr} will be performed at the fabrication facilities for the
\dword{dune} detector components. The review has been added due to the
larger number of components required and the increased number of
fabrication facilities. The goal of these reviews is to ensure
consistent fabrication processes between the facilities. If an issue
is identified at one facility it can be communicated to the other
applicable facilities to prevent the same issue occurring.

For \dword{protodune}, installation was performed at CERN under the
guidance of CERN policies and procedures. Installation of the
\dword{dune} detector at \surf will fall under similar procedures in the
\dword{lbnf}/\dword{dune} \dword{qa} plan.
