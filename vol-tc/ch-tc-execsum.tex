\chapter{Executive Summary}
\label{vl:tc-execsum}


\dword{dune} \dword{tc} holds general responsibility for design and
construction of the \dword{dune} detectors. \dword{tc} supports
integration and installation of the detectors at the far site under
the direction of the \dword{jpo}.  As part of its responsibilities
\dword{tc} manages interfaces of \dword{dune} to \dword{fnal} as the host
laboratory, \dword{lbnf} and \dword{surf}, which houses and supports
the \dword{dune} experiment. This includes interfaces to other surface
facilities at or near \dword{surf} operated for installation and
operations activities by the \dword{jpo} and to certain other common test
facilities operated for the benefit of \dword{dune}. Led by the
\dword{dune} \dword{tcoord}, \dword{dune} \dword{tc} provides
oversight and organizational support to the consortia building the
\dword{dune} detector.  This includes project functions such as
scheduling, requirements, risk management, value engineering,
interfaces to national project management and reporting. These project
functions are coordinated with the \dword{jpo}.  The \dword{tc}
organization establishes and maintains standards and tools for all
drawings and schematics, for grounding and shielding, and for all \dword{dune}
subsystem-to-subsystem and \dword{dune}-to-\dword{lbnf} interfaces.
\dword{tc} includes \dword{lbnf-dune} \dword{esh} Manager
and the \dword{lbnf-dune} \dword{qa} Manager. \dword{dune}
\dword{tc} manages the design and production 
reviews, maintains the overall schedule and monitors progress.

Chapters that follow lay out the structure of the \dword{dune} \dword{tc},
describe \dword{dune} facilities, and detail project-related functions
and methodologies. At the stage of this early draft to the  \dword{lbnc},
\dword{dune} \dword{tc} functional examples are most fully developed
for the \dword{sp} \dword{detmodule}. The volume will be expanded to
encompass more functional examples associated with \dword{dp}
\dword{detmodule} by the final draft.

