\chapter{Executive Summary}
\label{vl:tc-execsum}

This volume describes how the activities required to design, construct,
fabricate, install, and commission the \dword{dune} Far Detector modules
are organized and managed.  The \dword{dune} Far Detector construction
project is one piece of the global \dword{lbnf-dune}, which encompasses
all of the facilities, supporting infrastructure, and detector elements
required to carry out the \dword{dune} science program.

The \dword{dune} Collaboration has direct responsibility for the design
and construction of the \dword{dune} detectors.  Different groupings
of collaboration institutes, referred to as ``consortia'', are assigned
responsibility for individual detector subsystems.  The activities
of the consortia are overseen and coordinated through the \dword{dune}
\dword{tc} organization headed by the \dword{dune} \dword{tcoord}. 
The \dword{tc} organization provides project support functions such as
safety coordination, engineering integration, change control, document
management, scheduling, risk management, and technical review planning.
\dword{dune} \dword{tcoord} manages internal, subsystem-to-subsystem
interfaces and is responsible for ensuring the proper integration of
the different subsystems.

\dword{dune} \dword{tc} works closely with the support teams of its
\dword{lbnf-dune} partners within the framework of the \dword{jpo} to
ensure coherency in project support functions across the entire global
enterprise.  \dword{lbnf-dune} \dword{esh} and \dword{qa} Managers are
embedded both within the \dword{dune} \dword{tc} organization and the
\dword{jpo} to ensure that the \dword{dune} \dword{esh} and \dword{qa}
programs are consistent across the project and meet \dword{lbnf-dune}
requirements.  Engineering standards and documentation requirements
are established centrally through the \dword{jpo}, which also takes
responsibilty for common engineering tools used to model detector
components and create production drawings.  The \dword{jpo} manages
external \dword{dune} detector interfaces with \dword{lbnf} and is
responsible for ensuring proper integration of the detector elements
within the facilities and supporting infrastructure.

The \dword{jpo} organization will evolve to incorporate the on-site
team responsible for coodinating on-site integration and installation
activities at \dword{surf} under the direction of the \dword{ipd}.
Detector integration and integration activities are supported by the
\dword{dune} consortia, which maintain responsibility for ensuring
proper installation and commissioning of their subsystems.  External
\dword{dune} interfaces with the on-site integration and installation
activites are managed through the \dword{jpo}.

The ordering of the subsequent chapters is structed to provide first
additional detail regarding the organizational strucutres summarized
here; second overiews of the facilities, supporting infrastrucutre,
and detectors for context; and third information on project-related
functions and methodologies utilized by \dword{dune} \dword{tcoord}
focusing on the areas of integration engineering, technical reviews,
\dword{qa}, and safety oversight.  Due to their more advanced stage of
development, functional examples presented here are focused primarily
on the \dword{sp} \dword{detmodule}.
