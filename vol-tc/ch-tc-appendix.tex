%%%%%%%%%%%%%%%%%%%%%%%%%%%%%%%%
\section{Interface Documents}
\label{sec:fdsp-coord-interface}

\forlbnc{This section will include the interface matrix, with links to
  the interface documents and discuss the interface matrix and
  documents.}


%%%%%%%%%%%%%%%%%%%%%%%%%%%%%%%%
\section{Cost}
\label{sec:fdsp-coord-cost}

\forlbnc{This section will discuss the cost estimate.}

%%%%%%%%%%%%%%%%%%%%%%%%%%%%%%%%
\section{Schedule}
\label{sec:fdsp-coord-controls}

A series of tiered milestones have been developed for the \dword{dune}
project. The spokespersons and host laboratory director are
responsible for the tier 0 milestones. Three tier 0 milestones have
been defined and the dates set:
\begin{enumerate}
\item Start main cavern excavation \hspace{2.58in} 2020
\item Start \dword{detmodule}~1 installation \hspace{2.1in} 2024
\item Start operations of \dwords{detmodule} \#1--2 with beam \hspace{0.8in} 2028
\end{enumerate}
These dates will be revisited this spring after the \dword{tdr} is reviewed. The
\dword{tcoord} and \dword{lbnf} project manager hold the Tier 1
milestones; these milestones will be defined in advance of the
\dword{tdr} review. The consortia themselves hold the Tier 2
milestones.

Table~\ref{tab:DUNE_schedule} provides a high level version of the
\dword{dune} milestones from the \dword{lbnf-dune} schedule.
\begin{dunetable}
[Schedule]{p{0.65\textwidth}p{0.25\textwidth}}{tab:DUNE_schedule}{Schedule}   
Milestone & Date   \\ \toprowrule
  RRB Approval of Technical Design Review                       &  \\ \colhline
  Construction of steel frame for Cryostat \#2 complete         &  \\ \colhline
\rowcolor{dunepeach} Start of \dword{pdsp}-II installation& \startpduneiispinstall      \\ \colhline
\rowcolor{dunepeach} Start of \dword{pddp}-II installation& \startpduneiidpinstall      \\ \colhline
 \dword{prr} dates &      \\ \colhline
\rowcolor{dunepeach}South Dakota Logistics Warehouse available& \sdlwavailable      \\ \colhline
\rowcolor{dunepeach}Beneficial occupancy of cavern 1 and \dword{cuc}& \cucbenocc      \\ \colhline
  Construction of steel frame for Cryostat \#1 complete         &  \\ \colhline
\rowcolor{dunepeach} \dword{cuc} counting room accessible& \accesscuccountrm      \\ \colhline
\rowcolor{dunepeach}Top of \dword{detmodule} \#1 cryostat accessible& \accesstopfirstcryo      \\ \colhline
\rowcolor{dunepeach}Start of \dword{detmodule} \#1 TPC installation& \startfirsttpcinstall      \\ \colhline
\rowcolor{dunepeach}End of \dword{detmodule} \#1 TPC installation& \firsttpcinstallend      \\ \colhline
  Cryostat \#1 ready for filling                                &  \\ \colhline
  \textbf{Detector \#1 ready for operations}                    & \textbf{} \\ \colhline
\rowcolor{dunepeach}Top of \dword{detmodule} \#2 accessible& \accesstopsecondcryo      \\ \colhline
 \rowcolor{dunepeach}Start of \dword{detmodule} \#2 TPC installation& \startsecondtpcinstall      \\ \colhline
\rowcolor{dunepeach}End of \dword{detmodule} \#2 TPC installation& \secondtpcinstallend      \\ \colhline
  Cryostat \#2 ready for filling                                &  \\ \colhline
  \textbf{Detector \#2 ready for operations}                    & \textbf{} \\
\end{dunetable}
To monitor progress, \dword{tc} will maintain the \dword{lbnf-dune} schedule that
links all consortium schedules and contains milestones for each
consortia.  The schedules will go under change control after each
consortium agrees to the milestone dates and the \dword{tdr} is
approved.

To ensure that the \dword{dune} detector remains on schedule,
\dword{tc} will monitor schedule status from each consortium and organize
reviews of schedules and risks as appropriate.  As schedule problems
arise, \dword{tc} will work with affected consortium to resolve the
problems. If problems cannot be solved, the \dword{tc} will take the issue to the
\dword{tb} and \dword{exb}.

A monthly report with input from all consortia will be published by
\dword{tc}. This will include updates on consortium and \dword{tc}
technical progress against the schedule.


%%%%%%%%%%%%%%%%%%%%%%%%%%%%%%%%
\section{Requirements}
\label{sec:fdsp-coord-requirements}

The scientific goals of \dword{dune} as described in \dword{dune}
\dword{tdr} Volume~\volnumberexec:~\voltitleexec include
\begin{itemize}
\item a comprehensive program of neutrino oscillation measurements
  including the search for CP violation
\item measurement of $\nu_{e}$ flux from a core-collapse supernova within our
  galaxy should one occur during \dword{dune} operations
\item searching for baryon number violation
\end{itemize}
These goals motivate a number of key detector requirements: drift
field, electron lifetime, system noise, photon detector light yield
and time resolution. The \dword{exb} has approved a list of high
level detector specifications, including those listed above. These are
maintained in edms-xxxx, and the high level requirements with
significant impact on physics are highlighted in
Table~\ref{tab:dunephysicsreqs}.
\begin{dunetable}
  [\dword{dune} physics-related specifications owned by \dword{exb}]
  {p{0.025\textwidth}p{0.06\textwidth}p{0.2\textwidth}p{0.35\textwidth}p{0.15\textwidth}p{0.1\textwidth}}
  {tab:dunephysicsreqs}
  {\dword{dune} physics-related specifications owned by \dword{exb}}
  ID & System & Parameter & Physics Requirement Driver & Requirement & Goal \\ \toprowrule
  1   & HVS    & Minimum drift field &  Limit recombination, diffusion and space charge impacts on particle ID. Establish adequate \dword{s/n} for tracking. & >\SI{250}{V/cm} & \spmaxfield \\ \colhline
  2   & CE     & System noise & The noise specification is driven by pattern recognition and two-track separation.  & <\SI{1000}{enc} & ALARA \\ \colhline
  3   & PDS    & Light yield  & The light yield shall be sufficient to measure time of events with visible energy above 200 MeV.  Goal is 10\% energy measurement for visible energy of 10 MeV.  & >\SI{0.5}{pe/MeV} & >\SI{5}{pe/MeV}  \\ \colhline
  4   & PDS    & Time resolution  & The time resolution of the photon detection system shall be sufficient to assign a unique event time.  & $<\,\SI{1}{\micro\second}$ & $<\,\SI{100}{\nano\second}$  \\ \colhline
  5   & all    & liquid argon purity & The LAr purity shall be sufficient to enable drift e- lifetime of 3 (10)ms & $<$\,\SI{100}{ppt} & $<$\,\SI{30}{ppt} \\ \colhline
\end{dunetable}
Eleven other significant specifications owned by the \dword{exb} are
listed in Table~\ref{tab:dunephysicsspecs} along with another twelve
high level engineering specifications.
\begin{dunetable}
  [\dword{dune} physics-related specifications owned by \dword{exb}]
  {p{0.025\textwidth}p{0.06\textwidth}p{0.2\textwidth}p{0.35\textwidth}p{0.15\textwidth}p{0.1\textwidth}}
  {tab:dunephysicsspecs}
  {\dword{dune} high level system specifications owned by \dword{exb}}
  ID & System & Parameter & Physics Requirement Driver & Requirement & Goal \\ \toprowrule
  6   & APA & Gaps between APAs  & minimize events lost due to vertex in gaps between APAs (15mm on same support beam, 30mm on adjacent beams) & <\SI{30}{mm} & <\SI{15}{mm} \\ \colhline
  7   & DSS & Drift field uniformity & tolerance on drift field due to component location & $<\,\SI{1}{\%}$  &   \\ \colhline
  8   & APA & wire angles  & 0$^\circ$ collection, $\pm$35.7$^\circ$ induction &  &  \\ \colhline
  9   & APA & wire spacing  & \SI{4.669}{mm} for U,V; \SI{4.790}{mm} for X,G &  &  \\ \colhline
  10  & APA & wire position tolerance  & & $\pm\,\SI{0.5}{mm}$  &  \\ \colhline
  11  & HVS & Drift field uniformity & tolerance on drift field due to HVS system & $<\,\SI{1}{\%}$  &  \\ \colhline
  12  & HVS & Cathode power supply ripple & very small compared to intrinsic electronics noise & $<\,\SI{100}{enc}$ &   \\ \colhline
  13  & CE & Frontend peaking time  & optimize vertex resolution & \SI{1}{\micro\second} &  \\ \colhline
  14  & CE & Signal saturation  & largest signals occur with multiple protons in the primary vertex & 500k $e^-$ &  \\ \colhline
  15  & cryo & LAr N$_2$ contamination  & optical attenuation length in liquid argon with 50~ppm of N$_2$ contamination is roughly 3~m & $<\,\SI{25}{ppm}$ &  \\ \colhline
  16  & all & Detector dead time  & risk of missing a supernova burst if all operating cryostats are offline & $<\,\SI{0.5}{\%}$ &  \\ \colhline
\end{dunetable}
The high level \dword{dune} requirements that drive the \dword{lbnf} design are
maintained in DocDB-112 and under change control. These are owned by
the \dword{dune} \dword{tc} and the \dword{lbnf} project manager.

Lower level detector specifications are held by the consortia and
described in the \dword{dune} \dword{tdr} \dword{dsp}
Volume~\volnumbersp\ and \dword{ddp} Volume~\volnumberdp\ chapters for
each consortium. A complete list of detector specifications is
provided in Chapter~\ref{sec:fdsp-app-requirements}.

%%%%%%%%%%%%%%%%%%%%%%%%%%%%%%%%
\section{Risks}
\label{sec:fdsp-coord-risks}

\forlbnc{\dword{dune} is in the process of updating the Risk Registry
  as part of wirting the \dword{tdr}. Each consortia, along with
  \dword{tc} overall and installation are at various stages of
  updates. The last update was in spring 2018 before \dword{protodune}
  ran. Our intention is to validate each consortia updated risk before
  final submission of the \dword{tdr}.}

\dword{dune} initiated a risk registry in 2018 (available in
DocDB-6443). This document includes tabs for consortia risks and
\dword{tc} risks. It includes a summary tab for the most significant
overall \dword{dune} risks.  This registry has been updated for the
\dword{tdr} and the full listing can be found in Appendix
Section~\ref{sec:fdsp-app-risk}. It is expected to be updated
approximately twice a year. The last update was in early 2018 before
ProtoDUNE was completed and the next update is planned for late
2019. Several risks associated with ProtoDUNE will
be retired. \dword{lbnf} and \dword{dune}-US would like \dword{dune} to update and
expand this risk register to allow a Monte Carlo analysis of cost and
schedule risks to the \dword{us} project resulting from international
\dword{dune} risks. This request is under consideration as it may be
useful for other national projects as well.
Successfully operating \dword{protodune} retired many \dword{dune}
risks in \dword{dune}. This includes most risks associated with the
technical design, production processes, \dword{qa}, integration and
installation. Residual risks remain relating to design and production
modifications associated with scaling to \dword{dune}, mitigations to
known installation and performance issues in \dword{protodune},
underground installation at \surf and organizational growth.

The highest technical risks include development of a system to
deliver \SI{600}{kV} to the \dual cathode; general delivery of the
required \dword{hv}; cathode and \dword{fc} discharge to the cryostat
membrane; noise levels, particularly for the \dword{ce}; %cold TPC electronics,
number of dead channels; lifetime of components surpassing \dunelifetime{}; %20 years,
\dword{qc} of all components; verification of improved \dword{lem}
performance; verification of new cold  \dword{adc} and  \dword{coldata} performance;
argon purity; electron drift lifetime; \phel light yield;
incomplete calibration plan; and incomplete connection of design to
physics. Other major risks include insufficient funding, optimistic
production schedules, incomplete plans for integration, testing and installation.

One update to the risk registry since 2018 has been for \dword{tc}, in which some
risks  associated with  DUNE  integration and  installation have  been
added (see Volume~\volnumbersp:~\voltitlesp, Table...)

In addition to installation related risks \dword{tc} is developing its
own set of overall project risks not captured by conortia.  Key risks
for \dword{tc} to manage include the following:
\begin{enumerate}
\item Consortia leave too much scope unaccounted for and too much falls
  to  the \dword{comfund}.
\item Insufficient organizational systems are put into place to
  ensure that this complex international mega-science project,
  including \dword{tc}, \fnal as host laboratory, \surf, DOE, and all international
  partners continue to work together successfully to ensure
  appropriate processes and services are provided for the success of
  the project.
\item Inability of \dword{tc} to obtain sufficient personnel resources to
  ensure that \dword{tc} can oversee and coordinate all project tasks.  While the \dword{us}, 
  as host country, has a special responsibility to \dword{tc}, personnel resources should
  be directed to \dword{tc} from each collaborating country. 
\end{enumerate}

The consortia have provided preliminary versions of risk analyses that
have been collected on the \dword{tc} webpage (DocDB-6443). These have
been developed into an overall risk register that will be monitored
and maintained by \dword{tc} in coordination with the consortia. This
full set of risks can be found in Appendix
Section~\ref{sec:fdsp-app-risk}.

%%%%%%%%%%%%%%%%%%%%%%%%%%%%%%%%
\section{Full DUNE Requirements}
\label{sec:fdsp-app-requirements}

%\input{vol-tc/ch-tc-requirements}
\fixme{The first tables in each section, the top-level requirements for SP and DP, respectively, repeat the first five specs. I've asked Brett to look at the code. Anne}

%%%%%%%%%%%%%%%%%%%%%%%%%%%%%%%
\subsection{Single-phase}
\label{sec:tc-req-sp}

% This file is generated, any edits may be lost.
\begin{footnotesize}
%\begin{longtable}{p{0.14\textwidth}p{0.13\textwidth}p{0.18\textwidth}p{0.22\textwidth}p{0.20\textwidth}}
\begin{longtable}{p{0.12\textwidth}p{0.18\textwidth}p{0.17\textwidth}p{0.25\textwidth}p{0.16\textwidth}}
\caption{Specifications for SP-FD \fixmehl{ref \texttt{tab:spec:SP-FD}}} \\
  \rowcolor{dunesky}
       Label & Description  & Specification \newline (Goal) & Rationale & Validation \\  \colhline


   \newtag{SP-FD-1}{ spec:min-drift-field }  & Minimum drift field  &  $>$\,\SI{250}{ V/cm} \newline ( $>\,\SI{500}{ V/cm}$ ) &  Lessens impacts of $e^-$-Ar recombination, $e^-$ lifetime, $e^-$ diffusion and space charge. &  ProtoDUNE \\ \colhline
    
   
  \newtag{SP-FD-2}{ spec:system-noise }  & System noise  &  $<\,\SI{1000}\,e^-$ &  Provides $>$5:1 S/N on induction planes for  pattern recognition and two-track separation. &  ProtoDUNE and simulation \\ \colhline
    
   
  \newtag{SP-FD-3}{ spec:light-yield }  & Light yield  &  $>\,\SI{20}{PE/MeV}$ (avg), $>\,\SI{0.5}{PE/MeV}$ (min) &  Gives PDS energy resolution comparable that of the TPC for 5-7 MeV SN $\nu$s, and allows tagging of $>\,\SI{99}{\%}$ of nucleon decay backgrounds with light at all points in detector. &  Supernova and nucleon decay events in the FD with full simulation and reconstruction. \\ \colhline
    
    \\ \rowcolor{dunesky} \newtag{SP-FD-4}{ spec:time-resolution-pds } & Name: Time resolution \\
    Description & The time resolution of the photon detection system shall be less than 1 microsecond in order to assign a unique event time.   \\  \colhline
    Specification (Goal) &  $<\,\SI{1}{\micro\second}$  ( $<\,\SI{100}{\nano\second}$ ) \\   \colhline
    Rationale &   Enables \SI{1}{mm} position resolution for \SI{10}{MeV} SNB candidate events for instantaneous rate $<\,\SI{1}{m^{-3}ms^{-1}}$.  \\ \colhline
    Validation &   \\
   \colhline

   \newtag{SP-FD-5}{ spec:lar-purity }  & Liquid argon purity  &  $<$\,\SI{100}{ppt} \newline ($<\,\SI{30}{ppt}$) &  Provides $>$5:1 S/N on induction planes for  pattern recognition and two-track separation. &  Purity monitors and cosmic ray tracks \\ \colhline
    
    \\ \rowcolor{dunesky} \newtag{SP-FD-6}{ spec:apa-gaps } & Name: Gaps between APAs  \\
    Description & The gap size between APAs shall minimize loss of fiducial volume and distortion of charge collection.   \\  \colhline
    Specification &  $<\,\SI{15}{mm}$ between APAs on same support beam; $<\,\SI{30}{mm}$ between APAs on different support beams \\   \colhline
    Rationale &   Maintains fiducial volume.  Simplified contruction.  \\ \colhline
    Validation & ProtoDUNE  \\
   \colhline

    \\ \rowcolor{dunesky} \newtag{SP-FD-7}{ spec:misalignment-field-uniformity } & Name: Drift field uniformity due to component alignment \\
    Description & Misalignments of the various TPC components shall not introduce drift-field nonuniformities beyond those specified in the HVS requirements.   \\  \colhline
    Specification &  $<\,1\,$\% throughout volume \\   \colhline
    Rationale &   Maintains APA, CPA,  FC orientation and shape.  \\ \colhline
    Validation & ProtoDUNE  \\
   \colhline

    
   
  \newtag{SP-FD-8}{ spec:apa-wire-angles }  & APA wire angles  &  \SI{0}{\degree} for collection wires, \SI{35.7}{\degree} for induction wires &  Minimize inter-APA dead space. &  Engineering calculation \\ \colhline
    
    
   
  \newtag{SP-FD-9}{ spec:apa-wire-spacing }  & APA wire spacing  &  \SI{4.669}{mm} for U,V; \SI{4.790}{mm} for X,G &  Enables 100\% efficient MIP detection, \SI{1.5}{cm} $yz$ vertex resolution. &  Simulation \\ \colhline
    
   
  \newtag{SP-FD-10}{ spec:apa-wire-pos-tolerance }  & APA wire position tolerance  &  $\pm\,\SI{0.5}{mm}$ &  Interplane electron transparency; $dE/dx$, range, and MCS calibration. &  ProtoDUNE and simulation \\ \colhline
    
    \\ \rowcolor{dunesky} \newtag{SP-FD-11}{ spec:hvs-field-uniformity } & Name: Drift field uniformity due to HVS \\
    Description & Design of TPC cathode and FC components shall ensure uniform field.  Production tolerances shall be set so as to maintain flatness of component surfaces and, by extension, the shape of the drift field volume.   \\  \colhline
    Specification &  $<\,\SI{1}{\%}$ throughout volume \\   \colhline
    Rationale &   High reconstruction efficiency.  \\ \colhline
    Validation & ProtoDUNE and simulation  \\
   \colhline

   
  \newtag{SP-FD-12}{ spec:hv-ps-ripple }  & Cathode HV power supply ripple contribution to system noise  &  $<\,\SI{100}e^-$ &  Maximize live time; maintain high S/N. &  Engineering calculation, in situ measurement,   ProtoDUNE \\ \colhline
    
    
   \newtag{SP-FD-13}{ spec:fe-peak-time }  & Front-end peaking time  &  \SI{1}{\micro\second} \newline ( Adjustable so as to see saturation in less than \SI{10}{\%} of beam-produced events ) &  Vertex resolution; optimized for \SI{5}{mm} wire spacing. &  ProtoDUNE and simulation \\ \colhline
    
   
  \newtag{SP-FD-14}{ spec:sp-signal-saturation }  & Signal saturation level  &  \num{500000} electrons &  Maintain calorimetric performance for multi-proton final state. &  Simulation \\ \colhline
    
    \\ \rowcolor{dunesky} \newtag{SP-FD-15}{ spec:lar-n-contamination } & Name: LAr nitrogen contamination \\
    Description & The nitrogen contamination in the LAr shall remain below 25 ppm in order not to significantly affect the number of photons that reach the detectors (for both fast and late light components).   \\  \colhline
    Specification &  $<\,\SI{25}{ppm}$ \\   \colhline
    Rationale &   Maintain \SI{0.5}{PE/MeV} PDS sensitivity required for triggering proton decay near cathode.  \\ \colhline
    Validation & In situ measurment  \\
   \colhline

   
  \newtag{SP-FD-16}{ spec:det-dead-time }  & Detector dead time  &  $<\,\SI{0.5}{\%}$ &  Meet physics goals in timely fashion. &  ProtoDUNE \\ \colhline
    
    \\ \rowcolor{dunesky} \newtag{SP-FD-17}{ spec:cathode-resistivity } & Name: Cathode resistivity \\
    Description & The cathode resistivity shall ensure that in the event of an HV discharge, the release of the large stored energy is spread out over time.    \\  \colhline
    Specification (Goal) &  $>\,\SI{1}{\mega\ohm/square}$  ( $>\,\SI{1}{\giga\ohm/square}$ ) \\   \colhline
    Rationale &   Detector damage prevention.  \\ \colhline
    Validation & ProtoDUNE  \\
   \colhline

    
   
  \newtag{SP-FD-18}{ spec:cryo-monitor-devices }  & Cryogenic monitoring devices  &   &  Constrain uncertainties on detection efficiency, fiducial volume. &  ProtoDUNE \\ \colhline
    
    
   
  \newtag{SP-FD-19}{ spec:adc-sampling-freq }  & ADC sampling frequency  &  $\sim\,\SI{2}{\mega\hertz}$ &  Match \SI{1}{\micro\second} shaping time. &  Nyquist requirement and design choice \\ \colhline
    
    
   \newtag{SP-FD-20}{ spec:adc-number-of-bits }  & Number of ADC bits  &  \num{12} bits \newline ( \num{13} bits ) &  ADC noise contribution negligible (low end); match signal saturation specification (high end). &  Engineering calculation and design choice \\ \colhline
    
   
  \newtag{SP-FD-21}{ spec:ce-power-consumption }  & Cold electronics power consumption   &  $<\,\SI{50}{ mW/channel} $ &  No bubbles in LAr to redice HV discharge risk. &  ProtoDUNE \\ \colhline
    
   
  \newtag{SP-FD-22}{ spec:data-rate-to-tape }  & Data rate to tape  &  $<\,\SI{30}{PB/year}$ &  Cost.  Bandwidth. &  ProtoDUNE \\ \colhline
    
   
  \newtag{SP-FD-23}{ spec:sn-trigger }  & Supernova trigger  &  $>\,\SI{90}{\%}$ efficiency for SNB within \SI{100}{kpc} &  $>\,$90\% efficiency for SNB within 100 kpc &  Simulation and bench tests \\ \colhline
    
    \\ \rowcolor{dunesky} \newtag{SP-FD-24}{ spec:local-e-fields } & Name: Local electric fields \\
    Description & The integrated detector design shall minimize potential pathways for HV discharges.   \\  \colhline
    Specification &  $<\,\SI{30}{kV/cm}$ \\   \colhline
    Rationale &   Maximize live time; maintain high S/N.  \\ \colhline
    Validation & ProtoDUNE  \\
   \colhline

   
  \newtag{SP-FD-25}{ spec:non-fe-noise }  & Non-FE noise contributions  &  $<<\,\SI{1000}{enc} $ &  High S/N for high reconstruction efficiency. &  Engineering calculation and ProtoDUNE \\ \colhline
    
    \\ \rowcolor{dunesky} \newtag{SP-FD-26}{ spec:lar-impurity-contrib } & Name: LAr impurity contributions from components \\
    Description & Contributions to LAr contamination from detector components, through outgassing or other processes, shall remain << 30 ppt so as to avoid significantly increasing the nominal level of contamination.   \\  \colhline
    Specification &  $<<\,\SI{30}{ppt} $ \\   \colhline
    Rationale &   Maintain HV operating range for high live time fraction.  \\ \colhline
    Validation & ProtoDUNE  \\
   \colhline

    \\ \rowcolor{dunesky} \newtag{SP-FD-27}{ spec:radiopurity } & Name: Introduced radioactivity \\
    Description & Introduced radioactivity shall be less than that from 39Ar.   \\  \colhline
    Specification &  less than that from $^{39}$Ar \\   \colhline
    Rationale &   Maintain low radiological backgrounds for SNB searches.  \\ \colhline
    Validation & ProtoDUNE and assays during construction  \\
   \colhline

    
   
  \newtag{SP-FD-28}{ spec:dead-channels }  & Dead channels  &  $<\,\SI{1}{\%}$ &  Contingency for possible efficiency loss for $>\,$20 year operation.  &  ProtoDUNE \\ \colhline
    


\label{tab:specs:SP-FD}
\end{longtable}
\end{footnotesize}

% This file is generated, any edits may be lost.

% It defines macros which expand to corresponding
% specification values for subsystem SP-APA



\begin{longtable}{p{0.14\textwidth}p{0.13\textwidth}p{0.18\textwidth}p{0.22\textwidth}p{0.20\textwidth}}
\caption{Specifications for SP-APA \fixmehl{ref \texttt{tab:spec:SP-APA}}} \\
  \rowcolor{dunesky}
       Label & Description  & Specification \newline (Goal) & Rationale & Validation \\  \colhline

   \newtag{SP-FD-1}{ spec:min-drift-field }  & Minimum drift field  &  $>$\,\SI{250}{ V/cm} \newline ( $>\,\SI{500}{ V/cm}$ ) &  Lessens impacts of $e^-$-Ar recombination, $e^-$ lifetime, $e^-$ diffusion and space charge. &  ProtoDUNE \\ \colhline
    
   
  \newtag{SP-FD-2}{ spec:system-noise }  & System noise  &  $<\,\SI{1000}\,e^-$ &  Provides $>$5:1 S/N on induction planes for  pattern recognition and two-track separation. &  ProtoDUNE and simulation \\ \colhline
    
   
  \newtag{SP-FD-3}{ spec:light-yield }  & Light yield  &  $>\,\SI{20}{PE/MeV}$ (avg), $>\,\SI{0.5}{PE/MeV}$ (min) &  Gives PDS energy resolution comparable that of the TPC for 5-7 MeV SN $\nu$s, and allows tagging of $>\,\SI{99}{\%}$ of nucleon decay backgrounds with light at all points in detector. &  Supernova and nucleon decay events in the FD with full simulation and reconstruction. \\ \colhline
    
    \\ \rowcolor{dunesky} \newtag{SP-FD-4}{ spec:time-resolution-pds } & Name: Time resolution \\
    Description & The time resolution of the photon detection system shall be less than 1 microsecond in order to assign a unique event time.   \\  \colhline
    Specification (Goal) &  $<\,\SI{1}{\micro\second}$  ( $<\,\SI{100}{\nano\second}$ ) \\   \colhline
    Rationale &   Enables \SI{1}{mm} position resolution for \SI{10}{MeV} SNB candidate events for instantaneous rate $<\,\SI{1}{m^{-3}ms^{-1}}$.  \\ \colhline
    Validation &   \\
   \colhline

   \newtag{SP-FD-5}{ spec:lar-purity }  & Liquid argon purity  &  $<$\,\SI{100}{ppt} \newline ($<\,\SI{30}{ppt}$) &  Provides $>$5:1 S/N on induction planes for  pattern recognition and two-track separation. &  Purity monitors and cosmic ray tracks \\ \colhline
    
    \\ \rowcolor{dunesky} \newtag{SP-FD-6}{ spec:apa-gaps } & Name: Gaps between APAs  \\
    Description & The gap size between APAs shall minimize loss of fiducial volume and distortion of charge collection.   \\  \colhline
    Specification &  $<\,\SI{15}{mm}$ between APAs on same support beam; $<\,\SI{30}{mm}$ between APAs on different support beams \\   \colhline
    Rationale &   Maintains fiducial volume.  Simplified contruction.  \\ \colhline
    Validation & ProtoDUNE  \\
   \colhline

    \\ \rowcolor{dunesky} \newtag{SP-FD-7}{ spec:misalignment-field-uniformity } & Name: Drift field uniformity due to component alignment \\
    Description & Misalignments of the various TPC components shall not introduce drift-field nonuniformities beyond those specified in the HVS requirements.   \\  \colhline
    Specification &  $<\,1\,$\% throughout volume \\   \colhline
    Rationale &   Maintains APA, CPA,  FC orientation and shape.  \\ \colhline
    Validation & ProtoDUNE  \\
   \colhline

    
   
  \newtag{SP-FD-8}{ spec:apa-wire-angles }  & APA wire angles  &  \SI{0}{\degree} for collection wires, \SI{35.7}{\degree} for induction wires &  Minimize inter-APA dead space. &  Engineering calculation \\ \colhline
    
    
   
  \newtag{SP-FD-9}{ spec:apa-wire-spacing }  & APA wire spacing  &  \SI{4.669}{mm} for U,V; \SI{4.790}{mm} for X,G &  Enables 100\% efficient MIP detection, \SI{1.5}{cm} $yz$ vertex resolution. &  Simulation \\ \colhline
    
   
  \newtag{SP-FD-10}{ spec:apa-wire-pos-tolerance }  & APA wire position tolerance  &  $\pm\,\SI{0.5}{mm}$ &  Interplane electron transparency; $dE/dx$, range, and MCS calibration. &  ProtoDUNE and simulation \\ \colhline
    

    \\ \rowcolor{dunesky} \newtag{SP-APA-1}{ spec:apa-unit-size } & Name: APA unit size \\
    Description & Overall dimensions of a single anode plane assembly   \\  \colhline
    Specification &  \SI{6.0}{m} tall $\times$ \SI{2.3}{m} wide \\   \colhline
    Rationale &   Maximum size allowed for fabrication, transportation, and installation.   \\ \colhline
    Validation & ProtoDUNE-SP   \\
   \colhline

   
  \newtag{SP-APA-2}{ spec:apa-active-area }  & Active area  &  Maximize total active area. &  Maximize area for data collection  &  ProtoDUNE-SP  \\ \colhline
    
   
  \newtag{SP-APA-3}{ spec:apa-wire-tension }  & Wire tension  &  \SI{6}{N} $\pm$ \SI{1}{N} &  Prevent contact beween wires and minimize  break risk &  ProtoDUNE-SP \\ \colhline
    
    \\ \rowcolor{dunesky} \newtag{SP-APA-4}{ spec:apa-bias-voltage } & Name: Wire plane bias voltages \\
    Description & APAs should produce optimal and uniform induction and collection signal shapes.   \\  \colhline
    Specification &  The setup, including boards, must hold 150\% of max operating voltage. \\   \colhline
    Rationale &   Headroom in case adjustments needed  \\ \colhline
    Validation & E-field simulation sets wire bias voltages. ProtoDUNE-SP confirms performance.  \\
   \colhline

    
   
  \newtag{SP-APA-5}{ spec:apa-frame-planarity }  & Frame planarity (twist limit)  &  $<$\SI{5}{mm} &  APA transparency.  Ensures wire plane spacing change of $<$0.5 mm.  &  ProtoDUNE-SP \\ \colhline
    
    
   
  \newtag{SP-APA-6}{ spec:apa-bad-channels }  & Missing/unreadable channels  &  $<$1\%, with a goal of $<$0.5\% &  Reconstruction efficiency &  ProtoDUNE-SP \\ \colhline
    


\label{tab:specs:SP-APA}
\end{longtable}
% This file is generated, any edits may be lost.

\begin{longtable}{p{0.14\textwidth}p{0.13\textwidth}p{0.18\textwidth}p{0.22\textwidth}p{0.20\textwidth}}
\caption{Specifications for SP-HV \fixmehl{ref \texttt{tab:spec:SP-HV}}} \\
  \rowcolor{dunesky}
       Label & Description  & Specification \newline (Goal) & Rationale & Validation \\  \colhline

   \newtag{SP-FD-1}{ spec:min-drift-field }  & Minimum drift field  &  $>$\,\SI{250}{ V/cm} \newline ( $>\,\SI{500}{ V/cm}$ ) &  Lessens impacts of $e^-$-Ar recombination, $e^-$ lifetime, $e^-$ diffusion and space charge. &  ProtoDUNE \\ \colhline
    
   
  \newtag{SP-FD-2}{ spec:system-noise }  & System noise  &  $<\,\SI{1000}\,e^-$ &  Provides $>$5:1 S/N on induction planes for  pattern recognition and two-track separation. &  ProtoDUNE and simulation \\ \colhline
    
   
  \newtag{SP-FD-3}{ spec:light-yield }  & Light yield  &  $>\,\SI{20}{PE/MeV}$ (avg), $>\,\SI{0.5}{PE/MeV}$ (min) &  Gives PDS energy resolution comparable that of the TPC for 5-7 MeV SN $\nu$s, and allows tagging of $>\,\SI{99}{\%}$ of nucleon decay backgrounds with light at all points in detector. &  Supernova and nucleon decay events in the FD with full simulation and reconstruction. \\ \colhline
    
    \\ \rowcolor{dunesky} \newtag{SP-FD-4}{ spec:time-resolution-pds } & Name: Time resolution \\
    Description & The time resolution of the photon detection system shall be less than 1 microsecond in order to assign a unique event time.   \\  \colhline
    Specification (Goal) &  $<\,\SI{1}{\micro\second}$  ( $<\,\SI{100}{\nano\second}$ ) \\   \colhline
    Rationale &   Enables \SI{1}{mm} position resolution for \SI{10}{MeV} SNB candidate events for instantaneous rate $<\,\SI{1}{m^{-3}ms^{-1}}$.  \\ \colhline
    Validation &   \\
   \colhline

   \newtag{SP-FD-5}{ spec:lar-purity }  & Liquid argon purity  &  $<$\,\SI{100}{ppt} \newline ($<\,\SI{30}{ppt}$) &  Provides $>$5:1 S/N on induction planes for  pattern recognition and two-track separation. &  Purity monitors and cosmic ray tracks \\ \colhline
    
    \\ \rowcolor{dunesky} \newtag{SP-FD-11}{ spec:hvs-field-uniformity } & Name: Drift field uniformity due to HVS \\
    Description & Design of TPC cathode and FC components shall ensure uniform field.  Production tolerances shall be set so as to maintain flatness of component surfaces and, by extension, the shape of the drift field volume.   \\  \colhline
    Specification &  $<\,\SI{1}{\%}$ throughout volume \\   \colhline
    Rationale &   High reconstruction efficiency.  \\ \colhline
    Validation & ProtoDUNE and simulation  \\
   \colhline

   
  \newtag{SP-FD-12}{ spec:hv-ps-ripple }  & Cathode HV power supply ripple contribution to system noise  &  $<\,\SI{100}e^-$ &  Maximize live time; maintain high S/N. &  Engineering calculation, in situ measurement,   ProtoDUNE \\ \colhline
    
   
  \newtag{SP-FD-16}{ spec:det-dead-time }  & Detector dead time  &  $<\,\SI{0.5}{\%}$ &  Meet physics goals in timely fashion. &  ProtoDUNE \\ \colhline
    
    \\ \rowcolor{dunesky} \newtag{SP-FD-17}{ spec:cathode-resistivity } & Name: Cathode resistivity \\
    Description & The cathode resistivity shall ensure that in the event of an HV discharge, the release of the large stored energy is spread out over time.    \\  \colhline
    Specification (Goal) &  $>\,\SI{1}{\mega\ohm/square}$  ( $>\,\SI{1}{\giga\ohm/square}$ ) \\   \colhline
    Rationale &   Detector damage prevention.  \\ \colhline
    Validation & ProtoDUNE  \\
   \colhline

    \\ \rowcolor{dunesky} \newtag{SP-FD-24}{ spec:local-e-fields } & Name: Local electric fields \\
    Description & The integrated detector design shall minimize potential pathways for HV discharges.   \\  \colhline
    Specification &  $<\,\SI{30}{kV/cm}$ \\   \colhline
    Rationale &   Maximize live time; maintain high S/N.  \\ \colhline
    Validation & ProtoDUNE  \\
   \colhline


   
  \newtag{SP-HV-1}{ spec:power-supply-stability }  & Maximize power supply stability  &  $>\,\SI{90}{\%}$ uptime &  Collect data over long period with high uptime. &  ProtoDUNE \\ \colhline
    
    
   \newtag{SP-HV-2}{ spec:hv-connection-redundancy }  & Provide redundancy in all \dword{hv} connections.  &  Two-fold \newline ( Four-fold ) &  Avoid interrupting data collection. &  Assembly QC \\ \colhline
    


\label{tab:specs:SP-HV}
\end{longtable}
% This file is generated, any edits may be lost.
\begin{footnotesize}
%\begin{longtable}{p{0.14\textwidth}p{0.13\textwidth}p{0.18\textwidth}p{0.22\textwidth}p{0.20\textwidth}}
\begin{longtable}{p{0.12\textwidth}p{0.18\textwidth}p{0.17\textwidth}p{0.25\textwidth}p{0.16\textwidth}}
\caption{Specifications for SP-ELEC \fixmehl{ref \texttt{tab:spec:SP-ELEC}}} \\
  \rowcolor{dunesky}
       Label & Description  & Specification \newline (Goal) & Rationale & Validation \\  \colhline

   
  \newtag{SP-FD-2}{ spec:system-noise }  & System noise  &  $<\,\SI{1000}\,e^-$ &  Provides $>$5:1 S/N on induction planes for  pattern recognition and two-track separation. &  ProtoDUNE and simulation \\ \colhline
    
    
   \newtag{SP-FD-13}{ spec:fe-peak-time }  & Front-end peaking time  &  \SI{1}{\micro\second} \newline ( Adjustable so as to see saturation in less than \SI{10}{\%} of beam-produced events ) &  Vertex resolution; optimized for \SI{5}{mm} wire spacing. &  ProtoDUNE and simulation \\ \colhline
    
   
  \newtag{SP-FD-14}{ spec:sp-signal-saturation }  & Signal saturation level  &  \num{500000} electrons &  Maintain calorimetric performance for multi-proton final state. &  Simulation \\ \colhline
    
    
   
  \newtag{SP-FD-19}{ spec:adc-sampling-freq }  & ADC sampling frequency  &  $\sim\,\SI{2}{\mega\hertz}$ &  Match \SI{1}{\micro\second} shaping time. &  Nyquist requirement and design choice \\ \colhline
    
    
   \newtag{SP-FD-20}{ spec:adc-number-of-bits }  & Number of ADC bits  &  \num{12} bits \newline ( \num{13} bits ) &  ADC noise contribution negligible (low end); match signal saturation specification (high end). &  Engineering calculation and design choice \\ \colhline
    
   
  \newtag{SP-FD-21}{ spec:ce-power-consumption }  & Cold electronics power consumption   &  $<\,\SI{50}{ mW/channel} $ &  No bubbles in LAr to redice HV discharge risk. &  ProtoDUNE \\ \colhline
    
   
  \newtag{SP-FD-25}{ spec:non-fe-noise }  & Non-FE noise contributions  &  $<<\,\SI{1000}{enc} $ &  High S/N for high reconstruction efficiency. &  Engineering calculation and ProtoDUNE \\ \colhline
    
    
   
  \newtag{SP-FD-28}{ spec:dead-channels }  & Dead channels  &  $<\,\SI{1}{\%}$ &  Contingency for possible efficiency loss for $>\,$20 year operation.  &  ProtoDUNE \\ \colhline
    

   \newtag{SP-ELEC-1}{ spec:num-FE-baselines }  & Number of baselines in the front-end amplifier  &  2.0 \newline ( 2.0 ) &  Use a single type of amplifier for both induction and collection wires &  ProtoDUNE \\ \colhline
    
   \newtag{SP-ELEC-2}{ spec:gain-FE-amplifier }  & Gain of the front-end amplifier  &  $\sim\SI{20}{mV/fC}$ \newline (Adjustable in the range \SIrange{5}{25}{mV/fC}) &  The gain of the FE amplifier is obtained from the maximum charge to be observed without saturation and from the operating voltage of the amplifier, that depends on the technology choice. &   \\ \colhline
    
   \newtag{SP-ELEC-3}{ spec:syncronization-CE }  & System synchronization  &  \SI{50}{ns} \newline ( \SI{10}{ns} ) &  The dispersion of the sampling times on different wires of the APA should be much smaller than the sampling time (500 ns) and give a negligible contribution to the hit resolution. &   \\ \colhline
    
   
  \newtag{SP-ELEC-4}{ spec:num-channels-FEMB }  & Number of channels per front-end motherboard  &   &  The total number of wires on one side of an APA, 1,280, must be an integer multiple of the number of channels on the FEMBs. &  Design \\ \colhline
    
   \newtag{SP-ELEC-5}{ spec:FEMB-data-link }  & Number of links between the FEMB and the WIB  &  \num{4} at \SI{1.28}{Gbps} \newline ( \num{2} at \SI{2.56}{Gbps} ) &  Balance between reducing the number of links and reliability and power issues when increasing the data transmission speed. &  ProtoDUNE, Laboratory measurements on bit error rates \\ \colhline
    
   
  \newtag{SP-ELEC-6}{ spec:cold-cables-xsec }  & Cross section of cold cables  &  \SI{2.5}{inches} &  Avoid the need for further changes to the APA frame and for routing the cables along the cryostat walls &  Tests on APA frame prototypes \\ \colhline
    
   
  \newtag{SP-ELEC-7}{ spec:WIB-data-link }  & Data transmission speed between the WIB and the DAQ backend  &  \SI{10}{Gbps} &  Balance between cost and reduction of the number of optical fiber links for each WIB. &  ProtoDUNE, Laboratory measurements on bit error rates \\ \colhline
    
   
  \newtag{SP-ELEC-8}{ spec:cold-cables-xsec }  & Maximum diameter of conduit enclosing the cold cables while they are routed through the APA frame  &  \SI{6.35}{cm} (2.5") &  Avoid the need for further changes to the APA frame and for routing the cables along the cryostat walls &  Tests on APA frame prototypes \\ \colhline
    


\label{tab:specs:SP-ELEC}
\end{longtable}
\end{footnotesize} 
% This file is generated, any edits may be lost.

\begin{longtable}{p{0.14\textwidth}p{0.13\textwidth}p{0.18\textwidth}p{0.22\textwidth}p{0.20\textwidth}}
\caption{Specifications for SP-PDS \fixmehl{ref \texttt{tab:spec:SP-PDS}}} \\
  \rowcolor{dunesky}
       Label & Description  & Specification \newline (Goal) & Rationale & Validation \\  \colhline

   \newtag{SP-FD-1}{ spec:min-drift-field }  & Minimum drift field  &  $>$\,\SI{250}{ V/cm} \newline ( $>\,\SI{500}{ V/cm}$ ) &  Lessens impacts of $e^-$-Ar recombination, $e^-$ lifetime, $e^-$ diffusion and space charge. &  ProtoDUNE \\ \colhline
    
   
  \newtag{SP-FD-2}{ spec:system-noise }  & System noise  &  $<\,\SI{1000}\,e^-$ &  Provides $>$5:1 S/N on induction planes for  pattern recognition and two-track separation. &  ProtoDUNE and simulation \\ \colhline
    
   
  \newtag{SP-FD-3}{ spec:light-yield }  & Light yield  &  $>\,\SI{20}{PE/MeV}$ (avg), $>\,\SI{0.5}{PE/MeV}$ (min) &  Gives PDS energy resolution comparable that of the TPC for 5-7 MeV SN $\nu$s, and allows tagging of $>\,\SI{99}{\%}$ of nucleon decay backgrounds with light at all points in detector. &  Supernova and nucleon decay events in the FD with full simulation and reconstruction. \\ \colhline
    
    \\ \rowcolor{dunesky} \newtag{SP-FD-4}{ spec:time-resolution-pds } & Name: Time resolution \\
    Description & The time resolution of the photon detection system shall be less than 1 microsecond in order to assign a unique event time.   \\  \colhline
    Specification (Goal) &  $<\,\SI{1}{\micro\second}$  ( $<\,\SI{100}{\nano\second}$ ) \\   \colhline
    Rationale &   Enables \SI{1}{mm} position resolution for \SI{10}{MeV} SNB candidate events for instantaneous rate $<\,\SI{1}{m^{-3}ms^{-1}}$.  \\ \colhline
    Validation &   \\
   \colhline

   \newtag{SP-FD-5}{ spec:lar-purity }  & Liquid argon purity  &  $<$\,\SI{100}{ppt} \newline ($<\,\SI{30}{ppt}$) &  Provides $>$5:1 S/N on induction planes for  pattern recognition and two-track separation. &  Purity monitors and cosmic ray tracks \\ \colhline
    
    \\ \rowcolor{dunesky} \newtag{SP-FD-15}{ spec:lar-n-contamination } & Name: LAr nitrogen contamination \\
    Description & The nitrogen contamination in the LAr shall remain below 25 ppm in order not to significantly affect the number of photons that reach the detectors (for both fast and late light components).   \\  \colhline
    Specification &  $<\,\SI{25}{ppm}$ \\   \colhline
    Rationale &   Maintain \SI{0.5}{PE/MeV} PDS sensitivity required for triggering proton decay near cathode.  \\ \colhline
    Validation & In situ measurment  \\
   \colhline


    \\ \rowcolor{dunesky} \newtag{SP-PDS-1}{ spec:ly-uniformity } & Name: Light yield uniformity within the active volume. \\
    Description & The light yield in even the dimmest regions of the detector shall be sufficient to correctly associate scintillation light with events with energy > 200 MeV with efficiency >\SI{99}{\%}. The goal is to dramatically improve the uniformity to improve background rejection, triggering efficiency, and energy resolution.   \\  \colhline
    Specification (Goal) &  $>\,\SI{0.5}{pe/MeV}$  ( >\SI{4}{pe/MeV} ) \\   \colhline
    Rationale &   Fiducializing nucleon decay backgrounds with  $>\,\SI{99}{\%}$ efficiency at all points in the detector.  \\ \colhline
    Validation & Simulated nucleon decay events in the far detector.  \\
   \colhline

    \\ \rowcolor{dunesky} \newtag{SP-PDS-2}{ spec:spatial-localization } & Name: Spatial localization \\
    Description & Events inside the active volume shall be localized in the Y-Z plane  to within < \SI{2.5}{\meter} using light signals.   \\  \colhline
    Specification &  < \SI{2.5}{\meter} \\   \colhline
    Rationale &   Enables more accurate matching of photon detector and TPC signals.  \\ \colhline
    Validation & Supernova neutrino and nucleon decay simulation in the far detector.  \\
   \colhline

    
   
  \newtag{SP-PDS-3}{ spec:env-light-exposure }  & Environmental light exposure  &  \num{0} sunlight.  All other unfiltered sources: $<\,\num{30}$ minutes integrated across all exposures &  Prevent damage to wavelength-shifting coatings due to UV exposure &  ProtoDUNE, IU studies \\ \colhline
    
    \\ \rowcolor{dunesky} \newtag{SP-PDS-4}{ spec:env-humidity-limit } & Name: Environmental humidity limit \\
    Description & All working environments with exposed TPB coatings must maintain <\SI{50}{\%} Relative Humidity (RH) at  \SI{70}{\degree F}.   \\  \colhline
    Specification (Goal) &  < \SI{50}{\%} RH at \SI{70}{\degree F}  ( ALARA ) \\   \colhline
    Rationale &     \\ \colhline
    Validation &   \\
   \colhline

   
  \newtag{SP-PDS-5}{ spec:light-tightness }  & Light-tight cryostat  &  Cryostat light leaks responsible for $<\,\SI{10}{\%}$  of data transferred from PDS to DAQ &  Minimizing false triggers due to cryostat light leaks helps limit the data transfer rate to  \dshort{daq}. &  \dshort{pdsp} and \dshort{iceberg} \\ \colhline
    
   
  \newtag{SP-PDS-6}{ spec:ed-light }  & Light from electrical discharge  &  Light generated by HV system responsible for $<\,\SI{10}{\%}$ total trigger rate in DAQ &  Minimizing false triggers due to corona light from \dword{hv} discharge helps limit the data transfer rate to the \dword{daq}. &  \dword{pdsp} and \dword{iceberg} \\ \colhline
    
    \\ \rowcolor{dunesky} \newtag{SP-PDS-7}{ spec:mech-deflection } & Name: Mechanical deflection (static) \\
    Description & The PDS shall not deviate more than 5mm from nominal position in any APA orientation due to PD load   \\  \colhline
    Specification &  $<\,\SI{5}{\milli\meter}$ \\   \colhline
    Rationale &   Constrain PD motion (static and dynamic load) to avoid damaging APA  \\ \colhline
    Validation & PD FEA, ProtoDUNE, ICEBERG, Ash River integration tests, CERN pre-production integration test  \\
   \colhline

    \\ \rowcolor{dunesky} \newtag{SP-PDS-8}{ spec:apa-install } & Name: Clearance for installation through APA side tubes \\
    Description & PD modules must fit and be secured to the APA through slots in one side of the APA, as designed in concert with APA group.   \\  \colhline
    Specification &  $>$\SI{1}{\milli\meter} \\   \colhline
    Rationale &     \\ \colhline
    Validation &   \\
   \colhline

   
  \newtag{SP-PDS-9}{ spec:mech-compatibility }  & No mechanical interference with APA, SP-CE and SP-HV detector elements (clearance)  &  $>\,\SI{1}{\milli\meter}$ &  PD mounting and securing element tolerances must prevent interference with APA and CE cable bundles. &  Validation will occur in \dword{iceberg}, Ash River integration  tests, and the \dword{cern} pre-production integration tests. \\ \colhline
    
   
  \newtag{SP-PDS-10}{ spec:pds-cable }  & APA intrusion limit for PD cable routing   &  $<\,\SI{6}{\milli\meter}$ &  \dword{pd} modules must install into \dword{apa} frames following wire wrapping.  \dword{pd} modules must not occlude \dword{apa} side tubes. &  \dword{iceberg}, Ash River integration  tests, and the CERN pre-production integration tests \\ \colhline
    
   
  \newtag{SP-PDS-11}{ spec:pds-cablemate }  & PD cabling cannot limit upper-lower APA junction gap  &  \SI{0}{\milli\meter} separation mechanically allowed &  PD cable connections must not limit the minimum upper and lower APA separation. &  Validation will occur in \dword{iceberg}, Ash River integration  tests, and the \dword{cern} pre-production integration tests. \\ \colhline
    
   
  \newtag{SP-PDS-12}{ spec:pds-clearance }  & Maintain PD-APA clearance at LAr temperature.   &  $>\,\SI{0.5}{\milli\meter}$ &  \dword{pd} mounting frame and cable harness must accommodate thermal contraction of itself and \dword{apa} frame. &  Thermal modeling, \dword{protodune}, \dword{iceberg}, \dword{cern} pre-production integration tests \\ \colhline
    
   
  \newtag{SP-PDS-13}{ spec:pds-datarate }  & Data transfer rate from SP-PD to DAQ  &  $<\,\SI{8}{Gbps}$ &  \dword{pd} data transfer must not exceed \dword{daq} data throughput capability. &  Maximum bandwidth out of the \dword{pd} electronics is \SI{80}{Mbps} \\ \colhline
    
    \\ \rowcolor{dunesky} \newtag{SP-PDS-14}{ spec:pds-signaltonoise } & Name: Signal-to-noise in SP-PD \\
    Description & The signal-to-noise ratio (single PE pulse height / baseline noise RMS) shall be greater than 4.   \\  \colhline
    Specification &  $>4$ \\   \colhline
    Rationale &     \\ \colhline
    Validation &   \\
   \colhline

   
  \newtag{SP-PDS-15}{ spec:pds-darkrate }  & Dark noise rate in SP-PD  &  $<\,\SI{1}{kHz}$ &  Keep data rate within electronics bandwidth limits. &  Pre-production photosensor testing, \dword{pdsp}, \dword{iceberg} and \dword{pdsp2} \\ \colhline
    
   
  \newtag{SP-PDS-16}{ spec:pds-dynamicrange }  & Dynamic Range in SP-PD  &  $<\,\SI{20}{\%}$ &  Keep the rate of saturating channels low enough for effective mitigation. &  Pre-production photosensor testing, \dword{pdsp}, \dword{iceberg} and \dword{pdsp2} \\ \colhline
    


\label{tab:specs:SP-PDS}
\end{longtable}
% This file is generated, any edits may be lost.

\begin{longtable}{p{0.14\textwidth}p{0.13\textwidth}p{0.18\textwidth}p{0.22\textwidth}p{0.20\textwidth}}
\caption{Specifications for SP-CALIB \fixmehl{ref \texttt{tab:spec:SP-CALIB}}} \\
  \rowcolor{dunesky}
       Label & Description  & Specification \newline (Goal) & Rationale & Validation \\  \colhline

   \newtag{SP-FD-1}{ spec:min-drift-field }  & Minimum drift field  &  $>$\,\SI{250}{ V/cm} \newline ( $>\,\SI{500}{ V/cm}$ ) &  Lessens impacts of $e^-$-Ar recombination, $e^-$ lifetime, $e^-$ diffusion and space charge. &  ProtoDUNE \\ \colhline
    
   
  \newtag{SP-FD-2}{ spec:system-noise }  & System noise  &  $<\,\SI{1000}\,e^-$ &  Provides $>$5:1 S/N on induction planes for  pattern recognition and two-track separation. &  ProtoDUNE and simulation \\ \colhline
    
   
  \newtag{SP-FD-3}{ spec:light-yield }  & Light yield  &  $>\,\SI{20}{PE/MeV}$ (avg), $>\,\SI{0.5}{PE/MeV}$ (min) &  Gives PDS energy resolution comparable that of the TPC for 5-7 MeV SN $\nu$s, and allows tagging of $>\,\SI{99}{\%}$ of nucleon decay backgrounds with light at all points in detector. &  Supernova and nucleon decay events in the FD with full simulation and reconstruction. \\ \colhline
    
    \\ \rowcolor{dunesky} \newtag{SP-FD-4}{ spec:time-resolution-pds } & Name: Time resolution \\
    Description & The time resolution of the photon detection system shall be less than 1 microsecond in order to assign a unique event time.   \\  \colhline
    Specification (Goal) &  $<\,\SI{1}{\micro\second}$  ( $<\,\SI{100}{\nano\second}$ ) \\   \colhline
    Rationale &   Enables \SI{1}{mm} position resolution for \SI{10}{MeV} SNB candidate events for instantaneous rate $<\,\SI{1}{m^{-3}ms^{-1}}$.  \\ \colhline
    Validation &   \\
   \colhline

   \newtag{SP-FD-5}{ spec:lar-purity }  & Liquid argon purity  &  $<$\,\SI{100}{ppt} \newline ($<\,\SI{30}{ppt}$) &  Provides $>$5:1 S/N on induction planes for  pattern recognition and two-track separation. &  Purity monitors and cosmic ray tracks \\ \colhline
    
    \\ \rowcolor{dunesky} \newtag{SP-FD-7}{ spec:misalignment-field-uniformity } & Name: Drift field uniformity due to component alignment \\
    Description & Misalignments of the various TPC components shall not introduce drift-field nonuniformities beyond those specified in the HVS requirements.   \\  \colhline
    Specification &  $<\,1\,$\% throughout volume \\   \colhline
    Rationale &   Maintains APA, CPA,  FC orientation and shape.  \\ \colhline
    Validation & ProtoDUNE  \\
   \colhline

    
   
  \newtag{SP-FD-9}{ spec:apa-wire-spacing }  & APA wire spacing  &  \SI{4.669}{mm} for U,V; \SI{4.790}{mm} for X,G &  Enables 100\% efficient MIP detection, \SI{1.5}{cm} $yz$ vertex resolution. &  Simulation \\ \colhline
    
    \\ \rowcolor{dunesky} \newtag{SP-FD-11}{ spec:hvs-field-uniformity } & Name: Drift field uniformity due to HVS \\
    Description & Design of TPC cathode and FC components shall ensure uniform field.  Production tolerances shall be set so as to maintain flatness of component surfaces and, by extension, the shape of the drift field volume.   \\  \colhline
    Specification &  $<\,\SI{1}{\%}$ throughout volume \\   \colhline
    Rationale &   High reconstruction efficiency.  \\ \colhline
    Validation & ProtoDUNE and simulation  \\
   \colhline

    
   \newtag{SP-FD-13}{ spec:fe-peak-time }  & Front-end peaking time  &  \SI{1}{\micro\second} \newline ( Adjustable so as to see saturation in less than \SI{10}{\%} of beam-produced events ) &  Vertex resolution; optimized for \SI{5}{mm} wire spacing. &  ProtoDUNE and simulation \\ \colhline
    
   
  \newtag{SP-FD-16}{ spec:det-dead-time }  & Detector dead time  &  $<\,\SI{0.5}{\%}$ &  Meet physics goals in timely fashion. &  ProtoDUNE \\ \colhline
    
   
  \newtag{SP-FD-22}{ spec:data-rate-to-tape }  & Data rate to tape  &  $<\,\SI{30}{PB/year}$ &  Cost.  Bandwidth. &  ProtoDUNE \\ \colhline
    
   
  \newtag{SP-FD-23}{ spec:sn-trigger }  & Supernova trigger  &  $>\,\SI{90}{\%}$ efficiency for SNB within \SI{100}{kpc} &  $>\,$90\% efficiency for SNB within 100 kpc &  Simulation and bench tests \\ \colhline
    
    \\ \rowcolor{dunesky} \newtag{SP-FD-24}{ spec:local-e-fields } & Name: Local electric fields \\
    Description & The integrated detector design shall minimize potential pathways for HV discharges.   \\  \colhline
    Specification &  $<\,\SI{30}{kV/cm}$ \\   \colhline
    Rationale &   Maximize live time; maintain high S/N.  \\ \colhline
    Validation & ProtoDUNE  \\
   \colhline

   
  \newtag{SP-FD-25}{ spec:non-fe-noise }  & Non-FE noise contributions  &  $<<\,\SI{1000}{enc} $ &  High S/N for high reconstruction efficiency. &  Engineering calculation and ProtoDUNE \\ \colhline
    
    \\ \rowcolor{dunesky} \newtag{SP-FD-26}{ spec:lar-impurity-contrib } & Name: LAr impurity contributions from components \\
    Description & Contributions to LAr contamination from detector components, through outgassing or other processes, shall remain << 30 ppt so as to avoid significantly increasing the nominal level of contamination.   \\  \colhline
    Specification &  $<<\,\SI{30}{ppt} $ \\   \colhline
    Rationale &   Maintain HV operating range for high live time fraction.  \\ \colhline
    Validation & ProtoDUNE  \\
   \colhline

    \\ \rowcolor{dunesky} \newtag{SP-FD-27}{ spec:radiopurity } & Name: Introduced radioactivity \\
    Description & Introduced radioactivity shall be less than that from 39Ar.   \\  \colhline
    Specification &  less than that from $^{39}$Ar \\   \colhline
    Rationale &   Maintain low radiological backgrounds for SNB searches.  \\ \colhline
    Validation & ProtoDUNE and assays during construction  \\
   \colhline


   \newtag{SP-CALIB-1}{ spec:efield-calib-precision }  & Ionization laser electric field measurement precision  &  \SI{1}{\%} \newline ( $<$\SI{1}{\%} ) &  Electric field affects energy and position measurements. &  ProtoDUNE and external experiments. \\ \colhline
    
   \newtag{SP-CALIB-2}{ spec:efield-calib-coverage }  & Ionization laser \efield measurement coverage  &  $>\,\SI{75}{\%}$ \newline ( \SI{100}{\%} ) &  Allowable size of the uncovered detector regions is set by the highest reasonably expected field distortions, 4%. &  ProtoDUNE \\ \colhline
    
   \newtag{SP-CALIB-3}{ spec:efield-calib-granularity }  & Ionization laser \efield measurement  granularity  &  $<\,\SI{30\times30\times30}{\centi\meter}$ \newline ( $<\,\SI{10\times10\times10}{\centi\meter}$ ) &  Minimum measurable region is set by the maximum expected distortion and position reconstruction requirements. &  ProtoDUNE \\ \colhline
    
   \newtag{SP-CALIB-4}{ spec:laser-position-precision }  & Laser beam position precision  &  $~\SI{0.5}{\milli\radian}$ \newline ($<\SI{0.5}{\milli\radian}$) &  The necessary spatial precision does not need to be smaller than the APA wire gap. &  ProtoDUNE \\ \colhline
    
   \newtag{SP-CALIB-5}{ spec:neutron-source-coverage }  & Neutron source coverage  &  $>$\SI{75}{\%} \newline ( \SI{100}{\%} ) &  The coverage of the pulsed neutron system depends on the energy resolution requirements at low energy. &  Simulations \\ \colhline
    
   \newtag{SP-CALIB-6}{ spec:data-volume-laser }  & Ionization laser data volume per year (per 10 kt)  &  $>\SI{184}{TB/yr/10 kt}$ \newline ($>\SI{368}{TB/yr/10 kt}$) &  The laser data volume must allow the needed coverage and granularity. &  ProtoDUNE and simulations \\ \colhline
    
   \newtag{SP-CALIB-7}{ spec:data-volume-pns }  & Neutron source DAQ rate per year (per 10 kton)  &  $>$\SI{84}{TB/yr/10 kton} \newline ( $>$\SI{168}{TB/yr/10 kton} ) &  The coverage of the pulsed neutron system depends on the energy resolution requirements at low energy. &  Simulations \\ \colhline
    
   
  \newtag{SP-CALIB-8}{ spec:rate-gammas-source }  & Rate of 9 MeV capture gamma events in the proposed radioactive source  &  $<\,\SI{1}{\kilo\hertz}$ &  The source rate must be such that there is no more than one event per drift time. &  Lab tests \\ \colhline
    


\label{tab:specs:SP-CALIB}
\end{longtable}
% This file is generated, any edits may be lost.

\begin{longtable}{p{0.14\textwidth}p{0.13\textwidth}p{0.18\textwidth}p{0.22\textwidth}p{0.20\textwidth}}
\caption{Requirements and Specifications Relevant to the for SP-DAQ
  System} \\
  \rowcolor{dunesky}
       Label & Description  & Specification \newline (Goal) & Rationale & Validation \\  \colhline

   \newtag{SP-FD-1}{ spec:min-drift-field }  & Minimum drift field  &  $>$\,\SI{250}{ V/cm} \newline ( $>\,\SI{500}{ V/cm}$ ) &  Lessens impacts of $e^-$-Ar recombination, $e^-$ lifetime, $e^-$ diffusion and space charge. &  ProtoDUNE \\ \colhline
    
   
  \newtag{SP-FD-2}{ spec:system-noise }  & System noise  &  $<\,\SI{1000}\,e^-$ &  Provides $>$5:1 S/N on induction planes for  pattern recognition and two-track separation. &  ProtoDUNE and simulation \\ \colhline
    
   
  \newtag{SP-FD-3}{ spec:light-yield }  & Light yield  &  $>\,\SI{20}{PE/MeV}$ (avg), $>\,\SI{0.5}{PE/MeV}$ (min) &  Gives PDS energy resolution comparable that of the TPC for 5-7 MeV SN $\nu$s, and allows tagging of $>\,\SI{99}{\%}$ of nucleon decay backgrounds with light at all points in detector. &  Supernova and nucleon decay events in the FD with full simulation and reconstruction. \\ \colhline
    
    \\ \rowcolor{dunesky} \newtag{SP-FD-4}{ spec:time-resolution-pds } & Name: Time resolution \\
    Description & The time resolution of the photon detection system shall be less than 1 microsecond in order to assign a unique event time.   \\  \colhline
    Specification (Goal) &  $<\,\SI{1}{\micro\second}$  ( $<\,\SI{100}{\nano\second}$ ) \\   \colhline
    Rationale &   Enables \SI{1}{mm} position resolution for \SI{10}{MeV} SNB candidate events for instantaneous rate $<\,\SI{1}{m^{-3}ms^{-1}}$.  \\ \colhline
    Validation &   \\
   \colhline

   \newtag{SP-FD-5}{ spec:lar-purity }  & Liquid argon purity  &  $<$\,\SI{100}{ppt} \newline ($<\,\SI{30}{ppt}$) &  Provides $>$5:1 S/N on induction planes for  pattern recognition and two-track separation. &  Purity monitors and cosmic ray tracks \\ \colhline
    
   
  \newtag{SP-FD-12}{ spec:hv-ps-ripple }  & Cathode HV power supply ripple contribution to system noise  &  $<\,\SI{100}e^-$ &  Maximize live time; maintain high S/N. &  Engineering calculation, in situ measurement,   ProtoDUNE \\ \colhline
    
    
   \newtag{SP-FD-13}{ spec:fe-peak-time }  & Front-end peaking time  &  \SI{1}{\micro\second} \newline ( Adjustable so as to see saturation in less than \SI{10}{\%} of beam-produced events ) &  Vertex resolution; optimized for \SI{5}{mm} wire spacing. &  ProtoDUNE and simulation \\ \colhline
    
   
  \newtag{SP-FD-16}{ spec:det-dead-time }  & Detector dead time  &  $<\,\SI{0.5}{\%}$ &  Meet physics goals in timely fashion. &  ProtoDUNE \\ \colhline
    
    
   
  \newtag{SP-FD-19}{ spec:adc-sampling-freq }  & ADC sampling frequency  &  $\sim\,\SI{2}{\mega\hertz}$ &  Match \SI{1}{\micro\second} shaping time. &  Nyquist requirement and design choice \\ \colhline
    
    
   \newtag{SP-FD-20}{ spec:adc-number-of-bits }  & Number of ADC bits  &  \num{12} bits \newline ( \num{13} bits ) &  ADC noise contribution negligible (low end); match signal saturation specification (high end). &  Engineering calculation and design choice \\ \colhline
    
   
  \newtag{SP-FD-22}{ spec:data-rate-to-tape }  & Data rate to tape  &  $<\,\SI{30}{PB/year}$ &  Cost.  Bandwidth. &  ProtoDUNE \\ \colhline
    
   
  \newtag{SP-FD-23}{ spec:sn-trigger }  & Supernova trigger  &  $>\,\SI{90}{\%}$ efficiency for SNB within \SI{100}{kpc} &  $>\,$90\% efficiency for SNB within 100 kpc &  Simulation and bench tests \\ \colhline
    
   
  \newtag{SP-FD-25}{ spec:non-fe-noise }  & Non-FE noise contributions  &  $<<\,\SI{1000}{enc} $ &  High S/N for high reconstruction efficiency. &  Engineering calculation and ProtoDUNE \\ \colhline
    
    \\ \rowcolor{dunesky} \newtag{SP-FD-27}{ spec:radiopurity } & Name: Introduced radioactivity \\
    Description & Introduced radioactivity shall be less than that from 39Ar.   \\  \colhline
    Specification &  less than that from $^{39}$Ar \\   \colhline
    Rationale &   Maintain low radiological backgrounds for SNB searches.  \\ \colhline
    Validation & ProtoDUNE and assays during construction  \\
   \colhline

    
   
  \newtag{SP-FD-28}{ spec:dead-channels }  & Dead channels  &  $<\,\SI{1}{\%}$ &  Contingency for possible efficiency loss for $>\,$20 year operation.  &  ProtoDUNE \\ \colhline
    

    \\ \rowcolor{dunesky} \newtag{SP-DAQ-1}{ spec:trigger-high-energy } & Name: Off-beam High-energy Trigger \\
    Description & The detector shall trigger on the visible energy of underground physics events from decays or interactions within the active volume with high efficiency.   \\  \colhline
    Specification &  $>$\SI{100}{\MeV} \\   \colhline
    Rationale &   Driven by DUNE physics mission.  \\ \colhline
    Validation & Simulations  \\
   \colhline

   
  \newtag{SP-DAQ-2}{ spec:DAQ-throughput }  & DAQ storage throughput: The DAQ shall be able to store selected data at an average throughput of 10 Gb/s, with temporary peak throughput of 100 Gb/s.  &  10 Gb/s average storage throughput. 100 Gb/s peak temporary storage throguput per single phase detector module;  &  Average throughput estimated from physics and calibration requirements; peak throughput allowing for fast storage of SNB data ($\sim 10^4$ seconds to store 120 TB of data)  &  ProtoDUNE demonstrated steady storage at $\sim$ 40 Gb/s for a storage volume of 700 TB. Laboratory tests will allow to demonstrate the performance reach. \\ \colhline
    
    
   
  \newtag{SP-DAQ-3}{ spec:trigger-beam }  & Beam Trigger  &  $>$\SI{100}{\MeV} &  Driven by DUNE physics mission. &  Simulations, experience from past and ongoing experiments. \\ \colhline
    
   
  \newtag{SP-DAQ-4}{ spec:trigger-calibration }  & Calibration trigger: The DAQ shall provide the means to distribute time-synchronous commands to the calibration systems, in order to fire them, at a configurable rate and sequence and at configurable intervals in time. Those commands may be distributed during physics data taking or during special calibration data taking sessions. The DAQ shall trigger and acquire data at a fixed, configurable interval after the distribution of the commands, in order to capture the response of the detector to calibration stimuli.  &   &  Calibration is essential to attain required detector performance comprehension. &  Techniques for doing this have been run successfully in MicroBooNE and ProtoDUNE.  \\ \colhline
    
   
  \newtag{SP-DAQ-5}{ spec:trigger-calibration }  & Calibration trigger  &   &  Need to understand detector performance. &  Experience from past and ongoing experiments \\ \colhline
    
   
  \newtag{SP-DAQ-6}{ spec:data-verification }  & Data verification: The DAQ shall check integrity of data at every data transfer step. It shall only delete data from the local storage after confirmation that data have been correctly recorded to permanent storage.  &   &  Data integrity checking is fundamental to ensure data quality &   \\ \colhline
    
    
   
  \newtag{SP-DAQ-7}{ spec:deadtime }  & DAQ Deadtime  &   &  Driven by DUNE physics mission. &   \\ \colhline
    


\label{tab:rec-specs:SP-DAQ}
\end{longtable} 
% This file is generated, any edits may be lost.

\begin{longtable}{p{0.14\textwidth}p{0.13\textwidth}p{0.18\textwidth}p{0.22\textwidth}p{0.20\textwidth}}
\caption{Specifications for SP-CISC \fixmehl{ref \texttt{tab:spec:SP-CISC}}} \\
  \rowcolor{dunesky}
       Label & Description  & Specification \newline (Goal) & Rationale & Validation \\  \colhline

   \newtag{SP-FD-1}{ spec:min-drift-field }  & Minimum drift field  &  $>$\,\SI{250}{ V/cm} \newline ( $>\,\SI{500}{ V/cm}$ ) &  Lessens impacts of $e^-$-Ar recombination, $e^-$ lifetime, $e^-$ diffusion and space charge. &  ProtoDUNE \\ \colhline
    
   
  \newtag{SP-FD-2}{ spec:system-noise }  & System noise  &  $<\,\SI{1000}\,e^-$ &  Provides $>$5:1 S/N on induction planes for  pattern recognition and two-track separation. &  ProtoDUNE and simulation \\ \colhline
    
   
  \newtag{SP-FD-3}{ spec:light-yield }  & Light yield  &  $>\,\SI{20}{PE/MeV}$ (avg), $>\,\SI{0.5}{PE/MeV}$ (min) &  Gives PDS energy resolution comparable that of the TPC for 5-7 MeV SN $\nu$s, and allows tagging of $>\,\SI{99}{\%}$ of nucleon decay backgrounds with light at all points in detector. &  Supernova and nucleon decay events in the FD with full simulation and reconstruction. \\ \colhline
    
    \\ \rowcolor{dunesky} \newtag{SP-FD-4}{ spec:time-resolution-pds } & Name: Time resolution \\
    Description & The time resolution of the photon detection system shall be less than 1 microsecond in order to assign a unique event time.   \\  \colhline
    Specification (Goal) &  $<\,\SI{1}{\micro\second}$  ( $<\,\SI{100}{\nano\second}$ ) \\   \colhline
    Rationale &   Enables \SI{1}{mm} position resolution for \SI{10}{MeV} SNB candidate events for instantaneous rate $<\,\SI{1}{m^{-3}ms^{-1}}$.  \\ \colhline
    Validation &   \\
   \colhline

   \newtag{SP-FD-5}{ spec:lar-purity }  & Liquid argon purity  &  $<$\,\SI{100}{ppt} \newline ($<\,\SI{30}{ppt}$) &  Provides $>$5:1 S/N on induction planes for  pattern recognition and two-track separation. &  Purity monitors and cosmic ray tracks \\ \colhline
    
    \\ \rowcolor{dunesky} \newtag{SP-FD-15}{ spec:lar-n-contamination } & Name: LAr nitrogen contamination \\
    Description & The nitrogen contamination in the LAr shall remain below 25 ppm in order not to significantly affect the number of photons that reach the detectors (for both fast and late light components).   \\  \colhline
    Specification &  $<\,\SI{25}{ppm}$ \\   \colhline
    Rationale &   Maintain \SI{0.5}{PE/MeV} PDS sensitivity required for triggering proton decay near cathode.  \\ \colhline
    Validation & In situ measurment  \\
   \colhline

    
   
  \newtag{SP-FD-18}{ spec:cryo-monitor-devices }  & Cryogenic monitoring devices  &   &  Constrain uncertainties on detection efficiency, fiducial volume. &  ProtoDUNE \\ \colhline
    
   
  \newtag{SP-FD-25}{ spec:non-fe-noise }  & Non-FE noise contributions  &  $<<\,\SI{1000}{enc} $ &  High S/N for high reconstruction efficiency. &  Engineering calculation and ProtoDUNE \\ \colhline
    

   
  \newtag{SP-CISC-1}{ spec:inst-noise }  & Noise from Instrumentation devices  &  $\ll\,\SI{1000}\,e^- $ &  Max noise for 5:1 S/N for a MIP passing near cathode; per SBND and DUNE CE &  ProtoDUNE \\ \colhline
    
    \\ \rowcolor{dunesky} \newtag{SP-CISC-2}{ spec:inst-efield } & Name: Max. E field near instrumentation devices \\
    Description & The maximum field near instrumentation devices should be $<\,\SI{30}{kV/cm}$ to avoid dielectric breakdowns.   \\  \colhline
    Specification (Goal) &  $<\,\SI{30}{kV/cm}$  ( $<\,\SI{15}{kV/cm}$ ) \\   \colhline
    Rationale &   Significantly lower than max field of 30 kV/cm per DUNE HV   \\ \colhline
    Validation & 3D electrostatic simulation  \\
   \colhline

   \newtag{SP-CISC-3}{ spec:elec-lifetime-prec }  & Precision in electron lifetime  &  $<\,$1.4\% \newline ($<\,$1\%) &  Required for accurate charge reconstruction per DUNE-FD Task Force report. &  ProtoDUNE-SP and ITF \\ \colhline
    
   
  \newtag{SP-CISC-4}{ spec:elec-lifetime-range }  & Range in electron lifetime  &  \SIrange{0.04}{10}{ms} in cryostat, \SIrange{0.04}{30}{ms} inline &  Slightly beyond best values observed so far in other detectors.  &  ProtoDUNE-SP and CITF \\ \colhline
    
   \newtag{SP-CISC-11}{ spec:temp-repro }  & Precision: temperature reproducibility  &  $<\,\SI{5}{mK}$ \newline (\SI{2}{mK}) &  Enables validation of CFD models, which predicts gradients below 15 mK &  ProtoDUNE-SP and ITF \\ \colhline
    
   \newtag{SP-CISC-14}{ spec:temp-stability }  & Temperature stability  &  $<\,\SI{2}{mK}$ at all places and times \newline ( Match precision requirement at all places, at all times ) &  Measure the temp map with sufficient precision during the entire duration &  ProtoDUNE-SP \\ \colhline
    
    \\ \rowcolor{dunesky} \newtag{SP-CISC-27}{ spec:camera-cold-coverage } & Name: Coverage \\
    Description & The cold cameras are required to cover at least 80\% of the exterior of HV surfaces.   \\  \colhline
    Specification (Goal) &  $>\,$80\% of HV surfaces  ( \num{100}\% ) \\   \colhline
    Rationale &     \\ \colhline
    Validation &   \\
   \colhline

   \newtag{SP-CISC-51}{ spec:slowcontrol-alarm-rate }  & Slow control alarm rate  &  $<\,$150/day \newline ( $<\,$50/day ) &  Alarm rate low enough to allow response to every alarm. &  Detector module; depends on experimental conditions \\ \colhline
    
   \newtag{SP-CISC-52}{ spec:slowcontrol-num-vars }  & Total No. of variables  &  $>\,\num{150000}$ \newline (\SIrange{150000}{200000}{}) &  Scaled from ProtoDUNE-SP &  ProtoDUNE-SP and CITF \\ \colhline
    
    
   \newtag{SP-CISC-54}{ spec:slowcontrol-archive-rate }  & Archiving rate  &  \SI{0.02}{Hz} \newline ( Broad range \SI{1}{Hz} to \num{1} per few min. ) &  Archiving rate different for each variable, optimized to store important information  &  ProtoDUNE-SP \\ \colhline
    


\label{tab:specs:SP-CISC}
\end{longtable} 
% This file is generated, any edits may be lost.
\begin{footnotesize}
%\begin{longtable}{p{0.14\textwidth}p{0.13\textwidth}p{0.18\textwidth}p{0.22\textwidth}p{0.20\textwidth}}
\begin{longtable}{p{0.12\textwidth}p{0.18\textwidth}p{0.17\textwidth}p{0.25\textwidth}p{0.16\textwidth}}
\caption{Specifications for SP-TC \fixmehl{ref \texttt{tab:spec:SP-TC}}} \\
  \rowcolor{dunesky}
       Label & Description  & Specification \newline (Goal) & Rationale & Validation \\  \colhline

   \newtag{SP-FD-1}{ spec:min-drift-field }  & Minimum drift field  &  $>$\,\SI{250}{ V/cm} \newline ( $>\,\SI{500}{ V/cm}$ ) &  Lessens impacts of $e^-$-Ar recombination, $e^-$ lifetime, $e^-$ diffusion and space charge. &  ProtoDUNE \\ \colhline
    
   
  \newtag{SP-FD-2}{ spec:system-noise }  & System noise  &  $<\,\SI{1000}\,e^-$ &  Provides $>$5:1 S/N on induction planes for  pattern recognition and two-track separation. &  ProtoDUNE and simulation \\ \colhline
    
   
  \newtag{SP-FD-3}{ spec:light-yield }  & Light yield  &  $>\,\SI{20}{PE/MeV}$ (avg), $>\,\SI{0.5}{PE/MeV}$ (min) &  Gives PDS energy resolution comparable that of the TPC for 5-7 MeV SN $\nu$s, and allows tagging of $>\,\SI{99}{\%}$ of nucleon decay backgrounds with light at all points in detector. &  Supernova and nucleon decay events in the FD with full simulation and reconstruction. \\ \colhline
    
    \\ \rowcolor{dunesky} \newtag{SP-FD-4}{ spec:time-resolution-pds } & Name: Time resolution \\
    Description & The time resolution of the photon detection system shall be less than 1 microsecond in order to assign a unique event time.   \\  \colhline
    Specification (Goal) &  $<\,\SI{1}{\micro\second}$  ( $<\,\SI{100}{\nano\second}$ ) \\   \colhline
    Rationale &   Enables \SI{1}{mm} position resolution for \SI{10}{MeV} SNB candidate events for instantaneous rate $<\,\SI{1}{m^{-3}ms^{-1}}$.  \\ \colhline
    Validation &   \\
   \colhline

   \newtag{SP-FD-5}{ spec:lar-purity }  & Liquid argon purity  &  $<$\,\SI{100}{ppt} \newline ($<\,\SI{30}{ppt}$) &  Provides $>$5:1 S/N on induction planes for  pattern recognition and two-track separation. &  Purity monitors and cosmic ray tracks \\ \colhline
    

   
  \newtag{SP-TC-1}{ spec:logistics-material-handling }  & Compliance with the SURF Material Handling Specification for all material transported underground  &  SURF Material Handling Specification &  Loads must fit in the shaft be lifted safely. &  Visual and documentation check \\ \colhline
    
   
  \newtag{SP-TC-2}{ spec:logistics-shipping-coord }  & Coordination of shipments with CMGC; DUNE to schedule use of Ross Shaft  &  2 wk notice to CMGC &  Both DUNE and CMGC need to use Ross Shaft &  Deliveries will be rejected \\ \colhline
    
   
  \newtag{SP-TC-3}{ spec:logistics-materials-buffer }  & Maintain materials buffer at logistics facility in SD   &  $>1$ month &  Prevent schedule delays in case of shipping or customs delays &  Documentatation and progress reporting \\ \colhline
    
   
  \newtag{SP-TC-4}{ spec:apa-storage-sd }  & APA stroage at logistics facility in SD  &  700 m$^2$ &  Store APAs during lag between production and installation &  Agree upon space needs \\ \colhline
    
   
  \newtag{SP-TC-5}{ spec:cleanroom-specification }  & Installation cleanroom Specificaiton  &  ISO 8 &  Reduce dust (contains U/Th) to prevent induced radiological background in detector &  Monitor air purity \\ \colhline
    
   
  \newtag{SP-TC-6}{ spec:cleanroom-uv-filters }  & UV filter in ITF and installation cleanrooms for PDS sensor protection  &  na &  Prevent damage to PD coatings  &  Visual or spectrographic inspection \\ \colhline
    


\label{tab:specs:SP-TC}
\end{longtable}
\end{footnotesize} % aka iic or install

%%%%%%%%%%%%%%%%%%%%%%%%%%%%%%%
\subsection{Dual-phase}
\label{sec:tc-req-dp}

\fixme{not all the DP req tables have been provided yet}

% This file is generated, any edits may be lost.
\begin{footnotesize}
%\begin{longtable}{p{0.14\textwidth}p{0.13\textwidth}p{0.18\textwidth}p{0.22\textwidth}p{0.20\textwidth}}
\begin{longtable}{p{0.12\textwidth}p{0.18\textwidth}p{0.17\textwidth}p{0.25\textwidth}p{0.16\textwidth}}
\caption{Specifications for DP-FD \fixmehl{ref \texttt{tab:spec:DP-FD}}} \\
  \rowcolor{dunesky}
       Label & Description  & Specification \newline (Goal) & Rationale & Validation \\  \colhline


   \newtag{DP-FD-1}{ spec:dp-min-drift-field }  & Minimum drift field  &  $>$\,\SI{250}{V/cm} \newline ( $>\,\SI{500}{V/cm}$ ) &  Lessens impacts of $e^-$-Ar recombination, $e^-$ lifetime, $e^-$ diffusion and space charge. &  ProtoDUNE \\ \colhline
    
   
  \newtag{DP-FD-2}{ spec:dp-system-noise }  & System noise  &  $<\,\SI{1000}\,e^-$ &  Studies suggest that a minimum of 5:1 S/N on individual strip measurements allows for sufficient reconstruction performance. &  ProtoDUNE and simulation \\ \colhline
    
   
  \newtag{DP-FD-3}{ spec:dp-light-yield }  & Light yield  &  $>\,\SI{1}{PE/MeV}$ (at anode), $>\,\SI{5}{PE/MeV}$ (avg  over active volume) &  Enable drift position determination of \dshort{ndk} candidates. Enable \dshort{pds}-based triggering on galactic SNBs. &  Full sim/reco of \dshort{ndk}, \dshort{snb} $\nu$ and radiological events. \\ \colhline
    
   \newtag{DP-FD-4}{ spec:time-resolution-pds }  & Time resolution  &  $<\,\SI{1}{\micro\second}$ \newline ( $<\,\SI{100}{\nano\second}$ ) &  Enables \SI{1}{mm} position resolution for \SI{10}{MeV} SNB candidate events for instantaneous rate $<\,\SI{1}{m^{-3}ms^{-1}}$. &   \\ \colhline
    
   \newtag{DP-FD-5}{ spec:lar-purity }  & Liquid argon purity  &  $<$\,\SI{100}{ppt} \newline ($<\,\SI{30}{ppt}$) &  Provides $>$5:1 S/N on induction planes for  pattern recognition and two-track separation. &  Purity monitors and cosmic ray tracks \\ \colhline
    
   \newtag{DP-FD-6}{ spec:crp-gaps }  & Gaps between CRPs   &  $<\,\SI{30}{mm}$ between adjacent CRPs \newline (\SI{6}{mm} (achieved among groups of adjacent CRPs)) &   &   \\ \colhline
    
   
  \newtag{DP-FD-7}{ spec:dp-misalignment-field-uniformity }  & Drift field uniformity due to component alignment  &  $<\,1\,$\% throughout volume &  Maintains TPC and  FC orientation and shape. &  ProtoDUNE \\ \colhline
    
   \newtag{DP-FD-8}{ spec:crp-eff-gain }  & CRP effective gain  &  \num{6} \newline (E.g., $\sim\,\num{20}$) &   &   \\ \colhline
    
   
  \newtag{DP-FD-9}{ spec:dp-crp-strip-spacing }  & CRP strips spacing  &  $<\,\SI{4.7}{mm}$ &  Enables 100\% efficient MIP detection, \SI{1.5}{cm} $yz$ vertex resolution. &  Simulation \\ \colhline
    
   
  \newtag{DP-FD-10}{ spec:crp-planarity }  & CRP planarity  &  $\pm\,\SI{0.75}{mm}$ &   &   \\ \colhline
    
   
  \newtag{DP-FD-11}{ spec:dp-hvs-field-uniformity }  & Drift field uniformity due to HVS  &  $<\,\SI{1}{\%}$ throughout volume &  High reconstruction efficiency. &  ProtoDUNE and simulation \\ \colhline
    
   
  \newtag{DP-FD-12}{ spec:dp-hv-ps-ripple }  & Cathode HV power supply ripple contribution to system noise  &  $<\,\SI{100}e^-$ &  Maximize live time; maintain high S/N. &  Engineering calculation, in situ measurement,   ProtoDUNE \\ \colhline
    
   \newtag{DP-FD-13}{ spec:dp-fe-peak-time }  & Front-end peaking time  &  \SI{1}{\micro\second} \newline ( Adjustable so as to see saturation in less than \SI{10}{\%} of beam-produced events ) &  Vertex resolution &   \\ \colhline
    
   
  \newtag{DP-FD-14}{ spec:dp-signal-saturation }  & Signal saturation level  &  \num{7500000} electrons &  Maintain calorimetric performance for multi-proton final state; takes into account an effective CRP gain of 20 in the DP signal dynamics. &  Simulation \\ \colhline
    
   
  \newtag{DP-FD-15}{ spec:dp-lar-n-contamination }  & LAr nitrogen contamination  &  $<\,\SI{25}{ppm}$ &  Maintain \SI{0.5}{PE/MeV} PDS sensitivity required for triggering proton decay near cathode. &   \\ \colhline
    
   
  \newtag{DP-FD-16}{ spec:dp-det-dead-time }  & Detector dead time  &  $<\,\SI{0.5}{\%}$ &  Meet physics goals in timely fashion. &  ProtoDUNE \\ \colhline
    
   \newtag{DP-FD-17}{ spec:dp-cathode-resistivity }  & Cathode resistivity  &  $>\,\SI{1}{\mega\ohm/square}$ \newline ($>\,\SI{1}{\giga\ohm/square}$) &  Detector damage prevention. &  ProtoDUNE \\ \colhline
    
   
  \newtag{DP-FD-18}{ spec:dp-cryo-monitor-devices }  & Cryogenic monitoring devices  &   &  Constrain uncertainties on detection efficiency, fiducial volume. &  ProtoDUNE \\ \colhline
    
   
  \newtag{DP-FD-19}{ spec:dp-adc-sampling-freq }  & ADC sampling frequency  &  $\sim\,\SI{2.5}{\mega\hertz}$ &  Match \SI{1}{\micro\second} shaping time. &  Nyquist requirement and design choice \\ \colhline
    
   
  \newtag{DP-FD-20}{ spec:dp-adc-number-of-bits }  & Number of ADC bits  &  \num{12} bits &  ADC noise contribution negligible (low end); match signal saturation specification (high end). &  Engineering calculation and design choice \\ \colhline
    
   
  \newtag{DP-FD-21}{ spec:dp-ce-power-consumption }  & TPC analog cold FE electronics power consumption   &  $<\,\SI{50}{ mW/channel} $ &  No bubbles in LAr to reduce HV discharge risk. &  Bench test \\ \colhline
    
   
  \newtag{DP-FD-22}{ spec:dp-data-rate-to-tape }  & Data rate to tape  &  $<\,\SI{30}{PB/year}$ &  Cost.  Bandwidth. &  ProtoDUNE \\ \colhline
    
   
  \newtag{DP-FD-23}{ spec:dp-sn-trigger }  & Supernova trigger  &  $>\,\SI{95}{\%}$ efficiency for a SNB producing at least 60 interactions with a neutrino energy >10 MeV in 12 kt of active detector mass during the first 10 seconds of the burst. &  $>\,$90\% efficiency for SNB within 100 kpc &  Simulation and bench tests \\ \colhline
    
   
  \newtag{DP-FD-24}{ spec:dp-local-e-fields }  & Local electric fields  &  $<\,\SI{30}{kV/cm}$ &  Maximize live time; maintain high S/N. &  ProtoDUNE \\ \colhline
    
   
  \newtag{DP-FD-25}{ spec:de-non-fe-noise }  & Non-FE noise contributions  &  $<<\,\SI{1000}\,e^- $ &  High S/N for high reconstruction efficiency. &  Engineering calculation and ProtoDUNE \\ \colhline
    
   
  \newtag{DP-FD-26}{ spec:dp-lar-impurity-contrib }  & LAr impurity contributions from components  &  $<<\,\SI{30}{ppt} $ &  Maintain HV operating range for high live time fraction. &  ProtoDUNE \\ \colhline
    
   
  \newtag{DP-FD-27}{ spec:dp-radiopurity }  & Introduced radioactivity  &  less than that from $^{39}$Ar &  Maintain low radiological backgrounds for SNB searches. &  ProtoDUNE and assays during construction \\ \colhline
    
   
  \newtag{DP-FD-28}{ spec:dp-dead-channels }  & Dead channels  &  $<\,\SI{1}{\%}$ &  Contingency for possible efficiency loss for $>\,$20 year operation.  All DP electronics are accessible. &  ProtoDUNE \\ \colhline
    
   \newtag{DP-FD-29}{ spec:dp-det-uptime }  & Detector uptime  &  $>\,$98\% \newline ($>\,$99\%) &  Meet physics goals in timely fashion. &  ProtoDUNE \\ \colhline
    
   \newtag{DP-FD-30}{ spec:dp-det-mod-uptime }  & Individual detector module uptime  &  $>\,$90\% \newline ($>\,$95\%) &  Meet physics goals in timely fashion. &  ProtoDUNE \\ \colhline
    


\label{tab:specs:DP-FD}
\end{longtable}
\end{footnotesize}

% This file is generated, any edits may be lost.
\begin{footnotesize}
%\begin{longtable}{p{0.14\textwidth}p{0.13\textwidth}p{0.18\textwidth}p{0.22\textwidth}p{0.20\textwidth}}
\begin{longtable}{p{0.12\textwidth}p{0.18\textwidth}p{0.17\textwidth}p{0.25\textwidth}p{0.16\textwidth}}
\caption{Specifications for DP-HV \fixmehl{ref \texttt{tab:spec:DP-HV}}} \\
  \rowcolor{dunesky}
       Label & Description  & Specification \newline (Goal) & Rationale & Validation \\  \colhline

   \newtag{DP-FD-1}{ spec:dp-min-drift-field }  & Minimum drift field  &  $>$\,\SI{250}{V/cm} \newline ( $>\,\SI{500}{V/cm}$ ) &  Lessens impacts of $e^-$-Ar recombination, $e^-$ lifetime, $e^-$ diffusion and space charge. &  ProtoDUNE \\ \colhline
    
   
  \newtag{DP-FD-11}{ spec:dp-hvs-field-uniformity }  & Drift field uniformity due to HVS  &  $<\,\SI{1}{\%}$ throughout volume &  High reconstruction efficiency. &  ProtoDUNE and simulation \\ \colhline
    
   
  \newtag{DP-FD-12}{ spec:dp-hv-ps-ripple }  & Cathode HV power supply ripple contribution to system noise  &  $<\,\SI{100}e^-$ &  Maximize live time; maintain high S/N. &  Engineering calculation, in situ measurement,   ProtoDUNE \\ \colhline
    
   \newtag{DP-FD-17}{ spec:dp-cathode-resistivity }  & Cathode resistivity  &  $>\,\SI{1}{\mega\ohm/square}$ \newline ($>\,\SI{1}{\giga\ohm/square}$) &  Detector damage prevention. &  ProtoDUNE \\ \colhline
    
   
  \newtag{DP-FD-24}{ spec:dp-local-e-fields }  & Local electric fields  &  $<\,\SI{30}{kV/cm}$ &  Maximize live time; maintain high S/N. &  ProtoDUNE \\ \colhline
    

   \newtag{DP-HV-1}{ spec:hvdb-redundancy }  & Provide redundancy in HV distribution  &  $>\,num{2}$ HVDB chain \newline (\num{12} HVDB chains) &  Ensure the HV connections to the detector &  ProtoDUNE and calculations \\ \colhline
    


\label{tab:specs:DP-HV}
\end{longtable}
\end{footnotesize}
% This file is generated, any edits may be lost.

\begin{longtable}{p{0.14\textwidth}p{0.13\textwidth}p{0.18\textwidth}p{0.22\textwidth}p{0.20\textwidth}}
\caption{Specifications for DP-PDS \fixmehl{ref \texttt{tab:spec:DP-PDS}}} \\
  \rowcolor{dunesky}
       Label & Description  & Specification \newline (Goal) & Rationale & Validation \\  \colhline

   \newtag{SP-FD-1}{ spec:min-drift-field }  & Minimum drift field  &  $>$\,\SI{250}{ V/cm} \newline ( $>\,\SI{500}{ V/cm}$ ) &  Lessens impacts of $e^-$-Ar recombination, $e^-$ lifetime, $e^-$ diffusion and space charge. &  ProtoDUNE \\ \colhline
    
   
  \newtag{SP-FD-2}{ spec:system-noise }  & System noise  &  $<\,\SI{1000}\,e^-$ &  Provides $>$5:1 S/N on induction planes for  pattern recognition and two-track separation. &  ProtoDUNE and simulation \\ \colhline
    
   
  \newtag{SP-FD-3}{ spec:light-yield }  & Light yield  &  $>\,\SI{20}{PE/MeV}$ (avg), $>\,\SI{0.5}{PE/MeV}$ (min) &  Gives PDS energy resolution comparable that of the TPC for 5-7 MeV SN $\nu$s, and allows tagging of $>\,\SI{99}{\%}$ of nucleon decay backgrounds with light at all points in detector. &  Supernova and nucleon decay events in the FD with full simulation and reconstruction. \\ \colhline
    
    \\ \rowcolor{dunesky} \newtag{SP-FD-4}{ spec:time-resolution-pds } & Name: Time resolution \\
    Description & The time resolution of the photon detection system shall be less than 1 microsecond in order to assign a unique event time.   \\  \colhline
    Specification (Goal) &  $<\,\SI{1}{\micro\second}$  ( $<\,\SI{100}{\nano\second}$ ) \\   \colhline
    Rationale &   Enables \SI{1}{mm} position resolution for \SI{10}{MeV} SNB candidate events for instantaneous rate $<\,\SI{1}{m^{-3}ms^{-1}}$.  \\ \colhline
    Validation &   \\
   \colhline

   \newtag{SP-FD-5}{ spec:lar-purity }  & Liquid argon purity  &  $<$\,\SI{100}{ppt} \newline ($<\,\SI{30}{ppt}$) &  Provides $>$5:1 S/N on induction planes for  pattern recognition and two-track separation. &  Purity monitors and cosmic ray tracks \\ \colhline
    

   
  \newtag{DP-PDS-1}{ spec:hit-relative-timing }  & Relative timing accuracy among hits  &  $<\,\SI{100}{ns RMS}$ &  Enable effective clustering of \dword{pmt} signals based on relative hit timing information. &  Full sim/reco of \dword{ndk}, \dword{snb} $\nu$ and radiological events. \\ \colhline
    
   
  \newtag{DP-PDS-2}{ spec:hit-snr }  & Hit signal-to-noise ratio  &  $>\,\num{5}$ &  Efficiently reconstruct single-\phel hits while rate of electronics noise hits remains manageable. &  Single-\phel and baseline noise \dword{rms} measurements in $3\times1\times1$ prototype. \\ \colhline
    
   
  \newtag{DP-PDS-3}{ spec:hit-relative-timing }  & Relative timing accuracy among hits  &  $<\,\SI{100}{ns RMS}$ &  Enable effective clustering of PMT signals based on relative hit timing information. &  Full simulation/reconstruction of NDK, SN $\nu$ and radiological events. \\ \colhline
    
   
  \newtag{DP-PDS-4}{ spec:hit-relative-timing }  & Relative timing accuracy among hits  &  $<\,\SI{100}{ns RMS}$ &  Enable effective clustering of \dword{pmt} signals based on relative hit timing information. &  Full sim/reco of \dword{ndk}, \dword{snb} $\nu$ and radiological events. \\ \colhline
    
   
  \newtag{DP-PDS-5}{ spec:pmt-dark-rate }  & PMT dark count rate  &  $<\,\SI{100}{kHz}$ &  Dark counts should have negligible effect on clustering algorithm and PDS-based calorimetry. &  Characterization of PMTs at cryogenic temperatures prior to installation. \\ \colhline
    
   
  \newtag{DP-PDS-6}{ spec:pds-dynamic-range }  & Dynamic range per channel  &  $>\,\SI{200}{PE}$ &  Avoid hit saturation for energy depositions near cathode plane.  &  Full simulation/reconstruction of beam $\nu$ interactions near cathode plane.  \\ \colhline
    
   \newtag{DP-PDS-7}{ spec:time-resolution-dp-pds }  & Time resolution  &  $<\,\SI{1}{\micro\second}$ \newline ( $<\,\SI{100}{\nano\second}$ ) &  Enables \SI{1}{mm} position resolution for \SI{10}{MeV} SNB candidate events for instantaneous rate $<\,\SI{1}{m^{-3}ms^{-1}}$. &   \\ \colhline
    


\label{tab:specs:DP-PDS}
\end{longtable}
% This file is generated, any edits may be lost.
\begin{footnotesize}
%\begin{longtable}{p{0.14\textwidth}p{0.13\textwidth}p{0.18\textwidth}p{0.22\textwidth}p{0.20\textwidth}}
\begin{longtable}{p{0.12\textwidth}p{0.18\textwidth}p{0.17\textwidth}p{0.25\textwidth}p{0.16\textwidth}}
\caption{Specifications for DP-CISC \fixmehl{ref \texttt{tab:spec:DP-CISC}}} \\
  \rowcolor{dunesky}
       Label & Description  & Specification \newline (Goal) & Rationale & Validation \\  \colhline

   \newtag{DP-FD-1}{ spec:dp-min-drift-field }  & Minimum drift field  &  $>$\,\SI{250}{V/cm} \newline ( $>\,\SI{500}{V/cm}$ ) &  Lessens impacts of $e^-$-Ar recombination, $e^-$ lifetime, $e^-$ diffusion and space charge. &  ProtoDUNE \\ \colhline
    
   \newtag{DP-FD-5}{ spec:lar-purity }  & Liquid argon purity  &  $<$\,\SI{100}{ppt} \newline ($<\,\SI{30}{ppt}$) &  Provides $>$5:1 S/N on induction planes for  pattern recognition and two-track separation. &  Purity monitors and cosmic ray tracks \\ \colhline
    
   
  \newtag{DP-FD-15}{ spec:dp-lar-n-contamination }  & LAr nitrogen contamination  &  $<\,\SI{25}{ppm}$ &  Maintain \SI{0.5}{PE/MeV} PDS sensitivity required for triggering proton decay near cathode. &   \\ \colhline
    
   
  \newtag{DP-FD-16}{ spec:dp-det-dead-time }  & Detector dead time  &  $<\,\SI{0.5}{\%}$ &  Meet physics goals in timely fashion. &  ProtoDUNE \\ \colhline
    
   
  \newtag{DP-FD-18}{ spec:dp-cryo-monitor-devices }  & Cryogenic monitoring devices  &   &  Constrain uncertainties on detection efficiency, fiducial volume. &  ProtoDUNE \\ \colhline
    
   
  \newtag{DP-FD-24}{ spec:dp-local-e-fields }  & Local electric fields  &  $<\,\SI{30}{kV/cm}$ &  Maximize live time; maintain high S/N. &  ProtoDUNE \\ \colhline
    
   
  \newtag{DP-FD-25}{ spec:de-non-fe-noise }  & Non-FE noise contributions  &  $<<\,\SI{1000}\,e^- $ &  High S/N for high reconstruction efficiency. &  Engineering calculation and ProtoDUNE \\ \colhline
    

   
  \newtag{DP-CISC-1}{ spec:inst-noise }  & Noise from Instrumentation devices  &  $\ll\,\SI{1000}\,e^- $ &  Max noise for 5:1 S/N for a MIP passing near cathode; per SBND and DUNE CE &  ProtoDUNE \\ \colhline
    
   \newtag{DP-CISC-2}{ spec:inst-efield }  & Max. E field near instrumentation devices  &  $<\,\SI{30}{kV/cm}$ \newline ( $<\,\SI{15}{kV/cm}$ ) &  Significantly lower than max field of 30 kV/cm per DUNE HV  &  3D electrostatic simulation \\ \colhline
    
   \newtag{DP-CISC-3}{ spec:elec-lifetime-prec }  & Precision in electron lifetime  &  $<\,$1.4\% \newline ( $<\,$1\% ) &  Required for accurate charge reconstruction per DUNE-FD Task Force report. &  ProtoDUNE and CITF \\ \colhline
    
   \newtag{DP-CISC-4}{ spec:elec-lifetime-range }  & Range in electron lifetime  &  \SIrange{0}{10}{ms} (\SIrange{0}{30}{ms}) \newline (\SIrange{0}{10}{ms} (\SIrange{0}{30}{ms})) &  Slightly more than best values so far observed in other detectors. &  ProtoDUNE and CITF \\ \colhline
    
   \newtag{DP-CISC-11}{ spec:temp-repro }  & Precision: temperature reproducibility  &  $<\,\SI{5}{mK}$ \newline (\SI{2}{mK}) &  Allows validating CFD models that predict gradients less than 15 mK. &  ProtoDUNE and CITF \\ \colhline
    
   \newtag{DP-CISC-14}{ spec:temp-stability }  & Temperature stability  &  $<\,\SI{2}{mK}$ at all places and times \newline ( Match precision requirement at all places, at all times ) &  Measures temperature map with sufficient precision for the duration of thermometer operations. &  ProtoDUNE \\ \colhline
    
   \newtag{DP-CISC-27}{ spec:camera-cold-coverage }  & Cold camera coverage  &  $>\,$80\% of HV surfaces \newline ( \num{100}\% ) &  Enables detailed inspection of issues near HV surfaces. &  Calculated from location, validated in prototypes. \\ \colhline
    
   \newtag{DP-CISC-51}{ spec:slowcontrol-alarm-rate }  & Slow control alarm rate  &  $<\,$150/day \newline ($<\,$50/day) &  Keeps rate low enough to allow response to every alarm. &  Detector module; depends on experimental conditions \\ \colhline
    
   \newtag{DP-CISC-52}{ spec:slowcontrol-num-vars }  & Total No. of variables  &  $>\,\num{150000}$ \newline (\SIrange{150000}{200000}{}) &  Scaled from ProtoDUNE &  ProtoDUNE and CITF \\ \colhline
    
   \newtag{DP-CISC-54}{ spec:slowcontrol-archive-rate }  & Archiving rate  &  \SI{0.02}{Hz} \newline ( Broad range \SI{1}{Hz} to \num{1} per few min. ) &  Archiving rate differs by variable, optimized to store important information &  ProtoDUNE \\ \colhline
    


\label{tab:specs:DP-CISC}
\end{longtable}
\end{footnotesize}



%%%%%%%%%%%%%%%%%%%%%%%%%%%%%%%%
\section{Full DUNE Risks}
\label{sec:fdsp-app-risk}

%%%%%%%%%%%%%%%%%%%%%%%%%%%%%%%
\subsection{Single-phase}
\label{sec:tc-risks-sp}


% risk table values for subsystem SP-FD-APA
\begin{longtable}{p{0.18\textwidth}p{0.20\textwidth}p{0.32\textwidth}p{0.02\textwidth}p{0.02\textwidth}p{0.02\textwidth}} 
\caption{Risks for SP-FD-APA \fixmehl{ref \texttt{tab:risks:SP-FD-APA}}} \\
\rowcolor{dunesky}
ID & Risk & Mitigation & P & C & S  \\  \colhline
RT-SP-APA-01 & Loss of key personnel & Implement succession planning and formal project documentation & L & L & M \\  \colhline
RT-SP-APA-02 & Delay in finalisation of APA frame design & Close oversight on prototypes and interface issues & L & L & M \\  \colhline
RT-SP-APA-03 & One additional pre-production APA may be necessary & Close oversight on approval of designs, commissioning of tooling and assembly procedures & L & L & L \\  \colhline
RT-SP-APA-04 & APA winder construction takes longer than planned & Detailed plan to stand up new winding machines at each facility & M & L & M \\  \colhline
RT-SP-APA-05 & Poor quality of APA frames and/or inaccuracy in the machining of holes and slots & Clearly specified requirements and seek out backup vendors & L & L & M \\  \colhline
RT-SP-APA-06 & Insufficient scientific manpower at APA assembly factories & Get institutional commitments for requests of necessary personnel in research grants & M & M & L \\  \colhline
RT-SP-APA-07 & APA production quality does not meet requirements & Close oversight on assembly procedures & L & M & M \\  \colhline
RT-SP-APA-08 & Materials shortage at factory & Develop and execute a supply chain management plan & M & L & L \\  \colhline
RT-SP-APA-09 & Failure of a winding machine - Drive chain parts failure & Regular maintenance and availability of spare parts & L & L & L \\  \colhline
RT-SP-APA-10 & APA assembly takes longer time than planned  & Estimates based on protoDUNE. Formal training of every tech/operator & L & M & M \\  \colhline
RT-SP-APA-11 & Loss of one APA due to an accident & Define handling procedures supported by engineering notes & M & L & L \\  \colhline
RT-SP-APA-12 & APA transport box inadequate & Construction and test of prototype transport boxes & L & L & M \\  \colhline
RO-SP-APA-01 & Reduction of the APA assembly time & Improvements in the winding head and wire tension mesurements & M & M & M \\  \colhline
 &  &  &  &  &  \\  \colhline

\label{tab:risks:SP-FD-APA}
\end{longtable}

% risk table values for subsystem SP-FD-PD
\begin{footnotesize}
%\begin{longtable}{p{0.18\textwidth}p{0.20\textwidth}p{0.32\textwidth}p{0.02\textwidth}p{0.02\textwidth}p{0.02\textwidth}}
\begin{longtable}{P{0.18\textwidth}P{0.20\textwidth}P{0.32\textwidth}P{0.02\textwidth}P{0.02\textwidth}P{0.02\textwidth}} 
\caption[Risks for SP-FD-PD]{Risks for SP-FD-PD (P=probability, C=cost, S=schedule) More information at \dword{riskprob}. \fixmehl{ref \texttt{tab:risks:SP-FD-PD}}} \\
\rowcolor{dunesky}
ID & Risk & Mitigation & P & C & S  \\  \colhline
RT-SP-PD -01 & Additional photosensors and engineering required to ensure PD modules collect enough light to meet system physics performance specifications. & Extensive validation of \dword{xarapu} design to demonstrate they meet specification. & L & M & L \\  \colhline
RT-SP-PD-02 & Improvements to active ganging/front end electronics required to meet the specified 1~$\mu$s time resolution. & Extensive validation of photosensor ganging/front end electronics design to demonstrate they meet specification. & L & L & L \\  \colhline
RT-SP-PD-03 & Evolutions in the design of the photon detectors due to validation testing experience require modifications of the TPC elements at a late time. & Extensive validation of \dword{xarapu} design to demonstrate they meet specification and control of PD/APA interface. & L & L & L \\  \colhline
RT-SP-PD-04 & Cabling for PD and CE within the \dword{apa} frame or during the 2-APA assembly/installation procedure require additional engineering/development/testing. & Validation of PD/APA/CE cable routing in prototypes at Ash River. & L & L & L \\  \colhline
RT-SP-PD-05 & Experience with validation prototypes shows that the mechanical design of the PD is not adequate to meet system specifications. & Early validation of \dword{xarapu} prototypes and system interfaces to catch problems ASAP. & L & L & L \\  \colhline
RT-SP-PD-06 & pTB WLS filter coating not sufficiently stable, contaminates \dword{lar}. & Mechanical acceleration of coating wear.  Long-term tests of coating stability. & L & L & L \\  \colhline
RT-SP-PD-07 & Photosensors fail due to multiple cold cycles or extended cryogen exposure. & Execute testing program for cryogenic operation of photosensors including mutiple cryogenic immersion cycles. & L & L & L \\  \colhline
RT-SP-PD-08 & SiPM active ganging cold amplifiers fail or degrade detector performance. & Validation testing if photosensor ganging in multiple test beds. & L & L & L \\  \colhline
RT-SP-PD-09 & Previously undetected electro-mechanical interference discovered during integration. & Validation of electromechanical designin Ash River tests and at \dword{pdsp2}. & L & L & L \\  \colhline
RT-SP-PD-10 & Design weaknesses manifest during module logistics-handling. & Validation of shipping packaging and handling prior to shipping.  Inspection of modules shipped to site immediately upon receipt. & L & L & L \\  \colhline
RT-SP-PD-11 & PD/CE signal crosstalk. & Validation in \dword{pdsp}, \dword{iceberg} and \dword{pdsp2}. & L & L & L \\  \colhline
RT-SP-PD-12 & Lifetime of \dword{pd} components outside cryostat. & Specification of environmental controls to mitigate detector aging. & L & L & L \\  \colhline

\label{tab:risks:SP-FD-PD}
\end{longtable}
\end{footnotesize}

% risk table values for subsystem SP-FD-TPC
\begin{longtable}{p{0.18\textwidth}p{0.20\textwidth}p{0.32\textwidth}p{0.02\textwidth}p{0.02\textwidth}p{0.02\textwidth}} 
\caption{Risks for SP-FD-TPC \fixmehl{ref \texttt{tab:risks:SP-FD-TPC}}} \\
\rowcolor{dunesky}
ID & Risk & Mitigation & P & C & S  \\  \colhline
RT-SP-TPC-01 & Cold ASIC(s) not meeting specifications & Multiple developments, use of appropriate design rules for operation in LAr & M & M & L \\  \colhline
RT-SP-TPC-02 & Delay in the availability of ASICs and FEMBs & Increase pool of spares for long lead items, multiple QC sites for ASICs, appropriate measure against ESD damage, monitoring of yields & L & L & L \\  \colhline
RT-SP-TPC-03 & Damage to the FEMBs / cold cables during or after integration with the APAs & Redesign of the FEMB/cable connection, use of CE boxes, ESD protections, early integration tests & M & L & L \\  \colhline
RT-SP-TPC-04 & Cold cables cannot be run through the APAs frames & Redesign of APA frame, integration tests at Ash River and at CERN, further reduction of cable plant & M & L & L \\  \colhline
RT-SP-TPC-05 & Delay and/or damage to the TPC electronics components on the top of the cryostat & Sufficient spares, early production and installation, ESD protection measures & L & L & L \\  \colhline
RT-SP-TPC-06 & Interfaces between TPC electronics and other consortia not adequately defined & Early integration tests, second run of ProtoDUNE with pre-production components & L & L & L \\  \colhline
RT-SP-TPC-07 & Insufficient number of spares & Early start of production, close monitoring of usage of components, larger stocks of components with long lead times & L & L & L \\  \colhline
RT-SP-TPC-08 & Loss of key personnel & Distributed development of ASICs, increase involved of university groupos, training of younger personnel & L & L & M \\  \colhline
RT-SP-TPC-09 & Excessive noise observed during detector commissioning & Enforce grounding rules, early integration tests, second run of ProtoDUNE with pre-production components, cold box testing at SURF & M & L & M \\  \colhline
RT-SP-TPC-10 & Lifetime of components in the LAr & Design rules for cryogenic operation of ASICs, measurement of lifetime of components, reliability studies & M & n/a & n/a \\  \colhline
RT-SP-TPC-11 & Lifetime of components on the top of the cryostat & Use of filters on power supplies, stockpiling of components that may become obsolete, design rules to minimize parts that need to be redesigned / refabricated & L & M & L \\  \colhline

\label{tab:risks:SP-FD-TPC}
\end{longtable} % this is tpc elec

% risk table values for subsystem SP-FD-CAL
\begin{longtable}{p{0.18\textwidth}p{0.20\textwidth}p{0.32\textwidth}p{0.02\textwidth}p{0.02\textwidth}p{0.02\textwidth}} 
\caption{Risks for SP-FD-CAL \fixmehl{ref \texttt{tab:risks:SP-FD-CAL}}} \\
\rowcolor{dunesky}
ID & Risk & Mitigation & P & C & S  \\  \colhline
RT-SP-CAL-01 & Inadequate baseline design & Early detection allows R\&D of alternative designs accommodated through multipurpose ports & L & M & M \\  \colhline
RT-SP-CAL-02 & Inadequate engineering or production quality & Dedicated small scale tests and full prototyping at ProtoDUNE; pre-installation QC & L & M & M \\  \colhline
RT-SP-CAL-03 & Laser impact on PDS & Mirror movement control to avoid direct hits; turn laser off in case of PDS saturation & L & L & L \\  \colhline
RT-SP-CAL-04 & Laser positioning system stops working & QC at installation time, redundancy in available targets, including passive, alternative methods & L & L & L \\  \colhline
RT-SP-CAL-05 & Laser beam misaligned & Additional (visible) laser for alignment purposes & M & L & L \\  \colhline
RT-SP-CAL-06 & The neutron anti-resonance is much less pronounced & Dedicated measurements at LANL and test at ProtoDUNE & L & L & L \\  \colhline
RT-SP-CAL-07 & Neutron activation of the moderator and cryostat & Neutron activation studies and simulations & L & L & L \\  \colhline
RT-SP-CAL-08 & Neutron yield not high enough & Simulations and tests at ProtoDUNE & L & M & M \\  \colhline
RT-SP-CAL-09 & Neutrons do not reach detector center & Alternative, movable design and simulations & L & L & L \\  \colhline

\label{tab:risks:SP-FD-CAL}
\end{longtable}

% risk table values for subsystem SP-FD-JPO
\begin{longtable}{p{0.15\textwidth}p{0.13\textwidth}p{0.13\textwidth}p{0.28\textwidth}p{0.06\textwidth}p{0.06\textwidth}p{0.06\textwidth}} 
\caption{Specification for SP-FD-JPO \fixmehl{ref \texttt{tab:specs:SP-FD-JPO}}} \\
\rowcolor{dunesky}
ID & Risk & Label & Mitigation & Prob ability & Cost Impact & Sched ule Impact \\  \colhline
RT-JPO-001 & Personnel injury & jpo-person-injury & Follow established safety plans. & M & L & H \\  \colhline
RT-JPO-002 & Shipping delays & jpo-shipping-delay & Plan one month buffer to store  materials locally. Provide logistics manual. & H & L & L \\  \colhline
RT-JPO-003 & Missing components cause delays & jpo-missing-components & Use detailed inventory system to verify availability of  necessary components.  & H & L & L \\  \colhline
RT-JPO-004 & Import, export, visa issues  & jpo-import-visa & Dedicated \dword{fnal} \dword{sdsd}division will expedite import/export and visa-related issues. & H & M & M \\  \colhline
RT-JPO-005 & Lack of available labor  & jpo-labor-avail & Hire early and use Ash River setup to train \dword{jpo} crew. & L & L & L \\  \colhline
RT-JPO-006 & Parts do not fit together & jpo-cannot-assemble & Generate \threed model, create interface drawings, and prototype detector assembly. & H & L & L \\  \colhline
RT-JPO-007 & Cryostat damage & jpo-cryostat-damage & Use cryostat false floor and temporary protection. & L & L & M \\  \colhline
RT-JPO-008 & Weather closes SURF & jpo-weather-delay & Plan for \dword{surf} weather closures & H & L & L \\  \colhline
RT-JPO-009 & Detector failure during \cooldown & jpo-cooldown-failure & Cold test individual components then cold test \dword{apa} assemblies immediately before installation. & L & H & H \\  \colhline

\label{tab:risks:SP-FD-JPO}
\end{longtable} % aka tc or iic

%%%%%%%%%%%%%%%%%%%%%%%%%%%%%%%
\subsection{Dual-phase}
\label{sec:tc-risks-dp}


% risk table values for subsystem DP-FD-HV
\begin{longtable}{p{0.18\textwidth}p{0.20\textwidth}p{0.32\textwidth}p{0.02\textwidth}p{0.02\textwidth}p{0.02\textwidth}} 
\caption{Risks for DP-FD-HV \fixmehl{ref \texttt{tab:risks:DP-FD-HV}}} \\
\rowcolor{dunesky}
ID & Risk & Mitigation & P & C & S  \\  \colhline
RT-DP-HV-01 & Broken resistors or varistors on voltage divider boards & Redundancy of resistors, varistors, and \dwords{hvdb}.  & L & L & L \\  \colhline
RT-DP-HV-02 & \efield uniformity is not adequate for muon momentum reconstruction & Regularly map out field using a laser calibration sysem. & L & L & L \\  \colhline
RT-DP-HV-03 & \efield is below specification during stable operations & Improve purity by more aggressive filtering. & M & M & L \\  \colhline
RT-DP-HV-04 & Space charge from positive ions distorting the \efield beyond expectation & Minimize insulators facing cryostat wall ground. & M & M & L \\  \colhline
RT-DP-HV-05 & Damage to \dword{ce} in event of discharge & Minimize the energy released in a short time using highly resistive connections. & L & L & L \\  \colhline
RT-DP-HV-06 & Energy stored in FC (in DP) is suddenly discharged & Delay energy discharge by connecting neighboring Al profiles with resistive sheaths.  & L & L & L \\  \colhline
RT-DP-HV-07 & Detector components are damaged during shipment to the far site & Make sufficient spares and increase the number of shipping boxes.  & L & L & L \\  \colhline
RT-DP-HV-08 & Damages (scratches, bending) to aluminum profiles of Field Cage modules & Make sufficient spares and increase the number of shipping boxes.  & L & L & L \\  \colhline
RT-DP-HV-09 & Bubbles from heat in PMTs or resistors cause HV discharge & A large area of cathode consists of high resistance rods, delaying the energy release.   & L & L & L \\  \colhline
RT-DP-HV-10 & Free hanging frames can swing in the fluid flow &  & L & L & L \\  \colhline
RT-DP-HV-11 & FRP/ Polyethene/ laminated Kapton component lifetime is less than expected &  & L & L & L \\  \colhline
RT-DP-HV-12 & Lack of collaboration effort on this HV system & Continue recruiting collaborators. & L & L & L \\  \colhline
RT-DP-HV-13 & International funding level for DP HVC too low & Employ cost saving measures and  recruit collaborators. & L & L & L \\  \colhline
RT-DP-HV-14 & Underground installation is more labor intensive or slower than expected & Increase labor contingency and refine labor cost estimates. Further improve installation procedure. & L & L & L \\  \colhline

\label{tab:risks:DP-FD-HV}
\end{longtable}

% risk table values for subsystem DP-FD-PDS
\begin{footnotesize}
%\begin{longtable}{p{0.18\textwidth}p{0.20\textwidth}p{0.32\textwidth}p{0.02\textwidth}p{0.02\textwidth}p{0.02\textwidth}}
\begin{longtable}{P{0.18\textwidth}P{0.20\textwidth}P{0.32\textwidth}P{0.02\textwidth}P{0.02\textwidth}P{0.02\textwidth}} 
\caption[DP PDS risks]{Risks for DP-FD-PDS (P=probability, C=cost, S=schedule) More information at \dshort{riskprob}. \fixmehl{ref \texttt{tab:risks:DP-FD-PDS}}} \\
\rowcolor{dunesky}
ID & Risk & Mitigation & P & C & S  \\  \colhline
RT-DP-PDS-01 & Insufficient light yield due to inefficient PDS design & Increase PMT photo-cathode coverage and/or WLS reflector foils coverage. & L & M & L \\  \colhline
RT-DP-PDS-02 & Poor coating quality for \dshort{tpb} coated surfaces and \dshort{lar} contamination by \dshort{tpb} & Test quality and ageing properties of TPB coating techniques. Elaborate improved techniques if needed. & L & L & L \\  \colhline
RT-DP-PDS-03 & \dshort{pmt} channel loss due to faulty \dshort{pmt} base design & Optimize clustering algorithms. Improve \dshort{pmt} base design from analysis of possible failure modes in \dshort{pddp}. & L & L & L \\  \colhline
RT-DP-PDS-04 & Bad \dshort{pmt} channel due to faulty connection between \dshort{hv}/signal cable and \dshort{pmt} base & Optimize clustering algorithms. Connectivity tests in \lntwo prior to installation. & L & L & L \\  \colhline
RT-DP-PDS-05 & \dshort{pmt} signal saturation & Tuning of \dshort{pmt} gain. In worst case, redesign front-end to adjust to analog input range of ADC. & M & L & L \\  \colhline
RT-DP-PDS-06 & Excessive electronics noise to distinguish \dshort{lar} scintillation light & Measurement of noise levels during commissioning prior to \lar filling. Modifications to grounding, shielding, or power distribution schemes. & M & L & L \\  \colhline
RT-DP-PDS-07 & Availability of resources for work at the installation/integration site less than planned & Move people temporarily from institutions involved in the \dshort{pds} consortium to the integration/installation site. & L & L & L \\  \colhline
RT-DP-PDS-08 & Damage of \dshorts{pmt} during shipment to the experiment site & Special packaging to avoid possible \dshort{pmt} damage during shipment. Contingency of 10\% spare  \dshorts{pmt}. & L & L & L \\  \colhline
RT-DP-PDS-09 & Damage of optical fibers during installation & Fibers will be last DP-PDS item to be installed. Detailed documentation for all DP-PDS installation tasks.   & L & L & L \\  \colhline
RT-DP-PDS-10 & Excessive exposure to ambient light of \dshort{tpb} coated surfaces, resulting in degraded performance & \dshort{tpb} coated surfaces temporarily covered until cryostat closing. Detailed installation procedure to minimize exposure to ambient light. & L & L & L \\  \colhline
RT-DP-PDS-11 & \dshort{pmt} implosion during \dshort{lar} filling & No mitigation necessary, considering \SI{7}{bar} pressure rating of \dshorts{pmt} and experience with same/similar \dshorts{pmt} in other large liquid detectors. & L & L & L \\  \colhline
RT-DP-PDS-12 & Insufficient light yield due to poor \dshort{lar} purity & Procurement of \dshort{lar} from the manufacturer will require less than 3 ppM in nitrogen. & M & L & L \\  \colhline
RT-DP-PDS-13 & \dshort{pmt} channel or \dshort{pds} sector loss due to failures in \dshort{hv}/signal rack & Ease of maintenance outside cryostat and availability of spares for all components of at least one \dshort{hv}/signal rack. & L & L & L \\  \colhline
RT-DP-PDS-14 & Unstable response of the photon detection system over the lifetime of the experiment & Channel-level instabilities corrected via light calibration system. Detector-level instabilities corrected via cosmic-ray muon calibration data. & L & L & L \\  \colhline
RT-DP-PDS-15 & Bubbles from heat in \dshorts{pmt} or resistors cause \dshort{hv} discharge of the cathode & Verify the power density of the \dshort{pmt} bases are within specifications. Monitor and interlock \dshort{pmt} power supply currents. & L & L & L \\  \colhline
RT-DP-PDS-16 & Reflector/\dshort{wls} panel assemblies together with the \dshort{fc} walls can swing in the fluid flow & Allow appropriate open areas within/between reflector/\dshort{wls} panel assemblies to minimize drag. & L & L & L \\  \colhline

\label{tab:risks:DP-FD-PDS}
\end{longtable}
\end{footnotesize}



