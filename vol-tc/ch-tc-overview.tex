\chapter{Overview}
\label{vl:tc-overview}

\fixme{I would call this chapter ``Management and Responsibilties'' or some such and reorganize as ``who'' and ``what''. Currently it's a mixture that's not as clear as it could be}

%\section{(from Anne) The Organizational Structure}
%\label{sec:exb}

\fixme{Suggest some intro with a chart} 


%\section{Consortia}
\section{DUNE Far Detector Consortia}
\label{sec:consortia}


Construction of the \dword{dune} far \dwords{detmodule} is carried out by
``consortia of collaboration institutions'' who assume responsibility
for detector subsystems.  Each consortium plans and
executes the construction, installation and commissioning of its 
subsystem.


A total
of eleven \dword{fd} consortia have been formed to cover the
subsystems required for the two \dword{lartpc} designs, \dword{sp} and \dword{dp}, currently under
consideration.  Three  consortia pursue subsystems specific to
the \dword{sp} design (\dword{apa}, \dword{ce}, and \dword{pds}) and another three consortia pursue designs for \dword{dp}
specific subsystems (\dword{crp}, TPC electronics and a \dword{dp} \dword{pds}).  An additional five consortia are responsible for subsystems common to both detector
technologies; these are \dword{hv}, \dword{daq}, \dword{cisc}, calibration and computing.



A consortium leader
and a technical lead manage each consortium.  The consortium leader chairs an institutional
board composed of one representative from each of the contributing 
institutions.  Significant 
consortium decisions, such as technology selections and assignment of
responsibilities among the institutions, are first passed
through this board,  then passed as
recommendations to the \dword{dune} \dword{exb} for formal collaboration approval. 


In many cases, multiple institutions within a consortium, potentially supported by different funding agencies, share responsibility for a particular subsystem deliverable. 
\dword{dune} expects each participating funding agency to manage its
own internal project responsibilities.  The consortium technical lead coordinates the separate internal projects. This person also chairs a consortium project management board, composed of the managers of each internal project, that oversees the
interconnections between the different efforts.


%\section{(suggested) Executive Board}

The \dword{dune} \dword{exb} is the primary
collaboration decision-making body and as such includes
representatives from all major areas of activity within the
collaboration.  All collaboration decisions, especially those with
potential impact on the \dword{dune} scientific program or connected with the
assignment of institutional responsibilities, pass through the
\dword{exb}.  \dword{exb} decisions are expected to be
achieved through consensus.  In cases where consensus cannot be
obtained, decision-making responsibility passes to the
co-spokespersons.

The consortium leader represents the consortium on the \dword{exb}.


\section{Technical Coordination (TCN)}
\label{sec:tc}

Because the consortia operate as self-managed entities, a strong
\dword{tc} organization is required to ensure 
overall integration of the detector elements and successful
execution of the detector construction project.  \dword{tc}
responsibility includes project oversight,
systems engineering, quality assurance, and safety.  \dword{tc}
provides support to the \dword{jpo} (see
Section~\ref{sec:pm}) for planning and executing the required detector
integration and installation activities in the nearby surface
facilities and underground detector caverns at \surf.


\dword{tc} is headed by the \dword{tcoord}, who is an 
\dword{fnal} employee and is appointed jointly by the \dword{fnal}  director and
the \dword{dune} co-spokespersons.  A deputy technical coordinator is
selected from within the collaboration to assist the \dword{tc}. % with carrying out their responsibilities.



The \dword{tc} organization supports the work of the consortia and
takes responsibility for integration of the detector
subsystems.  The organization includes teams focused on project
coordination, detector integration, and installation support.  The
project coordination team is led by a lead project controls
specialist, a \dword{qa} manager, and an \dword{esh} manager.  The
detector integration team is directed by a lead mechanical and lead
electrical engineer, and includes an online computing coordinator.
The installation support team is headed by the coordinators for
activities associated with the integration of detector components on
the surface and installation of components in the underground areas.
Each of the three teams incorporates additional personnel that support
these individuals in carrying out their areas of responsibility.

\section{Project Management}
\label{sec:pm}

\fixme{I'm not clear on the distinction between board meetings:  consortium board meetings with all consortium leads? Then tech board meetings are at a higher level? Who is involved? Then PMBs with funding agencies (since they are responsible for managing their projects?}

The \dword{tcoord} manages the overall detector construction
project through regular Technical Board meetings with the consortia leadership
teams and members of the \dword{tc} organization.  These
board meetings provide the primary forums for required interactions
between the consortia leadership teams.

Technical Board meetings are used to evaluate consortia design
decisions with potential impacts on overall detector performance,
ensure that interfaces between the different subsystems are well
understood and documented, and monitor the overall construction
project to identify and address both technical and interface issues as
they arise.

Project board meetings are used to ensure that the scopes of each
consortium are fully documented with assigned institutional
responsibilities, develop and manage risks held within a global
project registry, review and manage project change requests, and
monitor the status of the overall detector construction schedule.

Any decisions generated through these board meetings are passed to the
\dword{dune} \dword{exb} as recommendations for formal approval.
Depending on the specific agenda items %to be discussed at these meetings, 
for a meeting, the \dword{tcoord} will invite additional members of
the collaboration with specific knowledge or particular expertise to
participate.  In addition, for major decisions, the \dword{tcoord}
will officially appoint three internal collaboration
referees with no direct conflicts of interest to engage in the
process.

\section{Global Project Partners}
\label{sec:partners}

\fixme{with this section heading, I'd expect something about international funding agencies...this info could go up top in my suggested section}

The \dword{lbnf} project is responsible for providing conventional
facilities and supporting infrastructure (cryostats and cryogenic
systems)  that house the \dword{dune} \dword{fd} modules. \dword{lbnf} is a \dword{us} DOE
project incorporating in-kind contributions from international
partners and is headed by the \dword{lbnf} project director, who also serves as
the \dword{fnal} deputy director for \dword{lbnf}.  Conversely, \dword{dune} is an
international project 
directed by the \dword{dune} collaboration management team, with some \dword{us} DOE contributions. 

The \dword{efig} is responsible for
high-level coordination between the \dword{lbnf} and \dword{dune}
projects.  The \dword{efig} is co-chaired by the \dword{lbnf} project director and the
integration project director, who is appointed by and reports to the
\dword{fnal}  director, and is responsible for the integration and
installation of the detector modules and for supporting the  \dword{lbnf}
infrastructure in the underground areas at \dword{surf}, post-excavation.  The
integration project director is connected to the facilities and
detector construction projects through their ex-officio positions on
the \dword{lbnf} project management board and \dword{dune} \dword{exb},
respectively.

\dword{efig} leadership incorporates the four members of the \dword{dune}
collaboration management team (co-spokespersons, \dword{tcoord} and
\dword{rcoord}).  The \dword{efig} is responsible for steering the global
project and operates via consensus.  %In the event that an issue arises for which 
If consensus cannot be achieved for a given issue, responsibility for
resolving the issue is passed to the \dword{fnal} director.

The \dword{efig} is supported by the \dword{jpo}, headed by the
integration project director.  \dword{jpo} functions include global project
configuration and integration, installation planning and coordination,
scheduling, safety assurance, technical review planning and oversight,
development of partner agreements, and financial reporting.  The \dword{jpo}
teams covering each of these areas are formed from the members of the
\dword{lbnf} project and \dword{dune} \dword{tc} teams with
responsibilities in these same areas. %towards these same items.  
For example, the \dword{jpo} team
responsible for building the fully integrated model of the detector
within its supporting infrastructure and the surrounding facility
includes members of the \dword{lbnf} project and \dword{dune} \dword{tc}
teams responsible for integrating the individual elements.

\section{Far Site Integration and Installation}
\label{sec:far_site}

All the collaboration organizations described in this chapter share responsibilities for for far site integration and installation. 
 
As mentioned in Section~\ref{sec:partners}, the integration project director has
responsibility for the overall planning and execution of activities
associated with component integration and installation, both in the
 underground detector caverns at \dword{surf} and in the nearby surface facilities. 
These activities are necessarily supported by the \dword{dune} consortia who
work with the coordinators of these efforts to develop the overall
plan for assembling and installing the detectors.  The consortia are
also responsible for providing the expert personnel and specialized
equipment necessary to integrate, install, and commission their
detector subsystems.

\fixme{following pgraph is a bit disjointed}
\dword{dune} \dword{tc} provides support to the core \dword{jpo} team that plans and coordinates installation. 
In particular, the coordinators and lead technicians associated with the
detector integration and installation efforts are members of the
\dword{tc} organization.  The core team also includes
riggers and other personnel responsible for the transportation and
movement of materials both to and within the \dword{surf} facility.  These
team members are associated with the \dword{fnal} \fixme{sdsd}, which is responsible providing host laboratory functions at
the far site.  \fixme{sdsd} members of the \dword{jpo} core team responsible for
component integration and installation include the rigging teams
responsible for moving materials in and out of the shaft, through the
underground tunnels, and within the detector caverns.  \fixme{sdsd} 
supports team members responsible for safety oversight and logistics
planning at the far site.


The \dword{jpo} team responsible for installation planning and coordination is
also responsible for the specification and procurement of
common infrastructure.  Common infrastructure includes detector pieces
such as racks, cable trays, cryostat flanges, as well as the
mechanical structure that supports the detector components from the
top of the cryostat.  It also includes general items required for
detector integration and installation, such as clean rooms, cranes,
scaffolding, and personnel lifts. \dword{dune} \dword{tc}
provides engineering resources to the \dword{jpo} team to support the needed
design efforts. Specialized installation hardware associated with
specific subsystems is provided by the consortia.

\section{Technical Coordination (TCN) Resources}
\label{sec:tc_resources}

Resources for the personnel and activities associated with \dword{tc}
are provided through a mixture of common collaboration funds and
support from the host nation.  During the construction phase of the
experiment, \dword{dune} will collect an annual membership fee from
each institution based on the number of Ph.D scientists %within 
contributing to the
collaboration.  The collected funds will be used primarily for
supporting the personnel %included within 
on the \dword{tc} team.  Funding
agencies will have the option of directly providing team members in
lieu of %cash 
monetary contributions.

The acquisition of common infrastructure will be supported through
additional collaboration common funds collected from the funding
agencies as a percentage of the value of their contributed detector
deliverables.  As in the case above, \dword{dune} will give funding agencies %will be given
the option of directly providing equivalently valued infrastructure
items as opposed to %actual cash 
monetary contributions.  Host laboratory
functions provided by the SDSD will be supported directly by the host nation
funding agency, DOE.

All project partners will sign \dwords{mou} that specify which detector
deliverables each %will be provided by each of the 
supporting funding %agencies.  
will provide. The \dword{mou} will also specify the required common fund
contributions from each  participating funding agency.  The
\dword{dune} resource coordinator will be responsible for managing and
reporting on all common fund contributions.
