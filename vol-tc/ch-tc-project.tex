\chapter{Reviews}
\label{vl:tc-review}

The \dword{tc} reviews all stages of detector development and works
with each consortium to arrange reviews of the design (\dword{cdrev},
\dword{pdr} and \dword{fdr}), production (\dword{prr} and
\dword{ppr}) and \dword{orr} of their system.  These reviews provide
information to the \dword{tb} to help in evaluating technical
decisions.  Review reports are tracked by \dword{tc} and provide
guidance on the key issues that require engineering oversight by the
\dword{tc} engineering team. \Dword{tc} maintains a calendar of
\dword{dune} reviews.

\Dword{tc} works with consortia leaders to review all detector
designs.  As part of the \dword{tdr} development \dwords{pdr} are
planned for each subsystem in advance of the \dword{tdr} followed by
\dwords{fdr} after the \dword{tdr}.  All major technology decisions
will be reviewed before down-select.  \Dword{tc} may form task forces
as needed to address specific issues that require more in depth
review.


Producing detector elements begins only after
successful \dwords{prr}. Regular production progress
reviews will be held once production starts. The \dwords{prr}
will typically include a review of the production of \textit{Module 0}, the
first module produced at the facility. \Dword{tc} will work with
consortia leaders on all production reviews.

\Dword{tc} coordinates technical documents for the LBNC
technical design review.

The review process is an important part of the \dword{dune} QA process
as described in Section~\ref{sec:verification}, both for
design and production.

The review process has been in place since 2016 with various reviews
of \dword{protodune} components and has continued into the first \dword{dune}
reviews in 2018. Past reviews and scheduled reviews are in the
\dword{dune} Indico at https://indico.fnal.gov/category/586.
Review reports are in DocDB-1584.

\section{Design Reviews}

The \dword{dune} design review process is described in DocDB-9664
and is consistent with the \fnal review process described in
http://eshq.fnal.gov/manuals/feshm. Design reviews for \dword{protodune} were held for each
major system. Because the schedule was extremely tight for \dword{protodune}, only a single design review
was held for each system.

The successful operation of \dword{protodune} means \dword{dune} is at
a very advanced state of design. The strategy goiong forward is to
hold \dword{cdrev} for systems with significant changes from
\dword{protodune}. These systems include the \dword{dss}, \dword{pds} and
\dword{daq}. All systems will go through \dword{pdr} to review
design changes from \dword{protodune} and \dword{fdr} after the
\dword{tdr}.

\dword{tc} has established an Engineering Safety Committee with
mechanical and electrical engineering experts from collaborating
institutions to develop processes and procedures to evaluate engineering designs using accepted international safety
standards. The current status of international code equivalencies is
discussed further in Section~\ref{sec:esh_codes}. The codes and
standards to which each system is designed will be reviewed as part of
the \dword{pdr} and \dword{fdr}.

\section{Production Reviews}

Once the designs are finished, production reviews will be held
before significant funds are authorized for large production
runs. These reviews are closely coordinated with the QA team. The
expectation is that a module 0 be produced and presented as part of the \dword{prr}.

Once production has started, \dword{tc} will schedule \dword{ppr}
as appropriate to monitor production schedule and quality.

\section{Operations Reviews}

Operation readiness reviews (
http://eshq.fnal.gov/manuals/feshm/\#series2000) are the final safety
check out before equipment can be operated.

\section{Review Tracking}

Tracking and controling review recommendations is part of the review
process. Later review committees assess recommendations from earlier reviews. \dword{tc} assures that
the consortia respond to review recommendations and 
works with the consortia to make sure the responses are appropriately documented and
implemented. Reports from \dword{dune} reviews are maintained in
DocDB-1584 along with the list of recommendations.


%%%%%%%%%%%%%%%%%%%%%%%%%%%%%%%%
\section{Lessons Learned}
\label{sec:fdsp-coord-lessons}

A detailed list of lessons learned from construction and operation of
\dword{pdsp} is in DocDB-8255. These lessons have driven planning for
\dword{dune} and have led to some design changes in
\dword{dune}. Lessons learned will continue to be updated throughout
the final design stage and into production. The methodologies are
described in Section~\ref{sec:lessons_learned}.


%%%%%%%%%%%%%%%%%%%%%%%%%%%%%%%%
\section{Reporting}
\label{sec:fdsp-coord-reporting}

The \dword{dune} project has published regular monthly reports since
the final design and construction of \dword{protodune} began in
earnest in summer 2016. \Dword{tc} will continue to compile and publish these
reports. Reporting will expand to include monthly reports against the
\dword{ims}. The \dword{dune} project provides regular reports to the
LBNC at reviews several times a year. The \dword{dune} project
produces reports from design, production and operations reviews.

