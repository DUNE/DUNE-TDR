\chapter{Environment, Safety and Health}
\label{vl:tc-ESH}


A strong \dword{esh} program is essential to the successful completion
of \dword{lbnf} which will house \dword{dune} at \surf and hosted by
\fnal.  The \dword{lbnf}/\dword{dune} is an internationally designed,
coordinated, and funded through collaborator laboratories,
universities both domestic and international.  It will comprise the
world's highest intensity neutrino beam at \fnal, and the
infrastructure necessary to support the experimental detector(s) at
\surf, in Lead, South Dakota.  The project is committed to ensuring a
safe work environment for \dword{lbnf}/\dword{dune} workers at all
institutions and to protecting the public from hazards associated with
construction and operation of \dword{lbnf}/\dword{dune}.  Accidents and
injuries are preventable, and it is important that we work together to
establish an injury free workplace.  Finally, all work will be
performed in a manner that preserves the quality of the environment
and prevents property damage.

\fnal and \dword{dune} are committed to supporting its research and
operations by protecting the health and safety of staff, the community
and the environment, as stated in the \dword{lbnf}/\dword{dune}
\dword{ieshp}. The \dword{esh} Program
is in compliance with applicable standards and Local, State and
Federal legal requirements through \fnal's Work Smart Set of Standards
and the contract between Fermi Research Alliance, LLC (FRA) and the
\dword{doe} Office of Science.  In order to implement the
ISM Plan, \fnal, as the host laboratory, established the South Dakota
Services Division (SDSD).  SDSD has the responsibly for
\dword{lbnf}/\dword{dune} operations at the \surf.

The program strives for the prevention of injury or illness and
continual improvement in safety and health management and performance.
To the maximum extent practicable, all hazards shall be eliminated,
substituted or minimized through engineering or administrative
controls.  Where engineering or administrative controls are not
feasible, personal protective equipment (PPE) shall be utilized.

The \dword{esh} Management System is
designed to work hand in hand with the Emergency Management System to
protect the public, the worker and the environment; ensure compliance
with the Contract; and to improve \fnal's and \dword{dune}'s ability to
meet or exceed customer expectations; thereby, executing the
scientific mission.  \fnal uses a set of elements to plan, direct,
control, coordinate, assure and improve how \dword{esh} policies, objectives,
processes and procedures are established, implemented, monitored and
achieved.

The \fnal facilities are further subject to the requirements of the
\dword{doe} Workers Safety and Health Program, Title 10, Code Federal
Regulations (CFR) Part 851 (10 CFR 851).  These requirements are
promulgated through the \fnal Director's Policy Manual\footnote{\fnal
  Director's Policy Manual is:
  http://www.fnal.gov/directorate/Policy\_Manual.html}, and the \fnal
\dword{esh} Manual\footnote{\fnal ES\&H Manual is:
  http://esh.fnal.gov/xms/ESHQ-Manuals/FESHM} (FESHM) which align with
the \surf \dword{esh} Manual.

\section{Hazard Analysis Report}

One of the key elements of an effective \dword{esh} program is the hazard
identification process. Hazard identification produces a list of
hazards present within a facility allowing these hazards to be
screened, and those of concern, managed through a suitable set of
controls.

The \dword{lbnf}/\dword{dune} project completed a Hazard Analysis Report (HAR) to
assure that identified hazards are mitigated early in the evolution of
the design.  The focus of the report is on process hazards rather than
activity hazards that are typically covered in a job hazard analysis.
The HAR has been completed to identify the hazards anticipated to be
encountered during the project's construction and operational phases.

The HAR then looks at the consequences of the hazard to establish a
pre-mitigation risk category. Proposed mitigations are applied to
hazards of concern to mitigate risks and then establishes a
post-mitigation risk category.

As the \dword{dune} design matures, the HAR will be updated to ensure
that all hazards have been properly identified and controlled through
design and safety management system programs.  In addition, some
sections of the HAR are used to meet the safety requirements as
defined in 10 CFR 851 and \dword{doe} Order 420.2C, Safety of
Accelerator Facilities.  Table~\ref{tab:hazards} summarizes these
hazards.  The sections following the table describe the hazards, which
are most applicable to \dword{dune} activities, in more detail and the
design and operational controls used to mitigate these hazards. The
results of these evaluations confirm that the potential risks from
construction, operations and maintenance are acceptable.

\begin{dunetable}
  [Hazard List] {|p{0.3\textwidth}|p{0.3\textwidth}|p{0.3\textwidth}|}
  {tab:hazards} {List of identified hazards}
  HA-1 (Construction) & HA-2 (Natural Phenomena) & HA-3 (Environmental)   \\ \toprowrule
  Site Clearing, Excavation, Mining, Tunneling (explosives), Vertical/Horizontal Conveyance Systems,
  Confined space, Heavy Equipment, Work at Elevations (steel, roofing), Material Handling (rigging)
  Utility interfaces, (electrical, steam, chilled water), Slips/trips/falls, Weather related conditions
  Scaffolding, Transition to Operations, Radiation Generating Devices &
  Seismic, Flooding, Wind, Lightning, Tornado &
  Construction impacts,
  Storm water discharge (construction and operations), Operations impacts, Soil and groundwater activation/contamination,
  Tritium contamination, Air activation, Cooling water activation (HVAC and Machine),
  Oils/chemical leaks or spills, Discharge/emission points (atmospheric/ground)\\ \colhline
  HA-4 (Waste) & HA-5 (Fire) & HA-6 (Electrical)   \\ \toprowrule
  Construction Phase, Facility maintenance, Experimental Operations, Industrial, Hazardous, Radiological &
  Facility Occupancy Classification, Construction Materials, Storage, Flammable/combustible liquids,
  Flammable gasses, Egress/access, Electrical, Lightning, Welding/cutting/brazing work, Smoking  &
  Facility, Experimental, Job built Equipment, Low Voltage/High Current, High Voltage/High Power,
  Maintenance, Arc flash, Electrical shock, Cable tray overloading/mixed utilities, Exposed 110V,
  Stored energy (capacitors \& inductors), Be in contactors   \\ \colhline
  HA-7 (Mechanical) & HA-8 (Cryo/ODH) & HA-9 (Confined Space)   \\ \toprowrule
  Construction Tools, Machine Shop Tools, Industrial Vehicles, Drilling, Cutting, Grinding,
  Pressure/Vacuum Vessels and Lines, High Temp Equipment (Bakeouts) &
  Thermal, Cryogenic systems, Pressure, Handling and Storage,
  Liquid argon/nitrogen spill/leak, Use of inert gases (argon, nitrogen, helium), Specialty gases &
  Sumps, Utility Chases        \\ \colhline
  HA-11 (Chemical) & HA-14 (Laser) & HA-15 (Material Handling)   \\ \toprowrule
  Toxic, Compressed gas, Combustibles, Explosives, Flammable gases, Lead (shielding), Cryogenic &
  Alignment Laser, Testing and Calibration, Magnetic Fields, Calibration \& Testing &
  Overhead cranes/hoists, Fork trucks, Manual material handling, Delivery area distribution,
  Manual movement of materials, Hoisting \& Rigging, Lead, Beryllium Windows,Oils, Solvents, Acids,
  Cryogens, Compressed Gases   \\ \colhline
  HA-16 (Experimental Ops) &  &    \\ \toprowrule
  Electrical equipment, Water Hazard, Working from heights (scaffolding/lifts), Transportation of hazardous materials,
  Liquid Argon/Nitrogen, Chemicals (Corrosive, Reactive, Flammable), Elevations, Ionizing radiation,
  Ozone production, Slips, trips, falls, Machine tools/hand tools, Stray static magnetic fields, Research gasses (Inert, Flammable) &
  &   \\ \colhline
\end{dunetable}


\subsection{Construction Hazards (\dword{lbnf}/\dword{dune} HA-1)}

The project will use the laboratories existing Work Planning and
Control process along with a Construction Project Safety and Health
Plan to communicate these policies and procedures as required by \dword{doe}
Order 413.3b. The typical installation and construction hazards
anticipated for the \dword{lbnf}/\dword{dune} project include the following:
\begin{itemize}
 \item Site Clearing
 \item Excavation
 \item Vertical/Horizontal Conveyance Systems
 \item Confined space
 \item Heavy Equipment Operations
 \item Work at Elevations (steel erection, roofing)
 \item Material Handling (rigging)
 \item Utility interfaces (electrical, chilled water, ICW, natural gas)
 \item Slips/trips/falls
 \item Weather related conditions
 \item Scaffolding
 \item Transition to Operations
 \item Radiation Generating Devices.
\end{itemize}

To reduce risks from construction hazards, \fnal will use
engineered and approved excavation and fall protection systems.  Heavy
equipment will utilize required safety controls. \fnal's
construction safety oversight program includes periodic evaluation of
the construction site and construction activities, hazard analysis for
all subcontractor activities, and frequent \dword{esh} communications at the
subcontractor's daily tool box meetings.

\subsection{Natural Phenomena (\dword{lbnf}/\dword{dune} HA-2)}

The \dword{lbnf}/\dword{dune} design will be governed by the International Building
Code (IBC), 2015 Edition and \dword{doe} Standard (STD)-1020, 2016 Edition,
Natural Phenomena Hazard Analysis and Design Criteria for \dword{doe}
Facilities, was utilize for guidance in the design for meeting the
natural phenomena hazard requirements.  The IBC specifies design
criteria for wind loading, snow loading, and seismic events.

\dword{lbnf}/\dword{dune} also was determined to be a low hazard Performance Category
1 facility as per \dword{doe} STD-1021-93. \dword{lbnf}/\dword{dune} areas will contain small
quantities of activated, radioactive, and hazardous chemical
materials. Should a Natural Phenomenon Hazard cause significant
damage, the impact will be mission related and will not pose a hazard
to the public or the environment.

\subsection{Environmental Hazards (\dword{lbnf}/\dword{dune} HA-3)}

Environmental hazards from \dword{dune} include the potential for releasing
chemicals to soil, groundwater, surface water, air, or sanitary sewer
system which if not controlled could exceed regulatory limits.

\fnal maintains an Environmental Management System equivalent to
ISO 14001 consisting of programs for protecting the environment,
assuring compliance with applicable environmental regulations and
standards and avoiding adverse environmental impacts through an effort
of continual improvement.  These programs are documented in the 8000
and 11000 series of chapters in the FESHM.  The environmental
mitigation plan will also meet federal and state regulations.


\subsection{Waste Hazards (\dword{lbnf}/\dword{dune} HA-4)}

Waste related hazards from \dword{dune} include the potential for releasing
waste materials (oils, solvents, chemicals and radioactive material)
to the environment, injury of personnel, and a possible reactive or
explosive event. Typical initiators will be transportation accidents,
incompatible materials, insufficient packaging/labeling, failure of
the packaging, and a natural phenomenon.

During the installation and operation of \dword{dune} it is anticipated that
minimal quantities of hazardous materials will be used. Such materials
include paints, epoxies, solvents, oils and lead in the form of
shielding. There are no current or anticipated activities at \dword{dune} that
would expose workers to levels of contaminants (dust, mists or fumes)
above regulatory limits.

The ESH\&Q Section Industrial Hygiene Group and Hazard Control
Technology Team provides program management and guidance to
collaborators who are subject to
waste-related hazards.  Their staff assists with identifying workplace
hazards, assists with identifying controls, and monitors
implementation. Industrial hygiene hazards will be evaluated,
identified, and mitigated as part of the work planning and control
hazard assessment process.

\subsection{Fire Hazards (\dword{lbnf}/\dword{dune} HA-5)}

The probability of a fire at \dword{lbnf}/\dword{dune} is very low,
similar to that of present neutrino detector operations for MINOS,
NoVA, and MicroBooNe at Fermilab.


Fire hazards have been evaluated and addressed in compliance with \dword{doe}
Order 420.1C, Facility Safety, Chapter II and \dword{doe}-STD 1066, Fire
Protection Design Criteria.  The intent of these documents is to meet
\dword{doe}'s Highly Protected Risk (HPR) approach to fire protection.  In
addition, the National Fire Protection Association (NFPA) Standard
520, Standard on Subterranean Spaces, was utilized to develop the
Basis for Design related to Fire Protection/Life Safety.

The combustible loads and the use of flammable and/or reactive
materials in the \dword{lbnf}/\dword{dune} facility are controlled in accordance with
the International Building Code (IBC) building occupancy
classification. Certain ancillary buildings outside the main structure
may be classified as higher hazard areas (``Use Group H'' occupancy),
including the gas cylinder and chemical storage rooms because they
hold more concentrated quantities of flammable or combustible
materials.  The ``control area'' concept used by IBC and NFPA standards,
such as 45, will be followed in hazardous chemical use and storage
areas to provide the greatest amount of flexibility and control of
materials by allowing inventory thresholds per control area.  The
\dword{lbnf}/\dword{dune} facility will be equipped with fire detection systems and
alarm systems that will monitor water flow in evident of suppression
activation and monitoring of control valves and detection systems.

Audible/visual alarm notification devices will alert building
occupants.  Manual pull stations for the fire alarms will be installed
at all building exits.  In accordance with NFPA 90A, the air handling
systems will have photoelectric smoke detectors.  Area smoke detection
will be provided in areas where there is highly sensitive electronic
equipment.  Combination audible/visual alarm notification devices will
be set up throughout the underground enclosures and service buildings
to alert occupants. All Fire Alarm signals will report through a
centralized system at \surf.  Fire alarm and supervisory signals will
be transmitted to the internal and external emergency responders via
the campus reporting system.

While designed-fixed fire protection systems afford an excellent level
of protection, additional strategies such as operational controls
including combustible materials minimization programs, adequately
fused power supplies, fire safety inspections, and Operational
Readiness Reviews will be utilized to further reduce fire hazards
within the facility in accordance with \dword{doe} HPR methodically.

Experimental cabling will meet the requirements of NFPA 70, National
Electrical Code, 2015 Edition.  Preferred cables are of the fire
resistive type, such as plenum (CMP), riser (CMR) and
general-purpose (CM, CMG, CMx) should be utilized.  When there is a
large investment in the cost of the equipment that will be in
experiment power or computer rack systems and when the equipment is
custom made (as opposed to off-the-shelf commercial electronics), a
device to detect faults or smoke in the system should be provided.
This device should also shut down the individual rack or racks when a
fault or smoke is detected.


\subsection{Electrical Hazards (\dword{lbnf}/\dword{dune} HA-6)}

\dword{lbnf}/\dword{dune} will have significant facility-related systems and
subsystems that produce or utilize high voltage, high current, or high
levels of stored energy, all of which can present electrical hazards
to personnel. Electrical hazards include electric shock and arc flash
from exposed conductors, defective and substandard equipment, lack of
training, or improper procedures.

\fnal has a well-established electrical safety program that
incorporates deenergizing equipment, isolation barriers, personal
protective equipment, and training. The cornerstone of the program is
the implementation of Lockout/Tagout (LOTO) in accordance with FESHM
Chapter 2100, \fnal Energy Control Program (Lockout/Tagout).

Design, installation, and operation of electrical equipment will be in
compliance with the National Electrical code (NFPA 70), applicable
parts of Title 29 Code of Federal Regulations, Parts 1910 and 1926,
NFPA 70E and \fnal electrical safety policies documented in the
FESHM 9000 series chapters. Equipment procured from outside vendors or
international in-kind partners will be either certified by a
nationally recognized testing laboratory (NRTL), conform to
international standards previously evaluated and deemed equivalent to
US standards, or inspected and accepted in accordance with \fnal's
electrical equipment inspection policies outlined in FESHM 9110,
Electrical Utilization Equipment Safety.


\subsection{Noise/Vibration/Thermal/Mechanical (\dword{lbnf}/\dword{dune} HA-7)}

Hazards from noise and vibration include overexposure of personnel to
American Conference of Governmental Industrial Hygienists (ACGIH) and
OSHA noise limits and permanent hearing loss, also known as Permanent
Threshold Shift (PTS). The vibration of equipment can contribute to
the noise levels, along with potential damage to or interference with
sensitive equipment.

\dword{lbnf}/\dword{dune} will incorporate a wide variety of equipment that will
produce a wide-range of noise and vibration. Support equipment, such
as pumps, motors, fans, machine shops, and general HVAC all contribute
to point source and overall ambient noise levels. While noise will
typically be below the ACGIH and OSHA 8-hour time weighted average,
certain areas with mechanical equipment could exceed that criterion
and will require periodic monitoring, posting, and the use of
PPE. Ambient background noise is a greater concern from the standpoint
of collaborator comfort, stress level and fatigue.

The detector facilities use a wide variety of noisy equipment. Pumps,
fans, and machine shop devices, among others, are possible sources of
noise levels that might exceed the \fnal noise action levels. FESHM
Chapter 4140, Hearing Conservation, contains requirements for reducing
noise and protecting personnel who may be exposed to excessive noise
levels. Warning signs are posted where hazardous noise levels may
arise, and hearing protection devices are readily
available. Methodologies to reduce noise and vibration will be
incorporated into the \dword{lbnf}/\dword{dune} design. These techniques include using
low noise/vibration producing equipment, especially for fans in the
HVAC equipment, isolating the noise producing equipment by segregating
or enclosing it, and using sound deadening materials on the walls and
ceilings of areas.

\subsection{Cryogenic/Oxygen Deficiency Hazard (\dword{lbnf}/\dword{dune} HA-8)}

The \dword{lbnf}/\dword{dune} project will use large volumes of liquid argon, nitrogen
and helium within the Far Site facilities. Cryogenic hazards will
include the potential for oxygen deficient atmospheres due to
catastrophic failure of the cryogenic systems, thermal (cold burn)
hazards from cryogenic components, and pressure hazards. Initiators
could include the failure or rupture of cryogenic systems from
overpressure, failure of insulating vacuum jackets, mechanical damage
or failure, deficient maintenance, or improper procedures.

The extreme low temperatures of cryogenic liquids and gases have a
significant adverse effect on the human body, as well as on inanimate
objects. These effects range from destroying human tissue to altering
the physical characteristics and properties of materials, such as
size, strength and flexibility of metals and other materials.

Although cryogens are used extensively at \fnal, there are strict
limitations on quantities that may be used within a facility. Uses
beyond defined limits require an Oxygen Deficiency Hazard (ODH)
analyses and the use of ventilation, oxygen deficiency monitoring, or
other controls.

Cryogenic systems are subject to the formal project review process,
which includes independent reviews by a subpanel of the Cryogenic
Safety Subcommittee in accordance with FESHM Chapter 5032, Cryogenic
System Review. The members of this panel have relevant knowledge in
applicable subject matter areas. They review the system safety
documentation, ODH analysis documentation, and the equipment before
new systems are permitted to begin the cool down process.

All piping systems and storage systems will be designed and installed
to comply with applicable FESHM 5000 series chapters, ASME and
American National Standards Institute (ANSI) standards.

\fnal has developed and successfully deployed ODH monitoring systems throughout the laboratory in support of its current cryogenic operations. The systems are designed to provide both local and remote alarms when atmospheres contain less than 19.5\% oxygen by volume.

\fnal has a mature training program to address cryogenic safety
hazards. Key program elements include ODH Training, pressurized gas
safety and general cryogenic safety.


\subsection{Confined Space Hazards (\dword{lbnf}/\dword{dune} HA-9)}

Hazards from confined spaces could result in death or injury due to
asphyxiation, compressive asphyxiation, smoke inhalation, or impact
with mechanical systems. Initiators would include failure of the
cryogenic systems releasing liquid, the release of gas, fire, or
failure of mechanical systems.

The \fnal confined space program is outlined in FESHM Chapter 4230,
Confined Spaces. \dword{lbnf}/\dword{dune} facilities will be incorporated into this
program. The emphasis at the \dword{lbnf}/\dword{dune} design phase will be to ensure
the minimum number of confined spaces are created. This is
accomplished by clear articulation of the definition of confined
spaces to facility designers to assure that adequate egress is
designed in to them, mechanical spaces are adequately sized, and,
wherever possible, a confined space not be created. During facility
operations, the existing campus confined space program, along with
appropriate labeling of confined spaces, work planning and control,
and entry permits will be used to control access to these spaces.


\subsection{Chemical/Hazardous Materials Hazards (\dword{lbnf}/\dword{dune} HA-11)}

The \dword{dune} facility anticipates a minimal use of chemical and hazardous
materials. Materials such as paints, epoxies, solvents, oils, and lead
shielding may be used during the construction and operations of the
facility. Exposure to these materials could result in injury, or
exposures that exceed regulatory limits. Initiators could be
experimental operations, transfer of material, failure of packaging,
improper marking/labeling, reactive or explosive event, improper
selection of or lack of, personal protective equipment (PPE), or a
natural phenomenon.

\fnal maintains an database of hazardous chemicals in compliance
with the requirements imposed by 10 CFR 851 and \dword{doe} Orders. In
addition to the inventory of chemicals at the facility, copies of the
respective manufacturer's Safety Data Sheets (SDSs) are
maintained. Reviews of the conventional safety aspects of the
facilities show that use of these chemicals does not warrant special
controls other than appropriate signs, procedures, appropriate use of
personal protective equipment, and hazard communication training. \dword{dune}
will also supply SDS documentation to the \surf \dword{esh} Department for all
chemicals and hazardous materials which arrive on site.

The industrial hygiene program, detailed in the FESHM 4000 series
chapters, addresses potential hazards to workers using such
materials. The program identifies how to evaluate workplace hazards
when planning work and the controls necessary to eliminate or mitigate
these hazards to an acceptable level.

Specific procedures are also in place for the safe handling, storing,
transporting, inspecting and disposing of hazardous materials. These
are contained in the FESHM 8000 and 10000 series chapters,
Environmental Protection and Material Handling and Transportation,
which describes the standards necessary to comply with the Code of
Federal Regulations, Occupational Safety and Health Standards, Hazard
Communication, Title 29 CFR, Part 1910.1200.


\subsection{Lasers \& Other Non-Ionizing Radiation Hazards (\dword{lbnf}/\dword{dune} HA-14)}

Production and delivery of Class 3B and Class 4, near-infrared, UV,
and visible lasers are required to be completely contained to
transport pipes or designated enclosures for the Class 3b and Class 4
lasers, thus creating a Laser Controlled Area (LCA). (This will be in
accordance with \fnal FESHM chapter 4260.)  Establishing the LCA
prevents areas surrounding the LCA from exceeding the Maximum
Permissible Exposure (MPE) as set by the \fnal Laser Safety Officer
(LSO).

\subsection{Material Handling Hazards (\dword{lbnf}/\dword{dune} HA-15)}

\dword{dune} will require a significant amount of manual and mechanical
material handling during the construction, installation and operations
phases.  The consequences of these hazards include serious injury or
death to equipment operators and bystanders, damage to equipment and
structures, and interruption of the program.  Additional material
handling hazards from forklift and tow cart operations include injury
to the operator or personnel in the area and contact with equipment or
structures. Cranes and hoists will be used during fabrication,
testing, removal, and installation of equipment. The error precursors
associated with this type of work include irregular shaped loads,
awkward load attachments, limited space, obscured sight lines, and
poor communication.  The material or equipment being moved is
typically one of a kind, potentially of high dollar or programmatic
value, and may not have dedicated lifting points or an obvious center
of gravity.

Lessons learned from across the \dword{doe} Complex and OSHA have been
evaluated and incorporated into the \fnal material handling
programs documented in the FESHM 10000 series chapters.  The
laboratory limits personnel who have access to mechanical material
handling equipment such as cranes and forklifts to those who have
successfully completed the laboratory's training programs and
demonstrated competence in operating this equipment.


\subsection{Experimental Operations (\dword{lbnf}/\dword{dune} HA-16)}

Experimental activity undertaken at \dword{lbnf}/\dword{dune} will be fully reviewed
under the Operational Readiness Clearance (ORC) process and other
Subject Matter Experts as needed (e.g., representatives from
Electrical Safety, Fire Safety, Environmental Compliance, Industrial
Hygiene, Cryogenic Safety, and Industrial Safety), to identify and
manage the hazards of each experimental operation. The shift leader
will ensure that all safety reviews take place for each
activity and that any issues are appropriately addressed. The ORC
process will document these reviews, covering the necessary controls
and management approval to proceed.

Typically, the ORC process evaluates the scope of the proposed
experimental activity, identifies the hazards and the controls to
mitigate them. It assures that collaborators are properly trained, and
qualified, hazardous material is kept to a minimum, engineering
controls are deployed as a preferred mitigation, and the personnel
protective equipment used is appropriate for the hazard.

\section{Work Planning and Controls}

The goal of the work planning and hazard analysis (HA) process is to
generate awareness about the hazards associated with work activities
and how they can be performed safely. Careful planning of a job helps
to assure that it is performed efficiently and safely. Work planning
ensures that the scope of the job is understood, appropriate materials
and tools are available, all hazards have been identified, mitigation
efforts established and all affected employees understand what is
expected of them. Hazard analysis is a critical part of work planning.
The Work Planning and Hazard Analysis program follows guidelines from
Chapters 2060 in FESHM.

An Installation \dword{esh} Plan will be developed
which will define a site specific set of \dword{esh} requirements and
responsibilities which personnel will be required to follow to perform
installation/construction activities at \surf.

\section{NEPA Compliance}

In compliance with the National Environmental Protection Act (NEPA)
and in accordance with \dword{doe} Policy 451.1, the
\dword{lbnf}/\dword{dune} project performed an evaluation of potential
environmental impacts during construction and operation of the
project.  An Environmental Assessment (EA) has been prepared to
evaluate the potential environmental impacts and the safety and health
hazards identified during the design, construction, and operating
phases of \dword{lbnf}/\dword{dune}.  The EA analyzed the potential
environmental consequences of the facility and compared them to the
consequences of a ``No Action Alternative''. The assessment included
detailed analysis of all potential environmental, safety, and health
hazards associated with construction and operation of the facility.
The Environmental Assessmnet (EA) has been completed and a FONSI
issued in September 2015.

\section{Codes/Standards Equivalencies}
\label{sec:esh_codes}

\dword{dune} will rely on significant contributions from International
Partners. In many cases, an International Partner will contribute
equipment for installation at \fnal that is built per one of the
International Standards or Directives. \fnal has established a
process, detailed in FESHM Chapter 2110, to establish code equivalency
between U.S. and International engineering design codes and
standards. This process enables the Laboratory to accept in kind
contributions from International partners or purchase equipment
designed per International standards while assuring an equivalent or
greater level of safety.

At the time of this writing, \fnal has completed the following code
equivalencies.
\begin{itemize}
 \item Pressure vessels designed per EN13445
 \item Structures designed per EN 1990, EN 1991, EN1993, EN 1999 (a
   subset of the ``Eurocodes''), and EN 14620.
 \item CE-marked pressure piping systems designed per PED 97/23 EN 13480.
 \item CE-marked relief valves designed to PED 2014/68/EU EN ISO 4126.
 \item CE-marked Electrical Equipment for Measurement, Control and
   Laboratory use designed per IEC 61010-1 and IEC 61010-2-030.
\end{itemize}

As necessary, the laboratory code equivalency process will be followed
to establish equivalency to other international codes and
standards. The current list of completed code equivalencies can be
found in the ESH\&Q Section Document Database doc-3303
(https://esh-docdbcert.fnal.gov/cgi-bin/cert/ShowDocument?docid=3303).

\section{\dword{esh} Requirements at Collaborating Laboratories and Institutions}

All work performed at collaborating institutions will be completed in
accordance with the collaborating institutions \dword{esh}
policies and program. Equipment and operating procedures
provided by the collaborating institution will conform to the \dword{dune}
project \dword{esh} and Integrated Safety Management policies and
procedures. The collaborating institution's \dword{esh} Department shall be
responsibility for providing ES\&H oversight for all work activities
carried out in collaborating institution facilities. \dword{lbnf}/\dword{dune}
personnel will also follow the \dword{esh} Manual and procedures of the
collaborative institutions.
