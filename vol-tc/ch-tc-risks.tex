%%%%%%%%%%%%%%%%%%%%%%%%%%%%%%%
\subsection{Single-phase}
\label{sec:tc-risks-sp}


% risk table values for subsystem SP-FD-APA
\begin{longtable}{p{0.18\textwidth}p{0.20\textwidth}p{0.32\textwidth}p{0.02\textwidth}p{0.02\textwidth}p{0.02\textwidth}} 
\caption{Risks for SP-FD-APA \fixmehl{ref \texttt{tab:risks:SP-FD-APA}}} \\
\rowcolor{dunesky}
ID & Risk & Mitigation & P & C & S  \\  \colhline
RT-SP-APA-01 & Loss of key personnel & Implement succession planning and formal project documentation & L & L & M \\  \colhline
RT-SP-APA-02 & Delay in finalisation of APA frame design & Close oversight on prototypes and interface issues & L & L & M \\  \colhline
RT-SP-APA-03 & One additional pre-production APA may be necessary & Close oversight on approval of designs, commissioning of tooling and assembly procedures & L & L & L \\  \colhline
RT-SP-APA-04 & APA winder construction takes longer than planned & Detailed plan to stand up new winding machines at each facility & M & L & M \\  \colhline
RT-SP-APA-05 & Poor quality of APA frames and/or inaccuracy in the machining of holes and slots & Clearly specified requirements and seek out backup vendors & L & L & M \\  \colhline
RT-SP-APA-06 & Insufficient scientific manpower at APA assembly factories & Get institutional commitments for requests of necessary personnel in research grants & M & M & L \\  \colhline
RT-SP-APA-07 & APA production quality does not meet requirements & Close oversight on assembly procedures & L & M & M \\  \colhline
RT-SP-APA-08 & Materials shortage at factory & Develop and execute a supply chain management plan & M & L & L \\  \colhline
RT-SP-APA-09 & Failure of a winding machine - Drive chain parts failure & Regular maintenance and availability of spare parts & L & L & L \\  \colhline
RT-SP-APA-10 & APA assembly takes longer time than planned  & Estimates based on protoDUNE. Formal training of every tech/operator & L & M & M \\  \colhline
RT-SP-APA-11 & Loss of one APA due to an accident & Define handling procedures supported by engineering notes & M & L & L \\  \colhline
RT-SP-APA-12 & APA transport box inadequate & Construction and test of prototype transport boxes & L & L & M \\  \colhline
RO-SP-APA-01 & Reduction of the APA assembly time & Improvements in the winding head and wire tension mesurements & M & M & M \\  \colhline
 &  &  &  &  &  \\  \colhline

\label{tab:risks:SP-FD-APA}
\end{longtable}

% risk table values for subsystem SP-FD-HV
\begin{footnotesize}
%\begin{longtable}{p{0.18\textwidth}p{0.20\textwidth}p{0.32\textwidth}p{0.02\textwidth}p{0.02\textwidth}p{0.02\textwidth}}
\begin{longtable}{x{0.18\textwidth}x{0.20\textwidth}x{0.32\textwidth}x{0.02\textwidth}x{0.02\textwidth}x{0.02\textwidth}} 
\caption[Risks for SP-FD-HV]{Risks for SP-FD-HV (P=probability, C=cost, S=schedule) More information at \dword{riskprob}. \fixmehl{ref \texttt{tab:risks:SP-FD-HV}}} \\
\rowcolor{dunesky}
ID & Risk & Mitigation & P & C & S  \\  \colhline
RT-SP-HV-01 & Open circuit on the field cage divider chain & Component selection and cold tests. Varistor protection. & L & L & L \\  \colhline
RT-SP-HV-02 & Damage to the resistive Kapton film on CPA & Careful visual inspection of panel surfaces.  Replace panel if scratches are deep and long  & L & L & L \\  \colhline
RT-SP-HV-03 & Sole source for Kapton resistive surface; and may go out of production & Another potential source of resistive Kapton identified. Possible early purchase if single source. & M & L & L \\  \colhline
RT-SP-HV-04 & Detector components are damaged during shipment to the far site  & Spare parts at  LW. FC/CPA modules can be swapped and replaced from factories in a few days. & L & L & L \\  \colhline
RT-SP-HV-05 & Damages (scratches, bending) to aluminum profiles of Field Cage modules & Require sufficent spare profiles for substitution. Alternate: local coating with epoxy resin. & L & L & L \\  \colhline
RT-SP-HV-06 & Electric field uniformity is not adequate for muon momentum reconstruction  & Redundant components; rigorous screening. Structure based on CFD. Calibration can map E-field. & L & L & L \\  \colhline
RT-SP-HV-07 & Electric field is below goal during stable operations & Improve the protoDUNE SP HVS design to reduce surface E-field and eliminate exterior insulators. & M & L & L \\  \colhline
RT-SP-HV-08 & Damage to CE in event of discharge  & HVS was designed to reduce discharge to a safe level. Higher resistivity cathode could optimize. & L & L & L \\  \colhline
RT-SP-HV-09 & Free hanging frames can swing in the fluid flow  & Designed for flow using fluid model; Deformation can be calibrated by lasers or cosmic rays. & L & L & L \\  \colhline
RT-SP-HV-10 & FRP/ Polyethene/ laminated Kapton component lifetime is less than expected & Positive experience in other detectors. Gain experience with LAr TPC's; exchangeable feedthrough. & L & L & L \\  \colhline
RT-SP-HV-11 & International funding level for SP HVS too low & Cost reduction through design optimization. Effort to increase international collaboration. & M & M & M \\  \colhline
RT-SP-HV-12 & Underground installation is more labor intensive or slower than expected & SWF contingency, full-scale trial before installation. Estimates based on ProtoDUNE experience. & L & L & L \\  \colhline

\label{tab:risks:SP-FD-HV}
\end{longtable}
\end{footnotesize}

% risk table values for subsystem SP-FD-TPCELEC
\begin{footnotesize}
%\begin{longtable}{p{0.18\textwidth}p{0.20\textwidth}p{0.32\textwidth}p{0.02\textwidth}p{0.02\textwidth}p{0.02\textwidth}}
\begin{longtable}{P{0.18\textwidth}P{0.20\textwidth}P{0.32\textwidth}P{0.02\textwidth}P{0.02\textwidth}P{0.02\textwidth}} 
\caption[Risks for SP-FD-TPCELEC]{Risks for SP-FD-TPCELEC (P=probability, C=cost, S=schedule) More information at \dword{riskprob}. \fixmehl{ref \texttt{tab:risks:SP-FD-TPCELEC}}} \\
\rowcolor{dunesky}
ID & Risk & Mitigation & P & C & S  \\  \colhline
RT-SP-TPC-001 & Cold ASIC(s) not meeting specifications & Multiple designs, use of appropriate design rules for operation in LAr & H  & M & L \\  \colhline
RT-SP-TPC-002 & Delay in the availability of ASICs and FEMBs & Increase pool of spares for long lead items, multiple QC sites for ASICs, appropriate measures against ESD, monitoring of yields & M & L & L \\  \colhline
RT-SP-TPC-003 & Damage to the FEMBs / cold cables during or after integration with the APAs & Redesign of the FEMB/cable connection, use of CE boxes, ESD protections, early integration tests & M & L & L \\  \colhline
RT-SP-TPC-004 & Cold cables cannot be run through the APAs frames & Redesign of APA frames, integration tests at Ash River and at CERN, further reduction of cable plant & L & L & L \\  \colhline
RT-SP-TPC-005 & Delay and/or damage to the TPC electronics components on the top of the cryostat & Sufficient spares, early production and installation, ESD protection measures & L & L & L \\  \colhline
RT-SP-TPC-006 & Interfaces between TPC electronics and other consortia not adequately defined & Early integration tests, second run of ProtoDUNE-SP with pre-production components & M & L & L \\  \colhline
RT-SP-TPC-007 & Insufficient number of spares & Early start of production, close monitoring of usage of components, larger stocks of components with long lead times & M & L & L \\  \colhline
RT-SP-TPC-008 & Loss of key personnel & Distributed development of ASICs, increase involved of university groups, training of younger personnel & H & L & M \\  \colhline
RT-SP-TPC-009 & Excessive noise observed during detector commissioning & Enforce grounding rules, early integration tests, second run of ProtoDUNE-SP with pre-production components, cold box testing at SURF & L & L & M \\  \colhline
RT-SP-TPC-010 & Lifetime of components in the LAr & Design rules for cryogenic operation of ASICs, measurement of lifetime of components, reliability studies & L & n/a & n/a \\  \colhline
RT-SP-TPC-011 & Lifetime of components on the top of the cryostat & Use of filters on power supplies, stockpiling of components that may become obsolete, design rules to minimize parts that need to be redesigned / refabricated & L & M & L \\  \colhline

\label{tab:risks:SP-FD-TPCELEC}
\end{longtable}
\end{footnotesize} % this is tpc elec

% risk table values for subsystem SP-FD-PD
\begin{footnotesize}
%\begin{longtable}{p{0.18\textwidth}p{0.20\textwidth}p{0.32\textwidth}p{0.02\textwidth}p{0.02\textwidth}p{0.02\textwidth}}
\begin{longtable}{P{0.18\textwidth}P{0.20\textwidth}P{0.32\textwidth}P{0.02\textwidth}P{0.02\textwidth}P{0.02\textwidth}} 
\caption[Risks for SP-FD-PD]{Risks for SP-FD-PD (P=probability, C=cost, S=schedule) More information at \dword{riskprob}. \fixmehl{ref \texttt{tab:risks:SP-FD-PD}}} \\
\rowcolor{dunesky}
ID & Risk & Mitigation & P & C & S  \\  \colhline
RT-SP-PD -01 & Additional photosensors and engineering required to ensure PD modules collect enough light to meet system physics performance specifications. & Extensive validation of \dword{xarapu} design to demonstrate they meet specification. & L & M & L \\  \colhline
RT-SP-PD-02 & Improvements to active ganging/front end electronics required to meet the specified 1~$\mu$s time resolution. & Extensive validation of photosensor ganging/front end electronics design to demonstrate they meet specification. & L & L & L \\  \colhline
RT-SP-PD-03 & Evolutions in the design of the photon detectors due to validation testing experience require modifications of the TPC elements at a late time. & Extensive validation of \dword{xarapu} design to demonstrate they meet specification and control of PD/APA interface. & L & L & L \\  \colhline
RT-SP-PD-04 & Cabling for PD and CE within the \dword{apa} frame or during the 2-APA assembly/installation procedure require additional engineering/development/testing. & Validation of PD/APA/CE cable routing in prototypes at Ash River. & L & L & L \\  \colhline
RT-SP-PD-05 & Experience with validation prototypes shows that the mechanical design of the PD is not adequate to meet system specifications. & Early validation of \dword{xarapu} prototypes and system interfaces to catch problems ASAP. & L & L & L \\  \colhline
RT-SP-PD-06 & pTB WLS filter coating not sufficiently stable, contaminates \dword{lar}. & Mechanical acceleration of coating wear.  Long-term tests of coating stability. & L & L & L \\  \colhline
RT-SP-PD-07 & Photosensors fail due to multiple cold cycles or extended cryogen exposure. & Execute testing program for cryogenic operation of photosensors including mutiple cryogenic immersion cycles. & L & L & L \\  \colhline
RT-SP-PD-08 & SiPM active ganging cold amplifiers fail or degrade detector performance. & Validation testing if photosensor ganging in multiple test beds. & L & L & L \\  \colhline
RT-SP-PD-09 & Previously undetected electro-mechanical interference discovered during integration. & Validation of electromechanical designin Ash River tests and at \dword{pdsp2}. & L & L & L \\  \colhline
RT-SP-PD-10 & Design weaknesses manifest during module logistics-handling. & Validation of shipping packaging and handling prior to shipping.  Inspection of modules shipped to site immediately upon receipt. & L & L & L \\  \colhline
RT-SP-PD-11 & PD/CE signal crosstalk. & Validation in \dword{pdsp}, \dword{iceberg} and \dword{pdsp2}. & L & L & L \\  \colhline
RT-SP-PD-12 & Lifetime of \dword{pd} components outside cryostat. & Specification of environmental controls to mitigate detector aging. & L & L & L \\  \colhline

\label{tab:risks:SP-FD-PD}
\end{longtable}
\end{footnotesize}

% risk table values for subsystem SP-FD-CAL
\begin{longtable}{p{0.18\textwidth}p{0.20\textwidth}p{0.32\textwidth}p{0.02\textwidth}p{0.02\textwidth}p{0.02\textwidth}} 
\caption{Risks for SP-FD-CAL \fixmehl{ref \texttt{tab:risks:SP-FD-CAL}}} \\
\rowcolor{dunesky}
ID & Risk & Mitigation & P & C & S  \\  \colhline
RT-SP-CAL-01 & Inadequate baseline design & Early detection allows R\&D of alternative designs accommodated through multipurpose ports & L & M & M \\  \colhline
RT-SP-CAL-02 & Inadequate engineering or production quality & Dedicated small scale tests and full prototyping at ProtoDUNE; pre-installation QC & L & M & M \\  \colhline
RT-SP-CAL-03 & Laser impact on PDS & Mirror movement control to avoid direct hits; turn laser off in case of PDS saturation & L & L & L \\  \colhline
RT-SP-CAL-04 & Laser positioning system stops working & QC at installation time, redundancy in available targets, including passive, alternative methods & L & L & L \\  \colhline
RT-SP-CAL-05 & Laser beam misaligned & Additional (visible) laser for alignment purposes & M & L & L \\  \colhline
RT-SP-CAL-06 & The neutron anti-resonance is much less pronounced & Dedicated measurements at LANL and test at ProtoDUNE & L & L & L \\  \colhline
RT-SP-CAL-07 & Neutron activation of the moderator and cryostat & Neutron activation studies and simulations & L & L & L \\  \colhline
RT-SP-CAL-08 & Neutron yield not high enough & Simulations and tests at ProtoDUNE & L & M & M \\  \colhline
RT-SP-CAL-09 & Neutrons do not reach detector center & Alternative, movable design and simulations & L & L & L \\  \colhline

\label{tab:risks:SP-FD-CAL}
\end{longtable}

% risk table values for subsystem SP-FD-DAQ
\begin{longtable}{p{0.18\textwidth}p{0.20\textwidth}p{0.32\textwidth}p{0.02\textwidth}p{0.02\textwidth}p{0.02\textwidth}} 
\caption{Risks for SP-FD-DAQ \fixmehl{ref \texttt{tab:risks:SP-FD-DAQ}}} \\
\rowcolor{dunesky}
ID & Risk & Mitigation & P & C & S  \\  \colhline
RT-SP-DAQ-01 & Detector noise specs not met & ProtoDUNE experience with noise levels and provisions for data processing redundancy in DAQ system (upstream DAQ FPGA/CPU and high level filter) & H & M & H  \\  \colhline
RT-SP-DAQ-02 & Externally-driven schedule change & Provisions for standalone testing and commissioning of production DAQ components, and schedule adjustment & H &  & M \\  \colhline
RT-SP-DAQ-03 & Lack of expert personnel & Resource-loaded plan for DAQ backed by institutional commitments, and schedule adjustment using float & H &  &  \\  \colhline
RT-SP-DAQ-04 & Power/space requirements exceed CUC capacity & Sufficient bandwidth to surface and move module 3/4 components to an expanded surface facility & H & M & H \\  \colhline
RT-SP-DAQ-05 & Excess fake trigger rate from instrumental effects & ProtoDUNE performance experience, and provisions for increase in event builder, high level filter, and upstream DAQ processing capacity, as needed & M & M & H  \\  \colhline
RT-SP-DAQ-06 & Calibration requirements exceed acceptable data rate & Provisions for increase in event builder and high level filter capacity, as neeed & M & L & M \\  \colhline
RT-SP-DAQ-07 & Cost/performance of hardware/computing excessive & Have prototyping and pre-production phases, reduce performance using margin or identify additional funds & M &  &  \\  \colhline
RT-SP-DAQ-08 & Optical components obsolete before production & Market survey, commercial standards, and design changes & H  &  &  \\  \colhline
RT-SP-DAQ-09 & Insufficient FPGA processing resources & ProtoDUNE and simulation experience, FPGA replacement, or sustain increased data rates & H  & H  & H  \\  \colhline
RT-SP-DAQ-10 & Insufficient throughput in computing system & Design allows for expansion; expand system capacity for throughput before commissioning & H  & M & H  \\  \colhline
RT-SP-DAQ-11 & Event builder throughput insufficient & Design allows for expansion; expand system capacity for throughput before commissioning & H  & M & M \\  \colhline
RT-SP-DAQ-12 & Additional data reduction steps required after event building & Provisions for increase in high level filter capacity, as needed & H  & L & M \\  \colhline
RT-SP-DAQ-13 & PDTS fails to scale for DUNE requirements & Hardware upgrade & M & L & H  \\  \colhline
RT-SP-DAQ-14 & Full remote operation of DAQ proves non-viable & Pre-production operation testing and allocation of surface or underground space for on-site operations & M &  & M \\  \colhline
RT-SP-DAQ-15 & DAQ system does not meet overall DUNE uptime specification & Extensive QA and exercise of failure mode recovery in extended ProtoDUNE runs, and replacement of components & H  &  & M \\  \colhline
RT-SP-DAQ-16 & External services do not allow reliable operation of DAQ system & Performance specifications for external services, prior to construction. & H  & L & H  \\  \colhline

\label{tab:risks:SP-FD-DAQ}
\end{longtable} 

% risk table values for subsystem SP-FD-CISC
\begin{longtable}{p{0.18\textwidth}p{0.20\textwidth}p{0.32\textwidth}p{0.02\textwidth}p{0.02\textwidth}p{0.02\textwidth}} 
\caption{Risks for SP-FD-CISC \fixmehl{ref \texttt{tab:risks:SP-FD-CISC}}} \\
\rowcolor{dunesky}
ID & Risk & Mitigation & P & C & S  \\  \colhline
RT-SP-CISC-001 & Baseline design from ProtoDUNEs for an instrumentation device is not adequate for DUNE far detectors & Focus on early problem discovery in ProtoDUNE so any needed redesigns can start as soon as possible. & L & M & L \\  \colhline
RT-SP-CISC-002 & Swinging of long instrumentation devices (T-gradient monitors or PrM system) & Add additional intermediate constraints to prevent swinging. & L & L & L \\  \colhline
RT-SP-CISC-003 & High E-fields near instrumentation devices cause dielectric breakdowns in \dword{lar} & CISC systems placed as far from cathode and FC as possible. & L & L & L \\  \colhline
RT-SP-CISC-004 & Light pollution from purity monitors and camera light emitting system & Use PrM lamp and camera lights outside PDS trigger window; cover PrM cathode to reduce light leakage. & L & L & L \\  \colhline
RT-SP-CISC-005 & Temperature sensors can induce noise in cold electronics & Check for noise before filling and remediate, repeat after filling. Filter or ground noisy sensors. & L & L  & L \\  \colhline
RT-SP-CISC-006 & Disagreement between lab and \em{in situ} calibrations for ProtoDUNE-SP dynamic T-gradient monitor & Investigate and improve both methods, particularly laboratory calibration. & M & L & L \\  \colhline
RT-SP-CISC-007 & Purity monitor electronics induce noise in TPC and PDS electronics. & Operate lamp outside TPC+PDS trigger window. Surround and ground light source with Faraday cage. & L & L & L \\  \colhline
RT-SP-CISC-008 & Discrepancies between measured temperature map and CFD simulations in ProtoDUNE-SP & Improve simulations with additional measurements inputs; use fraction of sensors to predict others   & L & L & L \\  \colhline
RT-SP-CISC-009 & Difficulty correlating purity and temperature in ProtoDUNE-SP impairs understanding cryo system. & Identify causes of discrepancy, modify design. Calibrate PrM differences, correlate with RTDs. & L & L & L \\  \colhline
RT-SP-CISC-010 & Cold camera R\&D fails to produce prototype meeting specifications \& safety requirements & Improve insulation and heaters. Use cameras in ullage or inspection cameras instead. & M & M & L \\  \colhline
RT-SP-CISC-011 & HV discharge caused by inspection cameras & Study E-field in and on housing and anchoring system. Test in HV facility. & L & L & L \\  \colhline
RT-SP-CISC-012 & HV discharge destroying the cameras & Ensure sufficient redundancy of cold cameras. Warm cameras are replaceable. & L & M & L \\  \colhline
RT-SP-CISC-013 & Insufficient light for cameras to acquire useful images & Test cameras with illumination similar to actual detector. & L & L & L \\  \colhline
RT-SP-CISC-014 & Cameras may induce noise in cold electronics & Continued R\&D work with grounding and shielding in realistic conditions. & L & L & L \\  \colhline
RT-SP-CISC-015 & Light attenuation in long optic fibers for purity monitors  & Test the max.\ length of usable fiber, optimize the depth of bottom PrM, number of fibers. & L & L & L \\  \colhline
RT-SP-CISC-016 & Longevity of purity monitors & Optimize PrM operation to avoid long running in low purity. Technique to protect/recover cathode. & L & L & L \\  \colhline
RT-SP-CISC-017 & Longevity: Gas analyzers and level meters may fail. & Plan for future replacement in case of failure or loss of sensitivity.  & M & M & L \\  \colhline
RT-SP-CISC-018 & Problems in interfacing  hardware devices (e.g. power supplies) with slow controls & Involve slow control experts in choice of hardware needing control/monitoring.
 & L & L & L \\  \colhline

\label{tab:risks:SP-FD-CISC}
\end{longtable} 

% risk table values for subsystem SP-FD-INST
\begin{footnotesize}
%\begin{longtable}{p{0.18\textwidth}p{0.20\textwidth}p{0.32\textwidth}p{0.02\textwidth}p{0.02\textwidth}p{0.02\textwidth}}
\begin{longtable}{P{0.18\textwidth}P{0.20\textwidth}P{0.32\textwidth}P{0.02\textwidth}P{0.02\textwidth}P{0.02\textwidth}} 
\caption[Risks for SP-FD-INST]{Risks for SP-FD-INST (P=probability, C=cost, S=schedule) More information at \dshort{riskprob}. \fixmehl{ref \texttt{tab:risks:SP-FD-INST}}} \\
\rowcolor{dunesky}
ID & Risk & Mitigation & P & C & S  \\  \colhline
RT-INST-01 & Personnel injury & Follow established safety plans. & M & L & H \\  \colhline
RT-INST-02 & Shipping delays & Plan one month buffer to store  materials locally. Provide logistics manual. & H & L & L \\  \colhline
RT-INST-03 & Missing components cause delays & Use detailed inventory system to verify availability of  necessary components.  & H & L & L \\  \colhline
RT-INST-04 & Import, export, visa issues  & Dedicated \dshort{fnal} \dshort{sdsd}division will expedite import/export and visa-related issues. & H & M & M \\  \colhline
RT-INST-05 & Lack of available labor  & Hire early and use Ash River setup to train \dshort{jpo} crew. & L & L & L \\  \colhline
RT-INST-06 & Parts do not fit together & Generate \threed model, create interface drawings, and prototype detector assembly. & H & L & L \\  \colhline
RT-INST-07 & Cryostat damage & Use cryostat false floor and temporary protection. & L & L & M \\  \colhline
RT-INST-08 & Weather closes SURF & Plan for \dshort{surf} weather closures & H & L & L \\  \colhline
RT-INST-09 & Detector failure during \cooldown & Cold test individual components then cold test \dshort{apa} assemblies immediately before installation. & L & H & H \\  \colhline

\label{tab:risks:SP-FD-INST}
\end{longtable}
\end{footnotesize}
 
%%
% risk table values for subsystem SP-FD-JPO
\begin{longtable}{p{0.15\textwidth}p{0.13\textwidth}p{0.13\textwidth}p{0.28\textwidth}p{0.06\textwidth}p{0.06\textwidth}p{0.06\textwidth}} 
\caption{Specification for SP-FD-JPO \fixmehl{ref \texttt{tab:specs:SP-FD-JPO}}} \\
\rowcolor{dunesky}
ID & Risk & Label & Mitigation & Prob ability & Cost Impact & Sched ule Impact \\  \colhline
RT-JPO-001 & Personnel injury & jpo-person-injury & Follow established safety plans. & M & L & H \\  \colhline
RT-JPO-002 & Shipping delays & jpo-shipping-delay & Plan one month buffer to store  materials locally. Provide logistics manual. & H & L & L \\  \colhline
RT-JPO-003 & Missing components cause delays & jpo-missing-components & Use detailed inventory system to verify availability of  necessary components.  & H & L & L \\  \colhline
RT-JPO-004 & Import, export, visa issues  & jpo-import-visa & Dedicated \dword{fnal} \dword{sdsd}division will expedite import/export and visa-related issues. & H & M & M \\  \colhline
RT-JPO-005 & Lack of available labor  & jpo-labor-avail & Hire early and use Ash River setup to train \dword{jpo} crew. & L & L & L \\  \colhline
RT-JPO-006 & Parts do not fit together & jpo-cannot-assemble & Generate \threed model, create interface drawings, and prototype detector assembly. & H & L & L \\  \colhline
RT-JPO-007 & Cryostat damage & jpo-cryostat-damage & Use cryostat false floor and temporary protection. & L & L & M \\  \colhline
RT-JPO-008 & Weather closes SURF & jpo-weather-delay & Plan for \dword{surf} weather closures & H & L & L \\  \colhline
RT-JPO-009 & Detector failure during \cooldown & jpo-cooldown-failure & Cold test individual components then cold test \dword{apa} assemblies immediately before installation. & L & H & H \\  \colhline

\label{tab:risks:SP-FD-JPO}
\end{longtable} % aka tc or iic

%%%%%%%%%%%%%%%%%%%%%%%%%%%%%%%
\subsection{Dual-phase}
\label{sec:tc-risks-dp}

For each risk, the risk probability, after taking into account the planned mitigation activities, is ranked as 
L (low $<\,$\SI{10}{\%}), 
M (medium \SIrange{10}{25}{\%}), or 
H (high $>\,$\SI{25}{\%}). 
The cost and schedule impacts are ranked as 
L (cost increase $<\,$\SI{5}{\%}, schedule delay $<\,$\num{2} months), 
M (\SIrange{5}{25}{\%} and 2--6 months, respectively) and 
H ($>\,$\SI{20}{\%} and $>\,$2 months, respectively).


% risk table values for subsystem DP-FD-CRP
\begin{footnotesize}
%\begin{longtable}{p{0.18\textwidth}p{0.20\textwidth}p{0.32\textwidth}p{0.02\textwidth}p{0.02\textwidth}p{0.02\textwidth}}
\begin{longtable}{x{0.18\textwidth}x{0.20\textwidth}x{0.32\textwidth}x{0.02\textwidth}x{0.02\textwidth}x{0.02\textwidth}} 
\caption[Risks for DP-FD-CRP]{Risks for DP-FD-CRP (P=probability, C=cost, S=schedule) More information at \dword{riskprob}. \fixmehl{ref \texttt{tab:risks:DP-FD-CRP}}} \\
\rowcolor{dunesky}
ID & Risk & Mitigation & P & C & S  \\  \colhline
RT-DP-CRP-01 & Poor quality of G10  frames and/or inaccuracy in the hole machining & Clearly specified requirements, followup and seek out backup vendors & L & L  & M \\  \colhline
RT-DP-CRP-02 & LEM production takes longer than expected & Define a production schedule allowing enough contingencies to limit the assembly impact   & L & L  & M \\  \colhline
RT-DP-CRP-03 & One of the CRP assembly site not ready on time & Close oversight on construction of tooling and preparation of assembly sites & L & M & H  \\  \colhline
RT-DP-CRP-04 & Materials shortage at production site & Develop and execute a supply chain management & M & L  & L \\  \colhline
RT-DP-CRP-05 & Failure of extraction grid winding machine & Regular maintenance and availability of spare parts & L & L  & L \\  \colhline
RT-DP-CRP-06 & CRP assembly takes longer time than planned & Estimates based on ProtoDUNE-DP. Formal training of every tech/operator at each site. & L & M & M \\  \colhline

\label{tab:risks:DP-FD-CRP}
\end{longtable}
\end{footnotesize}

% risk table values for subsystem DP-FD-HV
\begin{longtable}{p{0.18\textwidth}p{0.20\textwidth}p{0.32\textwidth}p{0.02\textwidth}p{0.02\textwidth}p{0.02\textwidth}} 
\caption{Risks for DP-FD-HV \fixmehl{ref \texttt{tab:risks:DP-FD-HV}}} \\
\rowcolor{dunesky}
ID & Risk & Mitigation & P & C & S  \\  \colhline
RT-DP-HV-01 & Broken resistors or varistors on voltage divider boards & Redundancy of resistors, varistors, and \dwords{hvdb}.  & L & L & L \\  \colhline
RT-DP-HV-02 & \efield uniformity is not adequate for muon momentum reconstruction & Regularly map out field using a laser calibration sysem. & L & L & L \\  \colhline
RT-DP-HV-03 & \efield is below specification during stable operations & Improve purity by more aggressive filtering. & M & M & L \\  \colhline
RT-DP-HV-04 & Space charge from positive ions distorting the \efield beyond expectation & Minimize insulators facing cryostat wall ground. & M & M & L \\  \colhline
RT-DP-HV-05 & Damage to \dword{ce} in event of discharge & Minimize the energy released in a short time using highly resistive connections. & L & L & L \\  \colhline
RT-DP-HV-06 & Energy stored in FC (in DP) is suddenly discharged & Delay energy discharge by connecting neighboring Al profiles with resistive sheaths.  & L & L & L \\  \colhline
RT-DP-HV-07 & Detector components are damaged during shipment to the far site & Make sufficient spares and increase the number of shipping boxes.  & L & L & L \\  \colhline
RT-DP-HV-08 & Damages (scratches, bending) to aluminum profiles of Field Cage modules & Make sufficient spares and increase the number of shipping boxes.  & L & L & L \\  \colhline
RT-DP-HV-09 & Bubbles from heat in PMTs or resistors cause HV discharge & A large area of cathode consists of high resistance rods, delaying the energy release.   & L & L & L \\  \colhline
RT-DP-HV-10 & Free hanging frames can swing in the fluid flow &  & L & L & L \\  \colhline
RT-DP-HV-11 & FRP/ Polyethene/ laminated Kapton component lifetime is less than expected &  & L & L & L \\  \colhline
RT-DP-HV-12 & Lack of collaboration effort on this HV system & Continue recruiting collaborators. & L & L & L \\  \colhline
RT-DP-HV-13 & International funding level for DP HVC too low & Employ cost saving measures and  recruit collaborators. & L & L & L \\  \colhline
RT-DP-HV-14 & Underground installation is more labor intensive or slower than expected & Increase labor contingency and refine labor cost estimates. Further improve installation procedure. & L & L & L \\  \colhline

\label{tab:risks:DP-FD-HV}
\end{longtable}

% risk table values for subsystem DP-FD-TPC
\begin{footnotesize}
%\begin{longtable}{p{0.18\textwidth}p{0.20\textwidth}p{0.32\textwidth}p{0.02\textwidth}p{0.02\textwidth}p{0.02\textwidth}}
\begin{longtable}{P{0.18\textwidth}P{0.20\textwidth}P{0.32\textwidth}P{0.02\textwidth}P{0.02\textwidth}P{0.02\textwidth}} 
\caption[Risks for DP-FD-TPC]{Risks for DP-FD-TPC (P=probability, C=cost, S=schedule) More information at \dshort{riskprob}. \fixmehl{ref \texttt{tab:risks:DP-FD-TPC}}} \\
\rowcolor{dunesky}
ID & Risk & Mitigation & P & C & S  \\  \colhline
RT-DP-TPC-01 & Component obsolescence over the experiment lifetime & Monitor component stocks and procure an adequate number of spares at the time of production & L & M & L \\  \colhline
RT-DP-TPC-02 & Modification to the LRO FE electronics due to evolution in design of PD design & A strict and timely following of the evolution of DP PDS & L & L & M \\  \colhline
RT-DP-TPC-03 & Damage to electronics due to HV discharges or other causes & FE analog electronics is protected with TVS diodes. Electronics can be easily replaced. & L & L & L \\  \colhline
RT-DP-TPC-04 & Problems with FE card extraction due to insufficient overhead clearance & Addressed by imposing a clearance requirement on \dshort{lbnf} & L & L & L \\  \colhline
RT-DP-TPC-05 & Overpressure in the \dshorts{sftchimney} & The \dshorts{sftchimney} are equipped with overpressure release valves & L & L & L \\  \colhline
RT-DP-TPC-06 & Leak of nitrogen inside the \dshort{dpmod} via cold flange & Monitor chimney pressure for leaks and switch to argon cooling in case of a leak & L & L & L \\  \colhline
RT-DP-TPC-07 & Data flow increase due to inefficient compression caused by higher noise & Have a sufficiently large (a factor of \num{5}) margin in the available bandwidth & L & L & L \\  \colhline
RT-DP-TPC-08 & Damage to \dshort{utca} crates due to presence of water on the roof of the cryostat & \dshort{lbnf} requirement that the cryostat top remains dry & L & L & L \\  \colhline
RT-DP-TPC-09 & Clogging ventilation system of \dshort{utca} crates due to bad air quality & \dshort{lbnf} requirement that the air quality is comparable to a standard industrial environment & L & L & L \\  \colhline

\label{tab:risks:DP-FD-TPC}
\end{longtable}
\end{footnotesize}


% risk table values for subsystem DP-FD-PDS
\begin{footnotesize}
%\begin{longtable}{p{0.18\textwidth}p{0.20\textwidth}p{0.32\textwidth}p{0.02\textwidth}p{0.02\textwidth}p{0.02\textwidth}}
\begin{longtable}{P{0.18\textwidth}P{0.20\textwidth}P{0.32\textwidth}P{0.02\textwidth}P{0.02\textwidth}P{0.02\textwidth}} 
\caption[DP PDS risks]{Risks for DP-FD-PDS (P=probability, C=cost, S=schedule) More information at \dshort{riskprob}. \fixmehl{ref \texttt{tab:risks:DP-FD-PDS}}} \\
\rowcolor{dunesky}
ID & Risk & Mitigation & P & C & S  \\  \colhline
RT-DP-PDS-01 & Insufficient light yield due to inefficient PDS design & Increase PMT photo-cathode coverage and/or WLS reflector foils coverage. & L & M & L \\  \colhline
RT-DP-PDS-02 & Poor coating quality for \dshort{tpb} coated surfaces and \dshort{lar} contamination by \dshort{tpb} & Test quality and ageing properties of TPB coating techniques. Elaborate improved techniques if needed. & L & L & L \\  \colhline
RT-DP-PDS-03 & \dshort{pmt} channel loss due to faulty \dshort{pmt} base design & Optimize clustering algorithms. Improve \dshort{pmt} base design from analysis of possible failure modes in \dshort{pddp}. & L & L & L \\  \colhline
RT-DP-PDS-04 & Bad \dshort{pmt} channel due to faulty connection between \dshort{hv}/signal cable and \dshort{pmt} base & Optimize clustering algorithms. Connectivity tests in \lntwo prior to installation. & L & L & L \\  \colhline
RT-DP-PDS-05 & \dshort{pmt} signal saturation & Tuning of \dshort{pmt} gain. In worst case, redesign front-end to adjust to analog input range of ADC. & M & L & L \\  \colhline
RT-DP-PDS-06 & Excessive electronics noise to distinguish \dshort{lar} scintillation light & Measurement of noise levels during commissioning prior to \lar filling. Modifications to grounding, shielding, or power distribution schemes. & M & L & L \\  \colhline
RT-DP-PDS-07 & Availability of resources for work at the installation/integration site less than planned & Move people temporarily from institutions involved in the \dshort{pds} consortium to the integration/installation site. & L & L & L \\  \colhline
RT-DP-PDS-08 & Damage of \dshorts{pmt} during shipment to the experiment site & Special packaging to avoid possible \dshort{pmt} damage during shipment. Contingency of 10\% spare  \dshorts{pmt}. & L & L & L \\  \colhline
RT-DP-PDS-09 & Damage of optical fibers during installation & Fibers will be last DP-PDS item to be installed. Detailed documentation for all DP-PDS installation tasks.   & L & L & L \\  \colhline
RT-DP-PDS-10 & Excessive exposure to ambient light of \dshort{tpb} coated surfaces, resulting in degraded performance & \dshort{tpb} coated surfaces temporarily covered until cryostat closing. Detailed installation procedure to minimize exposure to ambient light. & L & L & L \\  \colhline
RT-DP-PDS-11 & \dshort{pmt} implosion during \dshort{lar} filling & No mitigation necessary, considering \SI{7}{bar} pressure rating of \dshorts{pmt} and experience with same/similar \dshorts{pmt} in other large liquid detectors. & L & L & L \\  \colhline
RT-DP-PDS-12 & Insufficient light yield due to poor \dshort{lar} purity & Procurement of \dshort{lar} from the manufacturer will require less than 3 ppM in nitrogen. & M & L & L \\  \colhline
RT-DP-PDS-13 & \dshort{pmt} channel or \dshort{pds} sector loss due to failures in \dshort{hv}/signal rack & Ease of maintenance outside cryostat and availability of spares for all components of at least one \dshort{hv}/signal rack. & L & L & L \\  \colhline
RT-DP-PDS-14 & Unstable response of the photon detection system over the lifetime of the experiment & Channel-level instabilities corrected via light calibration system. Detector-level instabilities corrected via cosmic-ray muon calibration data. & L & L & L \\  \colhline
RT-DP-PDS-15 & Bubbles from heat in \dshorts{pmt} or resistors cause \dshort{hv} discharge of the cathode & Verify the power density of the \dshort{pmt} bases are within specifications. Monitor and interlock \dshort{pmt} power supply currents. & L & L & L \\  \colhline
RT-DP-PDS-16 & Reflector/\dshort{wls} panel assemblies together with the \dshort{fc} walls can swing in the fluid flow & Allow appropriate open areas within/between reflector/\dshort{wls} panel assemblies to minimize drag. & L & L & L \\  \colhline

\label{tab:risks:DP-FD-PDS}
\end{longtable}
\end{footnotesize}


% risk table values for subsystem DP-FD-CAL
\begin{longtable}{p{0.18\textwidth}p{0.20\textwidth}p{0.32\textwidth}p{0.02\textwidth}p{0.02\textwidth}p{0.02\textwidth}} 
\caption{Risks for DP-FD-CAL \fixmehl{ref \texttt{tab:risks:DP-FD-CAL}}} \\
\rowcolor{dunesky}
ID & Risk & Mitigation & P & C & S  \\  \colhline
RT-DP-CAL-01 & Inadequate baseline design & Early detection allows R\&D of alternative designs accommodated through multipurpose ports. & L & M & M \\  \colhline
RT-DP-CAL-02 & Inadequate engineering or production quality & Dedicated small-scale tests and full prototyping at ProtoDUNE; pre-installation QC. & L & M & M \\  \colhline
RT-DP-CAL-03 & Laser impact on PDS & Mirror movement control to minimize direct hits; intelock to keep laser off while PMTs are on. & L & L & L \\  \colhline
RT-DP-CAL-04 & Laser positioning system stops working & QC at installation time, redundancy in available targets, including passive, alternative methods. & L & L & L \\  \colhline
RT-DP-CAL-05 & Laser beam misaligned & Additional (visible) laser for alignment purposes. & M & L & L \\  \colhline
RT-DP-CAL-06 & The neutron anti-resonance is much less pronounced & Dedicated measurements at LANL and test at ProtoDUNE. & L & L & L \\  \colhline
RT-DP-CAL-07 & Neutron activation of the moderator and cryostat & Neutron activation studies and simulations. & L & L & L \\  \colhline
RT-DP-CAL-08 & Neutron yield not high enough & Simulations and tests at ProtoDUNE & L & M & M \\  \colhline
RT-DP-CAL-09 & Neutrons do not reach detector center & Alternative, movable design and simulations & L & L & L \\  \colhline

\label{tab:risks:DP-FD-CAL}
\end{longtable}

% risk table values for subsystem DP-FD-DAQ
\begin{footnotesize}
%\begin{longtable}{p{0.18\textwidth}p{0.20\textwidth}p{0.32\textwidth}p{0.02\textwidth}p{0.02\textwidth}p{0.02\textwidth}}
\begin{longtable}{P{0.18\textwidth}P{0.20\textwidth}P{0.32\textwidth}P{0.02\textwidth}P{0.02\textwidth}P{0.02\textwidth}} 
\caption[Risks for DP-FD-DAQ]{Risks for DP-FD-DAQ (P=probability, C=cost, S=schedule) More information at \dshort{riskprob}. \fixmehl{ref \texttt{tab:risks:DP-FD-DAQ}}} \\
\rowcolor{dunesky}
ID & Risk & Mitigation & P & C & S  \\  \colhline
RT-DP-DAQ-01 & Detector noise specs not met & ProtoDUNE experience with noise levels and provisions for data processing redundancy in DAQ system; ensure enough headroom of bandwidth to FNAL. & L & L & L \\  \colhline
RT-DP-DAQ-02 & Externally-driven schedule change & Provisions for standalone testing and commissioning of production DAQ components, and schedule adjustment & L & L & L \\  \colhline
RT-DP-DAQ-03 & Lack of expert personnel & Resource-loaded plan for DAQ backed by institutional commitments, and schedule adjustment using float & L & L & H \\  \colhline
RT-DP-DAQ-04 & Power/space requirements exceed CUC capacity & Sufficient bandwidth to surface and move module 3/4 components to an expanded surface facility & L & L & L \\  \colhline
RT-DP-DAQ-05 & Excess fake trigger rate from instrumental effects & ProtoDUNE performance experience, and provisions for increase in event builder and high level filter capacity, as needed; headroom in data link to FNAL. & L & L & L \\  \colhline
RT-DP-DAQ-06 & Calibration requirements exceed acceptable data rate & Provisions for increase in event builder and high level filter capacity, as neeed; headroom in data link to FNAL. & L & L & L \\  \colhline
RT-DP-DAQ-07 & Cost/performance of hardware/computing excessive & Have prototyping and pre-production phases, reduce performance using margin or identify additional funds & L & L & L \\  \colhline
RT-DP-DAQ-08 & PDTS fails to scale for DUNE requirements & Hardware upgrade & L & L & L \\  \colhline
RT-DP-DAQ-09 & WAN network & Extensive QA and development of failure mode recovery and automation, improved network connectivity, and personnel presence at SURF as last resort. & L & M & M \\  \colhline
RT-DP-DAQ-10 & Infrastructure & Design with redundancy, prior to construction, and improve power/cooling system. & M & M & L \\  \colhline
RT-DP-DAQ-11 & Custom electronics manifacturing issues & Diversify the manifacturers used for production; run an early pre-production and apply stringent QA criteria. & L & M & M \\  \colhline

\label{tab:risks:DP-FD-DAQ}
\end{longtable}
\end{footnotesize}


% risk table values for subsystem DP-FD-CISC
\begin{footnotesize}
\begin{longtable}{p{0.18\textwidth}p{0.20\textwidth}p{0.32\textwidth}p{0.02\textwidth}p{0.02\textwidth}p{0.02\textwidth}}
%\begin{longtable}{x{0.18\textwidth}x{0.20\textwidth}x{0.32\textwidth}x{0.02\textwidth}x{0.02\textwidth}x{0.02\textwidth}} 
\caption{Risks for DP-FD-CISC (P=probability, C=cost, S=schedule) (more information at \dword{riskprob}). \fixmehl{ref \texttt{tab:risks:DP-FD-CISC}}} \\
\rowcolor{dunesky}
ID & Risk & Mitigation & P & C & S  \\  \colhline
RT-DP-CISC-001 & Baseline design from ProtoDUNEs for an instrumentation device is not adequate for DUNE far detectors & Focus on early problem discovery in ProtoDUNE so any needed redesigns can start as soon as possible. & L & M & L \\  \colhline
RT-DP-CISC-002 & Swinging of long instrumentation devices (T-gradient monitors or PrM system) & Add additional intermediate constraints to prevent swinging. & L & L & L \\  \colhline
RT-DP-CISC-003 & High E-fields near instrumentation devices cause dielectric breakdowns in \dword{lar} & CISC systems shielded and placed as far from cathode and FC as possible. & L & L & L \\  \colhline
RT-DP-CISC-004 & Light pollution from purity monitors and camera light emitting system & Use PrM lamp and camera lights outside PDS trigger window; cover PrM cathode to reduce light leakage. & L & L & L \\  \colhline
RT-DP-CISC-005 & Temperature sensors can induce noise in cold electronics & Check for noise before filling and remediate, repeat after filling. Filter or ground noisy sensors. & L & L  & L \\  \colhline
RT-DP-CISC-006 & Disagreement between lab and \em{in situ} calibrations for ProtoDUNE-SP dynamic T-gradient monitor & Investigate and improve both methods, particularly laboratory calibration. & M & L & L \\  \colhline
RT-DP-CISC-007 & Purity monitor electronics induce noise in TPC and PDS electronics. & Operate lamp outside TPC+PDS trigger window. Surround and ground light source with Faraday cage. & L & L & L \\  \colhline
RT-DP-CISC-008 & Discrepancies between measured temperature map and CFD simulations in ProtoDUNE-SP & Improve simulations with additional measurements inputs; use fraction of sensors to predict others   & L & L & L \\  \colhline
RT-DP-CISC-009 & Difficulty correlating purity and temperature in ProtoDUNE-SP impairs understanding cryo system. & Identify causes of discrepancy, modify design. Calibrate PrM differences, correlate with RTDs. & L & L & L \\  \colhline
RT-DP-CISC-010 & Cold camera R\&D fails to produce prototype meeting specifications \& safety requirements & Improve insulation and heaters. Use cameras in ullage or inspection cameras instead. & M & M & L \\  \colhline
RT-DP-CISC-011 & HV discharge caused by inspection cameras & Study E-field in and on housing and anchoring system. Test in HV facility. & L & L & L \\  \colhline
RT-DP-CISC-012 & HV discharge destroying the cameras & Ensure sufficient redundancy of cold cameras. Warm cameras are replaceable. & L & M & L \\  \colhline
RT-DP-CISC-013 & Insufficient light for cameras to acquire useful images & Test cameras with illumination similar to actual detector. & L & L & L \\  \colhline
RT-DP-CISC-014 & Cameras may induce noise in cold electronics & Continued R\&D work with grounding and shielding in realistic conditions. & L & L & L \\  \colhline
RT-DP-CISC-015 & Light attenuation in long optic fibers for purity monitors  & Test the max.\ length of usable fiber, optimize the depth of bottom PrM, number of fibers. & L & L & L \\  \colhline
RT-DP-CISC-016 & Longevity of purity monitors & Optimize PrM operation to avoid long running in low purity. Technique to protect/recover cathode. & L & L & L \\  \colhline
RT-DP-CISC-017 & Longevity: Gas analyzers and level meters may fail. & Plan for future replacement in case of failure or loss of sensitivity.  & M & M & L \\  \colhline
RT-DP-CISC-018 & Problems in interfacing  hardware devices (e.g. power supplies) with slow controls & Involve slow control experts in choice of hardware needing control/monitoring.
 & L & L & L \\  \colhline

\label{tab:risks:DP-FD-CISC}
\end{longtable}
\end{footnotesize}

% risk table values for subsystem DP-FD-INST
\begin{footnotesize}
%\begin{longtable}{p{0.18\textwidth}p{0.20\textwidth}p{0.32\textwidth}p{0.02\textwidth}p{0.02\textwidth}p{0.02\textwidth}}
\begin{longtable}{P{0.18\textwidth}P{0.20\textwidth}P{0.32\textwidth}P{0.02\textwidth}P{0.02\textwidth}P{0.02\textwidth}} 
\caption[Risks for DP-FD-INST]{Risks for DP-FD-INST (P=probability, C=cost, S=schedule) More information at \dword{riskprob}. \fixmehl{ref \texttt{tab:risks:DP-FD-INST}}} \\
\rowcolor{dunesky}
ID & Risk & Mitigation & P & C & S  \\  \colhline
RT-DPINST-01 & Personnel injury & Follow the safety rules in force. & L & L & H \\  \colhline
RT-DPINST-02 & Cryostat damage during installation & Use temporary protection for the corrugated membrane. & L & L & M \\  \colhline
RT-DPINST-03 & Detector components damage during transport/installation & Handling should be done by trained people, following the handling instructions provided by the consortia and in presence of a technical expert. & L & M & M \\  \colhline
RT-DPINST-04 & Detector components failure during test & Only trained people should test equipments. Spare components must be available at the warehouse facility. & L & L & M \\  \colhline
RT-DPINST-05 & Components interferences & 3D model, survey at the construction sites, and full scale assembly tests. & M & L & L \\  \colhline
RT-DPINST-06 & Shipping delays/missing parts & Buffer material and detector components at the warehouse facility and use inventory tools to follow the fundamental items. & L & L & L \\  \colhline
RT-DPINST-07 & Lack of specialised/trained manpower & Hire and train personnel. Plan with all the underground stakeholders enough in advance the required personnel. & L & L & L \\  \colhline

\label{tab:risks:DP-FD-INST}
\end{longtable}
\end{footnotesize}
