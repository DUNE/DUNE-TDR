\chapter{Reviews}
\label{vl:tc-review}

\dword{tc} is responsible for reviewing all stages of detector
development and works with each consortium to arrange reviews of the
design (conceptual design review (CDR), \dword{pdr} and \dword{fdr}),
production (\dword{prr} and \dword{ppr}) and \dword{orr} of their
system.  These reviews provide input for the TB to evaluate technical
decisions.  Review reports are tracked by \dword{tc} and provide
guidance as to key issues that will require engineering oversight by
the \dword{tc} engineering team. \Dword{tc} will maintain a calendar
of \dword{dune} reviews.

\Dword{tc} works with consortia leaders to review all detector designs,
with expectation of a \dword{pdr}, followed by a \dword{fdr}.  All
major technology decisions will be reviewed prior to down-select.  \Dword{tc}
may form task forces as necessary for specific issues that need more
in-depth review.


Start of production of detector elements can commence only after
successful \dwords{prr}. Regular production progress
reviews will be held once production has commenced. The \dwords{prr}
will typically include review of the production of \textit{Module 0}, the
first such module produced at the facility. \Dword{tc} will work with
consortium leaders for all production reviews.

\Dword{tc} is responsible to coordinate technical documents for the LBNC
Technical Design Review.

The review process is an important part of the \dword{dune} QA process
as described in Section~\ref{sec:verification}, both in regards to
design and production.

The review process has been in place since 2016 with various reviews
of \dword{protodune} components and has continued into the first \dword{dune}
reviews in 2018. Past and scheduled reviews can be found in the
\dword{dune} Indico at https://indico.fnal.gov/category/586 .
Reports from the reviews are maintained in DocDB-1584.

\section{Design Reviews}

The \dword{dune} design review process is described in DocDB-9664,
which is consistent withthe \fnal review process described in
http://eshq.fnal.gov/manuals/feshm . Design reviews were held for each
major system for \dword{protodune}. Given the extreme schedule
pressure that \dword{protodune} was under only a single design review
was held for each system.

After the successful operation of \dword{protodune} \dword{dune} is at
a very advanced state of design. The strategy goiong forward to to
hold \dword{cdr} for systems that will have significant changes from
\dword{protodune}. These systems include the \dword{dss}, \dword{pds} and
\dword{daq}. All systems will go through \dword{pdr} to review
design changes after \dword{protodune} and \dword{fdr} after the
\dword{tdr}.

\dword{tc} has empowered an Engineering Safety Committee consisting of
mechanical and electrical engineering experts from collaborating
institutions to develop the processes and procedures for evaluation of
engineering designs in terms of accepted international safety
standards. The current status of international code equivalencies is
discussed further in Section~\ref{sec:esh_codes}. The codes and
standards to which each system is designed will be reviewed as part of
the \dword{pdr} and \dword{fdr}.

\section{Production Reviews}

Production reviews are scheduled after the designs are finalized and
before significant funds are authorized for large production
runs. These reviews are closely coordination with the QA team. The
expectation is that a ``module 0'' will have been produced and will be
presented as part of the \dword{prr}.

Once production has started then \dword{tc} will schedule \dword{ppr}
as appropriate to monitor production schedule and quality.

\section{Operations Reviews}

Operation readiness reviews (per
http://eshq.fnal.gov/manuals/feshm/\#series2000) Are the final safety
checkout before equipment can be operated.

\section{Review Tracking}

Tracking and control of review recommendations is part of the review
process. Recommendations from prior reviews are assessed by subsequent
review committees. \dword{tc} is responsible for assuring that
responses to review recommendations are generated by consortia and to
work with consortia to make sure they are appropriately documented and
implemented. Reports from \dword{dune} reviews are maintained in
DocDB-1584 along with the list of recommendations.

