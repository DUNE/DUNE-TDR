



{\LARGE
    \textbf{A Roadmap of the DUNE Technical Design Report}}

The \dword{dune} \dword{tdr} describes the proposed physics program,  
detector designs, and management structures and procedures at the technical design stage.  

The \dword{tdr} is composed of five volumes, as follows:

\begin{itemize}
\item Volume~\volnumberexec{}, \voltitleexec{} introduces and summarizes the entire \dword{tdr}. It also includes chapters and appendices on aspects of \dword{dune} not covered in the subsequent volumes, in particular, the \dword{nd} and the computing infrastructure.

\item Volume~\volnumberphysics{}, \voltitlephysics{} outlines the scientific objectives and describes the physics studies that  the \dword{dune} collaboration will undertake to address them.

\item Volume~\volnumbersp{}, \voltitlesp{} describes the \dword{sp} technology that will be used in the first \dword{fd} module, and the plans and procedures that are in place to construct and install  this \dword{spmod}.

\item Volume~\volnumberdp{}, \voltitledp{} describes the \dword{dp} technology that forms the basis for the design of the second \dword{fd} module, and the plans and procedures that will be needed to construct and install a \dword{dpmod}. 

\item Volume~\volnumbertc{}, \voltitletc{} describes the management structures, standards, and procedures in place to guide the safe and successful construction and installation of the first two \dword{fd} modules.
\end{itemize}

A few of the chapters present both the current design and one or more \textit{alternative designs} that are still under consideration. The alternative designs are in sections labeled as \textit{appendices} to clearly distinguish them from the current designs. 

Following its table of contents, each volume has a list of figures and a list of tables.

A volume-specific glossary is provided after the last chapter in each volume.  Some glossary terms are defined as abbreviations, other just as terms. Those defined as abbreviations expand to their full term upon first use in a chapter. Glossary terms in the text are hyperlinked  to their definitions in the PDF output files. 
Depending on the application used to display the PDF files, the glossary terms may appear highlighted with an underline.  No highlighting appears on printed pages.

Volume-specific references appear at the end of each volume. 


