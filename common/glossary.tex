% See dune-words.tex for detailed explanation.

% http://mirrors.ctan.org/macros/latex/contrib/glossaries/glossaries-user.pdf

% \usepackage[acronyms,toc]{glossaries}
\usepackage[toc]{glossaries}
\makeglossaries


% for terms with acronyms
\newcommand{\dshort}[1]{\glsentrytext{#1}}
\newcommand{\dshorts}[1]{\glsentryshortpl{#1}}
\newcommand{\dlong}[1]{\glsentrylong{#1}}
\newcommand{\dlongs}[1]{\glsentrylongpl{#1}}

% force the "first time" behavior
\newcommand{\dfirst}[1]{\glsfirst{#1}}
\newcommand{\dfirsts}[1]{\glsfirstplural{#1}}

\newcommand{\dword}[1]{\gls{#1}}
\newcommand{\dwords}[1]{\glspl{#1}}
\newcommand{\Dword}[1]{\Gls{#1}}
\newcommand{\Dwords}[1]{\Glspl{#1}}


% use this to define terms that do NOT have acronyms.
% \newduneword{label}{term}{description}
\newcommand{\newduneword}[3]{
    \newglossaryentry{#1}{
        text={#2},
        long={#2},
        name={\glsentrylong{#1}},
        first={\glsentryname{#1}},
        firstplural={\glsentrylong{#1}\glspluralsuffix},
        description={#3}
    }
}

% use this to define terms that DO have acronyms.
%                 1      2     3       4 
% \newduneabbrev{label}{abbrev}{term}{description}
\newcommand{\newduneabbrev}[4]{
  \newglossaryentry{#1}{
    text={#2},
    long={#3},
    shortplural={{#2}\glspluralsuffix},
    longplural={{#3}\glspluralsuffix{}},
    name={\glsentrylong{#1}{} (\glsentrytext{#1}{})},
    first={\glsentryname{#1}},
    firstplural={#3\glspluralsuffix{} (\glsentrytext{#1}\glspluralsuffix{})},
    description={#4}
  }
}

%% If plural needs special spelling besides adding an "s"
%                 1      2     3       4        5
% \newduneabbrev{label}{abbrev}{term}{terms}{description}
\newcommand{\newduneabbrevs}[5]{
  \newglossaryentry{#1}{
    text={#2},
    long={#3},
    plural={#4},
    shortplural={{#2}\glspluralsuffix},
    longplural={#4},
    name={\glsentrylong{#1}{} (\glsentrytext{#1}{})},
    first={#3 (#2)},
    firstplural={#4 (\glsentrytext{#1}\glspluralsuffix{})},
    description={#5}
  }
}


% tres meta
\newduneword{dword}{DUNE Word}{A term in the DUNE lexicon}

%%%%%%     START ADDING WORDS, IN ALPHABETICAL ORDER IF POSSIBLE!    %%%%%%%%
%near detector
\newduneabbrev{nd}{ND}{near detector}{Refers to the detector(s) %or more  generally the experimental site
 installed close to the neutrino source at \fnal }

%far detector
\newduneabbrev{fd}{FD}{far detector}{Refers to the \fdfiducialmass fiducial mass DUNE detector %or more generally the experimental 
  to be installed at the far site at \surf in
  Lead, SD, to be composed of four \nominalmodsize modules}

%single-phase
\newduneabbrev{sp}{SP}{single-phase}{Distinguishes one of the DUNE far detector technologies by the fact that it operates using argon in its liquid phase only
  %Distinguishes one of the four  \SI{10}{\kton} \glspl{detmodule} of the DUNE far detector by the  fact that it operates using argon in just its liquid phase}
  }

%dual-phase
\newduneabbrev{dp}{DP}{dual-phase}{Distinguishes one of the DUNE far detector technologies by the fact that it operates using argon %Distinguishes one of the four \SI{10}{\kton} \glspl{detmodule} of the DUNE far detector by the fact that it operates using argon 
 in both gas and liquid phases}

%photon detection system
\newduneabbrev{pds}{PDS}{photon detection system}{The detector 
  subsystem sensitive to light produced in the \lar }

%high voltage system
\newduneabbrev{hvs}{HVS}{high voltage system}{The detector 
  subsystem that provides the TPC drift field }

%time projection chamber
\newduneabbrev{tpc}{TPC}{time projection chamber}{The portion of each
  DUNE \gls{detmodule} that records ionization electrons
  after they drift away from a cathode through the \lar, and also
  through gaseous argon in a \dual module. 
  The activity is recorded by digitizing the waveforms of current
  induced on the anode as the distribution of ionization charge passes by
  or is collected on the electrode}

%anode plane assembly
\newduneabbrevs{apa}{APA}{anode plane assembly}{anode plane assemblies}{A unit of the \single
  detector module containing the elements sensitive to ionization in the \lar. 
  It contains two faces each of three planes of wires, and interfaces to the cold
  electronics and photon detection system} 

\newduneabbrev{awg}{AWG}{american wire gauge} {U.S. standard set of non-ferrous wire conductor sizes}

\newduneabbrev{ufer}{Ufer}{Concrete Encased Electrode} {U.S. National Electrical Code grounding method refered to as Concrete Encased Electrode}

%charge readout
\newduneabbrev{cro}{CRO}{charge readout}{The system for detecting
  ionization charge distributions in a \dual detector module}

%light readout
\newduneabbrev{lro}{LRO}{light readout}{The system for detecting
  scintillation photons in a \dual  detector module}

%safe high voltage
\newduneabbrev{shv}{SHV}{safe high voltage}{Type of bayonet mount
connector used on coaxial cables that has additional insulation 
compared to standard BNC and MHV connectots that makes it safer
for handling high voltage by preventing accidental contact with the
live wire connector in an unmated connector or plug}

%front-end
\newduneabbrev{fe}{FE}{front-end}{The front-end refers a point that is
  ``upstream'' of the data flow for a particular subsystem. 
  For example the front-end electronics is where the cold electronics
  meet the sense wires of the TPC and the front-end DAQ is where the
  DAQ meets the output of the electronics}

\newduneabbrev{daqrou}{RU}{DAQ readout units}{The first element in the data flow of the DAQ.}

\newduneabbrev{cots}{COTS}{commercial off-the-shelf}{Items, typically hardware such as 
computers, which may be purchased whole, without any custom design or fabrication and 
thus at normal consumer prices and availability.}

\newduneabbrev{i2c}{I2C}{inter-integrated circuit}{I$^2$C or I2C is a synchronous, 
multi-master, multi-slave, packet switched, single-ended, serial computer bus widely used 
for attaching lower-speed peripheral ICs to processors and microcontrollers in short-distance, 
intra-board communication.}

\newduneabbrev{spi}{SPI}{serial peripheral interface}{The serial peripheral interface is a 
synchronous serial communication interface specification used for short distance 
communication, primarily in embedded systems.}

\newduneabbrev{miso}{MISO}{master in slave out}{The master in slave out is a logic
signal on the \dword{spi} bus on which the data from the slave are transmitted once
a request from the master is received.}

\newduneabbrev{mosi}{MOSI}{master out slave in}{The master out slave in is a logic
signal on the \dword{spi} bus on which the data from the master is transmitted.}

\newduneabbrev{uart}{UART}{universal asynchrous receiver/transmitter}{A universal 
asynchronous receiver-transmitter is a computer hardware device for asynchronous 
serial communication in which the data format and transmission speeds are configurable.}

\newduneword{cr}{CR}{capacitance-resistance}

\newduneword{dc}{DC}{direct coupling}

\newduneword{ac}{AC}{capacitive coupling}

\newduneabbrev{pll}{PLL}{phase-locked loop}{A phase-locked loop is a control system that generates an
output signal whose phase is related to the phase of an input signal.}

\newduneword{fifo}{FIFO}{first-in-first-out}

\newduneword{tsmc}{TSMC}{Taiwan Semiconductor Manufacturing Company}

\newduneword{saci}{SACI}{\dword{slac} \dword{asic} Control Interface}

\newduneword{dnl}{DNL}{differential non-linearity}

\newduneword{inl}{INL}{integral non-linearity}

\newduneword{qfp}{QFP}{quad flat package}

\newduneabbrev{ams}{AMS}{Analog and mixed signal}{Verilog-AMS is a derivative of the Verilog hardware description language that includes analog and mixed-signal extensions (AMS) in order to define the behavior of analog and mixed-signal systems.}

\newduneabbrev{hepa}{HEPA}{high efficiency particulate air}{The high efficiency particulate air filters are a type of air filter that remove \num{99.97}\% of particles that have a size greater han or equal to \SI{0.3}{$\mu$m}.}

\newduneabbrev{uvm}{UVM}{universal verification methodology}{The universal verification methodology is a standardized methodology for verifying integrated circuit designs.}

\newduneword{pnp}{PNP}{Type of bipolar junction transistor consistning of a
layer of N-doped semiconductor sandwiched between two layers of P-doped material.}

\newduneabbrev{spice}{SPICE}{Simulation Program with Integrated Circuit Emphasis}{SPICE
(``Simulation Program with Integrated Circuit Emphasis'') is a general-purpose, 
open-source analog electronic circuit simulator. It is a program used in integrated 
circuit and board-level design to check the integrity of circuit designs and to 
predict circuit behavior.}

%analog digital converter
\newduneabbrev{adc}{ADC}{analog-to-digital converter}{A sampling of a voltage
  resulting in a discrete integer count corresponding in some way to
  the input}

%data acquisition
\newduneabbrev{daq}{DAQ}{data acquisition}{The data acquisition system
  accepts data from the detector FE electronics, buffers
  the data, performs a \gls{trigdecision}, builds events from the selected
  data and delivers the result to the offline \gls{diskbuffer}}

%detector module
\newduneword{detmodule}{detector module}{The entire DUNE far detector is
  segmented into four modules, each with a nominal \SI{10}{\kton}
  fiducial mass}

%detector unit
\newduneword{detunit}{detector unit}{A \gls{submodule} may be partitioned
  into a number of similar parts. 
  For example the single-phase TPC \gls{submodule} is made up of APA
  units}

%signal over noise ratio
\newduneword{snr}{\mbox{S/N}}{signal over noise ratio}

%secondary DAQ buffer
\newduneword{diskbuffer}{secondary DAQ buffer}{A secondary
  \dshort{daq} buffer holds a small subset of the full rate as
  selected by a \gls{trigcommand}. 
  This buffer also marks the interface with the DUNE Offline}

%online  monitoring
\newduneabbrev{om}{OM}{online monitoring}{Processes that run inside
  the DAQ on data ``in flight,'' specifically before landing on the
  offline disk buffer, and that provide feedback on the operation of
  the DAQ itself and the general health of the data it is marshalling}

%data quality monitoring
\newduneabbrev{dqm}{DQM}{data quality monitoring}{Analysis of the raw
  data to monitor the integrity of the data and the performance of the
  detectors and their electronics. This type of monitoring may be
  performed in real time, within the \gls{daq} system, or in later
  stages of processing, using disk files as input}

%DAQ dump buffer
\newduneword{dumpbuffer}{DAQ dump buffer}{This \dshort{daq} buffer
  accepts a high-rate data stream, in aggregate, from an associated
  \gls{submodule} sufficient to collect all data likely relevant to
  a potential \gls{snb}}

%Global Trigger Logic
\newduneabbrev{etl}{ETL}{external trigger logic}{Trigger processing
  that consumes \gls{detmodule} level \gls{trignote} information
  and other global sources of trigger input and emits
  \gls{trigcommand} information back to the \glspl{mtl}}

%trigger notification
\newduneword{trignote}{trigger notification}{Information provided by
  \gls{mtl} to \gls{etl} about \gls{trigdecision} its processing}

%trigger primitive
\newduneword{trigprimitive}{trigger primitive}{Information derived by
  the DAQ \gls{fe} hardware that describes a region of space (e.g.,
  one or several neighboring channels) and time (e.g., a contiguous set
  of ADC sample ticks) associated with some activity}

%external trigger candidate
\newduneword{externtrigger}{external trigger candidate}{Information
  provided to the \gls{mtl} about events external to a
  \gls{detmodule} so that it may be considered in forming
  \glspl{trigcommand}}

%Out-of-band trigger command dispatcher
\newduneabbrev{daqoob}{OOB dispatcher}{out-of-band trigger command
  dispatcher}{This component is responsible for dispatching a \gls{snb} dump
  command to all \glspl{daqfer} in the \gls{detmodule}}

%module trigger logic
\newduneabbrev{mtl}{MTL}{module trigger logic}{Trigger processing
  that consumes \gls{detunit} level \gls{trigcommand} information
  and emits \glspl{trigcommand}. 
  It provides the \gls{etl} with \glspl{trignote} and receives back any
  \glspl{externtrigger}}

%octant
\newduneword{octant}{octant}{Any of the eight parts into which 4$\pi$
  is divided by three mutually perpendicular axes. 
  In particular in referencing the value for the mixing angle
  $\theta_{23}$}

%sub-detector ??? %%%%%%%%%%%%%%%%%%%      Why not ``subdet''? (Anne)  %%%%%% ???????
\newduneword{submodule}{subdetector}{A detector unit of granularity less
  than one \gls{detmodule} such as the TPC of either a \single or \dual 
  module}

%trigger candidate
\newduneword{trigcandidate}{trigger candidate}{Summary information derived
  from the full data stream and representing a contribution toward
  forming a \gls{trigdecision}}

%trigger command
\newduneword{trigcommand}{trigger command}{Information derived from
  one or more \glspl{trigcandidate}  that directs elements of the
  \gls{detmodule} to read out a portion of the data stream}

%trigger command message
\newduneabbrev{tcm}{TCM}{trigger command message}{A message flowing
  down the trigger hierarchy from global to local context.  Also see \gls{tpm}}

%trigger decision
\newduneword{trigdecision}{trigger decision}{The process by which
  \glspl{trigcandidate} are converted into \glspl{trigcommand}}

%trigger primitive message
\newduneabbrev{tpm}{TPM}{trigger primitive message}{A message flowing
  up the trigger hierarchy from local to global context.  Also see \gls{tcm}}

\newduneabbrev{ipc}{IPC}{inter-process communication}{A system for software elements to exchange information between threads, local processes or across a data network.  An IPC system is typically specified in terms of protocols  composed of message types and their associated data schema.}

\newduneword{daqdispre}{discovery and presence}{As used in the context of the \gls{ipc}, a system which provides mechanisms for a node on a communication network to learn of the existence of peers and their identity (discovery) as well as determine if they are currently operational or have become unresponsive (presence).}

\newduneabbrev{pubsub}{PUB/SUB}{publish-subscribe communication pattern}{An \gls{ipc} communication pattern where one element, the publisher, sends data to all connected elements, the subscribers.  Each subscriber may connect to multiple publishers.  A variant is PUB/SUB with topics where a subscriber may register an identifier, the topic, to limit the information received to just an associated subset.}

%event builder
\newduneabbrev{eb}{EB}{event builder}{A software agent servicing one
  \gls{detmodule} by executing \glspl{trigcommand} by reading out
  the requested data}

\newduneabbrev{daqdfo}{DFO}{dataflow orchestration}{The process by which trigger commands are executed in parallel and asynchronous manner by the back-end output subsystem of the DAQ.}

% yes, this is a silly name
\newduneabbrev{daqfbi}{FBI}{front-end buffer interface}{The process which provides read-only access to data residing in a front-end's buffers to processes on the network.}

%cluster on board
\newduneabbrev{cob}{COB}{cluster on board}{An ATCA motherboard housing four RCEs}

%reconfigurable computing element
\newduneabbrev{rce}{RCE}{Reconfigurable Computing Element}{Data processor located outside of the cryostat on a \gls{cob} which contains \gls{fpga}, RAM and SSD resources, responsible for buffering data, producing trigger primitives, responding to triggered requests for data and sinking \gls{snb} dumps}

%bump on wire
\newduneabbrev{bow}{BOW}{Bump On Wire}{A working name for the front-end readout computing elements used in the nominal DAQ design to interface the \dual  crates to the DAQ front-end computers}

%advanced telecommunication computing architecture
\newduneabbrev{atca}{ATCA}{Advanced Telecommunications Computing
  Architecture}{An advanced computer architecture specification developed for the telecommunications, military, and aerospace industries that incorporates the latest trends high-speed interconnect technologies, next-generation processors, and improved reliability, availability and serviceability} 

%Micro Telecommunications Computing Architecture
\newduneabbrev{utca}{$\mu$TCA}{Micro Telecommunications Computing Architecture}{The computer architecture specification followed by the crates that house charge and light readout electronics in the dual-phase module} 

%user datagram protocol
\newduneabbrev{udp}{UDP}{user datagram protocol}{A simple,
  connectionless Internet protocol that supports data integrity
  checksums, requires no handshaking, and does not guarantee packet delivery}

%advanced meazzanine card
\newduneabbrev{amc}{AMC}{advanced mezzanine card}{Holds digitizing
  electronics and lives in \gls{utca} crates}

%radio frequency
\newduneabbrev{rf}{RF}{radio frequency}{Electromagnetic emissions
  that are within the (radio) frequency band of sensitivity of the detector
  electronics}

%field programmable gate array
\newduneabbrev{fpga}{FPGA}{field programmable gate array}{An
integrated circuit technology that allows the hardware to be reconfigured to
execute different algorithms after its manufacture and deployment}

\newduneabbrev{fmc}{FMC}{FPGA mezzanine card}{Boards holding FPGA and other integrated circuitry which attach to a motherboard.}

%Front-End Link eXchange
\newduneabbrev{felix}{FELIX}{Front-End Link eXchange}{A
  high-throughput interface between front-end and trigger electronics
  and the standard PCIe computer bus}

%DAQ partition
\newduneword{daqpart}{DAQ partition}{A cohesive and
 coherent collection of DAQ hardware and software working together to trigger and read out some portion of one detector module; it consists of an integral number of \glspl{daqfrag}. 
 Multiple DAQ partitions may operate simultaneously, but each instance operates independently}

%front-end computer
\newduneabbrev{fec}{FEC}{front-end computer}{The portion of one
  \gls{daqpart} that hosts the \gls{daqdr}, \gls{daqbuf} and
  \gls{daqds}.  It hosts the \gls{daqfer} and corresponding portion of the \gls{daqbuf}.}

%DAQ front-end fragment
\newduneword{daqfrag}{DAQ front-end fragment}{The portion of one
  \gls{daqpart} relating to a single \gls{fec} and corresponding to an
  integral number of \glspl{detunit}.  See also \gls{datafrag}}

%data fragment
\newduneword{datafrag}{data fragment}{A block of data read out from a single \gls{daqfrag} that
span a contiguous period of time as requested by a \gls{trigcommand}}

%DAQ front-end redout
\newduneabbrev{daqfer}{FER}{DAQ front-end readout}{The portion of a
  \gls{daqfrag} that accepts data from the detector electronics and
  provides it to the \gls{fec}.}

%DAQ data receiver
\newduneabbrev{daqdr}{DDR}{DAQ data receiver}{The portion of the
  \gls{daqfrag} that accepts data from the \gls{daqfer}, emits
  trigger candidates produced from the input trigger primitives, and
  forwards the full data stream to the \gls{daqbuf}}

%primary DAQ buffer
\newduneabbrev{daqbuf}{primary buffer}{DAQ primary buffer}{The portion
  of the \gls{daqfrag} that accepts full data stream from the
  corresponding \gls{detunit} and retains it sufficiently long for it
  to be available to produce a \gls{datafrag}}

%data selector
\newduneword{daqds}{data selector}{The portion of the \gls{daqfrag}
  that accepts \glspl{trigcommand} and returns the corresponding
  \gls{datafrag}.  Not to be confused with \gls{daqdsn}.}

\newduneword{daqdsn}{data selection}{The process of forming a trigger decision for selecting a subset of detector data for output by the DAQ from the content of the detector data itself.  Not to be confused with \gls{daqds}.}

\newduneabbrev{daqros}{RO}{DAQ readout sub-system}{The sub-system of the DAQ for accepting and buffering data input from detector electronics.}

\newduneabbrev{daqdss}{DS}{DAQ data selection sub-system}{The sub-system of the DAQ responsible for forming a trigger decision based on a portion of the input data stream.  The majority subset of the \gls{daqtrs}.}
\newduneabbrev{daqtrs}{TS}{DAQ trigger sub-system}{The sub-system of the DAQ responsible for forming a trigger decision.}
\newduneabbrev{daqbes}{BE}{DAQ back-end sub-system}{The portion of the DAQ which is generally toward its output end.  It is responsible for accepting and executing trigger commands and marshaling the data they address to output storage buffers.}
\newduneabbrev{daqtss}{TSS}{DAQ timing and synchronization sub-system}{The portion of the DAQ which provides for timing and synchronization to various components.}


%front-end mother board
\newduneabbrev{femb}{FEMB}{front-end mother board}{Refers a unit of
  the \gls{sp} cold electronics that contains the front-end amplifier
  and ADC ASICs covering 128 channels}

%application-specific integrated circuit
\newduneword{asic}{ASIC}{application-specific integrated circuit}

%low voltage
\newduneword{lv}{LV}{low voltage}

%iceberg
\newduneword{iceberg}{ICEBERG}{Integrated Cryostat and Electronics Built for Experimental Research Goals:
a new double wall cryostat build at Fermilab and installed in the Proton Assembly Building meant for 
liquid argon detector R\&D and for testing of DUNE detector components}

%coldadc
\newduneword{coldadc}{ColdADC}{newly developed 16-channels ASIC providing analog to digital conversion}

%coldata
\newduneword{coldata}{COLDATA}{a 64-channels control and communications ASIC}
%\newduneabbrev{coldata}{COLDATA}{a 64-channel control and communications ASIC}{A key component of the 128-channel \gls{femb} that provides a control and communication interface between cold \gls{lvds} channels and warm electronics external to the cryostat}

%cryo
\newduneword{cryo}{CRYO}{integrated ASIC including front-end circuitry providing signal amplification and pulse shaping, analog to digital conversion, and control and communication functionalities for 64 channels}

%liquid argon application-specific integrated circuit
\newduneword{larasic}{LArASIC}{A 16-channels front-end ASIC that provides signal amplification and pulse shaping}

%complementary metal-oxide-semiconductor
\newduneword{cmos}{CMOS}{Complementary metal-oxide-semiconductor}

%equivalent noise charge
\newduneword{enc}{ENC}{equivalent noise charge}

%equivalent number of bits
\newduneword{enob}{ENOB}{effective number of bits}

%dynamic range enhancement
\newduneword{dre}{DRE}{dynamic range enhancement}

%successive approximation register
\newduneword{sar}{SAR}{successive approximation register}

%protodune
\newduneword{protodune}{ProtoDUNE}{Either of the two DUNE prototype detectors constructed at CERN. % and operated in  a CERN test beam (expected fall 2018). 
  One prototype implements \single and the other \dual technology. }

%the single-phase ProtoDUNE detector
\newduneword{pdsp}{ProtoDUNE-SP}{The \single ProtoDUNE detector}

%the dual-phase ProtoDune detector
\newduneword{pddp}{ProtoDUNE-DP}{The \dual ProtoDUNE detector}

%WA105 dual-phase demonstrator
\newduneword{wa105}{WA105 DP demonstrator}{The \SI[product-units=power]{3x1x1}{m} WA105 dual-phase prototype detector at CERN}

%data aquisition event block --- includes dirty word: "event"
\newduneword{rawevent}{DAQ event block}{The unit of data output by the
  DAQ. 
  It contains trigger and detector data spanning a unique, contiguous
  time period and a subset of the detector channels.}

%solid-state disk
\newduneabbrev{ssd}{SSD}{solid-state disk}{Any storage device that
  may provide sufficient write throughput to receive, both collectively and
  distributed, the sustained full rate of data from a \gls{detmodule}
  for many seconds}

%high-level trigger --- fixme: this needs improvement
\newduneabbrev{hlt}{HLT}{high-level trigger}{This is actually a filter applied to data which has been triggered and aggregated in order to further reduce or characterize it.}

%particle identification
\newduneabbrev{pid}{PID}{particle ID}{Particle identification}

%readout window
\newduneword{readout window}{readout window}{A fixed, atomic and
  continuous period of time over which data from a \gls{detmodule}, in
  whole or in part, is recorded. 
  This period may differ based on the trigger that initiated the
  readout}

%zero-suppression
\newduneabbrev{zs}{ZS}{zero-suppression}{Used to delete some portion of a
  data stream that does not significantly deviate from zero or
  intrinsic noise levels. 
  It may be applied at different granularity from per-channel to per
  \dword{detunit}}

%run control ------- fixme: maybe another sentence
\newduneabbrev{rc}{RC}{run control}{The system for configuring,
  starting and terminating the DAQ}

\newduneabbrev{daqccm}{CCM}{DAQ control, configuration and monitoring sub-system}{A system for controlling, configuring and monitoring other systems in particular those that make up the DAQ where the CCM encompasses \gls{rc}.}

\newduneword{daqrun}{DAQ run}{A period of time over which relevant data taking conditions and DAQ configuration are asserted to be unchanged. 
  Multiple DAQ runs may occur simultaneously when multiple \glspl{daqpart} are active. 
  This term should not be confused with DUNE experiment or beam ``runs'' which typically span many DAQ runs.}
\newduneword{daqrunnum}{DAQ run number}{A monotonic increasing count which uniquely and globally identifies a \gls{daqrun}.}

%supernova neutrino burst  
\newduneabbrev{snb}{SNB}{supernova neutrino burst}{A prompt 
  increase in the flux of low-energy neutrinos emitted in the first few seconds of a core-collapse supernova.  It can also refer to a trigger command type that may be due to an SNB,
  or detector conditions that mimic its interaction signature}

%supernova burst and low energy
\newduneabbrev{snble}{SNB/LE}{supernova neutrino burst and low
  energy}{Supernova neutrino burst and low-energy physics program}

%supernova early warning system
\newduneabbrev{snews}{SNEWS}{SuperNova Early Warning System}{A global
  supernova neutrino burst trigger formed by a coincidence of SNB
  triggers collected from participating experiments}

%one pulse per second signal
\newduneabbrev{pps}{1PPS signal}{one-pulse-per-second signal}{An
  electrical signal with a fast rise time and that arrives in real
  time with a precise period of one second}

%spill location system
\newduneabbrev{sls}{SLS}{spill location system}{A system residing at
  the DUNE far detector site that provides information, possibly
  predictive, indicating periods of time when neutrinos are being
  produced by the \fnal Main Injector beam spills}

%warm interface board
\newduneabbrev{wib}{WIB}{warm interface board}{Digital electronics
  situated just outside the \single cryostat that receives digital data
  from the FEMBs over cold copper connections and sends it to the RCE
  FE readout hardware}

\newduneabbrev{gps}{GPS}{Global Positioning System}{A satellite-based system that provides a highly accurate \gls{pps} which may be used to synchronize clocks and determine the location.}

\newduneabbrev{ntp}{NTP}{Network Time Protocol}{A networking protocol which allows synchronizing of clocks to within a few \si{\milli\second} of a time standard on a local network and within a few tens of \si{\milli\second} over the Internet.} 

\newduneabbrev{irig}{IRIG}{inter-range instrumentation group}{A standards body which defined a time code standard for transferring timing information.}

%network interfce controller
\newduneabbrev{nic}{NIC}{network interface controller}{Hardware for controlling the interface to a communication network.  Typically, one that obeys the Ethernet protocol}

%warm interface electronics crate
\newduneabbrev{wiec}{WIEC}{warm interface electronics crate}{Crates mounted on the signal flanges that contain the warm interface boards}

%power and timing cards
\newduneabbrev{ptc}{PTC}{power and timing cards}{Cards that provide further processing and distribution of the signals entering and exiting the \single cryostat}
\newduneabbrev{ptb}{PTB}{power and timing backplane}{Backplane used to connect the \dwords{wib} and the \dword{ptc} on the \dword{wiec}. Also connects the \dword{ce} flange on the cryostat penetration.}

%silicon photomultipler
\newduneabbrev{sipm}{SiPM}{silicon photomultiplier}{A solid-state
  avalanche photodiode sensitive to single \phel signals}

%cryogenic instrumentation and slow control
\newduneabbrev{cisc}{CISC}{cryogenic instrumentation and slow controls}{A DUNE
  consortium responsible for the cryogenic instrumentation and slow controls components}

%FTE
\newduneword{fte}{FTE}{Full Time Equivalent. A unit of labor
  for the project. One year of work from one person.}

%consortium
\newduneword{consortium}{consortium}{A unit of organization in the
  DUNE project focused on one major component of the far detector}

%art 
\newduneword{art}{\textit{art}}{A software framework implementing an
  event-based execution paradigm} %http://art.fnal.gov/

%sequential access via metadata  
\newduneabbrev{sam}{SAM}{sequential
  access via metadata}{A data-handling system to store and retrieve
  files and associated metadata, including a complete record of the
  processing that has used the files}

%art data aquisition
\newduneword{artdaq}{\textit{art}DAQ}{A general-purpose \fnal data acquisition framework for distributed data combination and processing. 
It is designed to exploit the parallelism that is possible with
modern multi-core and networked computers}

%big-O notation
\newduneword{order}{$\mathcal{O}(n)$}{of order $n$}

%\newduneword{3d}{3D}{3 dimensional (also 1D, 2D, etc.)} % not phys 
%(Use defs.tex version}
%.........

%beamline
\newduneword{beamline}{beamline}{A sequence of control and monitoring devices used for the formation of a directed collection of particles}
%conceptual design report
\newduneabbrev{cdr}{CDR}{conceptual design report}{A formal project
  document %required by funding agencies
   that describes the experiment
  at a conceptual level}

%conventional facilities
\newduneabbrev{cf}{CF}{conventional facilities}{Pertaining to
  construction and operation of buildings or caverns and conventional infrastructure}

%charge parity
\newduneabbrev{cp}{CP}{charge parity}{Product of charge and parity
  transformations}

%product of charge, parity and time-reversal
\newduneword{cpt}{CPT}{product of charge, parity
  and time-reversal transformations}

%charge-parity symmetry violation
\newduneabbrev{cpv}{CPV}{charge-parity symmetry violation}{Lack of
  symmetry in a system before and after charge and parity
  transformations are applied}

%us department of energy
\newduneword{doe}{DOE}{U.S. Department of Energy}

\newduneword{us}{U.S.}{United States of America}

%deep underground neutrino experiment
\newduneword{dune}{DUNE}{Deep Underground Neutrino Experiment}

%environment, safety and health
\newduneword{esh}{ES\&H}{Environment, Safety and Health}

\newduneword{ppe}{PPE}{Personnel Protective Equipment}

\newduneword{odh}{ODH}{Oxygen Deficiency Hazard}

\newduneword{feshm}{FESHM}{Fermilab Environmental and Safety Manual}

%fine-grained tracker
\newduneabbrev{fgt}{FGT}{fine-grained tracker}{A near detector module ??}

%far site conventional facilities
\newduneabbrev{fscf}{FSCF}{far site conventional facilities}{The
  \gslpl{cf} at the DUNE far detector site, \surf}
  
%near site conventional facilities
\newduneabbrev{nscf}{NSCF}{near site conventional facilities}{The
  \gslpl{cf} at the DUNE near detector site, \fnal}

%grand unified theory
\newduneabbrevs{gut}{GUT}{grand unified theory}{grand unified theories}{A class of theories that unifies the electro-weak and strong forces}

%liquid argon
\newduneabbrev{lar}{LAr}{liquid argon}{The liquid phase of argon}

%liquid argon time-projection chamber
\newduneabbrev{lartpc}{LArTPC}{liquid argon time-projection chamber}{A
  class of detector technology that forms the basis for the DUNE far
  detector modules. 
  It typically entails observation of ionization activity by
  electrical signals and of scintillation by optical signals}

%long-baseline
\newduneabbrev{lbl}{LBL}{long-baseline}{Refers to the distance between the 
  neutrino source  and the far detector.  It can also refer to the distance between the near and far detectors. 
  The ``long'' designation is an approximate and relative distinction. For DUNE, this distance  (between \fnal and \surf) is approximately \SI{1300}{km}}

%long-baseline neutrino facility
\newduneabbrev{lbnf}{LBNF}{Long-Baseline Neutrino Facility}{The
  organizational entity responsible for developing the neutrino beam, the cryostats
  and cryogenics systems, and the conventional facilities for DUNE}

\newduneabbrev{lbnc}{LBNC}{Long-Baseline Neutrino Committee}{The
  organizational entity responsible for overseeing the LBNF and DUNE projects. The committee reports to the \fnal director.}

\newduneabbrev{ncg}{NCG}{Neutrino Cost Group}{The
  organizational entity responsible for reviewing and scrutinizing the DUNE project cost. The committee reports to the \fnal director.}

%mass hierarchy
\newduneabbrev{mh}{MH}{mass hierarchy}{Describes the separation
  between the mass squared differences related to the solar and
  atmospheric neutrino problems}

%fnal main injector
\newduneabbrev{mi}{MI}{\fnal Main Injector}{An accelerator at
  \fnal that provides a beam of high-energy protons that upon
  striking a target produce secondaries that decay to provide the
  neutrinos directed toward the DUNE far detector}

%protons on target
\newduneabbrev{pot}{POT}{protons on target}{Typically used as a unit
  of normalization for the number of protons striking the neutrino
  production target}

%quality assurance
\newduneabbrev{qa}{QA}{quality assurance}{The set of actions taken to provide confidence that quality requirements are fulfilled, and to detect and correct poor results}

%quality control
\newduneabbrev{qc}{QC}{quality control}{An aggregate of activities (such as design analysis and inspection for defects) performed to ensure adequate quality in manufactured products}

%standard model
\newduneabbrev{sm}{SM}{Standard Model}{Refers to a theory describing
  the interaction of elementary particles}

%technical design report
\newduneabbrev{tdr}{TDR}{technical design report}{A formal project
  document %required by funding agencies 
  that describes the experiment at a technical level}

%interim design report
\newduneabbrev{tp}{IDR}{interim design report}{An intermediate
milestone on the path to a full \gls{tdr}} % changed from ``technical proposal'' 6/6/2018

%%%%%%%%%%%%% PROJECT AND PHYSICS VOLUME list for acronyms below %%%%%%%%%%%%
\newduneabbrev{ckm}{CKM matrix}{Cabibbo-Kobayashi-Maskawa
  matrix}{Refers to the matrix describing the mixing between mass and
  weak eigenstates of quarks}

\newduneabbrev{cl}{CL}{confidence level}{Refers to a probability
  used to determine the value of a random variable given its
  distribution}

\newduneabbrev{pmns}{PMNS}{Pontecorvo-Maki-Nakagawa-Sakata}{A type of matrix that describes the mixing between mass and weak eigenstates of
  the neutrino}


%%%%%%%%%%%%%%%%%.....................

%%%%%%%%%%%%% PROJECT AND DETECTORS VOLUME list for acronyms below %%%%%%%%%%%%

% fixme: should not have degenerate definition.  This also should be an abbrev.
%\newduneword{blm}{BLM}{(in Volume 4) beamline measurement (system); (in Volume 3) beam loss monitor} (not used -- yet! Anne 10 May 2018)
%omit trailing period in newduneabbrev(s) and newduneword, glossaries package will append the end of sentence period - Ddm

\newduneabbrevs{cpa}{CPA}{cathode plane assembly}{cathode plane assemblies}{The component of the \single detector module that provides the drift HV cathode}

\newduneabbrev{fc}{FC}{field cage}{The component of a LArTPC that contains and shapes the applied \efield}

\newduneword{cpafc}{CPA/FC}{A pair of \gls{cpa} panels and the top and bottom \gls{fc} portions that attach to the pair; an intermediate assembly for installation into the \gls{spmod} }

\newduneabbrev{topfc}{top FC}{top field cage}{The horizontal portions of the \single FC on the top}

\newduneabbrev{botfc}{bottom FC}{bottom field cage}{The horizontal portions of the \single FC on the bottom}

\newduneabbrev{ewfc}{endwall FC}{endwall field cage}{The vertical portions of the \single FC near the wall}

\newduneabbrev{gp}{GP}{ground plane}{An electrode that is held to be
  electrically neutral relative to Earth ground voltage}

  \newduneword{gg}{ground grid}{An electrode that is held to be
  electrically neutral relative to Earth ground voltage  ??}


\newduneabbrev{alara}{ALARA}{as low as reasonably
  achievable}{Typically used with regard management of radiation
  exposure but may be used more generally. It means making every
  reasonable effort to maintain e.g., exposures, to as far below the
  limits as practical, consistent with the purpose for that the
  activity is undertaken}

\newduneabbrev{ecal}{ECAL}{electromagnetic calorimeter}{A detector
  component that measures energy deposition of traversing particles}

\newduneabbrev{hv}{HV}{high voltage}{Generally describes a voltage
  applied to drive the motion of free electrons through some media}

% can also use in the text: \dword{sp} \dword{detmodule} 
\newduneword{spmod}{SP module}{single-phase detector module}
\newduneword{dpmod}{DP module}{dual-phase detector module}
\newduneword{dsp}{DUNE-SP}{The \single DUNE detector}
\newduneword{ddp}{DUNE-DP}{The \dual DUNE detector}

\newduneabbrev{tcoord}{TC}{technical coordinator}{Responsible for organization of
the project effort. Is a member of the \dword{dune} management team}

\newduneabbrev{rcoord}{RC}{resource coordinator}{Responsible for financing of
the project effort. Is a member of the \dword{dune} management team}

\newduneword{tc}{TCN}{technical coordination}

\newduneabbrev{exb}{EB}{executive board}{The highest level DUNE
  decision making body for the collaboration}

\newduneabbrev{tb}{TB}{technical board}{The DUNE organization responsible for
  evaluating technical decisions}

\newduneabbrev{rrb}{RRB}{resource review board}{The organization of DUNE funding agencies responsible for funding decisions}


%%%%%%%%%%%%% PHYSICS AND DETECTORS VOLUME list for acronyms below %%%%%%%%%%%%

\newduneabbrev{cc}{CC}{charged current}{Refers to an interaction
  between elementary particles where a charged weak force carrier
  ($W^+$ or $W^-$) is exchanged}


\newduneabbrev{dis}{DIS}{deep inelastic scattering}{Refers to
  interaction of an elementary charged particle with a nucleus in an
  energy range where the interaction can be modeled as being with
  individual nucleons}

\newduneabbrev{fsi}{FSI}{final-state interactions}{Refers to
  interactions between elementary or composite particles subsequent to
  the initial, fundamental particle interaction, such as may occur as
  the products exit a nucleus}

\newduneword{geant4}{Geant4}{A
  software toolkit for the simulation of the passage of particles
  through matter using Monte Carlo methods}

\newduneabbrev{genie}{GENIE}{Generates Events for Neutrino Interaction
  Experiments}{Software providing an object-oriented neutrino
  interaction simulation resulting in kinematics of the products of
  the interaction}

\newduneabbrev{mc}{MC}{Monte Carlo}{Refers to a method of numerical
  integration that entails the statistical sampling of the integrand
  function. 
  Forms the basis for some types of detector and physics simulations}

\newduneabbrev{qe}{QE}{quasi-elastic}{Refers to interaction between
  elementary particles and a nucleus in an energy range where the
  interaction can be modeled as occurring between constituent quarks
  of one nucleon and resulting in no bulk recoil of the resulting
  nucleus}

%%%%%%%%%%%%%%%%%%%%%%%%% PROJECT VOLUME list for acronyms below %%%%%%%%%%%%%%%

\newduneabbrev{mou}{MoU}{memorandum of understanding}{A document
  summarizing an agreement between two or more parties}

\newduneabbrev{pip2}{PIP-II}{Proton Improvement Plan II}{A plan for
  improving the protons on target delivered for the DUNE neutrino
  production beam. 
  This is revision two of this plan and is planned to be followed by a PIP-III}

\newduneabbrev{sdsta}{SDSTA}{South Dakota Science and Technology
  Authority}{The legal entity that manages the Sanford Underground
  Research Facility, \surf, in Lead, S.D}
  
\newduneabbrev{sdsd}{SDSD}{The Fermilab South Dakota Services Division}{A Fermilab division responsible providing host laboratory functions at the far site}

\newduneabbrev{firus}{FIRUS}{Facility Information Reporting Utility System}
 {The safety system at \surf}

\newduneabbrev{bsi}{BSI}{Building and Site Infrastructure}
 {The work package for outfitting of the \dword{lbnf} underground infrastructure.}



\newduneabbrev{wbs}{WBS}{work breakdown structure}{An organizational
  project management tool by which the tasks to be performed are
  partitioned in a hierarchical manner}

%%%%%%%%%%%%%%%%%%%%%%%%% PHYSICS VOLUME list for acronyms below %%%%%%%%%%%%%%%
\newduneabbrev{br}{BR}{branching ratio}{A fractional probability for a
  decay of a composite particle to occur into some specified set or
  sets of products}
\newduneword{bsm}{BSM}{beyond the standard model}

\newduneabbrev{dm}{DM}{dark matter}{The term given to the unknown
  matter or force that explains measurements of motion of galaxies
  that are otherwise inconsistent with the amount of mass associated
  with observed amount of photon production}
  
  \newduneabbrev{bdm}{BDM}{dark matter}{Add definition}

\newduneabbrev{cern}{CERN}{European Organization for Nuclear
Research}{In French, Organisation europ\'{e}enne pour la recherche
nucl\'{e}aire, the leading particle physics laboratory in Europe and home
to the ProtoDUNEs}


\newduneabbrev{dsnb}{DSNB}{Diffuse Supernova Neutrino Background}{The
  term describing the pervasive, constant flux of neutrinos due to all
  past supernova neutrino bursts}

\newduneabbrev{espp}{ESPP}{European Strategy for Particle Physics}{The
European Strategy for Particle Physics is the cornerstone of Europe's
decision-making process for the long-term future of the
field. Mandated by the CERN Council, it is formed through a broad
consultation of the grass-roots particle physics community, it
actively solicits the opinions of physicists from around the world,
and it is developed in close coordination with similar processes in
the US and Japan in order to ensure coordination between regions and
optimal use of resources globally.}

\newduneabbrev{gartpc}{GArTPC}{gaseous argon time-projection
chamber}{A possible technology choice for the \dword{nd}.}


\newduneabbrev{globes}{GLoBES}{General Long-Baseline Experiment
  Simulator}{A software package for simulating energy spectra of
  neutrino flux, interaction and measured (after application of some
  model of a detector response) energy spectra}

\newduneword{snowglobes}{SNOwGLoBES}{SuperNova
Observatories with \dword{globes}. From the official description~\cite{snowglobes}: 
SNOwGLoBES is public software for computing interaction rates and distributions of observed quantities for supernova burst neutrinos in common detector materials. The intent is to provide a very simple and fast code and data package which can be used for tests of observability of physics signatures in current and future detectors, and for evaluation of relative sensitivities of different detector configurations. The event estimates are made using available cross-sections and parameterized detector responses. Water, argon, scintillator and lead-based configurations are included. The package makes use of GLoBES front-end software. SNOwGLoBES is not intended to replace full detector simulations; however output should be useful for many types of studies}


% are these really used anywhere?
%\newduneword{l/e}{L/E}{length-to-energy ratio}
%\newduneword{lri}{LRI}{long-range interactions}
%\newduneword{solarmass}{$M_{\odot}$}{solar mass}

\newduneabbrev{nc}{NC}{neutral current}{Refers to an interaction
  between elementary particles where a neutrally charged weak force carrier
  ($Z^0$) is exchanged}

\newduneabbrev{nh}{NH}{normal hierarchy}{Refers to the neutrino mass
  eigenstate ordering whereby the sign of the mass squared difference
  associated with the atmospheric neutrino problem is positive}

\newduneabbrev{ih}{IH}{inverted hierarchy}{Refers to the neutrino mass
  eigenstate ordering whereby the sign of the mass squared difference
  associated with the atmospheric neutrino problem is negative}

\newduneabbrev{msw}{MSW}{Mikheyev-Smirnov-Wolfenstein effect}{Explains
  the oscillatory behavior of neutrinos produced inside the sun as
  they traverse the solar matter}

\newduneabbrev{nsi}{NSI}{nonstandard interactions}{A general class of
  theory of elementary particles other than the Standard Model}



\newduneabbrev{pfive}{P5}{Particle Physics Project Prioritization
Panel}{The Particle Physics Project Prioritization Panel (P5) was a
subpanel of the High Energy Physics Advisory Panel (HEPAP). It completed
its Report, a ten-year strategic plan for high energy physics in the
U.S., in 2014. This report included a recommendation that ``host a world-leading neutrino
program that will have an optimized set of short- and long-baseline neutrino oscillation experiments, and its long-term focus
is a reformulated venture referred to here as the Long Baseline
Neutrino Facility (LBNF)''}

\newduneword{sme}{SME}{Standard-Model Extension}

\newduneabbrev{susy}{SUSY}{supersymmetry}{Theoretical symmetry between a fermion and a boson}

\newduneabbrev{wimp}{WIMP}{weakly-interacting massive particle}{A
  hypothesized particle that may be a component of dark matter}

%%%%%%%%%%%%%%%%%%%%%%%%% DETECTORS VOLUME list for acronyms below %%%%%%%%%%%%%%%

\newduneabbrev{ce}{CE}{cold electronics}{Refers to readout electronics that operate at cryogenic temperatures}

\newduneabbrev{crp}{CRP}{charge-readout plane}{In the \dual technology, a  collection of
  electrodes in a planar arrangement placed at a particular voltage
  relative to some applied \efield such that drifting electrons
  may be collected and their number and time may be measured}

\newduneabbrev{dram}{DRAM}{dynamic random access memory}{A computer memory technology}

% these should be abbrev and maybe combined.
\newduneword{fermilab}{\fnal}{Fermi National Accelerator Laboratory (in Batavia, IL, the Near Site)}
\newduneword{fnal}{\fnal}{Fermi National Accelerator Laboratory (in Batavia, IL, the Near Site)}
%should probably get rid of either fnal or fermilab here.
\newduneword{bnl}{BNL}{Brookhaven National Laboratory (in Upton, NY)}
\newduneword{slac}{SLAC}{SLAC National Accelerator Laboratory (in Menlo Park, CA)}
\newduneword{lbnl}{LBNL}{Lawrence Berkeley National Laboratory (in Berkeley, CA)}

\newduneabbrev{fs}{FS}{full stream}{Relates to a data stream that has not undergone selection, compression or other form of reduction}

\newduneabbrev{lem}{LEM}{large electron multiplier}{A micro-pattern detector suitable for use in ultra-pure argon vapor; LEMs consist of copper-clad PCB boards with sub-millimeter-size holes through which electrons undergo amplification}
%\newduneabbrev{lem}{LEM}{large electron multiplier}{A technique of electron multiplication via avalanche in sub-millimeter-size holes in a
%millimeter-thick printed circuit board where electric fields are understood}
\newduneabbrev{lng}{LNG}{liquefied natural gas}{Pertaining to natural gas in its liquid phase}

% this is pretty generic and isn't referenced.
%\newduneword{mesh}{mesh screen}{A fine mesh screen, glued directly to
%  the steel frame on both sides of each APA in the single-phase TPC,
% creates a uniform ground layer beneath the wire planes}

\newduneabbrev{mip}{MIP}{minimum ionizing particle}{Refers to a
  momentum traversing some medium such that the particle is losing
  near the minimum amount of energy per distance traversed} % some \mip and some \dword{mip}. If time, rectify. ??

%\newduneword{abc}{MTS}{Materials Test Stand}
%\newduneword{muid}{MuID}{muon identifier (detector)}

%\newduneword{abc}{OPERA}{Oscillation Project with Emulsion-Racking Apparatus (experiment at LNGS)}
%\newduneword{abc}{NND}{(used only in Volume 4 Chapter 7) near neutrino detector, same as ND}
%\newduneword{abc}{OD}{outer diameter}

% there is also pds
\newduneabbrev{pd}{PD}{photon detector}{Refers to the detector
  elements involved in measurement of number and arrival times of
  optical photons produced in a detector module} %volume}

\newduneabbrev{pmt}{PMT}{photomultiplier tube}{A device that makes use
  of the photoelectric effect to produce an electrical signal from the
  arrival of optical photons}

\newduneabbrev{ppm}{ppm}{parts per million}{A number equal to $10^{-6}$}
\newduneabbrev{ppb}{ppb}{parts per billion}{A number equal to $10^{-9}$}
\newduneabbrev{ppt}{ppt}{parts per trillion}{A number equal to $10^{-12}$}

\newduneword{rio}{RIO}{reconfigurable input output}

\newduneword{rpc}{RPC}{resistive plate chamber}
\newduneword{s/n}{S/N}{signal-to-noise (ratio)}
\newduneword{ssp}{SSP}{SiPM signal processor}
\newduneword{sbn}{SBN}{Short-Baseline Neutrino program (at \fnal)}
\newduneword{stt}{STT}{straw tube tracker}
%\newduneword{abc}{SURF (also Sanford Lab)}{Sanford Underground Research Facility (in Lead, SD, the Far Site)}
\newduneword{tr}{TR}{transition radiation}
%\newduneword{abc}{W}{Watt (also mW, kW, MW) }
%\newduneword{abc}{WA105}{Single-Phase LArTPC and the Long Baseline Neutrino Observatory Demonstration}
\newduneword{wire board}{wire board}{At the head end of the APA in the \single TPC, stacks of electronics boards referred to as ``wire boards'' are arrayed to anchor the wires.  They also provide the connection between the wires and the cold electronics} %?? long for a word. ??

\newduneabbrev{wls}{WLS}{wavelength shifting}{A material or process by
  which incident photons are absorbed by a material and photons are
  emitted at a different, typically longer, wavelength}

\newduneabbrev{tpb}{TPB}{tetra-phenyl butadiene}{A type of wavelength shifting material}

\newduneabbrev{sft}{SFT}{signal feedthrough}{A cryostat penetration allowing for the passage of cables or other extended parts}
\newduneabbrev{sftchimney}{SFT chimney}{signal feedthrough chimney}{In the \dual technology, a volume above the cryostat penetration used for a signal feedthrough}


\newduneword{catiroc}{CATIROC}{charge and time integrated readout chip}

\newduneabbrev{wr}{WR}{White Rabbit}{A component of the timing system that forwards clock signal and time-of-day reference data to the master timing unit}

\newduneabbrev{mch}{MCH}{MicroTCA Carrier Hub}{An network switching device}

\newduneabbrev{wrmch}{WR-MCH}{White Rabbit \gls{utca} Carrier Hub}{add def} %??

\newduneword{cmp}{CMP}{configuration management plan}

\newduneword{qap}{QAP}{quality assurance plan} %{A project management device for planning \gls{qa}}

\newduneword{ieshp}{IESHP}{integrated environmental, safety and health plan}%{Refers to the LBNF/DUNE project planning instrument}

\newduneword{dmp}{DMP}{data management plan} %{A project management device to state how the experimental data will be managed}

\newduneword{qam}{QAM}{quality assurance manager} %{The manager of \gls{qa} for the LBNF/DUNE project}

\newduneabbrev{dss}{DSS}{detector support system}{The system used to support the \single detector within the cryostat}

\newduneabbrev{ddss}{DDSS}{\dword{dune} detector safety system}{The system used to manage key aspects of detector safety}

\newduneabbrev{itf}{ITF}{integration and test facility}{A facility where various detector components will be tested prior to installation}

\newduneabbrev{lc}{LC}{logistics center}{A facility where \dword{lbnf} and \dword{dune} components will be received and transhipped to \dword{surf}}

\newduneabbrev{tco}{TCO}{temporary construction opening}{An opening in the side of a cryostat through which detector elements are brought into the cryostat; utilized during construction and installation}

\newduneabbrev{surf}{SURF}{Sanford Underground Research Facility}{The laboratory in South Dakota where \dword{lbnf} will be constructed}

\newduneabbrev{sit}{SIT}{surface installation team}{An organizational unit responsible for logistics and integration in South Dakota}

\newduneabbrev{uit}{UIT}{underground installation team}{An organizational unit responsible for installation in the underground area at the \surf site}

\newduneabbrev{cmgc}{CMGC}{construction manager/general contractor}{The organizational unit responsible for management of the construction of conventional facilities at the underground area at the \surf site}

\newduneabbrev{cdrev}{CDR}{conceptual design review}{A project management device by which a conceptual design is reviewed}
\newduneabbrev{pdr}{PDR}{preliminary design review}{A project management device by which an early design is reviewed}
\newduneabbrev{fdr}{FDR}{final design review}{A project management device by which a final design is reviewed}
\newduneabbrev{prr}{PRR}{production readiness review}{A project management device by which the production readiness is reviewed}
\newduneabbrev{orr}{ORR}{operational readiness review}{A project management device by which the operational readiness is reviewed}
\newduneabbrev{ppr}{PPR}{production progress review}{A project management device by which the progress of production is reviewed}
\newduneabbrev{edms}{EDMS}{engineering document management system}{A computerized system deveolped at CERN by which documents, drawings and models are managed}

\newduneword{wrgm}{WR grandmaster}{White Rabbit grandmaster}


%%%%% Software and computing %%%%

\newduneword{larsoft}{\larsoft}{Liquid Argon Software (\larsoft),  a shared base of physics software across \lartpc experiments}
\newduneword{nova}{\nova}{The \nova off-axis neutrino oscillation experiment at \fnal }
\newduneword{minerva}{\minerva}{The \minerva neutrino cross sections experiment at \fnal }
\newduneword{microboone}{\microboone}{The \lartpc-based \microboone neutrino oscillation experiment at \fnal }
\newduneword{sbnd}{SBND}{The Short-Baseline Near Detector experiment at \fnal}
\newduneword{nexo}{nEXO}{Enriched Xenon Observatory}
\newduneword{argoneut}{ArgoNeuT}{The ArgoNeuT test-beam experiment and \dword{lar} \dword{tpc} prototype at \fnal}
\newduneword{wirecell}{wire-cell}{A tomographic automated \threed neutrino event reconstruction method for \lartpc{}s}
\newduneword{ftslite}{F-FTS-lite}{Light-weight version of the \fnal File Transfer system used for rapid data transfers out of the online systems}
\newduneword{fts}{FTS}{File Transfer System developed at \fnal to catalog and move data to permanent storage}

%%% new ones that I haven't categorized (Anne)
\newduneword{35t}{35 ton prototype}{The 35 ton prototype cryostat and \gls{sp} detector built at \fnal before the \gls{protodune} detectors}

\newduneabbrev{cuc}{CUC}{central utility cavern}{The central underground cavern  containing utilities such as central cryogenics and other systems, and the underground data center and control room}
\newduneabbrev{cfd}{CFD}{computational fluid dynamics}{High performance computer-assisted modeling of fluid dynamical systems}
\newduneword{vuv}{VUV}{vacuum ultra-violet}
\newduneword{tallbo}{TallBo}{A cylindrical cryostat at \fnal primarily used for developing scintillation light collection technologies for \lartpc detectors}

\newduneword{root}{ROOT}{add definition}

\newduneabbrev{eos}{EOS}{EOS}{The XRootD based distributed file system developed by CERN}
\newduneabbrev{ehn1}{EHN1}{Experiment Hall North One}{Location at CERN of the ProtoDUNE experiments}
\newduneword{led}{LED}{Light-emitting diode}
\newduneword{rtd}{RTD}{Cryostat level and temperature monitoring devices}
\newduneword{swc}{SWC}{Software \& Computing}
\newduneabbrev{las}{LAS}{LEM-anode Sandwich}{In the \dual technology, a \gls{lem} and its corresponding anode are mounted together in a module called a LEM-anode sandwich}

\newduneword{roi}{ROI}{region of interest}
\newduneword{hpc}{HPC}{high-performance computing facilities; generally computing facilities emphasizing parallel computing with aggregate power of more than a teraflop}


\newduneword{comfund}{common fund}{The shared resources of the collaboration}
\newduneabbrev{ims}{IMS}{integrated master schedule}{A project management device consisting of linked tasks and milestones}

\newduneword{hvdb}{HVDB}{dual-phase HV divider board}
\newduneword{sas}{SAS}{A pass-through chamber used to ensure safe transfer of materials, avoiding contamination in both directions}

\newduneword{fea}{FEA}{finite element analysis}

\newduneword{fss}{FSS}{field shaping strips}
\newduneword{lvds}{LVDS}{low-voltage differential signaling}



%electrostatic discharge
\newduneword{esd}{ESD}{electrostatic discharge}%{ESD is the sudden flow of electricity between two electrically charged objects caused by contact, an electrical short, or dielectric breakdown. ESD can cause failure of electronic components such as integrated circuits}

\newduneabbrev{rp}{RP}{resistive panel}{Resistive panels form the constant potential surfaces for a \gls{spmod} \gls{cpa}; they are composed of a thin layer of carbon-impregnated Kapton and laminated to both sides of a \frfour sheet. }

\newduneword{uhmwpe}{UHMWPE}{ultra-high molecular weight polyethylene}

\newduneword{cts}{CTS}{Cryogenic Test System}
\newduneword{plc}{PLC}{programmable logic controller}

\newduneword{mppc}{MPPC}{\SI{6}{mm}$\times$\SI{6}{mm} Multi-Pixel Photon Counters produced by Hamamatsu\texttrademark{} Photonics K.K.}

\newduneabbrev{sfp}{SFP}{small form-factor pluggable (SFP)}{a particular standard for optical transceivers}

\newduneabbrev{minipod}{MiniPOD}{miniature parallel optical device}{a family of types of multi-channel optical transceivers}

\newduneword{ccc}{CCC}{configuration change command}
\newduneword{act}{ACT}{activation time stamp}
\newduneword{lcm}{LCM}{light calibration module}
\newduneword{lpm}{LPM}{light pulser module}
\newduneword{dac}{DAC}{digital-to-analog converter}
\newduneword{sarapu}{S-ARAPUCA}{ARAPUCA design with different wavelength shifter coatings on both faces of the dichroic filter window(s) of the cell}
\newduneword{xarapu}{X-ARAPUCA}{ARAPUCA design with different wavelength shifter coating on only the external face of the dichroic filter window(s) but with a wavelength shifter doped plate inside the cell}
\newduneword{feb}{FEB}{front-end board}

\newduneword{gdml}{GDML}{needs def}
\newduneword{lsnd}{LSND}{needs def}

\newduneword{cvn}{CVN}{convolutional visual network}
\newduneword{pandora}{Pandora}{needs def - software development kit...}

%Lisa added
\newduneabbrev{pma}{PMA}{Projection Matching Algorithm}{A reconstruction algorithm that combines \twod reconstructed objects to form a \threed representation}
\newduneabbrev{bdt}{BDT}{Boosted Decision Tree}{A method of multivariate analysis}
\newduneabbrev{cnn}{CNN}{Convolutional Neural Network}{A deep learning technique most commonly applied to analyzing visual imagery}
\newduneword{pdg}{PDG}{Particle Data Group}

% from CISC
\newduneword{pci}{PCI}{Peripheral Component Interconnect}

\newduneword{labview}{LabVIEW}{Laboratory Virtual Instrument Engineering Workbench is a system-design platform and development environment for a visual programming language from National Instruments}

\newduneword{pcb}{PCB}{Printed Circuit Board}

\newduneword{crio}{cRIO}{Compact Reconfigurable Input Output}

\newduneword{dcs}{DCS}{Distributed Communications System}

\newduneword{opc-ua}{OPC-UA}{OPC  Unified Architecture is a machine to machine communication protocol for industrial automation developed by the OPC Foundation. OPC stands for Object Linking and Embedding for Process Control}

\newduneword{cabangle}{Cabibbo angle}{add definition- from nu osc 01 or 02}
\newduneword{valor}{VALOR}{add def (from nu-osc 03)}
\newduneword{cafana}{CAFAna}{add def (from nu-osc 03)}
\newduneabbrev{pca}{PCA}{principal component analysis}{add def (from nu-osc 03)}
\newduneword{numi}{NuMI}{a set of facilities, collectively called ``Neutrinos at the Main Injector.''  The NuMI neutrino beamline target system converts an intense proton beam into a focused neutrino beam}
\newduneword{gibuu}{GiBUU}{Giessen Boltzmann-Uehling-Uhlenback Project; a unified theory and transport framework in the MeV and GeV energy regimes for elementary reactions on nuclei }
\newduneword{crpa}{CRPA}{add def (from nu-osc 05)}
\newduneword{rpa}{RPA}{add def (from nu-osc 05)}
\newduneword{t2k}{T2K}{T2K (Tokai to Kamioka) is a long-baseline neutrino experiment in Japan studying neutrino oscillations }
\newduneword{mptdet}{MPT detector}{multipurpose tracking detector}
\newduneword{hpg}{HPG}{high-pressure gas? add definition- from nu osc 06}
\newduneword{rhc}{RHC}{add def nu osc 07}
\newduneword{fhc}{FHC}{add def nu osc 07}
\newduneword{tdc}{TDC}{add def nu osc 07}
\newduneword{lariat}{LArIAT}{add def (from nu-osc 09)}
\newduneword{captain}{CAPTAIN}{add def (from nu-osc 09)}
\newduneword{dayabay}{Daya Bay}{add def (from nu-osc 10)}
\newduneword{nuwro}{NuWro}{neutrino interaction generatoradd def (from nu-osc 11)}
\newduneword{neut}{NEUT}{neutrino interaction generator, add def (from nu-osc 11)}
\newduneword{minos}{MINOS}{add def}
\newduneword{mpt}{MPT}{add def}


\newduneword{efig}{EFIG}{Experimental Facilities Interface Group}
\newduneabbrev{jpo}{JPO}{joint project office}{add def (tc vol, ch 2)}

\newduneword{ifbeam}{ifbeam}{Database that stores beamline information
indexed by timestamp}

\newduneabbrev{marley}{MARLEY}{Model of Argon Reaction Low Energy
Yields}{Developed at UC Davis, MARLEY is the first realistic model of
neutrino electron interactions on argon for enegies less than 50
MeV. This includes the energy range important for supernova burst
neutrinos and also solar 8--Boron neutrinos}

\newduneabbrev{es}{ES}{elastic scattering}{Events in which a neutrino
elastically scatters off of another particle}

